\documentclass[leqno,12pt]{article}
\setlength{\textheight}{23cm}
\setlength{\textwidth}{16cm}
\setlength{\oddsidemargin}{0cm}
\setlength{\evensidemargin}{0cm}
\setlength{\topmargin}{0cm}

\usepackage{hyperref}

\usepackage{amsmath,amssymb}
\usepackage{amsthm}

\def\address#1#2{\begingroup
\noindent\parbox[t]{9cm}{%
\small{\scshape\ignorespaces#1}\par\vskip1ex
\noindent\small{\itshape E-mail address}%
\/: #2\par\vskip4ex}\hfill%
\endgroup}%

\renewcommand{\baselinestretch}{1.2}
\renewcommand{\thefootnote}{}

\theoremstyle{plain}
\newtheorem{theorem}{\indent\sc Theorem}[section]
\newtheorem{lemma}[theorem]{\indent\sc Lemma}
\newtheorem{corollary}[theorem]{\indent\sc Corollary}
\newtheorem{proposition}[theorem]{\indent\sc Proposition}
\newtheorem{conjecture}[theorem]{\indent\sc Conjecture}

\newtheorem{maintheorem}{Theorem}
\renewcommand*{\themaintheorem}{\Roman{maintheorem}}

\theoremstyle{definition}
\newtheorem{definition}[theorem]{\indent\sc Definition}
\newtheorem{remark}[theorem]{\indent\sc Remark}

%%%%% Proof %%%%%
\renewcommand{\proofname}{\indent\sc Proof.}

\pagestyle{myheadings} \markright{Weil-\'{e}tale cohomology and duality for
  arithmetic schemes in negative weights}

\title{\uppercase{Weil-\'{e}tale cohomology and duality for arithmetic schemes
    in negative weights}}

\author{\textsc{Alexey Beshenov}}
\date{}

%%%%%%%%%%%%%%%%%%%%%%%%%%%%%%%%%%%%%%%%%%%%%%%%%%%%%%%%%%%%%%%%%%%%%%%%%%%%%%%%
% Here are my definitions for this paper

\usepackage{pdflscape} % Landscape diagram

\DeclareMathOperator{\Spec}{Spec}
\DeclareMathOperator{\Gal}{Gal}
\DeclareMathOperator{\Hom}{Hom}
\DeclareMathOperator{\im}{im}
\DeclareMathOperator{\Ext}{Ext}
\DeclareMathOperator{\Tot}{Tot}
\DeclareMathOperator{\coker}{coker}
\DeclareMathOperator{\rk}{rk}

\newcommand{\CC}{\mathbb{C}}
\newcommand{\FF}{\mathbb{F}}
\newcommand{\NN}{\mathbb{N}}
\newcommand{\QQ}{\mathbb{Q}}
\newcommand{\RR}{\mathbb{R}}
\newcommand{\ZZ}{\mathbb{Z}}

\renewcommand{\AA}{\mathbb{A}}
\renewcommand{\div}{\text{\it div}}
\newcommand{\tor}{\text{\it tor}}
\newcommand{\cotor}{\text{\it cotor}}

\newcommand{\dfn}{\mathrel{\mathop:}=}
\newcommand{\rdfn}{=\mathrel{\mathop:}}

\newcommand{\Wc}{\text{\it W,c}}
\newcommand{\ar}{\text{\it ar}}
\newcommand{\et}{\text{\it \'{e}t}}
\newcommand{\fg}{\text{\it fg}}

\newcommand{\iHom}{\underline{\Hom}}
\newcommand{\RHom}{R\!\Hom}

\newcommand{\DZ}{{\mathbf{D} (\ZZ)}}

\usepackage{tikz-cd}
\usetikzlibrary{arrows}
\usetikzlibrary{calc}
\usetikzlibrary{babel}
\usetikzlibrary{decorations.pathreplacing}
\usetikzlibrary{patterns}

\newcommand{\tikzpb}{\ar[phantom,pos=0.2]{dr}{\text{\large$\lrcorner$}}}
\newcommand{\tikzpbur}{\ar[phantom,pos=0.2]{dl}{\text{\large$\llcorner$}}}

%%%%%%%%%%%%%%%%%%%%%%%%%%%%%%%%%%%%%%%%%%%%%%%%%%%%%%%%%%%%%%%%%%%%%%%%%%%%%%%%

\begin{document}

\maketitle

%%%%%%%%%%%%%%% footnote %%%%%%%%%%%%%%%%
\footnote{ %2010 MSC numbers
2010 \textit{Mathematics Subject Classification}.
Primary 14F20; Secondary 14F42.}
\footnote{ %key words and phrases
  \textit{Key words and phrases}.
  Motivic cohomology, \'{e}tale cohomology, Weil-\'{e}tale cohomology.}

%%%%%%%%%%%%%%%%%%%%%%%%%%%%%%%%%%%%%%%%%

\begin{abstract}
  Flach and Morin constructed in \cite{Flach-Morin-2018} Weil-\'{e}tale
  cohomology $H^i_\Wc (X, \ZZ(n))$ for a proper, regular arithmetic scheme $X$
  (i.e. separated and of finite type over $\Spec \ZZ$) and $n \in \ZZ$. In the
  case when $n < 0$, we generalize their construction to an arbitrary arithmetic
  scheme $X$, thus removing the proper and regular assumption.  The construction
  assumes finite generation of suitable \'{e}tale motivic cohomology groups.
\end{abstract}

\section{Introduction}
\label{sec:introduction}

Stephen Lichtenbaum, in a series of papers
\cite{Lichtenbaum-2005,Lichtenbaum-2009-Euler-char,Lichtenbaum-2009-number-rings},
has envisioned a new cohomology theory for schemes, known as
\textbf{Weil-\'{e}tale cohomology}. The case of varieties over finite fields
$X/\FF_q$ was further studied by Geisser
\cite{Geisser-2004,Geisser-2006,Geisser-2010-arithmetic-homology}. Morin defined
in \cite{Morin-2014} the Weil-\'{e}tale cohomology with compact support
$H^i_\Wc (X, \ZZ)$ for $X \to \Spec\ZZ$ separated, of finite type, proper, and
regular. This construction was further generalized by Flach and Morin in
\cite{Flach-Morin-2018} to the groups $H^i_\Wc (X, \ZZ(n))$ with arbitrary
weights $n \in \ZZ$, under the same assumptions on $X$.

The aim of this paper is to remove the assumption that $X$ is proper and regular
and, following the ideas of \cite{Flach-Morin-2018}, to construct the groups
$H^i_\Wc (X, \ZZ(n))$ for any $X$ separated and of finite type over $\Spec\ZZ$
for the case of strictly negative weights $n < 0$.

As Flach and Morin already suggest in \cite[Remark 3.11]{Flach-Morin-2018},
we rework all their constructions in terms of cycle complexes $\ZZ^c (n)$, which
were considered by Geisser in \cite{Geisser-2010} in the context of arithmetic
duality theorems.

In a forthcoming paper we apply the results of this text to relate the
cohomology groups $H_\Wc^i (X, \ZZ(n))$ to the special value of the zeta
function $\zeta (X, s)$ at $s = n < 0$.

\subsection*{Notation and conventions}

\paragraph{Arithmetic schemes.}
In this work, an \textbf{arithmetic scheme} is a scheme $X$ that is separated
and of finite type over $\Spec \ZZ$.

\paragraph{Abelian groups.}
Let $A$ be an abelian group. For $m \ge 1$ we denote by ${}_m A$ its $m$-torsion
subgroup, and by $A_m$ the quotient $A/mA$:
$$0 \to {}_m A \to A \xrightarrow{\times m} A \to A_m \to 0$$
We denote by $A_\div$ (resp. $A_\tor$) the maximal divisible subgroup
(resp. maximal torsion subgroup), and by $A_\cotor$ the quotient $A/A_\tor$.

% For brevity, we will write $\Hom (A,B)$ for $\Hom_\ZZ (A,B)$, $\Ext (A,B)$ for
% $\Ext_\ZZ^1 (A,B)$, $A\otimes B$ for $A\otimes_\ZZ B$, and so on.

We say that $A$ is of \textbf{cofinite type} if it is $\QQ/\ZZ$-dual to a
finitely generated abelian group: $A = \Hom (B,\QQ/\ZZ)$ for a finitely
generated $B$.

\paragraph{Complexes.}
All our constructions take place in the derived category of abelian groups
$\DZ$. For our purposes, we introduce the following terminology. Recall first
that a complex of abelian groups $A^\bullet$ is \textbf{perfect} if it is
bounded (i.e. $H^i (A^\bullet) = 0$ for $|i| \gg 0$), and $H^i (A^\bullet)$ are
finitely generated abelian groups.

\begin{definition}
  \label{dfn:almost-of-(co)finite-type}
  A complex of abelian groups $A^\bullet$ is \textbf{almost perfect}
  if the cohomology groups $H^i (A^\bullet)$ are finitely generated, and
  bounded, except for possible finite $2$-torsion in arbitrarily high degree.
  That is, $H^i (A^\bullet) = 0$ for $i \ll 0$ and $H^i (A^\bullet)$ is finite
  $2$-torsion for $i \gg 0$.

  A complex of abelian groups $A^\bullet$ is of \textbf{cofinite type} if the
  cohomology groups $H^i (A^\bullet)$ are of cofinite type and bounded.

  A complex of abelian groups $A^\bullet$ is \textbf{almost of cofinite type}
  if the cohomology groups $H^i (A^\bullet)$ are of cofinite type and
  bounded, except for possible finite $2$-torsion in arbitrarily high
  degree.
\end{definition}

This terminology is ad hoc and was invented for this text, since such complexes
will appear frequently. Some basic observations about almost perfect and almost
cofinite type complexes are collected in
Appendix~\ref{app:homological-algebra}. We note that this finite $2$-torsion in
arbitrarily high degrees could be removed by working with the Artin--Verdier
topology $\overline{X}_\et$ instead of the usual \'{e}tale topology $X_\et$.
The general construction and basic properties of $\overline{X}_\et$ are treated
in \cite[Appendix~A]{Flach-Morin-2018}, but only for a \emph{proper and regular}
arithmetic scheme $X$. Our methods circumvent this restriction at the cost of
some technical hurdles with $2$-torsion.

\paragraph{\'{E}tale cohomology.}
For an arithmetic scheme $X$ and a complex of \'{e}tale sheaves
$\mathcal{F}^\bullet$, we denote by
\[ R\Gamma (X_\et, \mathcal{F}^\bullet) ~
  \text{(resp. }R\Gamma_c (X_\et, \mathcal{F}^\bullet), ~
  R\widehat{\Gamma}_c (X_\et, \mathcal{F}^\bullet)\text{)} \]
the complex that computes the corresponding cohomology, resp. cohomology with
compact support, and modified cohomology with compact support.  For the
convenience of the reader, we review the definitions in
Appendix~\ref{app:modified-cohomology-with-compact-support}. The purpose of
$R\widehat{\Gamma}_c (X_\et, \mathcal{F}^\bullet)$ is to take care of real
places $X (\RR)$. There exists a canonical projection
$R\widehat{\Gamma}_c (X_\et, \mathcal{F}^\bullet) \to R\Gamma_c (X_\et,
\mathcal{F}^\bullet)$, which is an isomorphism if $X (\RR) = \emptyset$.

\paragraph{$G$-equivariant sheaves and their cohomology.}
Let $\mathcal{X}$ be a topological space with an action of a discrete group $G$.
A \textbf{$G$-equivariant sheaf} $\mathcal{F}$ on $\mathcal{X}$ can be defined
as an espace \'{e}tal\'{e} $\pi\colon E\to \mathcal{X}$ with a $G$-action on $E$
such that the projection $\pi$ is $G$-equivariant (see e.g. \cite[\S II.6 +
pp.\,594]{MacLane-Moerdijk}). We denote by $\mathbf{Sh} (G, \mathcal{X})$ the
corresponding category.

The equivariant global sections are defined by
$$\Gamma (G,\mathcal{X},\mathcal{F}) = \mathcal{F} (\mathcal{X})^G,$$
with $G$ acting on
$\mathcal{F} (\mathcal{X}) = \{ s\colon \mathcal{X}\to E \mid \pi\circ s = id_\mathcal{X} \}$
via $(g\cdot s) (x) = g\cdot s (g^{-1}\cdot x)$. The corresponding
\textbf{$G$-equivariant cohomology} is given by the right derived functors of
$\Gamma (G,\mathcal{X},-)$.

More details on $G$-equivariant sheaves can be found in
\cite[Chapitre~2]{Morin-these}. For our modest purposes, it suffices to know
that any $G$-module $A$ gives rise to the corresponding abelian $G$-equivariant
constant sheaf. The latter corresponds to the espace \'{e}tal\'{e}
$\mathcal{X}\times A \to \mathcal{X}$, where $A$ is endowed with the discrete
topology.

\paragraph{$G_\RR$-equivariant cohomology of $X (\CC)$.}
Given an arithmetic scheme $X$, we denote by $X (\CC)$ the set of complex points
of $X$ endowed with the analytic topology. It carries the natural action of the
Galois group $G_\RR \dfn \Gal (\CC/\RR)$.

We consider the $G_\RR$-modules
\[ \ZZ (n) \dfn (2\pi i)^n\,\ZZ, \quad
  \QQ (n) \dfn (2\pi i)^n\,\QQ, \quad
  \QQ/\ZZ (n) \dfn \QQ (n) / \ZZ (n) \]
as constant $G_\RR$-equivariant sheaves on $X (\CC)$.

Then $R\Gamma_c (X (\CC), A (n))$ for $A = \ZZ, \QQ, \QQ/\ZZ$ (the complex that
computes singular cohomology with compact support of $X (\CC)$ with
coefficients in $A (n)$) is a complex of $G_\RR$-modules, and we can further
take the group cohomology (resp. Tate cohomology):
\begin{align*}
  R\Gamma_c (G_\RR, X (\CC), A (n)) & \dfn R\Gamma (G_\RR, R\Gamma_c (X (\CC), A (n))),\\
  R\widehat{\Gamma}_c (G_\RR, X (\CC), A (n)) & \dfn R\widehat{\Gamma} (G_\RR, R\Gamma_c (X (\CC), A (n))).
\end{align*}
By definition, this is the \textbf{$G_\RR$-equivariant cohomology}
(resp. \textbf{$G_\RR$-equivariant Tate cohomology})
\textbf{with compact support} of $X (\CC)$ with coefficients in $A (n)$.

\paragraph{Motivic cohomology $H^i (X_\et, \ZZ^c (n))$.}
Our construction is based on motivic cohomology defined in terms of complexes
of sheaves $\ZZ^c (n)$ on $X_\et$. We follow the notation of
\cite{Geisser-2010}.

Briefly, for $i \ge 0$ we consider the algebraic simplex
$$\Delta^i = \Spec \ZZ[t_0,\ldots,t_i]/(\sum_i t_i - 1).$$
We fix a negative weight $n \le 0$. Let $z_n (X,i)$ be the free abelian group
generated by the closed integral subschemes $Z \subset X \times \Delta^i$ of
dimension $n + i$ that intersect the faces properly. Then $z_n (X, \bullet)$ is
a (homological) complex of abelian groups whose differentials are given by the
alternating sum of intersections with the faces. We consider the (cohomological)
complex of \'{e}tale sheaves
$$\ZZ^c (n) \dfn z_n (\text{\textvisiblespace}, -\bullet) [2n].$$

The boundedness from below of $\ZZ^c(n)$ is not known in general; it is a
variant of the Beilinson--Soul\'{e} vanishing conjecture. To work
unconditionally with the derived functors, we use $K$-injective resolutions
\cite{Spaltenstein-1988,Serpe-2003} (resp. $K$-flat resolutions for the derived
tensor products).

To avoid any confusion, we use cohomological numbering for all complexes
in this paper, so we set
$$H^i (X_\et, \ZZ^c(n)) \dfn H^i (R\Gamma (X_\et, \ZZ^c(n))).$$
(\cite{Geisser-2010} uses homological numbering.)

If $X$ is \emph{proper, regular and of pure dimension $d$}, then for $n \le 0$
there exists an isomorphism
\begin{equation}
  \label{eqn:Zc(n)-vs-Z(d-n)}
  H^i (X_\et, \ZZ^c(n)) \cong H^{2d+i} (X_\et, \ZZ (d-n)),
\end{equation}
where the right-hand side is the ``usual'' motivic cohomology defined for
positive weights; see the original Bloch's paper \cite{Bloch-1986} for the case
of varieties, and also \cite{Geisser-2004-Dedekind,Geisser-2005} for the
definitions and properties over $\Spec \ZZ$.

\subsection*{Assumptions}

\paragraph{Weights.}
In this paper, $n < 0$ always denotes a strictly negative integer,
which will be the weight in the cohomology groups $H^i_\Wc (X, \ZZ(n))$.

\paragraph{Finite generation conjecture.}
Our construction of the Weil-\'{e}tale cohomology groups $H^i_\Wc (X,\ZZ(n))$
uses the following assumption.

\begin{conjecture}
  $\mathbf{L}^c (X_\et,n)$: for an arithmetic scheme $X$ and $n < 0$,
  the cohomology groups $H^i (X_\et, \ZZ^c (n))$ are finitely generated for all
  $i \in \ZZ$.
\end{conjecture}

See Proposition~\ref{prop:Lc-Xet-n-vs-L-Xet-d-n} for the precise relation of
$\mathbf{L}^c (X_\et,n)$ to other conjectures that appear in the literature.
We refer to \S\ref{sec:known-cases-of-Lc-Xet-n} for the cases where the
conjecture is known.

\subsection*{Main results}

Before outlining the construction of Weil-\'{e}tale cohomology, we state the
main results of this paper that make it possible. One of our main objects is the
following complex of abelian sheaves $\ZZ (n)$ on $X_\et$.

\begin{definition}[{\cite[\S 3.1]{Flach-Morin-2018}}, {\cite[\S 7]{Geisser-2004}}]
  \label{dfn:sheaf-Z(n)}
  Let $X$ be an arithmetic scheme and $n < 0$. For a prime $p$, consider
  the localization $X [1/p]$, and let $\mu_{p^r}$ be the sheaf of $p^r$-th
  roots of unity on $X [1/p]$. We define the twist of $\mu_{p^r}$ by $n$
  as
  $$\mu_{p^r}^{\otimes n} = \iHom_{X[1/p]} (\mu_{p^r}^{\otimes (-n)}, \ZZ/p^r\ZZ).$$

  Now $\ZZ (n)$ is the complex of sheaves on $X_\et$ given by
  \[ \ZZ (n) = \QQ/\ZZ (n) [-1],
  \quad \text{where }
  \QQ/\ZZ (n) = \bigoplus_p \varinjlim_r j_{p!} \mu_{p^r}^{\otimes n}, \]
  and $j_p$ is the canonical open immersion $X[1/p] \to X$.
\end{definition}

The above sheaves $\ZZ (n)$ should not be confused with cycle complexes;
the latter are $\ZZ^c (n)$ in the context of this paper.
In \S\ref{sec:arithmetic-duality-theorem} we prove the following arithmetic
duality theorem relating the two.

\begin{maintheorem}
  \label{theorem-I}
  Assuming Conjecture $\mathbf{L}^c (X_\et,n)$, there is a quasi-isomorphism
  \[ R\widehat{\Gamma}_c (X_\et, \ZZ (n)) \xrightarrow{\cong}
    \RHom (R\Gamma (X_\et, \ZZ^c (n)), \QQ/\ZZ [-2]). \]
\end{maintheorem}

The second result we need is related to the following morphism of complexes.

\begin{definition}
  \label{dfn:u-infty}
  We define
  $v_\infty^*\colon R\Gamma_c (X_\et, \QQ/\ZZ (n)) \to R\Gamma_c (G_\RR, X
  (\CC), \QQ/\ZZ (n))$ as the morphism in the derived category $\DZ$ induced by
  the comparison of \'{e}tale and analytic topology
  \[ \Gamma_c (X_\et, \QQ/\ZZ (n)) \to
  \Gamma_c (G_\RR, X (\CC), \alpha^* \QQ/\ZZ (n)) \cong
  \Gamma_c (G_\RR, X (\CC), \QQ/\ZZ (n)) \]
  (see Proposition~\ref{prop:inverse-image-gamma} and
  \ref{propn:image-of-Q/Zn-under-alpha}). Then we let
  $u_\infty^*\colon R\Gamma_c (X_\et, \ZZ(n)) \to R\Gamma_c (G_\RR, X (\CC), \ZZ (n))$
  be the composition
  \begin{multline*}
    R\Gamma_c (X_\et, \ZZ(n)) \dfn R\Gamma_c (X_\et, \QQ/\ZZ (n)) [-1]
    \xrightarrow{v_\infty^* [-1]} R\Gamma_c (G_\RR, X (\CC), \QQ/\ZZ (n)) [-1]
    \\ \to R\Gamma_c (G_\RR, X (\CC), \ZZ (n))
  \end{multline*}
  where the last arrow is induced by $\QQ/\ZZ (n) [-1] \to \ZZ (n)$, which comes
  from the distinguished triangle of constant $G_\RR$-equivariant sheaves
  $\ZZ (n) \to \QQ (n) \to \QQ/\ZZ (n) \to \ZZ (n) [1]$.
\end{definition}

Then \S\ref{sec:theorem-II} is devoted to the following result.

\begin{maintheorem}
  \label{theorem-II}
  The morphism $u_\infty^*$ is torsion, i.e. there exists a nonzero integer $m$
  such that $mu^*_\infty = 0$
\end{maintheorem}

\subsection*{Sketch of the construction of Weil-\'{e}tale cohomology}

Here we describe the structure of this paper, as well as our construction of the
Weil-\'{e}tale complexes $R\Gamma_\Wc (X, \ZZ (n))$.

First, \S\ref{sec:arithmetic-duality-theorem} is devoted to the proof of
Theorem~\ref{theorem-I}. Some of its consequences are deduced in
\S\ref{sec:consequences-of-theorem-I}. Namely, if we assume Conjecture
$\mathbf{L}^c (X_\et, n)$, then $R\Gamma (X_\et, \ZZ^c (n))$ is an almost
perfect complex, while $R\Gamma_c (X_\et, \ZZ (n))$ is almost of cofinite type
in the sense of Definition~\ref{dfn:almost-of-(co)finite-type}. For this, we
first make a small digression in \S\ref{sec:GR-equivariant-cohomology} to
analyze what kind of complexes we obtain for the $G_\RR$-equivariant cohomology
of $X (\CC)$.

Theorem~\ref{theorem-I} is used in \S\ref{sec:RGamma-fg} to define a morphism
$\alpha_{X,n}$ in the derived category (see Definition~\ref{def:RGamma-fg}),
and declare $R\Gamma_\fg (X, \ZZ(n))$ to be its cone:
\begin{multline*}
  \RHom (R\Gamma (X_\et, \ZZ^c (n)), \QQ [-2]) \xrightarrow{\alpha_{X,n}}
  R\Gamma_c (X_\et, \ZZ (n)) \to
  R\Gamma_\fg (X, \ZZ(n)) \\
  \to \RHom (R\Gamma (X_\et, \ZZ^c (n)), \QQ [-1])
\end{multline*}
The notation ``\emph{fg}'' comes from the fact that $R\Gamma_\fg (X, \ZZ(n))$ is
an almost perfect complex in the sense of
Definition~\ref{dfn:almost-of-(co)finite-type}. Thanks to specific properties of
the complexes involved, it turns out that $R\Gamma_\fg (X, \ZZ(n))$ is defined
up to a \emph{unique} isomorphism in the derived category (which is not normally
expected from a cone).

Then in \S\ref{sec:theorem-II} we establish Theorem~\ref{theorem-II}, and it is
used in \S\ref{sec:RGamma-Wc} to define Weil-\'{e}tale complexes
$R\Gamma_\Wc (X, \ZZ(n))$. To do this, we deduce from Theorem~\ref{theorem-II}
that $u_\infty^* \circ \alpha_{X,n} = 0$, which implies that there exists a
morphism in the derived category
$$i_\infty^*\colon R\Gamma_\fg (X, \ZZ (n)) \to R\Gamma_c (G_\RR, X (\CC), \ZZ(n))$$
---see \eqref{eqn:diagram-defining-everything} below. We choose a mapping fiber
of $i_\infty^*$ and call it $R\Gamma_\Wc (X, \ZZ (n))$, which turns out to be a
perfect complex.  Finally, in \S\ref{sec:known-cases-of-Lc-Xet-n} we consider
the cases of $X$ for which Conjecture $\mathbf{L}^c (X_\et, n)$ is known,
and hence our results hold unconditionally, and in
\S\ref{sec:comparison-with-FM} we verify that if $X$ is proper and regular, our
complex $R\Gamma_\Wc (X, \ZZ (n))$ is isomorphic to that constructed in
\cite{Flach-Morin-2018} by Flach and Morin.

There are two appendices to this paper: Appendix~\ref{app:homological-algebra}
collects some lemmas from homological algebra, and
Appendix~\ref{app:modified-cohomology-with-compact-support} gives an overview of
the definitions of \'{e}tale cohomology with compact support
$R\Gamma_c (X_\et, -)$ and its modified version
$R\widehat{\Gamma}_c (X_\et, -)$.

\vspace{1em}

The definition of $R\Gamma_\Wc (X,\ZZ(n))$ fits in the following commutative
diagram with distinguished triangles in the derived category $\DZ$:
\begin{equation}
  \label{eqn:diagram-defining-everything}
  \begin{tikzcd}[column sep=1em,font=\small]
    &[-3em] \RHom (R\Gamma (X_\et, \ZZ^c (n)), \QQ [-2]) \ar{d}{\alpha_{X,n}}[swap]{\text{Dfn.~\ref{def:RGamma-fg}}} \ar{r} &[-2.5em] 0 \ar{d} \\
    & R\Gamma_c (X_\et, \ZZ(n)) \ar{d}\ar{r}{u_\infty^*}[swap]{\text{Dfn.~\ref{dfn:u-infty}}} & R\Gamma_c (G_\RR, X (\CC), \ZZ(n))\ar{d}{id} \\
    R\Gamma_\Wc (X, \ZZ (n)) \ar{r} & R\Gamma_\fg (X, \ZZ(n)) \ar[dashed]{r}{i_\infty^*}\ar{d} & R\Gamma_c (G_\RR, X (\CC), \ZZ(n)) \ar{r} \ar{d} & R\Gamma_\Wc (X, \ZZ (n)) [1] \\
    & \RHom (R\Gamma (X_\et, \ZZ^c (n)), \QQ [-1]) \ar{r} & 0
  \end{tikzcd}
\end{equation}

Our construction follows \cite{Flach-Morin-2018}, and the resulting complex is
the same if $X$ is proper and regular, which is the assumption considered
by Flach and Morin. Here is a brief comparison between the notations.

\begin{center}
  \renewcommand{\arraystretch}{1.5}
  \begin{tabular}{cc}
    \hline
    \textbf{this paper} & \textbf{Flach--Morin} \\
    \hline
    {\renewcommand{\arraystretch}{1}\begin{tabular}{c} $X\to\Spec\ZZ$ \\ separated, of finite type \\ ~ \end{tabular}} & {\renewcommand{\arraystretch}{1}\begin{tabular}{c} $X\to\Spec\ZZ$ \\ separated, of finite type \\ proper, regular, equidimensional\end{tabular}} \\
    \hline
    $n < 0$ & $n \in \ZZ$ \\
    \hline
    {\renewcommand{\arraystretch}{1}\begin{tabular}{c} cycle complexes \\ $\ZZ^c (n)$ \end{tabular}} & {\renewcommand{\arraystretch}{1}\begin{tabular}{c} cycle complexes \\ $\ZZ (d-n)[2d]$, $d = \dim X$ \end{tabular}} \\
    \hline
    $R\Gamma_\fg (X, \ZZ(n))$ & {\renewcommand{\arraystretch}{1}\begin{tabular}{c} $R\Gamma_W (\overline{X}, \ZZ(n))$, \\ up to finite $2$-torsion \end{tabular}} \\
    \hline
    $R\Gamma_\Wc (X,\ZZ(n))$ & $R\Gamma_\Wc (X, \ZZ(n))$ \\
    \hline
  \end{tabular}
\end{center}

{\small
\subsection*{Acknowledgments}

This text is based on the results of my PhD thesis, carried out at the
Universit\'{e} de Bordeaux and Universiteit Leiden under the supervision of
Baptiste Morin and Bas Edixhoven. I am very grateful to them for their
support. I thank Stephen Lichtenbaum and Niranjan Ramachandran who kindly agreed
to act as reviewers for my thesis and provided me with many useful comments and
suggestions. I am also indebted to Matthias Flach, since the ideas of this paper
come from \cite{Flach-Morin-2018}. Moreover, the work of Thomas Geisser on
arithmetic duality \cite{Geisser-2010} is also crucial for this paper, and his
work on Weil-\'{e}tale cohomology for varieties over finite fields
\cite{Geisser-2004,Geisser-2006,Geisser-2010-arithmetic-homology} has been of
great influence for me. I thank Maxim Mornev for several fruitful
conversations. This paper was edited during my stay at the Center for Research
in Mathematics (CIMAT), Guanajuato, Mexico. I~am grateful personally to Pedro
Luis del \'{A}ngel and Xavier G\'{o}mez Mont for their hospitality.
Finally, I am indebted to the anonymous referee whose sharp and insightful
comments on an earlier draft helped to improve the exposition.}

%%%%%%%%%%%%%%%%%%%%%%%%%%%%%%%%%%%%%%%%%%%%%%%%%%%%%%%%%%%%%%%%%%%%%%%%%%%%%%%%

\section{Proof of Theorem~I}
\label{sec:arithmetic-duality-theorem}

At the heart of our constructions is an arithmetic duality theorem for cycle
complexes established by Thomas Geisser in \cite{Geisser-2010}. The purpose of
this section is to deduce Theorem~\ref{theorem-I} from Geisser's duality.
We would like to obtain a quasi-isomorphism of complexes
\[ R\widehat{\Gamma}_c (X_\et, \ZZ (n)) \xrightarrow{\cong}
  \RHom (R\Gamma (X_\et, \ZZ^c (n)), \QQ/\ZZ [-2]). \]

Here $R\widehat{\Gamma}_c (X_\et, \ZZ (n))$ denotes the modified \'{e}tale
cohomology with compact support, described in
Appendix~\ref{app:modified-cohomology-with-compact-support}. We note that
\cite{Geisser-2010} uses the notation ``$R\Gamma_c$'' for our
``$R\widehat{\Gamma}_c$'', but we take special care to distinguish the two
things, since we also need the usual \'{e}tale cohomology with compact support
$R\Gamma_c (X_\et, \ZZ (n))$.

We split our proof of Theorem~\ref{theorem-I} into two propositions.

\begin{proposition}
  For any $n < 0$ we have a quasi-isomorphism of complexes
  \begin{equation}
    \label{eqn:duality-quasi-isomorphism-1}
    R\widehat{\Gamma}_c (X_\et, \ZZ (n)) \cong
    \varinjlim_m \RHom (R\Gamma (X_\et, \ZZ/m\ZZ^c (n)), \QQ/\ZZ [-2]).
  \end{equation}

  \begin{proof}
    We unwind our definition of $\ZZ (n)$ for $n < 0$ and reduce everything to
    the results from \cite{Geisser-2010}. Since
    $\ZZ (n) \dfn \bigoplus_p \varinjlim_r j_{p!} \mu_{p^r}^{\otimes n} [-1]$,
    it suffices to show that for every prime $p$ and $r\ge 1$ there is a
    quasi-isomorphism of complexes
    \begin{equation}
      \label{eqn:duality-quasi-isomorphism-1-pr}
      R\widehat{\Gamma}_c (X_\et, j_{p!} \mu_{p^r}^{\otimes n} [-1]) \cong
      \RHom (R\Gamma (X_\et, \ZZ^c/p^r (n)), \QQ/\ZZ [-2]),
    \end{equation}
    and then pass to the corresponding filtered colimits.

    As in Definition~\ref{dfn:sheaf-Z(n)}, here $j_p$ denotes the canonical open
    immersion $j_p\colon X[1/p] \hookrightarrow X$. We further denote by $f$ the
    structure morphism $X\to \Spec \ZZ$ and by $f_p$ the structure morphism
    $X [1/p] \to \Spec \ZZ [1/p]$:

    \[ \begin{tikzcd}
        X [1/p]\ar[hookrightarrow]{r}{j_p}\ar{d}[swap]{f_p} & X\ar{d}{f} \\
        \Spec \ZZ [1/p]\ar[hookrightarrow]{r} & \Spec \ZZ
      \end{tikzcd} \]
  
    As we are going to change the base scheme, let us write $\Hom_X (-,-)$ for
    the $\Hom$ between sheaves on $X_\et$ and $\iHom_X (-,-)$ for the
    internal $\Hom$. Instead of $\Hom_{\Spec R}$, we will simply write
    $\Hom_R$.

    Applying various results from \cite{Geisser-2004} and \cite{Geisser-2010},
    we obtain a quasi-isomorphism of complexes of sheaves
    \[ R\iHom_X (j_{p!} \mu_{p^r}^{\otimes n} [-1], \ZZ^c_X (0)) \cong \hspace{4cm} \]
    \begin{align*}
      \hspace{2cm} & \cong R j_{p*} R\iHom_{X [1/p]} (\mu_{p^r}^{\otimes n} [-1], \ZZ^c_{X [1/p]} (0)) && \text{by \cite[Prop. 7.10~(c)]{Geisser-2010}} \\
      & \cong R j_{p*} R\iHom_{X[1/p]} (f_p^* \mu_{p^r}^{\otimes n} [-1], \ZZ^c_{X [1/p]} (0)) \\
      & \cong R j_{p*} R f^!_p R\iHom_{\ZZ [1/p]} (\mu_{p^r}^{\otimes n} [-1], \ZZ^c_{\ZZ [1/p]} (0)) && \text{by \cite[Prop. 7.10~(c)]{Geisser-2010}} \\
                                                                & \cong R j_{p*} R f^!_p R\iHom_{\ZZ [1/p]} (\mu_{p^r}^{\otimes n} [-1], \mathbb{G}_\mathrm{m} [1]) && \text{by \cite[Lemma~7.4]{Geisser-2010}} \\
      & \cong R j_{p*} R f^!_p R\iHom_{\ZZ [1/p]} (\mu_{p^r}^{\otimes n}, \mathbb{G}_\mathrm{m}) [2] \\
                                                                & \cong R j_{p*} R f^!_p \, \mu_{p^r}^{\otimes (1-n)} [2] \\
      & \cong R j_{p*} R f^!_p \, \Bigl(\ZZ_{\ZZ [1/p]}/p^r (1-n)\Bigr) [2] && \text{by \cite[Thm.~1.2]{Geisser-2004}} \\
                                                                & \cong R j_{p*} R f^!_p \, \ZZ^c_{\ZZ [1/p]}/p^r (n) && \text{by \eqref{eqn:Zc(n)-vs-Z(d-n)}} \\
                                                                & \cong R j_{p*} \ZZ^c_{X [1/p]} / p^r (n) && \text{by \cite[Prop.~7.10~(a)]{Geisser-2010}} \\
                                                                & \cong R j_{p*} j_p^*\ZZ^c_X/p^r (n) \cong \ZZ^c_X/p^r (n) && \text{by \cite[Thm.~7.2~(a), Prop.~2.3]{Geisser-2010}}
    \end{align*}

    After applying $R\Gamma (X_\et, -)$, we get a quasi-isomorphism of
    complexes of abelian groups
    \[
      \RHom (j_{p!} \mu_{p^r}^{\otimes n} [-1], \ZZ^c_X (0)) \cong
      R\Gamma (X_\et, \ZZ^c_X/p^r (n)).
    \]

    Now according to the duality \cite[Theorem~7.8]{Geisser-2010},
    \[
      \RHom (j_{p !} \mu_{p^r}^{\otimes n} [-1], \ZZ^c (0)) \cong
      \RHom (R\widehat{\Gamma}_c (X_\et, j_{p !} \mu_{p^r}^{\otimes n} [-1]), \QQ/\ZZ [-2]).
    \]

    What we end up with is a quasi-isomorphism
    \[ R\Gamma (X_\et, \ZZ^c/p^r (n)) \cong \RHom (R\widehat{\Gamma}_c (X_\et,
      j_{p !} \mu_{p^r}^{\otimes n} [-1]), \QQ/\ZZ [-2]). \]
    The groups $\widehat{H}^i_c (X_\et, j_{p!} \mu_{p^r}^{\otimes n} [-1])$ are
    finite (the sheaves $j_{p!} \mu_{p^r}^{\otimes n}$ are constructible),
    so applying $\RHom (-,\QQ/\ZZ [-2])$ yields
    \eqref{eqn:duality-quasi-isomorphism-1-pr}.
  \end{proof}
\end{proposition}

To conclude the proof of Theorem~\ref{theorem-I}, we identify the complex on the
right-hand side of \eqref{eqn:duality-quasi-isomorphism-1}. For this, we need
Conjecture $\mathbf{L}^c (X_\et, n)$.

\begin{proposition}
  \label{prop:a-quasi-isomorphism-with-dirlim}
  Assuming Conjecture $\mathbf{L}^c (X_\et, n)$, there is
  a quasi-isomorphism
  \[ \varinjlim_m \RHom (R\Gamma (X_\et, \ZZ/m\ZZ^c (n)), \QQ/\ZZ [-2]) \cong
    \RHom (R\Gamma (X_\et, \ZZ^c (n)), \QQ/\ZZ [-2]). \]

  \begin{proof}
    Consider short exact sequences
    \[ 0 \to H^i (X_\et, \ZZ^c (n))_m \to
      H^i (X_\et, \ZZ/m\ZZ^c (n)) \to
      {}_m H^{i+1} (X_\et, \ZZ^c (n)) \to 0 \]
    If we now take $\Hom (-,\QQ/\ZZ)$ and filtered colimits $\varinjlim_m$,
    we get
    \begin{multline}
      \label{eqn:short-exact-sequence-with-dirlim}
      0 \to \varinjlim_m \Hom ({}_m H^{i+1} (X_\et, \ZZ^c (n)), \QQ/\ZZ) \to \\
      \varinjlim_m \Hom (H^i (X_\et, \ZZ/m\ZZ^c (n)), \QQ/\ZZ) \to \\
      \varinjlim_m \Hom (H^i (X_\et, \ZZ^c (n))_m, \QQ/\ZZ) \to 0
    \end{multline}

    By Conjecture $\mathbf{L}^c (X_\et, n)$, the group
    $H^{i+1} (X_\et, \ZZ^c (n))$ is finitely generated, and hence
    the first $\varinjlim_m$ in the short exact sequence
    \eqref{eqn:short-exact-sequence-with-dirlim} vanishes, and we obtain
    isomorphisms
    \[ \varinjlim_m \Hom (H^i (X_\et, \ZZ^c (n))_m, \QQ/\ZZ) \xrightarrow{\cong}
      \varinjlim_m \Hom (H^i (X_\et, \ZZ/m\ZZ^c (n)), \QQ/\ZZ). \]
    It remains to note that the left-hand side is canonically isomorphic to
    $\Hom (H^i (X_\et, \ZZ^c (n)), \QQ/\ZZ)$, again thanks to the finite
    generation of $H^i (X_\et, \ZZ^c (n))$, under Conjecture
    $\mathbf{L}^c (X_\et, n)$.

    To see this, observe that if $A$ is a finitely generated abelian group,
    there is a canonical isomorphism
    $$\varinjlim_m \Hom (A_m, \QQ/\ZZ) \cong \Hom (A, \QQ/\ZZ)$$
    induced by $A \to A_m$, and then applying the functor $\Hom (-, \QQ/\ZZ)$
    and $\varinjlim_m$. Since $\QQ/\ZZ$ is a torsion group, any homomorphism
    $A\to \QQ/\ZZ$ is killed by some $m$, hence factors through $A_m$.
  \end{proof}
\end{proposition}

%%%%%%%%%%%%%%%%%%%%%%%%%%%%%%%%%%%%%%%%%%%%%%%%%%%%%%%%%%%%%%%%%%%%%%%%%%%%%%%%

\section{$G_\RR$-equivariant cohomology of $X (\CC)$}
\label{sec:GR-equivariant-cohomology}

We begin with some elementary homological algebra.

\begin{lemma}
  Let $A^\bullet$ be a perfect complex of $\ZZ G_\RR$-modules.

  \begin{enumerate}
  \item[1)] The complex $A^\bullet \otimes^\mathbf{L} \QQ/\ZZ$ is of cofinite
    type.

  \item[2)]
    $R\Gamma (G_\RR, A^\bullet \otimes \QQ) \cong (A^\bullet \otimes
    \QQ)^{G_\RR}$ is a perfect complex of $\QQ$-vector spaces, and the complex
    $R\widehat{\Gamma} (G_\RR, A^\bullet \otimes \QQ)$ is quasi-isomorphic to
    $0$.

  \item[3)]
    $R\widehat{\Gamma} (G_\RR, A^\bullet \otimes^\mathbf{L} \QQ/\ZZ) \cong
    R\widehat{\Gamma} (G_\RR, A^\bullet [+1])$, and these complexes have finite
    $2$-torsion cohomology.

  \item[4)] $R\Gamma (G_\RR, A^\bullet)$ is almost perfect, and
    $R\Gamma (G_\RR, A^\bullet \otimes^\mathbf{L} \QQ/\ZZ)$ is almost of
    cofinite type.
  \end{enumerate}

  \begin{proof}
    The universal coefficient theorem gives us short exact sequences
    $$0 \to H^i (A^\bullet)_m \to H^i (A^\bullet \otimes^\mathbf{L} \ZZ/m\ZZ) \to {}_m H^{i+1} (A^\bullet) \to 0$$
    The colimit of these over $m$ is
    $$0 \to H^i (A^\bullet) \otimes \QQ/\ZZ \to H^i (A^\bullet \otimes^\mathbf{L} \QQ/\ZZ) \to H^{i+1} (A^\bullet)_\tor \to 0$$
    Here $H^i (A^\bullet) \otimes \QQ/\ZZ$ is injective, hence the short exact
    sequence splits. We see that $H^i (A^\bullet \otimes^\mathbf{L} \QQ/\ZZ)$ is
    of cofinite type and vanishes for $|i| \gg 0$, i.e. that
    $A^\bullet \otimes^\mathbf{L} \QQ/\ZZ$ is of cofinite type.

    Let us now consider the spectral sequences
    \begin{align}
      \label{eqn:homological-lemma-ss-1} E_2^{pq} & = H^p (G_\RR, H^q (A^\bullet \otimes \QQ)) \Longrightarrow H^{p+q} (G_\RR, A^\bullet \otimes \QQ), \\
      \label{eqn:homological-lemma-ss-2} E_2^{pq} & = \widehat{H}^p (G_\RR, H^q (A^\bullet \otimes \QQ)) \Longrightarrow \widehat{H}^{p+q} (G_\RR, A^\bullet \otimes \QQ).
    \end{align}
    We recall that $H^p (G_\RR, -)$ are $2$-torsion groups for $p > 0$. Since
    $H^q (A^\bullet \otimes \QQ)$ are $\QQ$-vector spaces, it follows that
    $E_2^{pq} = 0$ for $p > 0$ in \eqref{eqn:homological-lemma-ss-1}, and the
    spectral sequence degenerates. Similarly, the Tate cohomology groups
    $\widehat{H}^p (G_\RR, H^q (A^\bullet \otimes \QQ))$ are trivial for
    \emph{all} $p$ for the same reasons, so that
    \eqref{eqn:homological-lemma-ss-2} is trivial. This proves part 2).

    Part 3) now follows from the distinguished triangle
    \[ R\widehat{\Gamma} (G_\RR, A^\bullet) \to
      R\widehat{\Gamma} (G_\RR, A^\bullet \otimes \QQ) \to
      R\widehat{\Gamma} (G_\RR, A^\bullet \otimes^\mathbf{L} \QQ/\ZZ) \to
      R\widehat{\Gamma} (G_\RR, A^\bullet) [1] \]

    Next, examining the spectral sequence
    $$E_2^{pq} = H^p (G_\RR, H^q (A^\bullet)) \Longrightarrow H^{p+q} (G_\RR, A^\bullet),$$
    we see that the groups $H^i (G_\RR, A^\bullet)$ are finitely generated, zero
    for $i \ll 0$, and torsion for $i \gg 0$. The latter is $2$-torsion. To see
    that, let $P_\bullet \twoheadrightarrow \ZZ$ be the bar-resolution of $\ZZ$
    by free $\ZZ G_\RR$-modules. Consider the morphism of complexes
    \[ \begin{tikzcd}
        \cdots\ar{r} & P_3\ar{r}\ar{d}{2} & P_2\ar{r}\ar{d}{2} & P_1\ar{r}\ar{d}{2} & P_0\ar{r}\ar{d}{2-N} & 0 \\
        \cdots\ar{r} & P_3\ar{r} & P_2\ar{r} & P_1\ar{r} & P_0\ar{r} & 0
      \end{tikzcd} \]
    where $N$ denotes the norm map. The proof of
    \cite[Theorem~6.5.8]{Weibel-1994} shows that the above morphism induces
    multiplication by $2$ on $H^i (G_\RR,-)$ for $i > 0$, and it is
    null-homotopic. Since $A^\bullet$ is bounded, we see that the above morphism
    induces multiplication by $2$ on $H^i (G_\RR, A^\bullet)$ for $i \gg 0$.

    Similarly, analyzing
    $$E_2^{pq} = H^p (G_\RR, H^q (A^\bullet \otimes^\mathbf{L} \QQ/\ZZ)) \Longrightarrow H^{p+q} (G_\RR, A^\bullet \otimes^\mathbf{L} \QQ/\ZZ).$$
    we see that $H^i (G_\RR, A^\bullet \otimes^\mathbf{L} \QQ/\ZZ)$ are groups of
    cofinite type. To see that these are finite $2$-torsion for $i \gg 0$,
    consider the triangle
    \[ R\Gamma (G_\RR, A^\bullet) \to
      R\Gamma (G_\RR, A^\bullet \otimes \QQ) \to
      R\Gamma (G_\RR, A^\bullet \otimes^\mathbf{L} \QQ/\ZZ) \to
      R\Gamma (G_\RR, A^\bullet) [1] \]
    Here $R\Gamma (G_\RR, A^\bullet \otimes \QQ)$ is bounded, and therefore
    $H^i (G_\RR, A^\bullet \otimes^\mathbf{L} \QQ/\ZZ) \cong H^{i+1} (G_\RR,
    A^\bullet)$ for $i \gg 0$.
  \end{proof}
\end{lemma}

\begin{proposition}
  \label{prop:equivariant-coho-of-X(C)}
  Let $X$ be an arithmetic scheme. Then $X (\CC)$ has the following types of
  complexes as its cohomology:

  \begin{center}
    \renewcommand{\arraystretch}{1.5}
    \begin{tabular}{|c|c|c|c|}
      \hline
      & $A=\ZZ$ & $A=\QQ$ & $A=\QQ/\ZZ$ \\
      \hline
      $R\Gamma_c (X (\CC), A(n))$ & perfect${}_{/\ZZ}$ & perfect${}_{/\QQ}$ & cofinite type \\
      \hline
      $R\Gamma_c (G_\RR, X (\CC), A (n))$ & {\renewcommand{\arraystretch}{0.75}\begin{tabular}{c} almost \\ perfect \end{tabular}} & perfect${}_{/\QQ}$ & {\renewcommand{\arraystretch}{0.75}\begin{tabular}{c} almost \\ cofinite type \end{tabular}} \\
      \hline
      $R\widehat{\Gamma}_c (G_\RR, X (\CC), A (n))$ & {\renewcommand{\arraystretch}{0.75}\begin{tabular}{c} finite \\ $2$-torsion \end{tabular}} & $\cong 0$ & {\renewcommand{\arraystretch}{0.75}\begin{tabular}{c} finite \\ $2$-torsion \end{tabular}} \\
      \hline
    \end{tabular}
  \end{center}

  \begin{proof}
    The perfectness of $R\Gamma_c (X (\CC), \ZZ (n))$ follows from the fact that
    $X (\CC)$ has the homotopy type of a finite CW-complex. This result goes
    back to van der Waerden \cite{van-der-Waerden-30}; more recent expositions
    (of more general results) can be found e.g. in \cite{Lojasiewicz-1964} and
    \cite{Hironaka-1974}.

    The rest is an application of the previous lemma to
    $R\Gamma_c (X(\CC), \ZZ(n))$.
  \end{proof}
\end{proposition}

Here is a comparison between the usual and Tate cohomology.

\begin{lemma}
  \label{lemma:Tate-vs-normal-cohomology-of-X(C)}
  For $i \ge 2d - 1$ there is an isomorphism of finite $2$-torsion groups
  \[ \widehat{H}^i_c (G_\RR, X (\CC), \ZZ(n)) \cong
    H^i_c (G_\RR, X (\CC), \ZZ(n)). \]

  \begin{proof}
    Consider the spectral sequences
    \begin{align*}
      \widehat{E}^{pq}_2 = \widehat{H}^p (G_\RR, H^q_c (X (\CC), \ZZ(n))) & \Longrightarrow
      \widehat{H}^i_c (G_\RR, X (\CC), \ZZ(n)), \\
      E^{pq}_2 = H^p (G_\RR, H^q_c (X (\CC), \ZZ(n))) & \Longrightarrow
      H^i_c (G_\RR, X (\CC), \ZZ(n)).
    \end{align*}
    Here $\widehat{H}^p (G_\RR, -) \cong H^p (G_\RR, -)$ for
    $p \ge 1$. Moreover, $H^q_c (X (\CC), \ZZ(n)) = 0$ for $q \ge 2d-1$, for
    the reasons of topological dimension of $X (\CC)$.
  \end{proof}
\end{lemma}

%%%%%%%%%%%%%%%%%%%%%%%%%%%%%%%%%%%%%%%%%%%%%%%%%%%%%%%%%%%%%%%%%%%%%%%%%%%%%%%%

\section{Some consequences of Theorem~I}
\label{sec:consequences-of-theorem-I}

Now we deduce some consequences from the duality Theorem~\ref{theorem-I}.

\begin{lemma}
  \label{lemma:morphism-hat-Hc(Xet,Z(n))->Hc(Xet,Z(n))}
  The canonical morphism
  $\phi^i\colon \widehat{H}^i_c (X_\et, \ZZ (n)) \to H^i_c (X_\et, \ZZ (n))$
  sits in a long exact sequence
  \begin{multline*}
    \cdots \to \widehat{H}^{i-1}_c (G_\RR, X (\CC), \ZZ (n)) \to
    \widehat{H}_c^i (X_\et, \ZZ(n)) \xrightarrow{\phi^i}
    H_c^i (X_\et, \ZZ(n)) \\
    \to \widehat{H}^i_c (G_\RR, X (\CC), \ZZ (n)) \to \cdots
  \end{multline*}
  where the groups $\widehat{H}^i_c (G_\RR, X (\CC), \ZZ (n))$ are finite
  $2$-torsion. In particular,
  \begin{enumerate}
  \item[$1)$] the kernel and cokernel of $\phi^i$ are finite $2$-torsion,

  \item[$2)$] if $X (\RR) = \emptyset$, then
    $R\widehat{\Gamma}_c (G_\RR, X (\CC), \ZZ (n)) = 0$ and
    $\widehat{H}^i_c (X_\et, \ZZ (n)) \cong H^i_c (X_\et, \ZZ (n))$.
  \end{enumerate}

  \begin{proof}
    The exact sequence follows from the definition of modified \'{e}tale cohomology
    with compact support and Artin's comparison theorem. This is proved in
    \cite[Lemma~6.14]{Flach-Morin-2018}. In particular, the argument shows that
    $R\widehat{\Gamma}_c (G_\RR, X (\CC), \ZZ (n)) \cong
    R\widehat{\Gamma} (G_\RR, v^* Rf_* \ZZ(n))$ where
    $v\colon \Spec \CC \to \Spec \ZZ$ and $f\colon X\to \Spec \ZZ$,
    and $R\widehat{\Gamma}_c (G_\RR, X (\CC), \ZZ (n)) = 0$ if
    $X (\RR) = \emptyset$.

    The fact that $\widehat{H}^i_c (G_\RR, X (\CC), \ZZ (n))$ are finite
    $2$-torsion is a part of Proposition~\ref{prop:equivariant-coho-of-X(C)}.
  \end{proof}
\end{lemma}

\begin{proposition}
  \label{prop:motivic-cohomology-duality-consequences}
  Let $X$ be an arithmetic scheme of dimension $d$ satisfying Conjecture
  $\mathbf{L}^c (X_\et,n)$ for $n < 0$.

  \begin{enumerate}
  \item[$1)$] If $X (\RR) = \emptyset$, then $H^i (X_\et, \ZZ^c (n)) = 0$ for
    $i > 1$ or $i < -2d$.

  \item[$2)$] In general, $H^i (X_\et, \ZZ^c (n)) = 0$ for $i < -2d$, and
    $H^i (X_\et, \ZZ^c (n))$ is a finite $2$-torsion group for $i > 1$.

  \item[$3)$] If $X/\FF_q$ is a variety over a finite field, then the groups
    $H^i (X_\et, \ZZ^c(n))$ are finite for all $i \in \ZZ$.
  \end{enumerate}

  In general, we have the following cohomology:
  \begin{center}
    \renewcommand{\arraystretch}{1.5}
    \begin{tabular}{|c|c|cl|cl|}
      \hline
      \textbf{groups} & \textbf{type} & \multicolumn{2}{c|}{$i \ll 0$} & \multicolumn{2}{c|}{$i \gg 0$} \\
      \hline
      $H^i (X_\et, \ZZ^c (n))$ & {\renewcommand{\arraystretch}{0.75}\begin{tabular}{c} finitely \\ generated \end{tabular}} & $0$ & for $i < -2d$ & {\renewcommand{\arraystretch}{0.75}\begin{tabular}{c} finite \\ $2$-torsion \end{tabular}} & for $i > 1$ \\
      \hline
      $\widehat{H}^i_c (X_\et, \ZZ (n))$ & cofinite & {\renewcommand{\arraystretch}{0.75}\begin{tabular}{c} finite \\ $2$-torsion \end{tabular}} & for $i < 1$ & $0$ & for $i > 2d + 2$ \\
      \hline
      $H^i_c (X_\et, \ZZ (n))$ & cofinite & $0$ & for $i < 1$ & {\renewcommand{\arraystretch}{0.75}\begin{tabular}{c} finite \\ $2$-torsion \end{tabular}} & for $i > 2d + 2$ \\
      \hline
    \end{tabular}
  \end{center}
  In particular, $R\Gamma (X_\et, \ZZ^c (n))$ is an almost perfect complex,
  while $R\Gamma_c (X_\et, \ZZ (n))$ is almost of cofinite type in the sense of
  Definition~{\rm\ref{dfn:almost-of-(co)finite-type}}.

  \begin{proof}
    If $X (\RR) = \emptyset$, then our duality Theorem~\ref{theorem-I} gives
    \[ \Hom (H^{2-i} (X_\et, \ZZ^c (n)), \QQ/\ZZ) \cong
      \widehat{H}^i_c (X_\et, \ZZ (n)) \stackrel{X(\RR)=\emptyset}{\cong}
      H^i_c (X_\et, \ZZ (n)). \]
    We have $H^i_c (X_\et, \ZZ (n)) = 0$ for $i < 1$ by the definition of
    $\ZZ (n)$, and $H^i_c (X_\et, \ZZ (n)) = H^{i-1} (X_\et, \QQ/\ZZ(n)) = 0$
    for $i > 2d + 2$ for the reasons of $\ell$-adic cohomological dimension
    \cite[Expos\'{e}~X, Th\'{e}or\`{e}me~6.2]{SGA4}. This proves part 1) of the proposition.

    In part 2), the group $H^i (X_\et, \ZZ^c (n))$ is finite $2$-torsion for
    $i > 1$, thanks to part 1) and
    Lemma~\ref{lemma:morphism-hat-Hc(Xet,Z(n))->Hc(Xet,Z(n))}. Moreover, we have
    $H^i (X_\et, \ZZ^c (n)) \cong H^i (X_\et, \QQ^c (n))$ for $i < -2d$
    according to \cite[Lemma~5.12]{Morin-2014}. Conjecture $L^c (X_\et, n)$
    implies that these groups are $\QQ$-vector spaces finitely generated over
    $\ZZ$, hence trivial.

    In part 3), the cohomology groups
    $H^i (X_\et, \ZZ (n)) = H^{i-1} (X_\et, \QQ/\ZZ (n))$ are finite for $n < 0$
    by \cite[Theorem~3]{Kahn-2003}.
  \end{proof}
\end{proposition}

\begin{remark}
  If $X$ is proper and regular of dimension $d$, then using
  \eqref{eqn:Zc(n)-vs-Z(d-n)}, we note that the Beilinson--Soul\'{e} vanishing
  conjecture (see, for example, \cite[\S 4.3.4]{Kahn-2005}) predicts that
  $H^i (X_\et, \ZZ^c (n)) = 0$ for $i < -2d$. Therefore, we proved this under
  Conjecture $\mathbf{L}^c (X_\et, n)$.
\end{remark}

%%%%%%%%%%%%%%%%%%%%%%%%%%%%%%%%%%%%%%%%%%%%%%%%%%%%%%%%%%%%%%%%%%%%%%%%%%%%%%%%

\section{Complex $R\Gamma_\fg (X, \ZZ(n))$}
\label{sec:RGamma-fg}

The purpose of this section is to define auxiliary complexes
$R\Gamma_\fg (X, \ZZ(n))$, which are used below in the construction of
Weil-\'{e}tale cohomology.

\begin{definition}
  \label{def:RGamma-fg}
  Assuming Conjecture $\mathbf{L}^c (X_\et,n)$, consider a morphism
  $\alpha_{X,n}$ in the derived category $\DZ$ given by the
  composition
  \[ \begin{tikzcd}[column sep=4em]
    \RHom (R\Gamma (X_\et, \ZZ^c (n)), \QQ[-2]) \ar{r}{\QQ \twoheadrightarrow \QQ/\ZZ}\ar{ddr}[swap]{\alpha_{X,n}} & \RHom (R\Gamma (X_\et, \ZZ^c (n)), \QQ/\ZZ[-2]) \\
    & R\widehat{\Gamma}_c (X_\et, \ZZ (n)) \ar{u}{\text{Theorem~\ref{theorem-I}}}[swap]{\cong} \ar{d}{\text{proj.}} \\
    & R\Gamma_c (X_\et, \ZZ (n))
  \end{tikzcd} \]

  Here the first arrow is induced by the canonical projection $\QQ \to \QQ/\ZZ$,
  and the last arrow is the canonical projection from the modified cohomology
  with compact support to the usual cohomology with compact support
  (see Appendix~\ref{app:modified-cohomology-with-compact-support}).

  We define the complex $R\Gamma_\fg (X, \ZZ(n))$ as a cone of $\alpha_{X,n}$:
  \begin{multline*}
    \RHom (R\Gamma (X_\et, \ZZ^c (n)), \QQ [-2]) \xrightarrow{\alpha_{X,n}}
    R\Gamma_c (X_\et, \ZZ (n)) \to
    R\Gamma_\fg (X, \ZZ(n)) \\
    \to \RHom (R\Gamma (X_\et, \ZZ^c (n)), \QQ [-1])
  \end{multline*}
  Further, we denote
  $$H^i_\fg (X, \ZZ (n)) \dfn H^i (R\Gamma_\fg (X, \ZZ (n))).$$
\end{definition}

\begin{remark}
  \label{rmk:alpha-X-n-determined-by-cohomology}
  Under Conjecture $\mathbf{L}^c (X_\et, n)$, the groups
  $H^i_c (X_\et, \ZZ (n))$ are of cofinite type by Theorem~\ref{theorem-I},
  while $\RHom (R\Gamma (X_\et, \ZZ^c (n)), \QQ [-2])$ is a complex of
  $\QQ$-vector spaces. Therefore, the morphism $\alpha_{X,n}$ is completely
  determined by the maps between cohomology groups
  \[ H^i (\alpha_{X,n})\colon
    \Hom (H^{2-i} (X_\et, \ZZ^c (n)), \QQ) \to
    H^i_c (X_\et, \ZZ (n)) \]
  ---see Lemma~\ref{lemma:morphisms-in-DAb-between-cplx-of-Q-vs-and-almost-cofinite-type-cplx}.
\end{remark}

\begin{remark}
  We note that our $R\Gamma_\fg (X, \ZZ (n))$ plays the same role as
  $R\Gamma_W (\overline{X}_\et, \ZZ (n))$ in
  \cite[Definition~3.6]{Flach-Morin-2018}. We use a different notation since
  Flach and Morin work with the Artin--Verdier topology and their complex
  $R\Gamma_W (\overline{X}_\et, \ZZ (n))$ is perfect, while our complex can have
  finite $2$-torsion in arbitrarily high degree.
\end{remark}

We first note that the definition simplifies when $X$ has no real places.

\begin{proposition}
  \label{prop:RGamma-fg-for-X(R)-empty}
  If $X (\RR) = \emptyset$, then
  \[ R\Gamma_\fg (X, \ZZ (n)) \cong
  \RHom (R\Gamma (X_\et, \ZZ^c (n)), \ZZ [-1]). \]

  \begin{proof}
    In this case
    $R\widehat{\Gamma}_c (X_\et, \ZZ (n)) \to R\Gamma_c (X_\et, \ZZ (n))$
    is the identity morphism, and therefore $\alpha_{X,n}$ sits in the following
    commutative diagram with distinguished columns:
    \[ \begin{tikzcd}
      \RHom (R\Gamma (X_\et, \ZZ^c (n)), \QQ [-2])\ar{d}{\alpha_{X,n}} \ar{r}{\mathrm{id}} & \RHom (R\Gamma (X_\et, \ZZ^c (n)), \QQ [-2])\ar{d} \\
      R\Gamma_c (X_\et, \ZZ (n))\ar{d} \ar{r}{\cong}[swap]{\text{Theorem~\ref{theorem-I}}} & \RHom (R\Gamma (X_\et, \ZZ^c (n)), \QQ/\ZZ [-2])\ar{d} \\
      R\Gamma_\fg (X, \ZZ (n))\ar{d} \ar[dashed]{r}{\cong} & \RHom (R\Gamma (X_\et, \ZZ^c (n)), \ZZ [-1])\ar{d} \\
      \RHom (R\Gamma (X_\et, \ZZ^c (n)), \QQ [-1]) \ar{r}{\mathrm{id}} & \RHom (R\Gamma (X_\et, \ZZ^c (n)), \QQ [-1])
    \end{tikzcd} \]
    Here the first column is our definition of $R\Gamma_\fg (X, \ZZ (n))$,
    and the second column is induced by the distinguished triangle
    $\ZZ \to \QQ \to \QQ/\ZZ \to \ZZ [1]$.
  \end{proof}
\end{proposition}

\begin{proposition}
  \label{prop:RGammafg-almost-perfect}
  Assuming Conjecture $\mathbf{L}^c (X_\et, n)$, the complex
  $R\Gamma_\fg (X, \ZZ (n))$ is almost perfect in the sense of
  Definition~{\rm\ref{dfn:almost-of-(co)finite-type}}, i.e. its cohomology
  groups $H^i_\fg (X, \ZZ (n))$ are finitely generated, trivial for $i \ll 0$,
  and $2$-torsion for $i \gg 0$.

  \begin{proof}
    By the definition of $R\Gamma_\fg (X, \ZZ (n))$, there are short exact
    sequences
    \[ 0 \to \coker H^i (\alpha_{X,n}) \to
      H^i_\fg (X, \ZZ (n)) \to
      \ker H^{i+1} (\alpha_{X,n}) \to 0 \]

    The morphism $\alpha_{X,n}$ is given at the level of cohomology by
    \begin{multline}
      \label{eqn:H(alpha-X-n)}
      H^i (\alpha_{X,n})\colon
      \Hom (H^{2-i} (X_\et, \ZZ^c (n)), \QQ) \xrightarrow{\psi^i}
      \Hom (H^{2-i} (X_\et, \ZZ^c (n)), \QQ/\ZZ) \xrightarrow{\cong} \\
      \widehat{H}^i_c (X_\et, \ZZ (n)) \xrightarrow{\phi^i} H^i_c (X_\et, \ZZ (n))
    \end{multline}
    where $H^{2-i} (X_\et, \ZZ^c (n))$ is a finitely generated abelian group
    according to $\mathbf{L}^c (X_\et, n)$.

    Here $\phi^i$ has a finite $2$-torsion kernel according to
    Lemma~\ref{lemma:morphism-hat-Hc(Xet,Z(n))->Hc(Xet,Z(n))}, and we observe
    that if $A$ is a finitely generated abelian group, then for a finite
    subgroup $T \subset \Hom (A, \QQ/\ZZ)$ the preimage under
    $\Hom (A, \QQ) \to \Hom (A, \QQ/\ZZ)$ is finitely generated. This
    justifies the finite generation of $\ker H^i (\alpha_{X,n})$ for all
    $i \in \ZZ$.

    For the morphism $\psi^i$ we have
    \begin{align*}
      \ker \psi^i & \cong \Hom (H^{2-i} (X_\et, \ZZ^c (n)), \ZZ), \\
      \coker \psi^i & \cong \Hom (H^{2-i} (X_\et, \ZZ^c (n))_\tor, \QQ/\ZZ),
    \end{align*}
    and these groups are finitely generated by $\mathbf{L}^c (X_\et, n)$. The
    composition of morphisms \eqref{eqn:H(alpha-X-n)} gives an exact sequence
    (ignoring the isomorphism in the middle)
    \[
      0 \to \ker\psi^i \to \ker (\phi^i\circ\psi^i) \to \ker\psi^i \to
      \coker\psi^i \to \coker (\phi^i\circ \psi^i) \to \coker\psi^i \to 0
    \]

    For $i \ll 0$ we have $H^i_c (X_\et, \ZZ (n)) = 0$, and therefore
    $$H^i_\fg (X, \ZZ (n)) \cong \Hom (H^{1-i} (X_\et, \ZZ^c (n)), \QQ) = 0,$$
    since the group $H^{1-i} (X_\et, \ZZ^c (n))$ is finite $2$-torsion for
    $i \ll 0$ by Proposition~\ref{prop:motivic-cohomology-duality-consequences}.

    For $i \gg 0$ we have $\Hom (H^{2-i} (X_\et, \ZZ^c (n)), \QQ) = 0$, so that
    $H^i_\fg (X, \ZZ (n)) \cong H^i_c (X_\et, \ZZ (n))$, which is finite
    $2$-torsion by
    Proposition~\ref{prop:motivic-cohomology-duality-consequences}.
  \end{proof}
\end{proposition}

\begin{proposition}
  \label{prop:RGamma-fg-well-defined}
  The complex $R\Gamma_\fg (X, \ZZ (n))$ is defined up to a unique isomorphism
  in the derived category $\DZ$.

  \begin{proof}
    The complex $\RHom (R\Gamma (X_\et, \ZZ^c (n)), \QQ [-2])$ consists of
    $\QQ$-vector spaces, and $R\Gamma_\fg (X, \ZZ (n))$ is almost perfect, so we
    are in the situation of Corollary~\ref{cor:TR3-TR1-with-uniqueness}.
  \end{proof}
\end{proposition}

\begin{proposition}
  \label{prop:tensoring-RGammafg-with-Z/m-and-Q}
  Suppose that Conjecture $\mathbf{L}^c (X_\et,n)$ holds and consider the
  distinguished triangle defining $R\Gamma_\fg (X, \ZZ (n))$:
  \begin{multline*}
    \RHom (R\Gamma (X_\et, \ZZ^c (n)), \QQ [-2]) \xrightarrow{\alpha_{X,n}}
    R\Gamma_c (X_\et, \ZZ (n)) \xrightarrow{f}
    R\Gamma_\fg (X, \ZZ (n)) \\
    \xrightarrow{g} \RHom (R\Gamma (X_\et, \ZZ^c (n)), \QQ [-1])
  \end{multline*}

  \begin{enumerate}
  \item[$1)$] The morphism $g$ induces an isomorphism
    \[ g\otimes \QQ\colon R\Gamma_\fg (X, \ZZ (n)) \otimes \QQ \xrightarrow{\cong}
      \RHom (R\Gamma (X_\et, \ZZ^c (n)), \QQ [-1]).\]

  \item[$2)$] For each $m \ge 1$ the morphism $f$ induces an
    isomorphism
    \[ f\otimes \ZZ/m\ZZ\colon
      R\Gamma_c (X_\et, \ZZ (n))\otimes^\mathbf{L} \ZZ/m\ZZ \xrightarrow{\cong}
      R\Gamma_\fg (X, \ZZ (n))\otimes^\mathbf{L} \ZZ/m\ZZ \]
    
  \item[$3)$] For any prime $\ell$ the morphism $f$ induces an isomorphism
    $$\varprojlim_r H_c^i (X_\et, \ZZ/\ell^r (n)) \cong H_\fg^i (X, \ZZ (n)) \otimes \ZZ_\ell.$$
  \end{enumerate}
  
  \begin{proof}
    The groups $H_c^i (X_\et, \ZZ (n))$ are all torsion, and therefore
    $R\Gamma_c (X_\et, \ZZ (n)) \otimes \QQ \cong 0$ in the derived
    category. Similarly, the complexes of $\QQ$-vector spaces
    $\RHom (R\Gamma (X_\et, \ZZ^c (n)), \QQ [\cdots])$ are killed by tensoring
    with $\ZZ/m\ZZ$.  This proves 1) and 2).

    Now 2) implies 3): by the finite generation of $H_\fg^i (X, \ZZ (n))$, we
    have
    \[ \varprojlim_r H_c^i (X_\et, \ZZ/\ell^r (n)) \stackrel{\text{2)}}{\cong}
      \varprojlim_r H_\fg^i (X, \ZZ/\ell^r (n)) \cong
      \varprojlim_r H_\fg^i (X, \ZZ (n))/\ell^r \cong
      H_\fg^i (X, \ZZ (n)) \otimes \ZZ_\ell. \qedhere \]
  \end{proof}
\end{proposition}

The groups $H_\fg^i (X, \ZZ (n))$ provide an integral model for $\ell$-adic
cohomology in the following sense (see also \cite[\S 8]{Geisser-2004}).

\begin{corollary}
  \label{cor:RGamma-fg-model-for-l-adic-cohomology}
  Let $X$ be an arithmetic scheme satisfying Conjecture
  $\mathbf{L}^c (X_\et, n)$ for $n < 0$. Then
  $$H_\fg^i (X, \ZZ (n)) \otimes \ZZ_\ell \cong H^i_c (X [1/\ell]_\et, \ZZ_\ell (n)),$$
  where the right-hand side denotes $\ell$-adic cohomology with compact support.

  \begin{proof}
    We have $\ZZ (n)/\ell^r \cong j_{\ell!} \mu_m^{\otimes n}$.
    Now by part 3) of the previous proposition,
    \[ H_\fg^i (X, \ZZ (n)) \otimes \ZZ_\ell \cong
      \varprojlim_r H_c^i (X_\et, j_{\ell!} \mu_{\ell^r}^{\otimes n}) \cong
      \varprojlim_r H_c^i (X [1/\ell]_\et, \mu_{\ell^r}^{\otimes n})
      \stackrel{\text{dfn}}{=} H_c^i (X [1/\ell]_\et, \ZZ_\ell (n)). \qedhere \]
  \end{proof}
\end{corollary}

%%%%%%%%%%%%%%%%%%%%%%%%%%%%%%%%%%%%%%%%%%%%%%%%%%%%%%%%%%%%%%%%%%%%%%%%%%%%%%%%

\section{Proof of Theorem~II}
\label{sec:theorem-II}

The aim of this section is to prove Theorem~\ref{theorem-II}. We recall that it
states that the morphism of complexes $u_\infty^*$, defined as the composition
\[ \begin{tikzcd}
  R\Gamma_c (X_\et, \ZZ(n)) \ar[equals]{d}\ar[dashed]{r}{u_\infty^*} & R\Gamma_c (G_\RR, X (\CC), \ZZ (n)) \\
  R\Gamma_c (X_\et, \QQ/\ZZ (n)) [-1] \ar{r}{v_\infty^* [-1]} & R\Gamma_c (G_\RR, X (\CC), \QQ/\ZZ (n)) [-1] \ar{u}
\end{tikzcd} \]
is torsion. Here the morphism
$v_\infty^*\colon R\Gamma_c (X_\et, \QQ/\ZZ (n)) \to R\Gamma_c (G_\RR, X (\CC), \QQ/\ZZ (n))$
is induced by the comparison functor
$\alpha^*\colon \mathbf{Sh} (X_\et) \to \mathbf{Sh} (G_\RR, X (\CC))$, as
explained in Proposition~\ref{prop:inverse-image-gamma}. We first ensure that
$\alpha^*$ identifies the sheaf $\QQ/\ZZ (n)$ on $X_\et$ from
Definition~\ref{dfn:sheaf-Z(n)} with the $G_\RR$-equivariant sheaf
$\QQ/\ZZ (n) \dfn \frac{(2\pi i)^n\,\QQ}{(2\pi i)^n\,\ZZ}$ on $X (\CC)$.

\begin{proposition}
  \label{propn:image-of-Q/Zn-under-alpha}
  For the sheaf $\QQ/\ZZ (n)$ on $X_\et$ we have an isomorphism of
  $G_\RR$-equivariant constant sheaves on $X (\CC)$
  $$\alpha^* \QQ/\ZZ (n) \cong \QQ/\ZZ (n).$$

  \begin{proof}
    We first compute that the functor $\alpha^*$ sends the sheaf
    $\mu_m^{\otimes n}$ on $X_\et$ to the constant $G_\RR$-equivariant sheaf
    $\frac{(2\pi i)^n\,\ZZ}{m\,(2\pi i)^n\,\ZZ}$ on $X(\CC)$:
    \begin{align*}
      \alpha^* \mu_m^{\otimes n} & \cong \mu_m (\CC)^{\otimes n} \dfn \iHom (\mu_m (\CC)^{\otimes (-n)}, \ZZ/m\ZZ) \\
                                 & \cong \frac{(2\pi i)^n\,\ZZ}{m\,(2\pi i)^n\,\ZZ}
    \end{align*}
    ---here the first isomorphism comes from the definition of $\alpha^*$ given
    in Appendix~\ref{app:modified-cohomology-with-compact-support}, and the
    second isomorphism comes from the corresponding isomorphism of
    $G_\RR$-modules.

    Since $\alpha^*$ preserves colimits
    (Lemma~\ref{lemma:alpha-preserves-colimits}), we have
    \[
      \alpha^* \QQ/\ZZ (n)
      = \alpha^* \Bigl(\bigoplus_p \varinjlim_r j_{p!} \mu_{p^r}^{\otimes n}\Bigr)
      \cong \varinjlim_m \alpha^* \mu_m^{\otimes n}
      \cong \varinjlim_m \frac{(2\pi i)^n\,\ZZ}{m\,(2\pi i)^n\,\ZZ}
      \cong \frac{(2\pi i)^n\,\QQ}{(2\pi i)^n\,\ZZ}.
      \qedhere
    \]
  \end{proof}
\end{proposition}

We proceed with our proof of Theorem~\ref{theorem-II}. This seems nontrivial;
our argument (motivated by \cite{Flach-Morin-2018}, where it is given for a
proper and regular $X$) is based on the following result about $\ell$-adic
cohomology.

\begin{proposition}
  \label{prop:l-adic-cohomology-key-lemma}
  Let $X$ be an arithmetic scheme and $n < 0$. Then for any prime $\ell$ we have
  $$(H^i_c (X_{\overline{\QQ},\text{\it \'{e}t}}, \QQ_\ell/\ZZ_\ell (n))^{G_\QQ})_\div = 0.$$

  \begin{proof}
    According to the basic results on $\ell$-adic cohomology
    \cite[Expos\'{e}~VI]{SGA5}, there exists a prime $p \ne \ell$ such that
    \begin{equation}
      \label{eqn:iso-pbc-Zl-Gal(QS/Q)}
      H^i_c (X_{\overline{\QQ}, \text{\it \'{e}t}}, \ZZ_\ell (n)) \cong H^i_c (X_{\overline{\FF_p}, \text{\it \'{e}t}}, \ZZ_\ell (n)).
    \end{equation}

    We denote by $I_p$ the inertia subgroup of the absolute Galois group
    $G_{\QQ_p}$:
    $$1 \to I_p \to G_{\QQ_p} \to G_{\FF_p} \to 1$$

    The isomorphism \eqref{eqn:iso-pbc-Zl-Gal(QS/Q)} is equivariant under the
    $G_{\QQ_p}$-action on the left-hand side and $G_{\FF_p}$-action on the
    right-hand side. We have
    \[ H^i_c (X_{\overline{\QQ}, \text{\it \'{e}t}}, \QQ_\ell/\ZZ_\ell (n))^{G_\QQ} \rightarrowtail
    H^i_c (X_{\overline{\QQ}, \text{\it \'{e}t}}, \QQ_\ell/\ZZ_\ell (n))^{G_{\QQ_p}/I_p}
    \cong H^i_c (X_{\overline{\FF_p}, \text{\it \'{e}t}}, \QQ_\ell/\ZZ_\ell (n))^{G_{\FF_p}}, \]
    so it suffices to show that
    $$(H^i_c (X_{\overline{\FF_p},\text{\it \'{e}t}}, \QQ_\ell/\ZZ_\ell (n))^{G_{\FF_p}})_\div = 0.$$

    The long exact sequence of $G_{\FF_p}$-modules
    \begin{multline*}
      \cdots \to
      H^i_c (X_{\overline{\FF_p},\text{\it \'{e}t}}, \ZZ_\ell (n)) \to
      H^i_c (X_{\overline{\FF_p},\text{\it \'{e}t}}, \QQ_\ell (n)) \to
      H^i_c (X_{\overline{\FF_p},\text{\it \'{e}t}}, \QQ_\ell/\ZZ_\ell (n)) \\
      \to H^{i+1}_c (X_{\overline{\FF_p},\text{\it \'{e}t}}, \ZZ_\ell (n)) \to
      \cdots
    \end{multline*}
    induces short exact sequences
    \begin{equation}
      \label{eqn:Zl-Ql-Ql/Zl-ses}
      0 \to H^i_c (X_{\overline{\FF_p},\text{\it \'{e}t}}, \ZZ_\ell (n))_\cotor \to
      H^i_c (X_{\overline{\FF_p},\text{\it \'{e}t}}, \QQ_\ell (n)) \to
      H^i_c (X_{\overline{\FF_p},\text{\it \'{e}t}}, \QQ_\ell/\ZZ_\ell (n))_\div \to 0
    \end{equation}

    According to \cite[Expos\'{e}~XXI, 5.5.3]{SGA7}, the eigenvalues of the
    geometric Frobenius acting on
    $H^i_c (X_{\overline{\FF_p},\text{\it \'{e}t}}, \QQ_\ell)$ are algebraic
    integers. After twisting $\QQ_\ell$ by $n$, the eigenvalues will lie in
    $p^{-n}\,\overline{\ZZ}$. Since $n < 0$ by our assumption, this implies
    that $1$ does not appear as an eigenvalue, and hence
    $$H^i_c (X_{\overline{\FF_p},\text{\it \'{e}t}}, \QQ_\ell (n))^{G_{\FF_p}} = 0.$$
    Thus, after taking the $G_{\FF_p}$-invariants in
    \eqref{eqn:Zl-Ql-Ql/Zl-ses}, we obtain
    \[
      0 \to (H^i_c (X_{\overline{\FF_p},\text{\it \'{e}t}}, \QQ_\ell/\ZZ_\ell (n))_\div)^{G_{\FF_p}} \to
      H^1 (G_{\FF_p}, H^i_c (X_{\overline{\FF_p},\text{\it \'{e}t}}, \ZZ_\ell (n))_\cotor) \to \cdots
    \]
    This gives a monomorphism between the maximal divisible subgroups
    \[ ((H^i_c (X_{\overline{\FF_p},\text{\it \'{e}t}}, \QQ_\ell/\ZZ_\ell (n))_\div)^{G_{\FF_p}})_\div \rightarrowtail
    H^1 (G_{\FF_p}, H^i_c (X_{\overline{\FF_p},\text{\it \'{e}t}}, \ZZ_\ell (n))_\cotor)_\div. \]
    However,
    $H^1 (G_{\FF_p}, H^i_c (X_{\overline{\FF_p},\text{\it \'{e}t}}, \ZZ_\ell (n))_\cotor)$
    is a finitely generated $\ZZ_\ell$-module, and therefore its
    maximal divisible subgroup is trivial. We conclude that
    \[ (H^i_c (X_{\overline{\FF_p},\text{\it \'{e}t}}, \QQ_\ell/\ZZ_\ell (n))^{G_{\FF_p}})_\div =
      ((H^i_c (X_{\overline{\FF_p},\text{\it \'{e}t}}, \QQ_\ell/\ZZ_\ell (n))_\div)^{G_{\FF_p}})_\div = 0. \qedhere \]
  \end{proof}
\end{proposition}

\begin{proof}[Proof of Theorem~\ref{theorem-II}]
  By Definition~\ref{dfn:u-infty}, this amounts to showing that the morphism
  $$v_\infty^*\colon R\Gamma_c (X_\et, \QQ/\ZZ (n)) \to R\Gamma_c (G_\RR, X (\CC), \QQ/\ZZ (n))$$
  is torsion. The complexes $R\Gamma_c (X_\et, \QQ/\ZZ (n))$ and
  $R\Gamma_c (G_\RR, X (\CC), \QQ/\ZZ (n))$ are almost of cofinite type by
  Proposition~\ref{prop:motivic-cohomology-duality-consequences} and
  Proposition~\ref{prop:equivariant-coho-of-X(C)} respectively.
  Therefore, according to Lemma~\ref{lemma:torsion-morphisms-in-D(Z)}, to show
  that
  $v^*_\infty\colon R\Gamma_c (X_\et, \QQ/\ZZ (n)) \to R\Gamma_c (G_\RR, X (\CC), \QQ/\ZZ (n))$
  is torsion, it suffices to show that the corresponding morphisms on the
  maximal divisible subgroups
  \[ H^i_c (v^*_\infty)_\div\colon H^i_c (X_\et, \QQ/\ZZ (n))_\div \to
     H^i_c (G_\RR, X (\CC), \QQ/\ZZ (n))_\div \]
  are trivial. The morphism $H^i_c (v^*_\infty)$ factors through
  $H^i_c (X_{\overline{\QQ}, \text{\it \'{e}t}}, \mu^{\otimes n})^{G_\QQ}$, where
  $\mu^{\otimes n}$ is the sheaf of all roots of unity on
  $X_{\overline{\QQ}, \text{\it \'{e}t}}$ twisted by $n$.
  So we have
  \[ \begin{tikzcd}[column sep=0pt]
    H^i_c (X_\et, \QQ/\ZZ (n))_\div\ar{rr}{H^i_c (v^*_\infty)_\div}\ar[dashed]{dr} && H^i_c (G_\RR, X (\CC), \QQ/\ZZ (n))_\div\\
    & \left(H^i_c (X_{\overline{\QQ}, \text{\it \'{e}t}}, \mu^{\otimes n})^{G_\QQ}\right)_\div\ar[dashed]{ur}
  \end{tikzcd} \]
  Now
  \begin{align*}
    \left(H^i_c (X_{\overline{\QQ}, \text{\it \'{e}t}}, \mu^{\otimes n})^{G_\QQ}\right)_\div & \cong
                                                                                               \left(\bigoplus_\ell H^i_c (X_{\overline{\QQ}, \text{\it \'{e}t}}, \QQ_\ell/\ZZ_\ell (n))^{G_\QQ}\right)_\div \\
                                                                                             & \cong
                                                                                               \bigoplus_\ell \left(H^i_c (X_{\overline{\QQ}, \text{\it \'{e}t}}, \QQ_\ell/\ZZ_\ell (n))^{G_\QQ}\right)_\div,
  \end{align*}
  where all the summands are trivial according by
  Proposition~\ref{prop:l-adic-cohomology-key-lemma}.
\end{proof}

%%%%%%%%%%%%%%%%%%%%%%%%%%%%%%%%%%%%%%%%%%%%%%%%%%%%%%%%%%%%%%%%%%%%%%%%%%%%%%%%

\section{Weil-\'{e}tale complex $R\Gamma_\Wc (X, \ZZ(n))$}
\label{sec:RGamma-Wc}

The aim of this section is to construct the Weil-\'{e}tale cohomology complexes
$R\Gamma_\Wc (X, \ZZ(n))$.

\begin{lemma}
  Let $X$ be an arithmetic scheme and $n < 0$. Assume Conjecture
  $\mathbf{L}^c (X_\et, n)$, so that the morphism $\alpha_{X,n}$ exists.
  Then $u_\infty^* \circ \alpha_{X,n} = 0$.

  \[ \begin{tikzcd}
    \RHom (R\Gamma (X, \ZZ^c (n)), \QQ [-2]) \ar{d}[swap]{\alpha_{X,n}}\ar{dr}{= 0} \\
      R\Gamma_c (X_\et, \ZZ (n)) \ar{r}{u_\infty^*} & R\Gamma_c (G_\RR, X (\CC), \ZZ (n))
    \end{tikzcd} \]

  \begin{proof}
    The morphism $\alpha_{X,n}$ is defined on a complex of $\QQ$-vector spaces,
    and $u_\infty^*$ is torsion by Theorem~\ref{theorem-II}.
  \end{proof}
\end{lemma}

\begin{definition}
  \label{dfn:i-infty}
  We let
  $i_\infty^*\colon R\Gamma_\fg (X, \ZZ (n)) \to R\Gamma_c (G_\RR, X (\CC), \ZZ (n))$
  be a morphism in $\DZ$ that gives a morphism of distinguished triangles
  \begin{equation}
    \label{eqn:triangle-defining-i-infty}
    \begin{tikzcd}
      \RHom (R\Gamma (X, \ZZ^c (n)), \QQ [-2]) \ar{d}[swap]{\alpha_{X,n}}\ar{r} & 0\ar{d} \\
      R\Gamma_c (X_\et, \ZZ (n)) \ar{r}{u_\infty^*}\ar{d} &  R\Gamma_c (G_\RR, X (\CC), \ZZ (n)) \ar{d}{id} \\
      R\Gamma_\fg (X, \ZZ (n)) \ar[dashed]{r}{i_\infty^*}\ar{d} & R\Gamma_c (G_\RR, X (\CC), \ZZ (n)) \ar{d} \\
      \RHom (R\Gamma (X, \ZZ^c (n)), \QQ [-1])\ar{r} & 0 \\
    \end{tikzcd}
  \end{equation}
\end{definition}

In fact, this makes $i_\infty^*$ independent of any choices.

\begin{proposition}
  \label{prop:uniqueness-of-i-infty}
  There is a unique morphism $i_\infty^*$ that fits in the
  diagram \eqref{eqn:triangle-defining-i-infty}.

  \begin{proof}
    We can apply Corollary~\ref{cor:TR3-TR1-with-uniqueness}, since
    $\RHom (R\Gamma (X, \ZZ^c (n)), \QQ [-2])$ is a complex of $\QQ$-vector
    spaces, and both
    $R\Gamma_\fg (X, \ZZ (n))$ and
    $R\Gamma_c (G_\RR, X (\CC), \ZZ (n))$
    are almost perfect by Proposition~\ref{prop:RGammafg-almost-perfect} and
    Proposition~\ref{prop:equivariant-coho-of-X(C)}.
  \end{proof}
\end{proposition}

\begin{proposition}
  \label{i-infty-is-torsion}
  The morphism $i_\infty^*$ is torsion.

  \begin{proof}
    Let us examine the morphism of distinguished triangles
    \eqref{eqn:triangle-defining-i-infty} that defines $i_\infty^*$; in
    particular, the commutative diagram
    \[ \begin{tikzcd}
      R\Gamma_c (X_\et, \ZZ (n)) \ar{r}\ar{d}[swap]{u_\infty^*} & R\Gamma_\fg (X, \ZZ (n))\ar{dl}{i_\infty^*} \\
      R\Gamma_c (G_\RR, X (\CC), \ZZ (n))
    \end{tikzcd} \]

    According to Corollary~\ref{cor:TR3-TR1-with-uniqueness}, the morphism
    \begin{multline*}
      \Hom_\DZ (R\Gamma_\fg (X,\ZZ (n)), R\Gamma_c (G_\RR, X (\CC), \ZZ (n))) \\
      \to
      \Hom_\DZ (R\Gamma_c (X_\et, \ZZ (n)), R\Gamma_c (G_\RR, X (\CC), \ZZ (n)))
    \end{multline*}
    induced by the composition with
    $R\Gamma_c (X_\et, \ZZ (n)) \to R\Gamma_\fg (X,\ZZ (n))$, is mono, and
    therefore
    \begin{multline*}
      \Hom_\DZ (R\Gamma_\fg (X,\ZZ (n)), R\Gamma_c (G_\RR, X (\CC), \ZZ (n)))\otimes \QQ \to\\
      \Hom_\DZ (R\Gamma_c (X_\et, \ZZ (n)), R\Gamma_c (G_\RR, X (\CC), \ZZ (n)))\otimes \QQ
    \end{multline*}
    is also mono. However, $u_\infty^*\otimes \QQ = 0$ by
    Theorem~\ref{theorem-II}, and this implies that $i_\infty^*\otimes \QQ = 0$.
  \end{proof}
\end{proposition}

We are now ready to define the Weil-\'{e}tale complexes.

\begin{definition}
  \label{dfn:RGammaWc}
  We let
  $R\Gamma_\Wc (X,\ZZ(n))$ be an object in the derived category
  $\DZ$ which is a mapping fiber of $i_\infty^*$:
  \[ R\Gamma_\Wc (X,\ZZ(n)) \to
  R\Gamma_\fg (X, \ZZ (n)) \xrightarrow{i_\infty^*}
  R\Gamma_c (G_\RR, X (\CC), \ZZ (n)) \to
  R\Gamma_\Wc (X,\ZZ(n)) [1] \]
  The \textbf{Weil-\'{e}tale cohomology with compact support} is given by
  $$H_\Wc^i (X, \ZZ (n)) \dfn H^i (R\Gamma_\Wc (X,\ZZ(n))).$$
\end{definition}

\begin{remark}
  Note that this defines $R\Gamma_\Wc (X,\ZZ(n))$ up to a non-unique isomorphism
  in $\DZ$, and the groups $H_\Wc^i (X, \ZZ (n))$ are also defined
  up to a non-unique isomorphism. In a continuation of this paper we will make
  use of the determinant $\det\nolimits_\ZZ R\Gamma_\Wc (X,\ZZ(n))$ in the
  sense of \cite{Knudsen-Mumford-1976}, which will be defined up to a canonical
  isomorphism.

  However, we recall from Proposition~\ref{prop:RGamma-fg-well-defined} that
  $R\Gamma_\fg (X, \ZZ (n))$ is defined up to a unique isomorphism in the
  derived category $\DZ$. If we could define
  $i_\infty^*\colon R\Gamma_\fg (X, \ZZ(n)) \to R\Gamma_c (G_\RR, X(\CC),
  \ZZ(n))$ as an explicit, genuine morphism of complexes (not just as a morphism
  in the derived category $\DZ$), this would give us a canonical and functorial
  definition for $R\Gamma_\Wc (X, \ZZ(n))$.
\end{remark}

\subsection*{Case of varieties over finite fields}

For varieties over finite fields, our Weil-\'{e}tale cohomology has a simple
description, and it is $\QQ/\ZZ$-dual to the arithmetic homology studied by
Geisser in \cite{Geisser-2010-arithmetic-homology}.

\begin{proposition}
  If $X$ is a variety over a finite field $\FF_q$, then assuming
  $\mathbf{L}^c (X,n)$, there is an isomorphism of complexes
  \begin{equation}
    \label{eqn:RGamma-Wc-over-finite-fields}
    R\Gamma_\Wc (X,\ZZ(n)) \cong \RHom (R\Gamma (X_\et, \ZZ^c (n)), \ZZ [-1]),
  \end{equation}
  and an isomorphism of finite groups
  \begin{align*}
    H^i_{W,c} (X, \ZZ (n)) & \cong
                             \Hom (H^{2-i} (X_\et, \ZZ^c (n)), \QQ/\ZZ)\\
                           & \cong
                             H^i_c (X_\et, \ZZ(n)) \\
                           & \cong
                             \Hom (H_{i-1}^c (X_\ar, \ZZ (n)), \QQ/\ZZ),
  \end{align*}
  where $H_\bullet^c (X_\ar, \ZZ (n))$ are the arithmetic homology groups
  defined in {\rm \cite[\S 3]{Geisser-2010-arithmetic-homology}}.

  \begin{proof}
    Under our assumptions, $X (\CC) = \emptyset$, and therefore
    $R\Gamma_c (G_\RR, X (\CC), \ZZ (n)) = 0$, so that
    $R\Gamma_\Wc (X,\ZZ(n)) \cong R\Gamma_\fg (X, \ZZ (n))$. Finally, by
    Proposition~\ref{prop:RGamma-fg-for-X(R)-empty}, we have an isomorphism
    $R\Gamma_\fg (X, \ZZ (n)) \cong \RHom (R\Gamma (X_\et, \ZZ^c (n)), \ZZ
    [-1])$.  We recall from
    Proposition~\ref{prop:motivic-cohomology-duality-consequences} that the
    groups $H^i (X_\et, \ZZ^c(n))$ are finite under our assumption.

    To relate this to Geisser's arithmetic homology, according to
    \cite[Theorem~3.1]{Geisser-2010-arithmetic-homology}, there is a long exact
    sequence
    \[ \cdots \to H_{i-1}^c (X_\et, \ZZ (n)) \to
      H_i^c (X_\ar, \ZZ (n)) \to CH_n (X, i-2n)_\QQ \to
      H_{i-2}^c (X_\et, \ZZ (n)) \to \cdots \]
    Here the homological notation means that
    \begin{align*}
      H_i^c (X_\et, \ZZ (n)) & = H^{-i} (X_\et, \ZZ^c (n)), \\
      CH_n (X, i-2n)_\QQ & = H_i^c (X_\et, \QQ (n)) = 0,
    \end{align*}
    and therefore
    $$H_i^c (X_\ar, \ZZ (n)) \cong H^{1-i} (X_\et, \ZZ^c (n)).$$

    Now \eqref{eqn:RGamma-Wc-over-finite-fields} gives
    \begin{equation}
      \label{eqn:RGamma-Wc-over-finite-fields-ss}
      E_2^{p,q} = \Ext_\ZZ^p (H^{1-q} (X_\et, \ZZ^c (n)), \ZZ) \Longrightarrow
      H^{p+q}_{W,c} (X, \ZZ (n)),
    \end{equation}
    and again, by finiteness of $H^{1-q} (X_\et, \ZZ^c (n))$, this spectral
    sequence is concentrated in $p = 1$, where the interesting terms are
    \[ \Ext_\ZZ^1 (H^{1-q} (X_\et, \ZZ^c (n)), \ZZ) \cong
      \Hom (H^{1-q} (X_\et, \ZZ^c (n)), \QQ/\ZZ), \]
    so that
    \[ H^{1+i}_{W,c} (X, \ZZ (n)) \cong
      \Hom (H^{1-i} (X_\et, \ZZ^c (n)), \QQ/\ZZ) \cong
      \Hom (H_i^c (X_\ar, \ZZ (n)), \QQ/\ZZ). \qedhere \]
  \end{proof}
\end{proposition}

\subsection*{Perfectness of the complex}

Our next aim is to verify that $R\Gamma_\Wc (X, \ZZ(n))$ is a perfect
complex. From now on we tacitly assume Conjecture $\mathbf{L}^c (X_\et,n)$.

\begin{lemma}
  The groups $H^i_\Wc (X, \ZZ(n))$ are finitely generated for all $i \in \ZZ$.

  \begin{proof}
    In the long exact sequence
    \begin{multline*}
      \cdots \to H^{i-1}_c (G_\RR, X (\CC), \ZZ (n)) \to
      H^i_\Wc (X,\ZZ(n)) \to
      H^i_\fg (X,\ZZ(n)) \\
      \xrightarrow{H^i (i_\infty^*)}
      H^i_c (G_\RR, X (\CC), \ZZ (n)) \to \cdots
    \end{multline*}
    the groups $H^i_c (G_\RR, X (\CC), \ZZ (n))$ and $H^i_\fg (X, \ZZ(n))$ are
    finitely generated by
    Proposition~\ref{prop:equivariant-coho-of-X(C)}, and
    Proposition~\ref{prop:RGammafg-almost-perfect}, respectively.
    This implies the finite generation of $H^i_\Wc (X, \ZZ(n))$.
  \end{proof}
\end{lemma}

\begin{lemma}
  One has $H^i_\Wc (X,\ZZ(n)) = 0$ for $i < 0$.
  % If we further assume that $H^1 (X_\et, \ZZ^c(n))$ is a finite group, then
  % $H^0_\Wc (X,\ZZ(n)) = 0$.

  \begin{proof}
    The definitions of $R\Gamma_\fg (X, \ZZ(n))$ and $R\Gamma_\Wc (X, \ZZ(n))$
    yield exact sequences
    \[ \begin{tikzcd}[row sep=1em,column sep=1em]
        &[-1em] H^{i-1}_c (G_\RR, X (\CC), \ZZ (n)) \ar{d} &[-1em] \\
        & H^i_\Wc (X, \ZZ(n)) \ar{d} \\
        H^i_c (X_\et, \ZZ(n)) \ar{r} & H^i_\fg (X, \ZZ(n)) \ar{r}\ar{d} & \Hom (H^{1-i} (X_\et, \ZZ^c (n)), \QQ) \ar{r} & H^{i+1}_c (X_\et, \ZZ(n)) \\
        & H^i_c (G_\RR, X (\CC), \ZZ (n))
      \end{tikzcd} \]
    If $i < 0$, then
    $H^i_c (X_\et, \ZZ(n)) = H^i_c (G_\RR, X (\CC), \ZZ (n)) = 0$.
    Moreover, $\Hom (H^{1-i} (X_\et, \ZZ^c (n)), \QQ) = 0$ for $i < 0$, since
    $H^{1-i} (X_\et, \ZZ^c (n))$ is finite $2$-torsion
    (Proposition~\ref{prop:motivic-cohomology-duality-consequences}).  We
    conclude that $H^i_\Wc (X, \ZZ(n)) = H^i_\fg (X, \ZZ(n)) = 0$ for $i < 0$.
  \end{proof}
\end{lemma}

For the vanishing of $H^i_\Wc (X, \ZZ(n))$ for $i \gg 0$, we first establish the
following auxiliary result.

\begin{lemma}
  Let $d = \dim X$. For each prime $\ell$ and $i \ge 2d$ we have
  \begin{equation}
    \label{eqn:l-adic-completion-of-H-Wc}
    H^i_\Wc (X, \ZZ (n)) \otimes \ZZ_\ell =
    \widehat{H}_c^i (X [1/\ell]_\et, \ZZ_\ell (n)),
  \end{equation}
  where the right-hand side is defined via
  $\varprojlim_r \widehat{H}_c^i (X [1/\ell]_\et, \mu_{\ell^r}^{\otimes n})$.

  \begin{proof}
    Consider the commutative diagram with distinguished rows and columns
    \[ \begin{tikzcd}[font=\footnotesize]
        {[R\Gamma (X_\et, \ZZ^c(n)), \QQ[-2]]} \ar{r}{\widehat{\alpha}_{X,n}} \ar{d}{id} & R\widehat{\Gamma}_c (X_\et, \ZZ(n)) \ar{d} \ar{r} & R\widehat{\Gamma}_\fg (X, \ZZ(n)) \ar{r}\ar{d} & {[+1]} \ar{d}{id} \\
        {[R\Gamma (X_\et, \ZZ^c(n)), \QQ[-2]]} \ar{r}{\alpha_{X,n}} \ar{d} & R\Gamma_c (X_\et, \ZZ(n)) \ar{r} \ar{d}{\widehat{u}^*_\infty} & R\Gamma_\fg (X, \ZZ(n)) \ar{r} \ar{d}{\widehat{i}^*_\infty} & {[+1]} \ar{d} \\
        0 \ar{r}\ar{d} & R\widehat{\Gamma}_c (G_\RR, X(\CC), \ZZ(n)) \ar{r}{id} \ar{d} & R\widehat{\Gamma}_c (G_\RR, X(\CC), \ZZ(n)) \ar{r} \ar{d} & 0 \ar{d} \\
        {[R\Gamma (X_\et, \ZZ^c(n)), \QQ[-1]]} \ar{r}{\widehat{\alpha}_{X,n} [1]} & R\widehat{\Gamma}_c (X_\et, \ZZ(n)) [1] \ar{r} & R\widehat{\Gamma}_\fg (X, \ZZ(n)) [1] \ar{r} & {[+2]}
      \end{tikzcd} \]
    Here $\widehat{u}^*_\infty$ (resp. $\widehat{i}^*_\infty$) is defined as the
    composition of the canonical morphism $u^*_\infty$ (resp. $i^*_\infty$) with
    the projection to the Tate cohomology
    \[ \pi\colon R\Gamma_c (G_\RR, X(\CC), \ZZ(n)) \to
      R\widehat{\Gamma}_c (G_\RR, X(\CC), \ZZ(n)). \]
    By Lemma~\ref{lemma:Tate-vs-normal-cohomology-of-X(C)}, $H^i (\pi)$ is an
    isomorphism for $i \ge 2d-1$. Therefore, the five-lemma applied to
    \[ \begin{tikzcd}
        R\Gamma_\Wc (X, \ZZ(n)) \ar{r}\ar{d}{f} & R\Gamma_\fg (X, \ZZ(n)) \ar{r}{i^*_\infty}\ar{d}{id} & R\Gamma_c (G_\RR, X(\CC), \ZZ(n)) \ar{r}\ar{d}{\pi} & {[+1]}\ar{d}{f [1]} \\
        R\widehat{\Gamma}_\fg (X, \ZZ(n)) \ar{r} & R\Gamma_\fg (X, \ZZ(n)) \ar{r}{\widehat{i}^*_\infty} & R\widehat{\Gamma}_c (G_\RR, X(\CC), \ZZ(n)) \ar{r} & {[+1]}
      \end{tikzcd} \]
    shows that for $i \ge 2d$ holds
    \[ H^i_\Wc (X, \ZZ(n)) \cong \widehat{H}^i_\fg (X, \ZZ(n)). \]
    As in Corollary~\ref{cor:RGamma-fg-model-for-l-adic-cohomology}, we have for a
    prime $\ell$
    \[ \widehat{H}^i_\fg (X, \ZZ(n)) \otimes \ZZ_\ell \cong \widehat{H}^i_c (X[1/\ell]_\et, \ZZ_\ell (n)). \qedhere \]
  \end{proof}
\end{lemma}

% \begin{proof}
%   Consider the commutative diagram
%   \[ \begin{tikzcd}[row sep=3em, column sep=3em]
%       R\Gamma_\Wc (X, \ZZ (n)) \otimes^\mathbf{L} \ZZ/\ell^r \ar{r} & R\Gamma_\fg (X, \ZZ(n)) \otimes^\mathbf{L} \ZZ/\ell^r \ar{r}{i_\infty^* \otimes \ZZ/\ell^r}\ar{d}[swap]{\cong} & R\Gamma_c (G_\RR, X(\CC), \ZZ(n)) \otimes^\mathbf{L} \ZZ/\ell^r\ar{d} \\
%       R\widehat{\Gamma}_c (X_\et, \ZZ (n)) \otimes^\mathbf{L} \ZZ/\ell^r \ar{r} & R\Gamma_c (X_\et, \ZZ (n)) \otimes^\mathbf{L} \ZZ/\ell^r \ar{r}\ar{ur}{u_\infty^* \otimes \ZZ/\ell^r} & R\widehat{\Gamma}_c (G_\RR, X(\CC), \ZZ(n)) \otimes^\mathbf{L} \ZZ/\ell^r
%     \end{tikzcd} \]

%   The left vertical arrow is the inverse of the canonical morphism
%   $R\Gamma_c (X_\et, \ZZ(n)) \to R\Gamma_\fg (X, \ZZ(n))$, which becomes a
%   quasi-isomorphism after tensoring with $\ZZ/\ell^r\ZZ$
%   (see Proposition~\ref{prop:tensoring-RGammafg-with-Z/m-and-Q}).
%   The mapping fiber of the top horizontal arrow is
%   $R\Gamma_\Wc (X, \ZZ (n)) \otimes^\mathbf{L} \ZZ/\ell^r$, while the
%   mapping fiber of the bottom horizontal arrow is modified \'{e}tale cohomology
%   with compact support
%   $R\widehat{\Gamma}_c (X_\et, \ZZ (n)) \otimes^\mathbf{L} \ZZ/\ell^r$
%   by \cite[Lemma~6.14]{Flach-Morin-2018}. Therefore, we have a comparison
%   morphism
%   \begin{equation}
%     \label{eqn:comparison-of-H-Wc-mod-lr}
%     R\Gamma_\Wc (X, \ZZ (n)) \otimes^\mathbf{L} \ZZ/\ell^r \to
%     R\widehat{\Gamma}_c (X_\et, \ZZ (n)) \otimes^\mathbf{L} \ZZ/\ell^r.
%   \end{equation}
%   We recall that the arrow
%   $R\Gamma_c (G_\RR, X(\CC), \ZZ(n)) \to R\widehat{\Gamma}_c (G_\RR, X(\CC), \ZZ(n))$
%   induces isomorphisms in cohomology $H^i$ for $i \ge 2d-1$
%   (see Lemma~\ref{lemma:Tate-vs-normal-cohomology-of-X(C)}).
%   Taking the limits $\varprojlim_r$ and applying the five-lemma,
%   we see that \eqref{eqn:comparison-of-H-Wc-mod-lr} induces the isomorphism
%   \eqref{eqn:l-adic-completion-of-H-Wc} for $i \ge 2d$.
% \end{proof}

\begin{corollary}
  One has $H^i_\Wc (X, \ZZ(n)) = 0$ for $i > 2d+1$.

  \begin{proof}
    It suffices to verify that $H^i_\Wc (X, \ZZ(n)) \otimes \ZZ_\ell = 0$
    for each prime $\ell$. Thanks to the isomorphism
    \eqref{eqn:l-adic-completion-of-H-Wc}, this reduces to
    $\widehat{H}_c^i (X [1/\ell]_\et, \ZZ_\ell (n)) = 0$ for $i > 2d+1$,
    which is true for the reasons of cohomological dimension
    \cite[Expos\'{e}~X, Th\'{e}or\`{e}me~6.2]{SGA4}. We note that if $\ell = 2$ and
    $X (\RR) \ne \emptyset$, then the usual \'{e}tale cohomology has finite
    $2$-torsion in arbitrarily high degrees. It is important that we consider
    here the \emph{modified} cohomology with compact support
    $\widehat{H}_c^i (-)$. To obtain the corresponding statement, combine the
    arguments from \cite[Expos\'{e}~X]{SGA4} with the well-known computations of
    modified cohomology for number fields; cf. \cite[Chapter~II]{Milne-ADT} and
    \cite{Artin-Verdier-1964}, \cite{Mazur-1973}.
  \end{proof}
\end{corollary}

Summarizing the above observations, we obtain the following result.

\begin{proposition}
  \label{prop:RGammaWc-perfect}
  Conjecture $\mathbf{L}^c (X_\et,n)$ implies that $R\Gamma_\Wc (X, \ZZ(n))$
  is a perfect complex. More precisely, $H^i_\Wc (X, \ZZ(n))$ are finitely
  generated groups, and $H^i_\Wc (X, \ZZ(n)) = 0$ for
  $i \notin [0, 2\dim X + 1]$.
\end{proposition}

\subsection*{Rational coefficients}

\begin{proposition}
  There is a non-canonical splitting
  \[ R\Gamma_\Wc (X, \ZZ(n)) \otimes \QQ \cong
    \RHom (R\Gamma (X_\et, \ZZ^c (n)), \QQ) [-1] \oplus
    R\Gamma_c (G_\RR, X (\CC), \QQ (n)) [-1]. \]

  \begin{proof}
    The distinguished triangle defining $R\Gamma_\Wc (X, \ZZ(n))$ becomes after
    tensoring with $\QQ$
    \begin{multline*}
      R\Gamma_\Wc (X, \ZZ (n))\otimes \QQ \to
      R\Gamma_\fg (X, \ZZ(n))\otimes \QQ \xrightarrow{i_\infty^*\otimes\QQ = 0}
      R\Gamma_c (G_\RR, X (\CC), \ZZ(n))\otimes \QQ \\
      \to R\Gamma_\Wc (X, \ZZ (n))\otimes \QQ [1]
    \end{multline*}
    which yields a non-canonical splitting \cite[Chapitre~II,
    Corollaire~1.2.6]{Verdier-thesis}
    \[ R\Gamma_\Wc (X, \ZZ (n))\otimes \QQ \cong
      R\Gamma_\fg (X, \ZZ(n))\otimes \QQ \oplus
      R\Gamma_c (G_\RR, X (\CC), \ZZ(n)) [-1]\otimes \QQ, \]
    and we have already established in
    Proposition~\ref{prop:tensoring-RGammafg-with-Z/m-and-Q} that
    \[ R\Gamma_\fg (X, \ZZ (n)) \otimes \QQ \cong
      \RHom (R\Gamma (X_\et, \ZZ^c (n)), \QQ) [-1]. \qedhere \]
  \end{proof}
\end{proposition}

%%%%%%%%%%%%%%%%%%%%%%%%%%%%%%%%%%%%%%%%%%%%%%%%%%%%%%%%%%%%%%%%%%%%%%%%%%%%%%%%

\section{Known cases of Conjecture $\mathbf{L}^c (X_\et,n)$}
\label{sec:known-cases-of-Lc-Xet-n}

Since the main constructions of this paper assume Conjecture
$\mathbf{L}^c (X_\et,n)$, we relate it here to other known conjectures about the
finite generation of \'{e}tale motivic cohomology, and also describe certain
schemes $X$ for which $\mathbf{L}^c (X_\et,n)$ holds unconditionally.

\vspace{1em}

Flach and Morin state in \cite{Flach-Morin-2018} a slightly different conjecture
$\mathbf{L} (X_\et, -)$ instead of our $\mathbf{L}^c (X_\et,-)$. Taking into
account the relation \eqref{eqn:Zc(n)-vs-Z(d-n)} for regular schemes, we can
reformulate their conjecture as follows.

\begin{conjecture}[{\cite[Conjecture~3.2; Lemma~3.3]{Flach-Morin-2018}}]
  $\mathbf{L} (X_\et, d-n)$: for a proper regular arithmetic scheme $X$ and
  $n < 0$, the groups $H^i (X_\et, \ZZ^c (n))$ are finitely generated for
  $i \le -2n+1$.
\end{conjecture}

A more precise conjectural description of \'{e}tale motivic cohomology is
\cite[Conjecture~4.12]{Geisser-2017}, which can be written as follows, again
using \eqref{eqn:Zc(n)-vs-Z(d-n)}:

\begin{conjecture}
  $\mathbf{L}' (X_\et, d-n)$: for a proper regular arithmetic scheme $X$ and
  $n < 0$, one has
  \[ H^i (X_\et, \ZZ^c (n)) = \begin{cases}
      \text{finitely generated}, & i \le -2n, \\
      \text{finite}, & i = -2n + 1, \\
      \text{cofinite type}, & i \ge -2n + 2.
    \end{cases} \]
\end{conjecture}

\begin{proposition}
  \label{prop:Lc-Xet-n-vs-L-Xet-d-n}
  Let $X$ be a proper regular arithmetic scheme of dimension $d$.
  Then for $n < 0$
  \[ \mathbf{L}^c (X_\et, n) \Longleftrightarrow
    \mathbf{L} (X_\et, d-n) \Longleftrightarrow
    \mathbf{L}' (X_\et, d-n). \]

  \begin{proof}
    The nontrivial implications are
    \[
      \mathbf{L} (X_\et, d-n) \Longrightarrow \mathbf{L}^c (X_\et, n),
      \quad
      \mathbf{L} (X_\et, d-n) \Longrightarrow \mathbf{L}' (X_\et, d-n).
    \]

    For the first implication, we note that by
    \cite[Proposition~3.4]{Flach-Morin-2018}, $\mathbf{L} (X_\et, d-n)$ implies
    the Artin--Verdier duality
    \[ H^i (X_\et, \ZZ (n)) \cong \Hom (H^{2-i} (X_\et, \ZZ^c (n)), \QQ/\ZZ)
      \text{ up to finite }2\text{-torsion}. \]
    Hence $H^i (X_\et, \ZZ^c (n))$ is finite $2$-torsion for $i \ge 2$, and in
    particular for $i > -2n + 1$.

    The second implication is also established in
    \cite[Proposition~3.4]{Flach-Morin-2018}.
  \end{proof}
\end{proposition}

We now list some special cases where Conjecture $\mathbf{L}^c (X_\et,n)$ is
known, and therefore gives unconditional results. We follow
\cite[\S 5]{Morin-2014} very closely. For an arithmetic scheme $X$, we formulate
the following conjecture, which is the conjunction of $\mathbf{L}^c (X_\et,n)$
for all $n < 0$.

\begin{conjecture}
  $\mathbf{L}^c (X_\et)$: the cohomology groups $H^i (X_\et, \ZZ^c (n))$ are
  finitely generated for all $i \in \ZZ$ and $n < 0$.
\end{conjecture}

This is similar to \cite[Definition~5.8]{Morin-2014}, with the only difference
that Morin also requires the finite generation of $H^i (X_\et, \ZZ^c (0))$ for
$i \le 0$. Conjecture $\mathbf{L}^c (X_\et)$ is known for number rings, and also
for certain varieties over finite fields. As in \cite{Soule-1984},
\cite{Geisser-2004}, and \cite{Morin-2014}, we consider the following class.

\begin{definition}
  Let $A (\FF_q)$ be the full subcategory of the category of smooth projective
  varieties over a finite field $\FF_q$ generated by products of curves and the
  following operations.
  \begin{enumerate}
  \item[1)] If $X$ and $Y$ lie in $A (\FF_q)$, then $X \sqcup Y$ lies
    $A (\FF_q)$.
  \item[2)] If $Y$ lies in $A (\FF_q)$ and there are morphisms $c\colon X\to Y$
    and $c'\colon Y\to X$ in the category of Chow motives such that
    $c'\circ c\colon X\to X$ is a multiplication by constant, then
    $X$ lies in $A (\FF_q)$.
  \item[3)] If $\FF_{q^m}/\FF_q$ is a finite extension and
    $X_{\FF_q^m} = X \times_{\Spec \FF_q} \Spec \FF_{q^m}$ lies in
    $A (\FF_{q^m})$, then $X$ lies in $A (\FF_q)$.
  \item[4)] If $X$ and $Y$ lie in $A (\FF_q)$, and $Y$ is a closed subscheme of
    $X$, then the blowup of $X$ along $Y$ lies in $A (\FF_q)$.
  \end{enumerate}
\end{definition}

The following is similar to \cite[Definition~5.9]{Morin-2014}.

\begin{definition}
  Let $\mathcal{L} (\ZZ)$ be the full subcategory of arithmetic schemes
  generated by the following objects:
  \begin{itemize}
  \item the empty scheme $\emptyset$,
  \item $\Spec \mathcal{O}_F$ for a number field $F$,
  \item varieties $X \in A (\FF_q)$ for any finite field $\FF_q$,
  \end{itemize}
  and the following operations.
  \begin{enumerate}
  \item[$\mathcal{L}$1)] Let $X$ be an arithmetic scheme, $Z \subset X$ a closed
    subscheme and $U \dfn X\setminus Z$ its open complement. If two of three
    schemes $X,Z,U$ lie in $\mathcal{L} (\ZZ)$, then the third also lies in
    $\mathcal{L} (\ZZ)$.

  \item[$\mathcal{L}$2)] A finite disjoint union
    $X = \coprod_{1 \le j \le p} X_j$ lies in $\mathcal{L} (\ZZ)$ if and only if
    each $X_j$ lies in $\mathcal{L} (\ZZ)$.

  \item[$\mathcal{L}$3)] If $V \to U$ is an affine bundle and $U$ lies in
    $\mathcal{L} (\ZZ)$, then $V$ also lies in $\mathcal{L} (\ZZ)$.

  \item[$\mathcal{L}$4)] If $\{ U_i \to X \}_{i \in I}$ is a finite surjective
    family of \'{e}tale morphisms such that each $U_{i_0,\ldots,i_p}$ lies in
    $\mathcal{L} (\ZZ)$, then $X$ also lies in $\mathcal{L} (\ZZ)$.
  \end{enumerate}
\end{definition}

\begin{proposition}
  Conjecture $\mathbf{L}^c (X_\et)$ holds for any arithmetic scheme
  $X \in \mathcal{L} (\ZZ)$.

  \begin{proof}
    See the argument in \cite[Proposition~5.10]{Morin-2014}.
  \end{proof}
\end{proposition}

Finally, we consider cellular schemes, as in \cite[\S 5.4]{Morin-2014}.

\begin{definition}
  Let $Y$ be a separated scheme of finite type over $\Spec k$ for a field
  $k$. We say that $Y$ \textbf{admits a cellular decomposition} if there exists
  a filtration of $Y$ by reduced closed subschemes
  $$Y^{red} = Y_N \supseteq Y_{N-1} \supseteq \cdots \supseteq Y_{-1} = \emptyset$$
  such that $Y_i\setminus Y_{i-1} \cong \AA^{r_i}_k$ is isomorphic to an affine
  space over $k$.

  We say that $Y$ is \textbf{geometrically cellular} if
  $Y_{\overline{k}} = Y \times_{\Spec k} \Spec \overline{k}$ admits a cellular
  decomposition. This is equivalent to the existence of a finite Galois
  extension $k'/k$ such that $Y_{k'}$ admits a cellular decomposition.

  Finally, given an $S$-scheme $X \to S$ that is separated and of finite type,
  we say that $X$ is \textbf{geometrically cellular} if for each $s \in S$ the
  corresponding fiber $X_s$ is geometrically cellular.
\end{definition}

\begin{proposition}
  Let $Y$ be a separated scheme of finite type over $\Spec \FF_q$.
  If $Y$ is geometrically cellular, then $X \in \mathcal{L} (\ZZ)$,
  and in particular Conjecture $\mathbf{L}^c (Y_\et)$ holds.

  If $X \to \Spec \mathcal{O}_F$ is a flat, separated scheme of finite type over
  the ring of integers of a number field, and $X$ is geometrically cellular,
  then $X \in \mathcal{L} (\ZZ)$, and in particular $\mathbf{L}^c (X_\et)$
  holds.
\end{proposition}

For a proof, we refer to \cite[Proposition~5.14]{Morin-2014}.

%%%%%%%%%%%%%%%%%%%%%%%%%%%%%%%%%%%%%%%%%%%%%%%%%%%%%%%%%%%%%%%%%%%%%%%%%%%%%%%%

\section{Comparison with the complex of Flach and Morin}
\label{sec:comparison-with-FM}

This paper is based on the ideas of Flach and Morin \cite{Flach-Morin-2018}, who
gave a similar construction of Weil-\'{e}tale cohomology
$R\Gamma_\Wc (X, \ZZ(n))$ for a \emph{proper and regular} arithmetic scheme $X$,
and for \emph{any integer weight} $n \in \ZZ$. In this section, we will go
through the definitions of \cite{Flach-Morin-2018}, to verify the following
claim.

\begin{proposition}
  \label{prop:comparison-with-FM}
  Let $X$ be a proper, regular arithmetic scheme, and $n < 0$. Assume
  Conjecture $\mathbf{L}^c (X_\et, n)$. Then the Weil-\'{e}tale complex
  $R\Gamma_\Wc (X, \ZZ(n))$ defined above in {\rm \S\ref{sec:RGamma-Wc}}
  is isomorphic to the corresponding complex defined in
  {\rm \cite{Flach-Morin-2018}}.
\end{proposition}

From now on we tacitly assume Conjecture $\mathbf{L}^c (X_\et, n)$, which is
also equivalent to the assumptions on motivic cohomology in
\cite{Flach-Morin-2018} (see
Proposition~\ref{prop:Lc-Xet-n-vs-L-Xet-d-n}). Flach and Morin consider the case
of a proper and regular arithmetic scheme $X$ of equal dimension $d$. In this
case, we can use the isomorphism \eqref{eqn:Zc(n)-vs-Z(d-n)} to reformulate
their constructions in terms of complexes $\ZZ^c (n)$.

Moreover, they work with the Artin--Verdier \'{e}tale topos $\overline{X}_\et$,
whose definition and basic properties can be found in
\cite[\S 6]{Flach-Morin-2018}. They consider a morphism
\[ \overline{\alpha}_{X,n}\colon
  \RHom (R\Gamma (X, \ZZ^c (n)), \QQ [-2]) \to
  R\Gamma (\overline{X}_\et, \ZZ (n)), \]
defined in a similar way to our $\alpha_{X,n}$ (Definition~\ref{def:RGamma-fg})
using a duality similar to our Theorem~\ref{theorem-I}.

The notation in \cite{Flach-Morin-2018} and in this paper is intentionally the
same for various objects and morphisms. However, in this section we will write,
for example, $\overline{\alpha}_{X,n}$ to denote the morphism of Flach and
Morin, to distinguish it from our $\alpha_{X,n}$, etc. An overline indicates
that the corresponding thing comes from \cite{Flach-Morin-2018} and has
something to do with the Artin--Verdier \'{e}tale topos.

\begin{lemma}
  The square
  \begin{equation}
    \label{eqn:alpha-vs-alpha-bar-square}
    \begin{tikzcd}
      \RHom (R\Gamma (X, \ZZ^c (n)), \QQ [-2]) \ar{r}{\overline{\alpha}_{X,n}}\ar{d}{id} & R\Gamma (\overline{X}_\et, \ZZ(n))\ar{d} \\
      \RHom (R\Gamma (X, \ZZ^c (n)), \QQ [-2]) \ar{r}{\alpha_{X,n}} & R\Gamma (X_\et, \ZZ(n))
    \end{tikzcd}
  \end{equation}
  commutes.

  \begin{proof}
    We recall from Remark~\ref{rmk:alpha-X-n-determined-by-cohomology} that
    $\alpha_{X,n}$ is determined by the maps at the level of cohomology
    $H^i (\alpha_{X,n})$. The same is true for $\overline{\alpha}_{X,n}$, for
    the same reasons. Now \cite[Theorem~3.5]{Flach-Morin-2018} defines
    \begin{multline*}
      H^i (\overline{\alpha}_{X,n})\colon
      \Hom (H^{2-i} (X, \ZZ^c (n)), \QQ) \xrightarrow{\cong}
      \Hom (H^{2-i} (\overline{X}_\et, \ZZ^c (n)), \QQ) \to \\
      \Hom (H^{2-i} (\overline{X}_\et, \ZZ^c (n)), \QQ/\ZZ) \xleftarrow{\cong}
      H^i (\overline{X}_\et, \ZZ (n)),
    \end{multline*}
    where the last isomorphism is the duality
    \cite[Corollary~6.26]{Flach-Morin-2018}. Similarly, our morphism
    $\alpha_{X,n}$ gives
    \begin{multline*}
      H^i (\alpha_{X,n})\colon
      \Hom (H^{2-i} (X, \ZZ^c (n)), \QQ) \xrightarrow{\cong}
      \Hom (H^{2-i} (X_\et, \ZZ^c (n)), \QQ) \to \\
      \Hom (H^{2-i} (X_\et, \ZZ^c (n)), \QQ/\ZZ) \xleftarrow{\cong}
      \widehat{H}^i_c (X_\et, \ZZ (n)) \to
      H^i (X_\et, \ZZ (n)).
    \end{multline*}

    The groups $\widehat{H}^i_c (X_\et, \ZZ (n))$ and
    $H^i (\overline{X}_\et, \ZZ (n))$ are different, but the duality in terms of
    $H^i (\overline{X}_\et, \ZZ (n))$ is induced precisely from the duality in
    terms of $\widehat{H}^i_c (X_\et, \ZZ (n))$
    (see \cite[Theorem~6.24]{Flach-Morin-2018}): we have a commutative diagram
    \[ \begin{tikzcd}
        R\widehat{\Gamma}_c (X_\et, \ZZ/m\ZZ(n)) \ar{d}\ar{r}{\cong} & \RHom (R\Gamma (X_\et, \ZZ/m\ZZ^c (n)), \QQ/\ZZ [-2])\ar{d} \\
        R\Gamma (\overline{X}_\et, \ZZ/m\ZZ(n)) \ar{r}{\cong} & \RHom (R\Gamma (\overline{X}_\et, \ZZ/m\ZZ^c (n)), \QQ/\ZZ [-2])
      \end{tikzcd} \]
    and the diagram
    \[ \begin{tikzcd}
        R\widehat{\Gamma}_c (X_\et, \ZZ(n)) \ar{r}\ar{d} & R\Gamma (X_\et, \ZZ(n)) \\
        R\Gamma (\overline{X}_\et, \ZZ(n) \ar{ur}
      \end{tikzcd} \]
    commutes as well. We see that the diagram we are interested in commutes:
    \[ \begin{tikzcd}[column sep=1.25em]
        \Hom (H^{2-i} (X, \ZZ^c (n)), \QQ) \ar{r}\ar{d}{id} \ar[bend left=15]{rrr}{H^i (\overline{\alpha}_{X,n})} & H^{2-i} (\overline{X}_\et, \ZZ^c (n))^D & & H^i (\overline{X}_\et, \ZZ(n))\ar{d} \ar{ll}[swap]{\cong} \\
        \Hom (H^{2-i} (X, \ZZ^c (n)), \QQ) \ar{r} \ar[bend right=15]{rrr}[swap]{H^i (\alpha_{X,n})} & H^{2-i} (X_\et, \ZZ^c (n))^D \ar[dashed]{u} & \widehat{H}^i_c (X_\et, \ZZ(n)) \ar{l}[swap]{\cong}\ar{r}\ar[dashed]{ur} & H^i (X_\et, \ZZ(n))
      \end{tikzcd} \]
    For brevity, $\Hom (A,\QQ/\ZZ)$ is denoted here by $A^D$.
  \end{proof}
\end{lemma}

Taking the cones of $\overline{\alpha}_{X,n}$ and $\alpha_{X,n}$, we obtain
respectively the complex $R\Gamma_W (\overline{X}, \ZZ (n))$ of Flach and Morin
\cite[Definition~3.6]{Flach-Morin-2018} and our complex
$R\Gamma_\fg (X, \ZZ(n))$ (Definition~\ref{def:RGamma-fg} above).

The square \eqref{eqn:alpha-vs-alpha-bar-square} induces the following diagram
with distinguished rows and columns (cf. \cite[Proposition~1.4.6]{Neeman-2001}):
\begin{equation}
  \label{eqn:cones-of-alphas}
  \begin{tikzcd}[column sep=1em,font=\small]
    {[R\Gamma (X, \ZZ^c (n)), \QQ [-2]]} \ar{r}{\overline{\alpha}_{X,n}}\ar{d}{id} & R\Gamma (\overline{X}_\et, \ZZ(n))\ar{d}\ar{r}{f} & R\Gamma_W (\overline{X}, \ZZ(n)) \ar{r}\ar[dashed]{d} & {[-1]}\ar{d}{id} \\
    {[R\Gamma (X, \ZZ^c (n)), \QQ [-2]]} \ar{r}{\alpha_{X,n}}\ar{d} & R\Gamma (X_\et, \ZZ(n)) \ar{r}{g}\ar{d} & R\Gamma_\fg (X, \ZZ(n)) \ar{r}\ar{d} & {[-1]}\ar{d} \\
    0\ar{r}\ar{d} & R\Gamma (X (\RR), \tau_{\ge n+1} R \widehat{\pi}_* \ZZ (n)) \ar{r}{id}\ar{d} & R\Gamma (X (\RR), \tau_{\ge n+1} R \widehat{\pi}_* \ZZ (n)) \ar{r}\ar{d} & 0\ar{d} \\
    {[R\Gamma (X, \ZZ^c (n)), \QQ [-1]]} \ar{r} & R\Gamma (\overline{X}_\et, \ZZ(n))[1]\ar{r}{f[1]} & R\Gamma_W (\overline{X}, \ZZ(n))[1] \ar{r} & {[0]}
  \end{tikzcd}
\end{equation}

% The complex $R\Gamma_W (\overline{X}, \ZZ(n))$ is perfect by
% \cite[Proposition~3.8]{Flach-Morin-2018}, unlike our complex
% $R\Gamma_\fg (X, \ZZ(n))$,
% which can have finite $2$-torsion in arbitrarily high degrees if
% $X (\RR) \ne \emptyset$. This is the price we pay for not working with
% the Artin--Verdier \'{e}tale topos.

Then \cite[Definition~3.23]{Flach-Morin-2018} considers a morphism
$\overline{u}^*_\infty$ defined via
\begin{equation}
  \label{eqn:definition-of-u-bar-infty}
  \begin{tikzcd}[column sep=1em]
    R\Gamma (\overline{X}_\et, \ZZ(n)) \ar{r}\ar[dashed]{d}[swap]{\exists}{\overline{u}_\infty^*} & R\Gamma (X_\et, \ZZ(n)) \ar{r}\ar{d}{u_\infty^*} & R\Gamma (X(\RR), \tau_{\ge n+1} R \widehat{\pi}_* \ZZ (n)) \ar{r}\ar{d}{id} & {[+1]} \ar[dashed]{d}{\overline{u}_\infty^* [1]} \\
    R\Gamma_W (X_\infty, \ZZ (n)) \ar{r} & R\Gamma (G_\RR, X (\CC), \ZZ (n)) \ar{r} & R\Gamma (X(\RR), \tau_{\ge n+1} R \widehat{\pi}_* \ZZ (n)) \ar{r} & {[+1]}
  \end{tikzcd}
\end{equation}
Here the complex $R\Gamma_W (X_\infty, \ZZ(n))$ is \emph{defined} via the bottom
triangle.

Then \cite[Proposition~3.24]{Flach-Morin-2018} and our
Proposition~\ref{prop:uniqueness-of-i-infty} above establish the existence and
uniqueness of morphisms $\overline{\iota}_\infty^*$ and $i_\infty^*$ which make
the triangles below commutative, and then the Weil-\'{e}tale complexes are
defined as mapping fibers of $\overline{\iota}_\infty^*$ and $i_\infty^*$:

\[ \begin{tikzcd}[column sep=1em]
    R\Gamma_\Wc (\overline{X}, \ZZ (n))\ar{d} & & R\Gamma_\Wc (X, \ZZ (n))\ar{d} \\
    R\Gamma_W (\overline{X}, \ZZ(n)) \ar[dashed]{d}{\overline{\iota}_\infty^*} & R\Gamma (\overline{X}_\et, \ZZ(n)) \ar{dl}{\overline{u}_\infty^*}\ar{l}[swap]{f} & R\Gamma_\fg (X, \ZZ(n))  \ar[dashed]{d}{\iota_\infty^*} & R\Gamma (X_\et, \ZZ(n)) \ar{dl}{u_\infty^*}\ar{l}[swap]{g} \\
    R\Gamma_W (X_\infty, \ZZ (n))\ar{d} & & R\Gamma (G_\RR, X (\CC), \ZZ (n))\ar{d} \\
    R\Gamma_\Wc (\overline{X}, \ZZ (n))[1] & & R\Gamma_\Wc (X, \ZZ (n))[1] \\
  \end{tikzcd} \]

In order to compare the two resulting complexes, we note that
$\overline{u}_\infty^*$ is only defined via
\eqref{eqn:definition-of-u-bar-infty}, so in the diagram below from
Figure~\ref{fig:comparison-with-FM}, we can first choose
$\overline{\iota}_\infty^*$ such that the front face gives a morphism of
triangles. Then we can \emph{declare} $\overline{u}_\infty^*$ to be the
composition $\overline{\iota}_\infty^* \circ f$. In this way everything
commutes, and we see that
$R\Gamma_\Wc (\overline{X}, \ZZ(n)) \cong R\Gamma_\Wc (X, \ZZ(n))$.

\vspace{1em}

This concludes the proof of Proposition~\ref{prop:comparison-with-FM}. \qed

\begin{landscape}
  \begin{figure}
    \[ \begin{tikzcd}[column sep=0.5em, row sep=3em,font=\small]
        R\Gamma_\Wc (\overline{X}, \ZZ(n)) \ar{dd}\ar[dashed]{rr}{\cong} &[-1.5em] &[-1.5em] R\Gamma_\Wc (X, \ZZ(n))\ar{rr} &[-1.5em] &[-1.5em] 0 \ar{rr} &[-2.5em] & {[+1]} \\
        & R\Gamma (\overline{X}_\et, \ZZ(n)) \ar{rr}\ar[near end]{dddl}{\overline{u}_\infty^*}\ar{dl}[swap]{f} & & R\Gamma (X_\et, \ZZ(n)) \ar[near end]{dddl}{u_\infty^*}\ar{dl}[swap]{g}\ar{rr} && R\Gamma (X (\RR), \tau_{\ge n+1} R\widehat{\pi}_* \ZZ (n))\ar{dl}[swap]{id}\ar[near end]{dddl}{id}\ar{rr} &&[1em] {[+1]}\ar{dddl}{\overline{u}_\infty^* [1]} \ar{dl}[swap]{f[1]} \\
        R\Gamma_W (\overline{X}, \ZZ(n)) \ar[dashed]{dd}[swap]{\overline{\iota}_\infty^*}\ar[crossing over]{rr} & & R\Gamma_\fg (X, \ZZ(n)) \ar{dd}[swap]{i_\infty^*}\ar[crossing over]{rr}\ar[<-,crossing over]{uu} && R\Gamma (X (\RR), \tau_{\ge n+1} R\widehat{\pi}_* \ZZ (n)) \ar{dd}[swap]{id}\ar[<-,crossing over]{uu} \ar[crossing over]{rr} && {[+1]}\ar[<-,crossing over]{uu}\ar[dashed]{dd}[swap]{\overline{\iota}_\infty^* [1]} \\
        \\
        R\Gamma_W (X_\infty, \ZZ (n)) \ar{rr}\ar{dd} & & R\Gamma (G_\RR, X (\CC), \ZZ (n)) \ar{rr}\ar{dd} && R\Gamma (X (\RR), \tau_{\ge n+1} R\widehat{\pi}_* \ZZ (n)) \ar{rr}\ar{dd} && {[+1]}\ar{dd} \\
        \\
        R\Gamma_\Wc (\overline{X}, \ZZ(n))[1] \ar[dashed]{rr}{\cong} && R\Gamma_\Wc (X, \ZZ(n))[1]\ar{rr} && 0 \ar{rr} && {[+2]}
      \end{tikzcd} \]

    \caption{Comparison of the Weil-\'{e}tale complexes from
      \cite{Flach-Morin-2018} and this paper, denoted
      $R\Gamma_\Wc (\overline{X}, \ZZ(n))$ and $R\Gamma_\Wc (X, \ZZ(n))$
      respectively. The top face of the prism comes from
      \eqref{eqn:cones-of-alphas}. The arrow $\overline{\iota}_\infty^*$ is
      chosen so that the front face is commutative. Then set
      $\overline{u}_\infty^* = \overline{\iota}_\infty^* \circ f$ so that the
      back face is commutative and corresponds to
      \eqref{eqn:definition-of-u-bar-infty}.}
    \label{fig:comparison-with-FM}
  \end{figure}
\end{landscape}

%%%%%%%%%%%%%%%%%%%%%%%%%%%%%%%%%%%%%%%%%%%%%%%%%%%%%%%%%%%%%%%%%%%%%%%%%%%%%%%%

\pagebreak
\appendix
\section{Some homological algebra}
\label{app:homological-algebra}

This appendix contains some basic results about the derived category of abelian
groups $\DZ$ which are used throughout the text. The following lemmas are
isolated from the proofs in \cite{Flach-Morin-2018}, with some modifications to
treat the $2$-torsion.

First, recall that every complex of abelian groups $A^\bullet$ (not necessarily
bounded) is quasi-isomorphic to its cohomology:
\begin{multline*}
  A^\bullet \cong \coprod_{i\in \ZZ} H^i (A^\bullet) [-i] \cong
  \prod_{i\in \ZZ} H^i (A^\bullet) [-i] \\
  = \Bigl(\cdots \to H^{i-1} (A^\bullet) \xrightarrow{0} H^i (A^\bullet)
  \xrightarrow{0} H^{i+1} (A^\bullet) \to \cdots\Bigr).
\end{multline*}
Here $\coprod$ and $\prod$ denote the coproduct and product in the category of
complexes, which coincide in this case. This gives us a useful expression for
morphisms in the derived category: since
$\Hom_\DZ (A,B [i]) \cong \Ext_\ZZ^i (A,B)$, and $\Ext_\ZZ^i (A,B) = 0$ for
$i > 1$, we obtain
\begin{align}
  \notag \Hom_\DZ (A^\bullet, B^\bullet) & \cong \Hom_\DZ (\coprod_{i\in\ZZ} H^i (A^\bullet) [-i], \prod_{j\in\ZZ} H^j (B^\bullet) [-j]) \\
  \notag & \cong \prod_{i\in \ZZ} \prod_{j\in \ZZ} \Hom_\DZ (H^i (A^\bullet), H^j (B^\bullet) [i-j]) \\
  \notag & \cong \prod_{i\in \ZZ} \left(\Hom (H^i (A^\bullet), H^i (B^\bullet)) \oplus \Ext (H^i (A^\bullet), H^{i-1} (B^\bullet))\right) \\
           \label{eqn:morphisms-in-D(Z)} & \cong \prod_{i\in \ZZ} \Hom (H^i (A^\bullet), H^i (B^\bullet)) \oplus \prod_{i\in \ZZ} \Ext (H^i (A^\bullet), H^{i-1} (B^\bullet)).
\end{align}

\begin{lemma}
  \label{lemma:morphisms-inDAb-not-divisible}
  ~

  \begin{enumerate}
  \item[$1)$] If $C^\bullet$ and $C'^\bullet$ are almost perfect in the sense of
    Definition~{\rm\ref{dfn:almost-of-(co)finite-type}}, then the group
    $\Hom_\DZ (C^\bullet, C'^\bullet)$ has no nontrivial
    divisible subgroups.

  \item[$2)$] If $A^\bullet$ is a complex such that $H^i (A^\bullet)$ are
    finite-dimensional $\QQ$-vector spaces and $C^\bullet$ is a complex such
    that $H^i (C^\bullet)$ are finitely generated abelian groups, then the group
    $\Hom_\DZ (A^\bullet, C^\bullet)$ is divisible.
  \end{enumerate}

  \begin{proof}
    In 1), we consider the decomposition \eqref{eqn:morphisms-in-D(Z)} for
    $\Hom_\DZ (C^\bullet, C'^\bullet)$, and observe that under
    our assumptions, both groups
    \[ \prod_{i\in\ZZ} \Hom (H^i (C^\bullet), H^i (C'^\bullet))
      \quad\text{and}\quad
      \prod_{i\in\ZZ} \Ext (H^i (C^\bullet), H^{i-1} (C'^\bullet))\]
    are of the form $G \oplus T$, where $G$ is a finitely generated abelian
    group and $T$ is $2$-torsion. From this we see that if
    $x \in \Hom_\DZ (C^\bullet, C'^\bullet)$ is divisible by all
    powers of $2$, then $x = 0$.

    Similarly, in part 2), we consider the decomposition
    \eqref{eqn:morphisms-in-D(Z)} for
    $\Hom_\DZ (A^\bullet, C^\bullet)$. Under our assumptions,
    $\Hom (H^i (A^\bullet), H^i (C^\bullet)) = 0$ for all $i$, and each
    $\Ext (H^i (A^\bullet), H^{i-1} (C^\bullet))$ is a direct sum of
    finitely many groups isomorphic to $\Ext (\QQ,\ZZ)$, which is
    divisible. Therefore, $\Hom_\DZ (A^\bullet, C^\bullet)$ is
    a direct product of divisible groups, hence divisible.
  \end{proof}
\end{lemma}

Recall that Verdier's axiom (TR1) states that every morphism
$v\colon A^\bullet \to B^\bullet$ can be completed to a distinguished triangle
$A^\bullet \xrightarrow{u} B^\bullet \xrightarrow{v} C^\bullet \xrightarrow{w}
A^\bullet [1]$. Axiom (TR3) states that for every commutative diagram with
distinguished rows
\begin{equation}
  \label{eqn:TR3-input}
  \begin{tikzcd}
    A^\bullet\ar{r}{u}\ar{d}{f} & B^\bullet\ar{r}{v}\ar{d}{g} & C^\bullet\ar{r}{w} & A^\bullet [1] \\
    A'^\bullet\ar{r}{u'} & B'^\bullet\ar{r}{v'} & C'^\bullet\ar{r}{w'} & A'^\bullet [1]
  \end{tikzcd}
\end{equation}
there exists some $h\colon C^\bullet \to C'^\bullet$, which gives a morphism of
distinguished triangles
\begin{equation}
  \label{eqn:TR3-output}
  \begin{tikzcd}
    A^\bullet\ar{r}{u}\ar{d}{f} & B^\bullet\ar{r}{v}\ar{d}{g} & C^\bullet\ar{r}{w}\ar[dashed]{d}{\exists h} & A^\bullet [1]\ar{d}{f [1]} \\
    A'^\bullet\ar{r}{u'} & B'^\bullet\ar{r}{v'} & C'^\bullet\ar{r}{w'} & A'^\bullet [1]
  \end{tikzcd}
\end{equation}

The cone $C^\bullet$ in (TR1) and the morphism $h$ in (TR3) are neither unique
nor canonical. Two different cones of the same morphism are necessarily
isomorphic, but the isomorphism between them is not unique, because it is
provided by (TR3). Let us recall a useful argument showing that things are
well-defined in some special cases.

\begin{lemma}[{$\approx$\cite[Proposition~1.1.9, Corollaire~1.1.10]{Beilinson-Bernstein-Deligne}}]
  \label{lemma:TR3-TR1-with-uniqueness-general-statement}

  Consider the derived category $\mathbf{D} (\mathcal{A})$ of an abelian
  category $\mathcal{A}$.

  \begin{enumerate}
  \item[$1)$] For a commutative diagram \eqref{eqn:TR3-input}, assume that the
    homomorphism of abelian groups
    \[ w^*\colon \Hom_{\mathbf{D} (\mathcal{A})} (A^\bullet [1], C'^\bullet) \to
      \Hom_{\mathbf{D} (\mathcal{A})} (C^\bullet, C'^\bullet) \]
    induced by $w$ is trivial. Then there exists a unique morphism
    $h\colon C^\bullet \to C'^\bullet$ that gives a morphism of triangles
    \eqref{eqn:TR3-output}.

  \item[$2)$] For a distinguished triangle
    $A^\bullet \xrightarrow{u} B^\bullet \xrightarrow{v} C^\bullet \xrightarrow{w} A^\bullet[1]$,
    assume that for any other cone $C'^\bullet$ of $u$ the morphism $w^*$ is
    trivial. Then the cone of $u$ is unique up to a unique isomorphism.
  \end{enumerate}

  \begin{proof}
    In 1), applying $\Hom_{\mathbf{D} (\mathcal{A})} (-, C'^\bullet)$ to the
    first distinguished triangle, we obtain an exact sequence of abelian groups
    \[ \Hom_{\mathbf{D} (\mathcal{A})} (A^\bullet [1], C'^\bullet) \xrightarrow{w^*}
      \Hom_{\mathbf{D} (\mathcal{A})} (C^\bullet, C'^\bullet) \xrightarrow{v^*}
      \Hom_{\mathbf{D} (\mathcal{A})} (B^\bullet, C'^\bullet). \]
    If $w^* = 0$, we conclude that $v^*$ is a monomorphism. This implies that
    there is a unique morphism $h$ such that $h\circ v = v'\circ g$. Now in 2),
    if $C^\bullet$ and $C'^\bullet$ are two different cones of $u$, we have a
    commutative diagram
    \[ \begin{tikzcd}
        A^\bullet\ar{r}{u}\ar{d}{id} & B^\bullet\ar{r}{v}\ar{d}{id} & C^\bullet\ar{r}{w}\ar[dashed]{d} & A^\bullet [1]\ar{d}{id} \\
        A^\bullet\ar{r}{u'} & B^\bullet\ar{r}{v'} & C'^\bullet\ar{r}{w'} & A^\bullet [1]
      \end{tikzcd} \]
    By the triangulated five-lemma, the dashed arrow is an isomorphism, and it
    is unique thanks to part 1).
  \end{proof}
\end{lemma}

Here is a special case that we need.

\begin{corollary}
  \label{cor:TR3-TR1-with-uniqueness}
  Consider the derived category $\DZ$.

  \begin{enumerate}
  \item[$1)$] Suppose we have a commutative diagram with distinguished rows
    \eqref{eqn:TR3-input}, where $A^\bullet$ is a complex such that
    $H^i (A^\bullet)$ are finite-dimensional $\QQ$-vector spaces and
    $C^\bullet$, $C'^\bullet$ are almost perfect complexes in the sense of
    Definition~{\rm\ref{dfn:almost-of-(co)finite-type}}. Then there exists a
    unique morphism ${h\colon C^\bullet \to C'^\bullet}$ which gives a morphism
    of triangles \eqref{eqn:TR3-output}.

  \item[$2)$] For a distinguished triangle
    $$A^\bullet \xrightarrow{u} B^\bullet \xrightarrow{v} C^\bullet \xrightarrow{w} A^\bullet[1]$$
    assume that $A^\bullet$ is a complex such that $H^i (A^\bullet)$ are
    finite-dimensional $\QQ$-vector spaces and $C^\bullet$ is an almost perfect
    complex. Then the cone of $u$ is unique up to a unique isomorphism.
  \end{enumerate}

  \begin{proof}
    In this situation, by Lemma~\ref{lemma:morphisms-inDAb-not-divisible}, the
    group $\Hom_\DZ (C^\bullet, C'^\bullet)$ has no nontrivial divisible
    subgroups, and $\Hom_\DZ (A^\bullet [1], C'^\bullet)$ is divisible. This
    means that there are no nontrivial homomorphisms
    $\Hom_\DZ (A^\bullet [1], C'^\bullet) \to \Hom_\DZ (C^\bullet, C'^\bullet)$,
    and we can apply
    Lemma~\ref{lemma:TR3-TR1-with-uniqueness-general-statement}.
  \end{proof}
\end{corollary}

\begin{lemma}
  \label{lemma:torsion-morphisms-in-D(Z)}
  Suppose that $A^\bullet$ and $B^\bullet$ are almost of cofinite type in the
  sense of Definition~{\rm\ref{dfn:almost-of-(co)finite-type}}. Then a morphism
  $f\colon A^\bullet\to B^\bullet$ is torsion (i.e. a torsion element in the
  group $\Hom_\DZ (A^\bullet, B^\bullet)$, i.e.  $f\otimes \mathbb{Q} = 0$) if
  and only if the morphisms $H^i (f)\colon H^i (A^\bullet) \to H^i (B^\bullet)$
  are torsion; that is, they are trivial on the maximal divisible subgroups:
  $$(H^i (f)_\div\colon H^i (A^\bullet)_\div \to H^i (B^\bullet)_\div) = 0.$$

  \begin{proof}
    Under our assumptions, the groups $H^i (A^\bullet)$ and
    $H^{i-1} (B^\bullet)$ appearing in \eqref{eqn:morphisms-in-D(Z)} are of
    cofinite type. We calculate that in this case,
    $\Ext (H^i (A^\bullet), H^{i-1} (B^\bullet))$ is finite.

    For $i \gg 0$, the groups $H^i (A^\bullet)$ and $H^{i-1} (B^\bullet)$ are
    finite $2$-torsion, and therefore
    $\Ext (H^i (A^\bullet), H^{i-1} (B^\bullet))$ is finite $2$-torsion as
    well. It follows that the whole product
    $\prod_{i\in \mathbb{Z}} \Ext (H^i (A^\bullet), H^{i-1} (B^\bullet))$ is of
    the form $G \oplus T$, where $G$ is finite and $T$ is (possibly infinite)
    $2$-torsion. We have therefore $(G \oplus T)\otimes \QQ = 0$.

    Similarly, the group
    $\prod_{i\in\ZZ} \Hom (H^i (A^\bullet), H^i (B^\bullet))$ consists of some
    part of the form $\widehat{\ZZ}^{\oplus r} \oplus G$, where $G$ is finite,
    and some possibly infinite $2$-torsion part, which is killed by tensoring
    with $\QQ$. It follows from \eqref{eqn:morphisms-in-D(Z)} that there is an
    isomorphism
    \begin{align*}
      \Hom_\DZ (A^\bullet, B^\bullet)\otimes \mathbb{Q} & \cong
      \prod_{i\in \mathbb{Z}} \Hom (H^i (A^\bullet), H^i (B^\bullet)) \otimes \mathbb{Q}, \\
      f\otimes\QQ & \mapsto (H^i (f) \otimes \QQ)_{i\in\ZZ}. \qedhere
    \end{align*}
  \end{proof}
\end{lemma}

\begin{lemma}
  \label{lemma:morphisms-in-DAb-between-cplx-of-Q-vs-and-almost-cofinite-type-cplx}
  If $A^\bullet$ is a complex of $\QQ$-vector spaces and $B^\bullet$ is a
  complex almost of cofinite type in the sense of
  Definition~{\rm\ref{dfn:almost-of-(co)finite-type}}, then there is an
  isomorphism of abelian groups
  \begin{align*}
    \Hom_\DZ (A^\bullet, B^\bullet) & \xrightarrow{\cong}
    \prod_{i\in \ZZ} \Hom (H^i (A^\bullet), H^i (B^\bullet)),\\
    f & \mapsto (H^i (f))_{i\in \ZZ}.
  \end{align*}

  \begin{proof}
    In the formula \eqref{eqn:morphisms-in-D(Z)}, if $H^i (A^\bullet)$ are
    $\QQ$-vector spaces and $H^{i-1} (B^\bullet)$ are groups of cofinite type,
    then the term $\Ext (H^i (A^\bullet), H^{i-1} (B^\bullet))$ vanishes.
  \end{proof}
\end{lemma}

%%%%%%%%%%%%%%%%%%%%%%%%%%%%%%%%%%%%%%%%%%%%%%%%%%%%%%%%%%%%%%%%%%%%%%%%%%%%%%%%

\section{Cohomology with compact support}
\label{app:modified-cohomology-with-compact-support}

For any arithmetic scheme $f\colon X\to \Spec \ZZ$ there exists a
\textbf{Nagata compactification}
\cite{Conrad-Deligne-Nagata,Conrad-Deligne-Nagata-erratum}
(see also \cite[Expos\'{e}~XVII]{SGA4})
\[ \begin{tikzcd}
X \ar[hookrightarrow]{rr}{j}\ar{dr}[swap]{f} & & \mathfrak{X} \ar{dl}{g} \\
 & \Spec \ZZ
\end{tikzcd} \]
where $j$ is an open immersion and $g$ is a proper morphism.

\begin{definition}
  Let $X$ be an arithmetic scheme and let $\mathcal{F}$ be an abelian torsion
  sheaf on $X_\et$. Then one defines the
  \textbf{cohomology with compact support} of $\mathcal{F}$ via the complex
  \begin{equation}
    \label{eqn:RGammac-via-extension-by-zero}
    R\Gamma_c (X_\et, \mathcal{F}) \dfn
    R\Gamma (\mathfrak{X}_\text{\it \'et}, j_! \mathcal{F}).
  \end{equation}
\end{definition}

For torsion sheaves, this does not depend on the choice of
$j\colon X \hookrightarrow \mathfrak{X}$, but here we would like to fix this
choice in order to compare cohomology with compact support on $X_\et$ with
the singular cohomology with compact support on $X (\CC)$.

\subsection*{Comparison with the analytic cohomology}

\begin{definition}
  Given a Nagata compactification $j\colon X\hookrightarrow \mathfrak{X}$,
  we consider the corresponding open immersion
  $j (\CC)\colon X (\CC) \to \mathfrak{X} (\CC)$,
  and for a sheaf $\mathcal{F}$ on $X (\CC)$ we define
  \[ \Gamma_c (X (\CC), \mathcal{F}) \dfn
  \Gamma (\mathfrak{X} (\CC), j (\CC)_! \mathcal{F}). \]
  Similarly, for a $G_\RR$-equivariant sheaf on $X (\CC)$ we define
  \[ \Gamma_c (G_\RR, X (\CC), \mathcal{F}) \dfn
  \Gamma (G_\RR, \mathfrak{X} (\CC), j (\CC)_! \mathcal{F}). \]
\end{definition}

The canonical reference for the comparison between \'{e}tale and singular
cohomology is \cite[Expos\'{e}~XI, \S 4]{SGA4}, so we borrow some definitions
and notations from there. Let $X$ be an arithmetic scheme.

\begin{enumerate}
\item The base change from $\Spec \ZZ$ to $\Spec \CC$ gives us a morphism of
  sites
  \begin{equation}
    \label{eqn:comparison-functor-gamma}
    \gamma\colon X_{\CC,\text{\it \'{e}t}} \to X_\et.
  \end{equation}

\item Let $X_\text{\it cl}$ be the site of \'{e}tale maps
  $f\colon U\to X (\CC)$. A covering family in $X_\text{\it cl}$ is a family of
  maps $\{ U_i \to U \}$ such that $U$ is the union of images of $U_i$.

  (We recall that in the analytic topology, $f\colon U\to X (\CC)$ is \textbf{\'{e}tale}
  if it is a \emph{local on the source homeomorphism}: for each $u \in U$ there
  exists an open neighborhood $u \ni V$ such that
  $\left.f\right|_V\colon V\to f(V)$ is a homeomorphism.)

  Since the inclusion of an open subset $U \subset X (\CC)$ is an \'{e}tale map,
  we have a fully faithful functor $X (\CC) \subset X_\text{\it cl}$, and the
  topology on $X (\CC)$ is induced by the topology on $X_\text{\it cl}$. This
  gives us a morphism of sites $\delta\colon X_\text{\it cl} \to X (\CC)$, which
  by the comparison lemma \cite[Expos\'{e}~III, Th\'{e}or\`{e}me~4.1]{SGA4}
  induces an equivalence of the corresponding categories of sheaves
  \begin{equation}
    \label{eqn:comparison-functor-delta}
    \delta_*\colon \mathbf{Sh} (X_\text{\it cl}) \to \mathbf{Sh} (X (\CC)).
  \end{equation}

\item A morphism of schemes $f\colon X'_\CC \to X_\CC$ over $\Spec \CC$ is
  \'{e}tale if and only if the map $f (\CC)\colon X' (\CC) \to X (\CC)$ is
  \'{e}tale \cite[Expos\'{e}~XII, Proposition~3.1]{SGA1}, and therefore the
  functor $X'_\CC \rightsquigarrow X' (\CC)$ gives us a morphism of sites
  \begin{equation}
    \label{eqn:comparison-functor-epsilon}
    \epsilon\colon X_\text{\it cl} \to X_{\CC,\text{\it \'{e}t}}.
  \end{equation}
\end{enumerate}

\begin{definition}
  We define the functor
  $$\alpha^*\colon \mathbf{Sh} (X_\et) \to \mathbf{Sh} (G_\RR, X (\CC))$$
  via the composition of \eqref{eqn:comparison-functor-gamma},
  \eqref{eqn:comparison-functor-epsilon}, \eqref{eqn:comparison-functor-delta}:
  \[ \begin{tikzcd}
      \mathbf{Sh} (X_\et) \ar{r}{\gamma^*} &
      \mathbf{Sh} (X_{\CC,\text{\it \'{e}t}}) \ar{r}{\epsilon^*} &
      \mathbf{Sh} (X_\text{\it cl}) \ar{r}{\delta_*}[swap]{\simeq} &
      \mathbf{Sh} (X (\CC))
    \end{tikzcd} \]
\end{definition}

As we start from a scheme over $\Spec \ZZ$ and base change to $\Spec \CC$, the
resulting sheaf on $X (\CC)$ is equivariant with respect to the complex
conjugation, hence an object in $\mathbf{Sh} (G_\RR, X (\CC))$. For the
definition of equivariant sheaves, we refer to the introduction.

\begin{lemma}
  \label{lemma:alpha-preserves-colimits}
  $\alpha^*$ preserves colimits.

  \begin{proof}
    $\alpha^*$ is the composition of the inverse image functors $\gamma^*$ and
    $\epsilon^*$ (which are left adjoint) and an equivalence $\delta_*$.
  \end{proof}
\end{lemma}

\begin{proposition}
  \label{prop:inverse-image-gamma}
  Given a sheaf $\mathcal{F}$ on $X_\et$, there exists a natural morphism
  $$\Gamma (X_\et, \mathcal{F}) \to \Gamma (G_\RR, X (\CC), \alpha^* \mathcal{F}),$$
  and similarly, for cohomology with compact support,
  $$\Gamma_c (X_\et, \mathcal{F}) \to \Gamma_c (G_\RR, X (\CC), \alpha^* \mathcal{F}).$$

  \begin{proof}
    If $j\colon X \hookrightarrow \mathfrak{X}$ is a Nagata compactification, we
    have the corresponding compactification
    $j (\CC)\colon X (\CC) \hookrightarrow \mathfrak{X} (\CC)$. The extension by
    zero morphism
    $j (\CC)_!\colon \mathbf{Sh} (X (\CC)) \to \mathbf{Sh} (\mathfrak{X} (\CC))$
    restricts to the subcategory of $G_\RR$-equivariant sheaves: if
    $\mathcal{F}$ is a $G_\RR$-equivariant sheaf on $X (\CC)$, then
    $j (\CC)_!  \mathcal{F}$ is a $G_\RR$-equivariant sheaf on
    $\mathfrak{X} (\CC)$. From the definition of $\alpha^*$, we see that that
    there is a commutative diagram
    \[ \begin{tikzcd}
        \mathbf{Sh} (X_\et) \ar{r}{\alpha^*}\ar{d}[swap]{j_!} & \mathbf{Sh} (G_\RR, X (\CC)) \ar{d}{j (\CC)_!} \\
        \mathbf{Sh} (\mathfrak{X}_\et) \ar{r}[swap]{\alpha^*_\mathfrak{X}} & \mathbf{Sh} (G_\RR, \mathfrak{X} (\CC))
      \end{tikzcd} \]
    ---this diagram commutes for representable \'{e}tale sheaves, and then every
    \'{e}tale sheaf is a colimit of representable sheaves, and $\alpha^*$,
    $j_!$, $\alpha^*_\mathfrak{X}$, $j (\CC)_!$ preserve colimits, as left
    adjoints.

    The morphism in question is given by
    \begin{multline*}
      \Gamma_c (X_\et, \mathcal{F}) \dfn \Gamma (\mathfrak{X}_\et, j_! \mathcal{F}) \to
      \Gamma (G_\RR, \mathfrak{X} (\CC), \alpha^*_\mathfrak{X} j_! \mathcal{F}) \\
      =
      \Gamma (G_\RR, \mathfrak{X} (\CC), j (\CC)_! \, \alpha^* \mathcal{F})
      \rdfn \Gamma_c (G_\RR, X (\CC), \alpha^* \mathcal{F}). \qedhere
    \end{multline*}
  \end{proof}
\end{proposition}

The morphism $\alpha$ is also discussed in \cite[Appendix~A]{Flach-Morin-2018},
but Flach and Morin work with proper schemes; the above remarks are to make sure
that everything works fine for compactifications.

\subsection*{Modified \'{e}tale cohomology}

Here we briefly review the
\textbf{modified \'{e}tale cohomology with compact support}
$R\widehat{\Gamma}_c (X_\et, -)$. It was introduced by Th.~Zink in
\cite[Appendix~2]{Haberland-1978} for the case of number rings
$X = \Spec \mathcal{O}_{K,S}$, and it is also discussed in
\cite[\S II.2]{Milne-ADT}. The general definition for $X \to \Spec\ZZ$
is treated in \cite[\S 6.7]{Flach-Morin-2018} and
\cite[\S 2]{Geisser-Schmidt-2018}.

Note that thanks to the Leray spectral sequence
$R\Gamma (\mathfrak{X}_\et, -) \cong R\Gamma (\Spec \ZZ_\et, -)\circ R g_*$,
we have
\begin{equation}
  \label{eqn:RGammac-via-derived-lower-shriek}
  R\Gamma_c (X_\et, \mathcal{F}) \dfn R\Gamma (\mathfrak{X}_\et, j_!\mathcal{F})
  \cong R\Gamma ((\Spec \ZZ)_\et, R f_! \mathcal{F}), \quad
  \text{where } Rf_! \mathcal{F} \dfn R g_* j_! \mathcal{F}.
\end{equation}

First we recall that for a finite group $G$ and a $G$-module $A$ the
corresponding group cohomology $H^i (G,A)$ (resp. Tate cohomology
$\widehat{H}^i (G,A)$) can be defined in terms of resolutions $P_\bullet$
(resp. complete resolutions $\widehat{P}_\bullet$) of $\ZZ$ by free
$\ZZ G$-modules (see e.g. \cite[Chapter~VI]{Brown-1994}). More
generally, if $A^\bullet$ is a bounded (cohomological) complex of
$G$-modules, we obtain a \emph{double complex} of abelian groups
$\Hom^{\bullet\bullet} (P_\bullet, A^\bullet)$ (resp.  $\Hom^{\bullet\bullet}
(\widehat{P}_\bullet, A^\bullet)$), and it makes sense to define the
corresponding \textbf{group hypercohomology}
(resp. \textbf{Tate hypercohomology}) via the complexes
\[ R\Gamma (G, A^\bullet) \dfn
\Tot^\oplus (\Hom^{\bullet\bullet} (P_\bullet, A^\bullet)), \quad
R\widehat{\Gamma} (G, A^\bullet) \dfn
\Tot^\oplus (\Hom^{\bullet\bullet} (\widehat{P}_\bullet, A^\bullet)). \]

Now if $\mathcal{F}$ is an abelian sheaf on $(\Spec \ZZ)_\et$, then the
corresponding \textbf{modified cohomology with compact support} is characterized
by the distinguished triangle
\[ R\widehat{\Gamma}_c ((\Spec \ZZ)_\et, \mathcal{F}) \to
R\Gamma ((\Spec \ZZ)_\et, \mathcal{F}) \to
R\widehat{\Gamma} (G_\RR, v^* \mathcal{F}) \to
R\widehat{\Gamma}_c ((\Spec \ZZ)_\et, \mathcal{F}) [1] \]
Here $v\colon \Spec \RR \to \Spec \ZZ$ is the canonical morphism, and
$v^* \mathcal{F}$ is the corresponding sheaf on $(\Spec \RR)_\et$, which can be
viewed as a $G_\RR$-module by \cite[Expos\'{e}~VII, 2.3]{SGA4}, and
$R\widehat{\Gamma} (G_\RR, v^* \mathcal{F})$ denotes the corresponding Tate
cohomology.

In general, given an arithmetic scheme $X \to \Spec \ZZ$ and a torsion abelian
sheaf $\mathcal{F}$ on $X_\et$, we choose a Nagata compactification as above and
set
\[ R\widehat{\Gamma}_c (X_\et, \mathcal{F}) \dfn
  R\widehat{\Gamma}_c ((\Spec \ZZ)_\et, R f_! \mathcal{F}). \]
We have a natural morphism
$$R\widehat{\Gamma}_c (X_\et, \mathcal{F}) \to R\Gamma_c (X_\et, \mathcal{F}),$$
which is an isomorphism if $X (\RR) = \emptyset$. In general, Tate cohomology
$\widehat{H}^i (G_\RR, -)$ is annihilated by multiplication by $2 = \# G_\RR$,
and therefore
$\widehat{H}^i_c (X_\et,\mathcal{F}) \to H^i_c (X_\et,\mathcal{F})$ has
$2$-torsion kernel and cokernel.

For canonicity and functoriality, I refer to \cite[\S 2]{Geisser-Schmidt-2018}.

%%%%%%%%%%%% References %%%%%%%%%%%%%

\begin{thebibliography}{99}

\providecommand{\bysame}{\leavevmode\hbox to3em{\hrulefill}\thinspace}

\bibitem{SGA4}
  \textsc{Michael Artin, Alexander Grothendieck, and Jean-Louis Verdier} (eds.),
  S{\'e}minaire de g{\'e}om{\'e}trie alg{\'e}brique du {B}ois-{M}arie
  1963--1964 ({SGA}~4): {T}h\'eorie des topos et cohomologie \'etale des
  sch\'emas, Lecture Notes in Mathematics, Vol. 269, 270, 305,
  Springer-Verlag, Berlin-New York, 1972--73, Avec la collaboration de
  {N}.~{B}ourbaki, {P}.~{D}eligne et {B}.~{S}aint-{D}onat.

\bibitem{Artin-Verdier-1964}
  \textsc{Michael Artin and Jean-Louis Verdier}, Seminar on \'{e}tale cohomology
  of number fields, Lecture notes prepared in connection with the seminars held
  at the summer institute on algebraic geometry. Whitney estate, {W}oods {H}ole,
  {M}assachusetts. July 6 -- {J}uly 31, 1964, Amer. Math. Soc., Providence, RI,
  1964.

\bibitem{Beilinson-Bernstein-Deligne}
  \textsc{A.~A. Beilinson, J.~Bernstein, and P.~Deligne}, Faisceaux pervers,
  Analysis and topology on singular spaces, {I} ({L}uminy, 1981),
  Ast\'{e}risque, vol. 100, Soc. Math. France, Paris, 1982, pp.~5--171.

\bibitem{Bloch-1986}
  \textsc{Spencer Bloch}, Algebraic cycles and higher {$K$}-theory, Adv. in
  Math. \textbf{61} (1986), no.~3, 267--304.

\bibitem{Brown-1994}
  \textsc{Kenneth~S. Brown}, Cohomology of groups, Graduate Texts in
  Mathematics, vol.~87, Springer-Verlag, New York, 1994, Corrected reprint of
  the 1982 original.

\bibitem{Conrad-Deligne-Nagata}
  \textsc{Brian Conrad}, Deligne's notes on {N}agata compactifications,
  J. Ramanujan Math. Soc. \textbf{22} (2007), no.~3, 205--257.

\bibitem{Conrad-Deligne-Nagata-erratum}
  \bysame, Erratum for ``{D}eligne's notes on {N}agata compactifications'',
  J. Ramanujan Math. Soc. \textbf{24} (2009), no.~4, 427--428.

\bibitem{SGA4-1-2}
  \textsc{Pierre Deligne}, Cohomologie \'etale, Lecture Notes in Mathematics, Vol.
  569, Springer-Verlag, Berlin-New York, 1977, S{\'e}minaire de
  G{\'e}om{\'e}trie Alg{\'e}brique du {B}ois-{M}arie {SGA}~4$\frac{1}{2}$, Avec
  la collaboration de {J}.~{F}.~{B}outot, {A}.~{G}rothendieck, {L}.~Illusie et
  {J}.~{L}.~{V}erdier.

\bibitem{Flach-Morin-2018}
  \textsc{Matthias Flach and Baptiste Morin}, Weil-\'{e}tale cohomology and
  zeta-values of proper regular arithmetic schemes, Doc. Math. \textbf{23}
  (2018), 1425--1560.

\bibitem{Geisser-2004-Dedekind}
  \textsc{Thomas Geisser}, Motivic cohomology over {D}edekind rings, Math. Z.
  \textbf{248} (2004), no.~4, 773--794.

\bibitem{Geisser-2004}
  \bysame, Weil-\'{e}tale cohomology over finite fields, Math. Ann.
  \textbf{330} (2004), no.~4, 665--692.

\bibitem{Geisser-2005}
  \bysame, Motivic cohomology, {$K$}-theory and topological cyclic
  homology, Handbook of {$K$}-theory. {V}ol. 1, 2, Springer, Berlin, 2005,
  pp.~193--234.

\bibitem{Geisser-2006}
  \bysame, Arithmetic cohomology over finite fields and special values of
  {$\zeta$}-functions, Duke Math. J. \textbf{133} (2006), no.~1, 27--57.

\bibitem{Geisser-2010-arithmetic-homology}
  \bysame, Arithmetic homology and an integral version of {K}ato's
  conjecture, J. Reine Angew. Math. \textbf{644} (2010), 1--22.

\bibitem{Geisser-2010}
  \bysame, Duality via cycle complexes, Ann. of Math. (2) \textbf{172}
  (2010), no.~2, 1095--1126.

\bibitem{Geisser-2017}
  \bysame, On the structure of \'{e}tale motivic cohomology, J. Pure Appl.
  Algebra \textbf{221} (2017), no.~7, 1614--1628.

\bibitem{Geisser-Schmidt-2018}
  \textsc{Thomas Geisser and Alexander Schmidt}, Poitou-{T}ate duality for
  arithmetic schemes, Compos. Math. \textbf{154} (2018), no.~9, 2020--2044.

\bibitem{SGA5}
  \textsc{Alexander Grothendieck} (ed.), S{\'e}minaire de g{\'e}om{\'e}trie
  alg{\'e}brique du {B}ois-{M}arie 1965--66 ({SGA}~5): {C}ohomologie
  $\ell$-adique et fonctions {$L$}, Lecture Notes in Mathematics, Vol. 589,
  Springer-Verlag, Berlin-New York, 1977, Avec la collaboration de I.~Bucur,
  C.~Houzel, L.~Illusie, J.-P.~Jouanolou et J.-P.~Serre.

\bibitem{SGA7}
  \textsc{Alexander Grothendieck, Pierre Deligne, and Nicholas Katz} (eds.),
  S{\'e}minaire de g{\'e}om{\'e}trie alg{\'e}brique du {B}ois-{M}arie
  1967--69 ({SGA}~7): Groupes de monodromie en g\'eom\'etrie alg\'ebrique,
  Lecture Notes in Mathematics, Vol. 288, 340, Springer-Verlag, Berlin-New
  York, 1972--73, Avec la collaboration de {M}.~{R}aynaud et {D}.{S}.~{R}im.

\bibitem{SGA1}
  \textsc{Alexander Grothendieck and Mich\`ele Raynaud} (eds.), S{\'e}minaire de
  g{\'e}om{\'e}trie alg{\'e}brique du {B}ois-{M}arie 1960--61 ({SGA}~4):
  Rev\^etements \'etales et groupe fondamental, Lecture Notes in Mathematics,
  Vol. 224, Springer-Verlag, Berlin-New York, 1971.

\bibitem{Haberland-1978}
  \textsc{Klaus Haberland}, Galois cohomology of algebraic number fields, VEB
  Deutscher Verlag der Wissenschaften, Berlin, 1978, With two appendices by
  Helmut Koch and Thomas Zink.

\bibitem{Hironaka-1974}
  \textsc{Heisuke Hironaka}, Triangulations of algebraic sets. Algebraic
  geometry (Proc. Sympos. Pure Math., Vol. 29, Humboldt State Univ., Arcata,
  Calif., 1974), pp. 165--185. Amer. Math. Soc., Providence, R.I., 1975.

\bibitem{Kahn-2003}
  \textsc{Bruno Kahn}, Some finiteness results for \'{e}tale cohomology,
  J. Number Theory \textbf{99} (2003), no.~1, 57--73.

\bibitem{Kahn-2005}
  \bysame, Algebraic {$K$}-theory, algebraic cycles and arithmetic
  geometry, Handbook of {$K$}-theory. {V}ol. 1, 2, Springer, Berlin, 2005,
  pp.~351--428.

\bibitem{Knudsen-Mumford-1976}
  \textsc{Finn~Faye Knudsen and David Mumford}, The projectivity of the moduli space
  of stable curves. {I}. {P}reliminaries on ``det'' and ``{D}iv'', Math.
  Scand. \textbf{39} (1976), no.~1, 19--55.

\bibitem{Lichtenbaum-2005}
  \textsc{Stephen Lichtenbaum}, The {W}eil-\'{e}tale topology on schemes over finite
  fields, Compos. Math. \textbf{141} (2005), no.~3, 689--702.

\bibitem{Lichtenbaum-2009-Euler-char}
  \bysame, Euler characteristics and special values of zeta-functions,
  Motives and algebraic cycles, Fields Inst. Commun., vol.~56, Amer. Math.
  Soc., Providence, RI, 2009, pp.~249--255.

\bibitem{Lichtenbaum-2009-number-rings}
  \bysame, The {W}eil-\'{e}tale topology for number rings, Ann. of Math.
  (2) \textbf{170} (2009), no.~2, 657--683.

\bibitem{Lojasiewicz-1964}
  \textsc{Stanis\l{}aw \L{}ojasiewicz}, Triangulation of semi-analytic sets. Annali della Scuola Normale Superiore di Pisa - Classe di Scienze, S\'{e}rie 3, Tome 18 (1964) no. 4, pp. 449--474.

\bibitem{MacLane-Moerdijk}
  \textsc{Saunders Mac Lane and Ieke Moerdijk},
  Sheaves in Geometry and Logic: A~first introduction to topos theory,
  Universitext, Springer-Verlag, New York, 1994,
  Corrected reprint of the 1992 edition.

\bibitem{Mazur-1973}
\textsc{Barry Mazur}, Notes on \'{e}tale cohomology of number fields, Ann. Sci.
  \'{E}cole Norm. Sup. (4) \textbf{6} (1973), 521--552 (1974).

\bibitem{Milne-ADT}
  \textsc{J.~S. Milne}, Arithmetic duality theorems, second ed., BookSurge, LLC,
  Charleston, SC, 2006.

\bibitem{Morin-these}
  \textsc{Baptiste Morin}, Sur le topos weil-\'{e}tale d'un corps de nombres, 2008,
  {PhD} thesis, {U}niversit\'{e} {B}ordeaux {I}.

\bibitem{Morin-2014}
  \bysame, Zeta functions of regular arithmetic schemes at {$s=0$}, Duke
  Math. J. \textbf{163} (2014), no.~7, 1263--1336.

\bibitem{Neeman-2001}
  \textsc{Amnon Neeman}, Triangulated categories, Annals of Mathematics Studies,
  vol. 148, Princeton University Press, Princeton, NJ, 2001.

\bibitem{Serpe-2003}
  \textsc{Christian Serp\'{e}}, Resolution of unbounded complexes in
  Grothendieck categories, J. Pure Appl. Algebra 177 (2003), no. 1, 103--112.

\bibitem{Soule-1984}
  \textsc{Christophe Soul\'{e}}, Groupes de {C}how et {$K$}-th\'{e}orie de
  vari\'{e}t\'{e}s sur un corps fini, Math. Ann. \textbf{268} (1984), no.~3,
  317--345.

\bibitem{Spaltenstein-1988}
  \textsc{Nicolas Spaltenstein}, Resolutions of unbounded complexes,
  Compositio Math. 65 (1988), no. 2, 121--154.

\bibitem{Verdier-thesis}
  \textsc{Jean-Louis Verdier}, Des cat\'{e}gories d\'{e}riv\'{e}es des
  cat\'{e}gories ab\'{e}liennes, Ast\'{e}risque (1996), no.~239, xii+253 pp.
  (1997), With a preface by Luc Illusie, Edited and with a note by Georges
  Maltsiniotis.

\bibitem{van-der-Waerden-30}
  \textsc{Bartel~L. van~der Waerden}, Topologische {B}egr\"undung des
  {K}alk\"uls der abz\"ahlenden {G}eometrie, Math. Ann. \textbf{102} (1930),
  no.~1, 337--362.

\bibitem{Weibel-1994}
  \textsc{Charles~A. Weibel}, An introduction to homological algebra, Cambridge
  Studies in Advanced Mathematics, vol.~38, Cambridge University Press,
  Cambridge, 1994.

\end{thebibliography}
\bigskip
%%%%%%%%%%%% Authors' addresses %%%%%%%%%%%%%
\address{
  Centro de Investigaci\'{o}n en Matem\'{a}ticas\\
  Callej\'{o}n de Jalisco, Col. Valenciana\\
  36023 Guanajuato\\
  M\'{e}xico}
{alexey.beshenov@cimat.mx}

\end{document}