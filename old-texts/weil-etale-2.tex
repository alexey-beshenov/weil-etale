\documentclass{article}

\usepackage[utf8]{inputenc}
\usepackage{fullpage}

\usepackage[titletoc]{appendix}
\usepackage[russian,english]{babel}

\usepackage{amsmath,amssymb}

\usepackage{pdflscape}

%%%%%%%%%%%%%%%%%%%%%%%%%%%%%%%%%%%%%%%%%%%%%%%%%%%%%%%%%%%%%%%%%%%%%%%%%%%%%%%%

\DeclareMathOperator{\Cone}{Cone}
\DeclareMathOperator{\coker}{coker}
\DeclareMathOperator{\Ext}{Ext}
\DeclareMathOperator{\fchar}{char}
\DeclareMathOperator{\Fil}{Fil}
\DeclareMathOperator{\Gal}{Gal}
\DeclareMathOperator{\Hom}{Hom}
\DeclareMathOperator{\im}{im}
\DeclareMathOperator{\Isom}{Isom}
\DeclareMathOperator{\ord}{ord}
\DeclareMathOperator{\Pic}{Pic}
\DeclareMathOperator{\rk}{rk}
\DeclareMathOperator{\Spec}{Spec}
\DeclareMathOperator{\Tot}{Tot}
\DeclareMathOperator{\vol}{vol}

%%%%%%%%%%%%%%%%%%%%%%%%%%%%%%%%%%%%%%%%%%%%%%%%%%%%%%%%%%%%%%%%%%%%%%%%%%%%%%%%

\newcommand{\CC}{\mathbb{C}}
\newcommand{\FF}{\mathbb{F}}
\newcommand{\NN}{\mathbb{N}}
\newcommand{\QQ}{\mathbb{Q}}
\newcommand{\RR}{\mathbb{R}}
\newcommand{\ZZ}{\mathbb{Z}}

\renewcommand{\AA}{\mathbb{A}}
\newcommand{\PP}{\mathbb{P}}

\DeclareMathOperator{\Gr}{Gr}

\newcommand{\bone}{1\!\!1}

\newcommand{\Parf}{\mathcal{P}\!\text{\it arf}}

% force \nolimits on \det:
\renewcommand{\det}{\operatorname{det}}

\renewcommand{\Re}{\operatorname{Re}}
\renewcommand{\emptyset}{\varnothing}

\newcommand{\ar}{\text{\it ar}}
\newcommand{\BM}{\text{\it BM}}
\newcommand{\DB}{{\mathcal{D}\text{-\foreignlanguage{russian}{Б}}}}
\newcommand{\dR}{\text{\it dR}}
\newcommand{\et}{\text{\it ét}}
\newcommand{\fg}{\text{\it fg}}
\newcommand{\fin}{\text{\it fin}}
\newcommand{\is}{\text{\it is}}
\newcommand{\red}{\text{\it red}}
\newcommand{\tors}{\text{\it tors}}
\newcommand{\Wc}{\text{\it W,c}}
\newcommand{\Zar}{\text{\it Zar}}

\newcommand{\dfn}{\mathrel{\mathop:}=}
\newcommand{\rdfn}{=\mathrel{\mathop:}}

\newcommand{\iHom}{\underline{\Hom}}
\newcommand{\RHom}{R\!\Hom}

%%%%%%%%%%%%%%%%%%%%%%%%%%%%%%%%%%%%%%%%%%%%%%%%%%%%%%%%%%%%%%%%%%%%%%%%%%%%%%%%

\usepackage{tikz-cd}
\usetikzlibrary{arrows}
\usetikzlibrary{calc}
\usetikzlibrary{babel}
\usetikzlibrary{decorations.pathreplacing}

\newcommand{\tikzpb}{\ar[phantom,pos=0.2]{dr}{\text{\large$\lrcorner$}}}
\newcommand{\tikzpbur}{\ar[phantom,pos=0.2]{dl}{\text{\large$\llcorner$}}}

%%%%%%%%%%%%%%%%%%%%%%%%%%%%%%%%%%%%%%%%%%%%%%%%%%%%%%%%%%%%%%%%%%%%%%%%%%%%%%%%

\usepackage[numbers]{natbib}

\usepackage[hidelinks]{hyperref}

\hypersetup{
    colorlinks,
    linkcolor={red!60!black},
    citecolor={blue!60!black},
    urlcolor={blue!80!black}
}

%%%%%%%%%%%%%%%%%%%%%%%%%%%%%%%%%%%%%%%%%%%%%%%%%%%%%%%%%%%%%%%%%%%%%%%%%%%%%%%%

\usepackage{amsthm}

\newtheoremstyle{myplain}
{\topsep}   % ABOVESPACE
{\topsep}   % BELOWSPACE
{\itshape}  % BODYFONT
{0pt}       % INDENT (empty value is the same as 0pt)
{\bfseries} % HEADFONT
{.}         % HEADPUNCT
{5pt plus 1pt minus 1pt} % HEADSPACE
{\thmnumber{#2}. \thmname{#1}\thmnote{ (#3)}}   % CUSTOM-HEAD-SPEC

\newtheoremstyle{mydefinition}
{\topsep}   % ABOVESPACE
{\topsep}   % BELOWSPACE
{\normalfont}  % BODYFONT
{0pt}       % INDENT (empty value is the same as 0pt)
{\bfseries} % HEADFONT
{.}         % HEADPUNCT
{5pt plus 1pt minus 1pt} % HEADSPACE
{\thmnumber{#2}. \thmname{#1}\thmnote{ (#3)}}   % CUSTOM-HEAD-SPEC

\theoremstyle{plain}
\newtheorem{maintheorem}{Theorem}
\renewcommand*{\themaintheorem}{\Roman{maintheorem}}
\newtheorem*{maintheorem*}{Main theorem}
\newtheorem*{thetheorem*}{Theorem}
\newtheorem*{proposition*}{Proposition}

\theoremstyle{myplain}
\newtheorem{theorem}{Theorem}[section]
\newtheorem{proposition}[theorem]{Proposition}
\newtheorem{lemma}[theorem]{Lemma}
\newtheorem{corollary}[theorem]{Corollary}

\theoremstyle{definition}
\newtheorem*{conjecture*}{Conjecture}

\theoremstyle{mydefinition}
\newtheorem{definition}[theorem]{Definition}
\newtheorem{conjecture}[theorem]{Conjecture}
\newtheorem{remark}[theorem]{Remark}
\newtheorem{example}[theorem]{Example}

%%%%%%%%%%%%%%%%%%%%%%%%%%%%%%%%%%%%%%%%%%%%%%%%%%%%%%%%%%%%%%%%%%%%%%%%%%%%%%%%

\usepackage[perpage,symbol]{footmisc}
\renewcommand{\thefootnote}{\ifcase\value{footnote}\or{*}\or{**}\or{***}\or{****}\fi}


\title{Weil-étale cohomology for arbitrary arithmetic schemes and $n < 0$.
  Part II: The special value conjecture}
\author{Alexey Beshenov}
% \date{November 21, 2020}

\AtEndDocument{%
  \par
  \medskip
  \begin{tabular}{@{}l@{}}%
    \\
    Alexey Beshenov \\
    Center for Research in Mathematics (CIMAT), Guanajuato, Mexico \\
    % E-mail: \texttt{alexey.beshenov@cimat.mx} \\
    URL: \url{https://cadadr.org/}
  \end{tabular}}

\numberwithin{equation}{section}

\begin{document}

\maketitle

\begin{abstract}
  Following the ideas of Flach and Morin \cite{Flach-Morin-2018}, we state a
  conjecture in terms of Weil-étale cohomology for the vanishing order and
  special value of the zeta function $\zeta (X,s)$ at $s = n < 0$, where $X$ is
  a separated scheme of finite type over $\Spec \ZZ$. We prove that the
  conjecture is compatible with closed-open decompositions of schemes and affine
  bundles, and as a consequence, that it holds for cellular schemes over certain
  $1$-dimensional bases.

  This is a continuation of author's preprint \cite{Beshenov-Weil-etale-1},
  which gives a construction of Weil-étale cohomology for $n<0$ under the
  mentioned assumptions on $X$.
\end{abstract}

\tableofcontents

%%%%%%%%%%%%%%%%%%%%%%%%%%%%%%%%%%%%%%%%%%%%%%%%%%%%%%%%%%%%%%%%%%%%%%%%%%%%%%%%

\section{Introduction}

Let $X$ be an \textbf{arithmetic scheme}, by which we will mean throughout this
paper that it is a separated scheme of finite type over $\Spec \ZZ$. Then the
corresponding zeta function is defined by
\[ \zeta (X,s) = \prod_{\substack{x \in X \\ \text{closed pt.}}}
  \frac{1}{1 - \#\kappa (x)^{-s}}, \]
where $\kappa (x) = \mathcal{O}_{X,x}/\mathfrak{m}_{X,x}$ denotes the residue
field of a point. The above product converges for $\Re s > \dim X$, and
conjecturally admits a meromorphic continuation to the whole complex plane.
For basic facts and conjectures about zeta functions of schemes, see Serre's
survey \cite{Serre-1965}.

Of particular interest are the so-called special values of $\zeta (X,s)$ at
integers $s = n \in \ZZ$. To define these, assume that $\zeta (X,s)$ admits a
meromorphic continuation around $s = n$. We will denote by
$d_n = \ord_{s=n} \zeta (X,s)$ the \textbf{vanishing order} of $\zeta (X,s)$ at
$s = n$. That is, $d_n > 0$ (resp. $d_n < 0$) if $\zeta (X,s)$ has a zero
(resp. pole) of order $d_n$ at $s = n$. The corresponding \textbf{special value}
of $\zeta (X,s)$ at $s = n$ is defined to be the leading nonzero coefficient of
the Taylor expansion:
$$\zeta^* (X,n) = \lim_{s \to n} (s - n)^{-d_n}\,\zeta (X,s).$$

Early on Stephen Lichtenbaum conjectured that both numbers
$\ord_{s = n} \zeta (X,s)$ and $\zeta^* (X,n)$ should have a cohomological
interpretation, related to étale motivic cohomology of $X$
(see e.g. \cite{Lichtenbaum-1984} for varieties over finite fields).
This is made more precise in Lichtenbaum's \textbf{Weil-étale program}.
In particular it suggests the existence of a cohomology theory
$H^i_\Wc (X, \ZZ (n))$, \textbf{Weil-étale cohomology} with compact support,
which encodes the information about the vanishing order and special value of
$\zeta (X,n)$ at $s = n$.

For Lichtenbaum's recent work on the topic, we refer to his papers
\cite{Lichtenbaum-2005,Lichtenbaum-2009-number-rings,Lichtenbaum-2009-Euler-char,Lichtenbaum-2021}.
The case of varieties over finite fields $X/\FF_q$ was further studied by
Thomas Geisser
\cite{Geisser-2004,Geisser-2006,Geisser-2010-arithmetic-homology},
and it is rather well understood now.

Matthias Flach and Baptiste Morin considered the case of proper, regular
arithmetic schemes $X$. In \cite{Flach-Morin-2012} they defined and studied the
corresponding Weil-étale topos. Later, Morin gave in \cite{Morin-2014} an
explicit construction of Weil-étale cohomology groups $H^i_\Wc (X, \ZZ (n))$ for
$n = 0$, and $X$ a proper, regular arithmetic scheme, under assumptions on
finite generation of suitable étale motivic cohomology groups. This construction
was further generalized by Flach and Morin in \cite{Flach-Morin-2018} to any
$n \in \ZZ$, again for proper and regular $X$.

Motivated by the work of Flach and Morin, the author constructed in
\cite{Beshenov-Weil-etale-1} the Weil-étale cohomology groups
$H^i_\Wc (X, \ZZ (n))$ for any arithmetic scheme $X$ (thus removing the
assumptions that $X$ is proper or regular) and $n < 0$ (which apparently
simplifies certain aspects of the theory). The construction relies on the
following assumption (see \cite[\S 8]{Beshenov-Weil-etale-1} for further details
and known cases).

\begin{conjecture*}
  $\mathbf{L}^c (X_\et,n)$: the cohomology groups $H^i (X_\et, \ZZ^c (n))$ are
  finitely generated for all $i \in \ZZ$.
\end{conjecture*}

Namely, assuming $\mathbf{L}^c (X_\et,n)$, we defined in
\cite{Beshenov-Weil-etale-1} perfect complexes of abelian groups
$R\Gamma_\Wc (X, \ZZ(n))$. This text is a continuation of
\cite{Beshenov-Weil-etale-1} and it explores the conjectural relation of our
Weil-étale cohomology to the special value of $\zeta (X,s)$ at $s = n < 0$.
Specifically, we make the following conjectures.

\begin{enumerate}
\item[1)] \textbf{Conjecture} $\mathbf{VO} (X,n)$:
  \emph{the vanishing order is given by}
  \[ \ord_{s=n} \zeta (X,s) =
    \sum_{i\in \ZZ} (-1)^i \cdot i \cdot \rk_\ZZ H_\Wc^i (X, \ZZ(n)). \]

\item[2)] A consequence of \textbf{Conjecture} $\mathbf{B} (X,n)$
  (see \S\ref{sec:regulator}):
  \emph{after tensoring with $\RR$, one obtains a long exact sequence of finite
    dimensional real vector spaces}
  \[ \cdots \to H_\Wc^{i-1} (X, \RR (n)) \xrightarrow{\smile\theta}
    H_\Wc^i (X, \RR (n)) \xrightarrow{\smile\theta}
    H_\Wc^{i+1} (X, \RR (n)) \to \cdots \]
  Here
  $H_\Wc^i (X, \RR (n)) = H_\Wc^i (X, \ZZ (n)) \otimes_\ZZ \RR =
  H^i (R\Gamma_\Wc (X, \ZZ (n)) \otimes_\ZZ \RR)$.

  It follows that there is a canonical isomorphism
  \[ \lambda\colon \RR \xrightarrow{\cong}
    (\det_\ZZ R\Gamma_\Wc (X, \ZZ (n))) \otimes \RR. \]
  Here $\det_\ZZ R\Gamma_\Wc (X, \ZZ (n))$ is the determinant of the
  perfect complex of abelian groups $R\Gamma_\Wc (X, \ZZ (n))$, in the sense of
  Knudsen and Mumford \cite{Knudsen-Mumford-1976}. In particular,
  $\det_\ZZ R\Gamma_\Wc (X, \ZZ (n))$ is a free $\ZZ$-module of rank
  $1$. For the reader's convenience, we include a brief overview of determinants
  in appendix~\ref{app:determinants}.

\item[3)] \textbf{Conjecture} $\mathbf{C} (X,n)$:
  \emph{the special value is determined up to sign by}
  \[ \lambda (\zeta^* (X, n)^{-1}) \cdot \ZZ =
    \det_\ZZ R\Gamma_\Wc (X, \ZZ (n)). \]
\end{enumerate}

If $X$ is proper and regular, then our construction of
$R\Gamma_\Wc (X, \ZZ (n))$ and the above conjectures agree with those of Flach
and Morin from \cite{Flach-Morin-2018}. Apart from removing the assumption that
$X$ is proper and regular, one novelty of this work is that we prove the
compatibility of the above conjectures with operations on schemes, in particular
closed-open decompositions $Z \not\hookrightarrow X \hookleftarrow U$, where
$Z \subset X$ is a closed subscheme and $U = X\setminus Z$ is the open
complement, and affine bundles $\AA_X^r = \AA_\ZZ^r \times X$.
(See proposition~\ref{prop:compatibility-of-VO(X,n)} and
theorem~\ref{thm:compatibility-of-C(X,n)}.) This gives a machinery that allows
one to start from the particular instances of schemes for which the conjectures
are known, and construct new schemes for which the conjectures hold as well.  As
an application, we prove in \S\ref{sec:unconditional-results} the following
result.

\begin{maintheorem*}
  Let $B$ be a $1$-dimensional arithmetic scheme, such that each of the generic
  points $\eta \in B$ satisfies one of the following properties:
  \begin{enumerate}
  \item[a)] $\fchar \kappa (\eta) = p > 0$;

  \item[b)] $\fchar \kappa (\eta) = 0$, and $\kappa (\eta)/\QQ$ is an abelian
    number field.
  \end{enumerate}
  If $X$ is a $B$-cellular arithmetic scheme with smooth quasi-projective fiber
  $X_{\red,\CC}$, then the conjectures $\mathbf{VO} (X,n)$ and
  $\mathbf{C} (X,n)$ hold unconditionally for any $n < 0$.
\end{maintheorem*}

In fact, this result will be established for a bigger class of arithmetic
schemes $\mathcal{C} (\ZZ)$; we refer to \S\ref{sec:unconditional-results} for
further details.

\subsection*{Outline of the paper}

In \S\ref{sec:regulator} we define the regulator morphism, based on the
construction of Kerr, Lewis, and Müller-Stach
\cite{Kerr-Lewis-Muller-Stach-2006}, and state the conjecture $\mathbf{B} (X,n)$
related to it.

Then \S\ref{sec:vanishing-order-conjecture} is devoted to the vanishing order
conjecture $\mathbf{VO} (X,n)$. We also explain why it is consistent with a
conjecture of Soulé and the vanishing orders that come from the expected
functional equation

In \S\ref{sec:special-value-conjecture} we state the special value conjecture
$\mathbf{C} (X,n)$.

We explain in \S\ref{sec:finite-fields} that for $X/\FF_q$ a variety over a
finite field, the special value conjecture $\mathbf{C} (X,n)$ is consistent with
the conjectures considered by Geisser in
\cite{Geisser-2004,Geisser-2006,Geisser-2010-arithmetic-homology}.
In particular, it holds for any smooth projective $X/\FF_q$, assuming the
conjecture $\mathbf{L}^c (X_\et,n)$. This also generalizes to all varieties
$X/\FF_q$ if we assume the resolution of singularities over $\FF_q$.

Then we prove in \S\ref{sec:compatibility-with-operations} that the conjectures
$\mathbf{VO} (X,n)$ and $\mathbf{C} (X,n)$ are compatible with basic operations
on schemes: disjoint unions, closed-open decompositions, and affine
bundles. Using these results, we conclude in \S\ref{sec:unconditional-results}
with a class of schemes for which the special value conjecture holds
unconditionally.

For convenience of the reader, the appendix \ref{app:determinants} briefly
reviews basic definitions and facts related to the determinants of complexes
that are relevant to this text.

\subsection*{Notation}

In this paper, $X$ will always denote an \textbf{arithmetic scheme} (separated,
of finite type over $\Spec \ZZ$), and $n$ will always be a strictly negative
integer.

We denote by $R\Gamma_\fg (X, \ZZ (n))$ and $R\Gamma_\Wc (X, \ZZ (n))$ the
complexes constructed in \cite{Beshenov-Weil-etale-1}, assuming the conjecture
$\mathbf{L}^c (X_\et,n)$ stated above. For $\QQ$-coefficients, we put
\begin{align*}
  R\Gamma_\fg (X, \QQ (n)) & \dfn
  R\Gamma_\fg (X, \ZZ (n)) \otimes_\ZZ^\mathbf{L} \QQ =
  R\Gamma_\fg (X, \ZZ (n)) \otimes_\ZZ \QQ, \\
  R\Gamma_\Wc (X, \QQ (n)) & \dfn
  R\Gamma_\Wc (X, \ZZ (n)) \otimes_\ZZ^\mathbf{L} \QQ =
  R\Gamma_\Wc (X, \ZZ (n)) \otimes_\ZZ \QQ.
\end{align*}
Accordingly,
\begin{align*}
  H^i_\fg (X, \QQ (n)) & \dfn
  H^i (R\Gamma_\fg (X, \QQ (n))) = H^i_\fg (X, \ZZ (n)) \otimes_\ZZ \QQ, \\
  H^i_\Wc (X, \QQ (n)) & \dfn
  H^i (R\Gamma_\Wc (X, \QQ (n))) = H^i_\Wc (X, \ZZ (n)) \otimes_\ZZ \QQ.
\end{align*}
Similarly, we define the corresponding complexes and cohomology with
$\RR$-coefficients.

By $X (\CC)$ we denote the space of complex points of $X$ with the usual
analytic topology. It carries a natural action of $G_\RR = \Gal (\CC/\RR)$ via
complex conjugation. For a subring $A \subseteq \RR$ we denote by $A (n)$ the
$G_\RR$-module $(2\pi i)^n\,A$, and also the corresponding constant
$G_\RR$-equivariant sheaf on $X (\CC)$.

We denote by $R\Gamma_c (X (\CC), A (n))$
(resp. $R\Gamma_c (G_\RR, X (\CC), A (n))$) the cohomology with compact
support (resp. $G_\RR$-equivariant cohomology with compact support) of $X$ with
$A (n)$-coefficients. For more details on $G_\RR$-equivariant cohomology,
see \cite{Beshenov-Weil-etale-1}. With real coefficients,
$H_c^i (G_\RR, X (\CC), \RR (n)) = H^i_c (X (\CC), \RR (n))^{G_\RR}$,
where the $G_\RR$-action on $H^i_c (X (\CC), \RR (n))$ naturally comes from the
corresponding action on $X (\CC)$ and $\RR (n)$.
The \textbf{Borel--Moore homology} is defined as dual to the cohomology with
compact support. In particular, we will be interested in
\begin{align*}
  R\Gamma_\BM (X (\CC), \RR (n)) & \dfn
  \RHom (R\Gamma_c (X (\CC), \RR (n)), \RR), \\
  R\Gamma_\BM (G_\RR, X (\CC), \RR (n)) & \dfn
  \RHom (R\Gamma_c (G_\RR, X (\CC), \RR (n)), \RR).
\end{align*}

\subsection*{Acknowledgments}

Parts of this work are based on the results of my PhD thesis, prepared in
Université de Bordeaux and Universiteit Leiden under supervision of Baptiste
Morin and Bas Edixhoven. I am deeply grateful for their support during working
on this project. I am also indebted to Matthias Flach, since the ideas of this
work come from \cite{Flach-Morin-2018}. I thank Stephen Lichtenbaum and Niranjan
Ramachandran who kindly accepted to be the referees for my thesis and provided
many useful comments and suggestions. Finally, I thank Pedro Luis del Ángel,
José Jaime Hernández Castillo, Diosel López Cruz, and Maxim Mornev for various
fruitful conversations.

This paper was edited while I was visiting Center for Research in Mathematics
(CIMAT), Guanajuato. I am grateful personally to Pedro Luis del Ángel and Xavier
Gómez Mont for their hospitality.

%%%%%%%%%%%%%%%%%%%%%%%%%%%%%%%%%%%%%%%%%%%%%%%%%%%%%%%%%%%%%%%%%%%%%%%%%%%%%%%%

\section{Regulator morphism}
\label{sec:regulator}

To state the special value conjecture, we need to define the regulator morphism
from motivic cohomology to Deligne-Beilinson (co)homology.
It was introduced by Bloch in \cite{Bloch-1986-Lefschetz}, and here we are going
to use the construction of Kerr, Lewis, and Müller-Stach from
\cite{Kerr-Lewis-Muller-Stach-2006}, which works on the level of complexes.
We will simply call it ``the KLM morphism''. It works under assumption that
$X_{\red,\CC}$ is a smooth quasi-projective variety.

For simplicity we will assume in this section that $X$ is reduced (the motivic
cohomology does not distinguish between $X$ and $X_\red$), and that $X_\CC$ is
connected of dimension $d_\CC$ (otherwise, the arguments below may be applied to
each connected component). We fix a compactification by a normal crossing
divisor
\[ \begin{tikzcd}
    X_\CC \ar[right hook->]{r}{j} & \overline{X}_\CC & D \ar[left hook->]{l}
  \end{tikzcd} \]
The KLM regulator takes form of a morphism in the derived category
\begin{equation}
  \label{eqn:KLM-morphism-1}
  z^p (X_\CC, -\bullet) \otimes \QQ \to
  {}' C_\mathcal{D}^{2p - 2d_\CC + \bullet} (\overline{X}_\CC, D, \QQ (p-d_\CC)).
\end{equation}

Here $z^p (X_\CC, -\bullet)$ denotes the Bloch's cycle complex
\cite{Bloch-1986}. We recall that by the definition, $z^p (X_\CC, i)$ is freely
generated by algebraic cycles $Z \subset X_\CC \times_{\Spec \CC} \Delta_\CC^i$
of codimension $p$ that intersect properly the faces of the algebraic simplex
$\Delta_\CC^i = \Spec \CC [t_0,\ldots,t_i]/(1 - \sum_j t_j)$. For us it will be
more convenient to pass to
$$z_{d_\CC - p} (X_\CC, i) = z^p (X_\CC, i),$$
generated by cycles $Z \subset X_\CC \times_{\Spec \CC} \Delta_\CC^i$ of
dimension $p+i$.

The complex ${}' C_\mathcal{D}^{\bullet} (\overline{X}_\CC, D, \QQ (k))$ on the
right hand side of \eqref{eqn:KLM-morphism-1} computes Deligne(--Beilinson)
homology, as defined by Jannsen \cite{Jannsen-1988}. In particular, taking
$p = d_\CC + 1 - n$, tensoring with $\RR$, and shifting by $2n$, we obtain
\begin{equation}
  \label{eqn:KLM-morphism-2}
  z_{n-1} (X_\CC, -\bullet) \otimes \RR [2n] \to
  {}' C_\mathcal{D}^{2 + \bullet} (\overline{X}_\CC, D, \RR (1-n)).
\end{equation}

\begin{remark}
  Some comments are in order.

  \begin{enumerate}
  \item Originally, the KLM morphism is defined using a cubical version of cycle
    complexes, but these are quasi-isomorphic to the usual simplicial cycle
    complexes (see \cite{Levine-1994}), so we do not make a distinction here.
    For a simplicial version, see also \cite{Kerr-Lewis-Lopatto-2018}.

  \item The KLM morphism is defined as a genuine morphism of complexes (not just
    a morphism in the derived category) on a certain subcomplex
    $z^r_\RR (X_\CC, \bullet) \subset z^r (X_\CC, \bullet)$. This inclusion is a
    quasi-isomorphism, if we pass rational coefficients. This is stated without
    tensoring with $\QQ$ in the original paper
    \cite{Kerr-Lewis-Muller-Stach-2006}, but the omission is acknowledged later
    in \cite{Kerr-Lewis-2007}. For our particular purposes, it will be enough to
    have a regulator with coefficients in $\QQ$, or in fact in $\RR$.

  \item The case of smooth quasi-projective $X_\CC$, where one has to consider a
    compactification by a normal crossing divisor as above, is treated in
    \cite[\S 5.9]{Kerr-Lewis-Muller-Stach-2006}.
  \end{enumerate}
\end{remark}

Now we make a little digression to identify the right hand side of
\eqref{eqn:KLM-morphism-2}. Under our assumption that $n < 0$, the Deligne
homology corresponds to Borel--Moore homology.

\begin{lemma}
  For any $n < 0$ there is a quasi-isomorphism
  \[ {}' C^\bullet_\mathcal{D} (\overline{X_\CC}, D, \RR (1-n)) \cong
    R\Gamma_\BM (X (\CC), \RR (n)) [-1] \dfn
    \RHom (R\Gamma_c (X (\CC), \RR (n)), \RR) [-1]. \]
  Further, this respects the natural actions of $G_\RR$ on both complexes.

  \begin{proof}
    From the proof of \cite[Theorem~1.15]{Jannsen-1988}, for any $k \in \ZZ$ we
    have a quasi-isomorphism
    \begin{equation}
      \label{eqn:Jannsen-Theorem-1.15}
      {}' C^\bullet_\mathcal{D} (\overline{X_\CC}, D, \RR (k)) \cong
      R\Gamma (\overline{X} (\CC), \RR (k + d_\CC)_{\DB, (\overline{X}_\CC,X_\CC)}) [2d_\CC],
    \end{equation}
    where
    \[ \RR (k + d_\CC)_{\DB, (\overline{X}_\CC,X_\CC)} =
      \Cone \Bigl(R j_* \RR (k + d_\CC)
      \oplus
      \Omega^{\geqslant k + d_\CC}_{\overline{X} (\CC)} (\log D)
      \xrightarrow{\epsilon - \iota}
      R j_* \Omega_{X (\CC)}^\bullet \Bigr) [-1] \]
    is the sheaf whose hypercohomology on $\overline{X} (\CC)$ gives
    Deligne-Beilinson cohomology (see \cite{Esnault-Viehweg-1988} for
    further details).

    Here $\Omega^\bullet_{\overline{X} (\CC)}$ denotes the usual de Rham complex
    of holomorphic differential forms, and
    $\Omega^\bullet_{\overline{X} (\CC)} (\log D)$ is the complex of forms with
    at most logarithmic poles along $D (\CC)$.
    The latter complex is filtered by subcomplexes
    $\Omega^{\geqslant \bullet}_{\overline{X} (\CC)} (\log D)$.
    The morphism
    $\epsilon\colon R j_* \RR (k) \to R j_* \Omega^\bullet_{X (\CC)}$ is induced
    by the canonical morphism of sheaves $\RR (k) \to \mathcal{O}_{X (\CC)}$,
    and $\iota$ is induced by the natural inclusion
    $\Omega^\bullet_{\overline{X} (\CC)} (\log D) \xrightarrow{\cong} j_*
    \Omega_{X (\CC)}^\bullet = R j_* \Omega_{X (\CC)}^\bullet$, which is a
    quasi-isomorphism of filtered complexes.

    We will be interested in the case of $k > 0$, when the part
    $\Omega^{\geqslant k + d_\CC}_{\overline{X} (\CC)} (\log D)$ disappears,
    and we obtain
    \begin{align}
      \notag \RR (k + d_\CC)_{\DB, (\overline{X}_\CC,X_\CC)} & \cong
                                     R j_* \Cone \Bigl(\RR (k + d_\CC)
                                     \xrightarrow{\epsilon}
                                     \Omega_{X (\CC)}^\bullet \Bigr) [-1] \\
      \notag & \cong R j_* \Bigl(\RR (k + d_\CC) \xrightarrow{\epsilon}
               \Omega_{X (\CC)}^\bullet [-1] \Bigr) \\
      \label{eqn:deligne-homology-1} & \cong R j_* \Bigl(\RR (k + d_\CC) \to \CC [-1] \Bigr) \\
      \label{eqn:deligne-homology-2} & \cong R j_* \RR (k + d_\CC - 1) [-1]
    \end{align}
    Here \eqref{eqn:deligne-homology-1} comes from the Poincaré lemma
    $\CC \cong \Omega_{X (\CC)}^\bullet$, and \eqref{eqn:deligne-homology-2}
    comes from the short exact sequence of $G_\RR$-modules
    $\RR (k + d_\CC) \rightarrowtail \CC \twoheadrightarrow \RR (k + d_\CC - 1)$.

    Coming back to \eqref{eqn:Jannsen-Theorem-1.15} for $k = 1-n$,
    we conclude that
    \begin{align*}
      {}' C^\bullet_\mathcal{D} (\overline{X_\CC}, D, \RR (1-n)) & \cong
      R\Gamma (X (\CC), \RR (d_\CC - n)) [2d_\CC-1] \\
      & \cong \RHom (R\Gamma_c (X (\CC), \RR (n)), \RR) [-1].
    \end{align*}
    Here the last isomorphism is Poincaré duality.
  \end{proof}
\end{lemma}

Now coming back to \eqref{eqn:KLM-morphism-2}, the above lemma allows us to
reinterpret the KLM morphism as
\begin{equation}
  \label{eqn:KLM-morphism-3}
  z_{n-1} (X_\CC, -\bullet) \otimes \RR [2n] \to
  R\Gamma_\BM (X (\CC), \RR (n)), \RR) [1].
\end{equation}

By the definition, we have
\begin{equation}
  \label{eqn:KLM-morphism-4}
  z_{n-1} (X_\CC, -\bullet) \otimes \RR [2n] =
  z_{n-1} (X_\CC, -\bullet) \otimes \RR [2n-2] [2] =
  \Gamma (X_{\CC,\et}, \RR^c (n-1)) [2],
\end{equation}
where the complex of sheaves $\RR^c (p)$ is defined by
$U \rightsquigarrow z_p (U, -\bullet) \otimes \RR [2p]$.
By étale cohomological descent \cite[Theorem~3.1]{Geisser-2010},
we have\footnote{We note that \cite[Theorem~3.1]{Geisser-2010} holds
  unconditionally, since the Beilinson--Lichtenbaum conjecture follows from
  the Bloch--Kato conjecture, which is now a theorem. See also
  \cite{Geisser-2004-Dedekind} where the consequences of Bloch--Kato for motivic
  cohomology are deduced.}
\begin{equation}
  \label{eqn:KLM-morphism-5}
  \Gamma (X_{\CC,\et}, \RR^c (n-1)) \cong R\Gamma (X_{\CC,\et}, \RR^c (n-1)).
\end{equation}
Finally, the base change from $X$ to $X_\CC$ naturally maps cycles
$Z \subset X \times \Delta_\ZZ^i$ of dimension $n$ to cycles in
$X_\CC \times_{\Spec \CC} \Delta_\CC^i$ of dimension $n-1$, so that there is a
morphism
\begin{equation}
  \label{eqn:KLM-morphism-6}
  R\Gamma (X_\et, \RR^c (n)) \to R\Gamma (X_{\CC,\et}, \RR^c (n-1)) [2].
\end{equation}

\begin{remark}
  Assuming that $X$ is flat of pure Krull dimension $d$, we have
  $\RR^c (n)^X = \RR (d-n)^X [2d]$, where $\RR (\bullet)$ is the usual cycle
  complex defined by $z^n (\text{\textvisiblespace}, -\bullet) [-2n]$.
  Similarly, $\RR^c (n)^{X_\CC} = \RR (d_\CC-n)^{X_\CC} [2d_\CC]$, with
  $d_\CC = d - 1$. With this renumbering, the morphism
  \eqref{eqn:KLM-morphism-6} becomes
  $$R\Gamma (X_\et, \RR (d-n)) [2d] \to R\Gamma (X_{\CC,\et}, \RR (d-n)) [2d].$$
  This probably looks more natural, but we do not impose extra assumptions on
  $X$ and work exclusively with complexes $A^c (\bullet)$ defined in terms of
  dimension of algebraic cycles, instead of $A (\bullet)$ defined in terms of
  codimension.
\end{remark}

\begin{definition}
  Given an arithmetic scheme $X$ with smooth quasi-projective $X_\CC$, and
  $n < 0$, consider the composition of morphisms
  \begin{multline*}
    R\Gamma (X_\et, \RR^c (n)) \xrightarrow{\text{\eqref{eqn:KLM-morphism-6}}}
    R\Gamma (X_{\CC,\et}, \RR^c (n-1)) [2] \stackrel{\text{\eqref{eqn:KLM-morphism-5}}}{\cong}
    \Gamma (X_{\CC,\et}, \RR^c (n-1)) [2] \\
    \stackrel{\text{\eqref{eqn:KLM-morphism-4}}}{=}
    z_{n-1} (X_\CC, -\bullet)_\RR [2n] \xrightarrow{\text{\eqref{eqn:KLM-morphism-3}}}
    R\Gamma_\BM (X (\CC), \RR (n)), \RR) [1].
  \end{multline*}
  Further, we take $G_\RR$-invariants, which gives us the
  \textbf{(étale) regulator}
  \[ Reg_{X,n}\colon R\Gamma (X_\et, \RR^c (n)) \to
    R\Gamma_\BM (G_\RR, X(\CC), \RR (n)) [1]. \]
\end{definition}

Now we state our conjecture about the regulator, which will play an important
role in everything that follows.

\begin{conjecture}
  $\mathbf{B} (X,n)$: given an arithmetic scheme $X$ with smooth
  quasi-projective $X_\CC$ and $n < 0$, the regulator morphism $Reg_{X,n}$
  induces a quasi-isomorphism of complexes of real vector spaces
  \[ Reg_{X,n}^\vee\colon R\Gamma_c (G_\RR, X (\CC), \RR (n)) [-1] \to
    \RHom (R\Gamma (X_\et, \ZZ^c (n)), \RR). \]
\end{conjecture}

\begin{remark}
  If $X/\FF_q$ is a variety over a finite field, then $X (\CC) = \emptyset$,
  so the regulator map in not interesting. Indeed, in our setting, its purpose
  is to take care of the archimedian places of $X$. In this case
  $\mathbf{B} (X,n)$ implies that $H^i (X_\et, \ZZ^c (n))$ are torsion groups.
  However, by \cite[Proposition~4.2]{Beshenov-Weil-etale-1}, the finite
  generation conjecture $\mathbf{L}^c (X_\et, n)$ already implies that
  $H^i (X_\et, \ZZ^c (n))$ are finite groups.
\end{remark}

\begin{remark}
  \label{rmk:regulator-is-defined-for-XC-smooth-quasi-proj}
  We reiterate that our construction of $Reg_{X,n}$ works for $X_{\red,\CC}$
  smooth quasi-projective. In everything that follows, whenever the regulator
  morphism or the conjecture $\mathbf{B} (X,n)$ is brought, we will tacitly
  assume this.  This is quite unfortunate, because the Weil-étale complexes
  $R\Gamma_\Wc (X, \ZZ(n))$ were constructed in \cite{Beshenov-Weil-etale-1} for
  any arithmetic scheme, assuming only $\mathbf{L}^c (X_\et,n)$. Defining the
  regulator for singular $X_{\red,\CC}$ is an interesting project for further
  work.
\end{remark}

%%%%%%%%%%%%%%%%%%%%%%%%%%%%%%%%%%%%%%%%%%%%%%%%%%%%%%%%%%%%%%%%%%%%%%%%%%%%%%%%

\section{Vanishing order conjecture}
\label{sec:vanishing-order-conjecture}

Assuming that $\zeta (X,s)$ admits a meromorphic continuation around
$s = n < 0$, we state the following conjecture for the vanishing order at
$s = n$.

\begin{conjecture}
  $\mathbf{VO} (X,n)$: one has
  \[ \ord_{s=n} \zeta (X,s) =
    \chi' (R\Gamma_\Wc (X, \ZZ (n))) \dfn
    \sum_{i \in \ZZ} (-1)^i \cdot i \cdot \rk_\ZZ H^i_\Wc (X, \ZZ (n)). \]
\end{conjecture}

We note that the right hand side makes sense assuming the conjecture
$\mathbf{L}^c (X_\et,n)$, under which $H^i_\Wc (X, \ZZ (n))$ are finitely
generated groups, trivial for $|i| \gg 0$
(see \cite[Proposition~7.12]{Beshenov-Weil-etale-1}).

\begin{remark}
  The conjecture $\mathbf{VO} (X,n)$ is similar to
  \cite[Conjecture~5.11]{Flach-Morin-2018}. When $X$ is proper and regular, then
  $\mathbf{VO} (X,n)$ is the same as the vanishing order conjecture of Flach and
  Morin. Namely, the latter takes form
  \begin{equation}
    \label{eqn:FM-vanishing-order}
    \ord_{s = n} \zeta (X,s) =
    \sum_{i\in \ZZ} (-1)^i \cdot i \cdot \dim_\RR H^i_{\ar,c} (X, \widetilde{\RR}(n)),
  \end{equation}
  where by definition,
  \[ R\Gamma_{\ar,c} (X, \widetilde{\RR}(n)) \dfn
    R\Gamma_c (X, \RR(n)) \oplus R\Gamma_c (X, \RR(n)) [-1]. \]
  Moreover, \cite[Proposition 4.14]{Flach-Morin-2018}, gives a distinguished
  triangle
  \[ R\Gamma_\dR (X_\RR/\RR) / \Fil^n [-2] \to
    R\Gamma_{\ar,c} (X, \widetilde{\RR}(n)) \to
    R\Gamma_\Wc (X, \ZZ(n)) \otimes \RR \to
    R\Gamma_\dR (X_\RR/\RR) / \Fil^n [-1] \]
  Therefore, in case of $n < 0$, we have
  $R\Gamma_{\ar,c} (X, \widetilde{\RR}(n)) \cong
  R\Gamma_\Wc (X, \ZZ(n)) \otimes \RR$, and
  \eqref{eqn:FM-vanishing-order} is exactly our conjecture $\mathbf{VO} (X,n)$.
\end{remark}

\begin{remark}
  The alternating sum in the formula is the so-called
  \textbf{secondary Euler characteristic} of the Weil-étale complex
  $R\Gamma_\Wc (X, \ZZ (n))$.  The easy calculations below show that the usual
  Euler characteristic of $R\Gamma_\Wc (X, \ZZ (n))$ vanishes, assuming
  conjectures $\mathbf{L}^c (X_\et,n)$ and $\mathbf{B} (X,n)$.
  See \cite{Ramachandran-2016} for more details about secondary Euler
  characteristic and its appearances in nature.
\end{remark}

Under the regulator conjecture, our conjectural vanishing order formula takes
form of the usual Euler characteristic of the equivariant cohomology
$R\Gamma_c (G_\RR, X(\CC), \RR (n))$ or motivic cohomology
$R\Gamma (X_\et, \ZZ^c (n)) [1]$.

\begin{proposition}
  \label{prop:VO(X,n)-assuming-B(X,n)}
  Assuming $\mathbf{L}^c (X_\et, n)$ and $\mathbf{B} (X,n)$, the conjecture
  $\mathbf{VO} (X,n)$ is equivalent to
  \begin{align*}
    \ord_{s=n} \zeta (X,s) & = \chi (R\Gamma_c (G_\RR, X(\CC), \RR (n))
    = \sum_{i \in \ZZ} (-1)^i \dim_\RR H^i_c (X(\CC), \RR (n))^{G_\RR} \\
                           & = -\chi (R\Gamma (X_\et, \ZZ^c (n)))
    = \sum_{i \in \ZZ} (-1)^{i+1} \rk_\ZZ H^i (X_\et, \ZZ^c (n)).
  \end{align*}
  Further, we have
  $$\chi (R\Gamma_\Wc (X, \ZZ(n))) = 0.$$

  \begin{proof}
    Thanks to \cite[Proposition~7.14]{Beshenov-Weil-etale-1}, the Weil-étale
    complex tensored with $\RR$ splits as
    \[ R\Gamma_\Wc (X,\RR (n)) \cong
      \RHom (R\Gamma (X_\et, \ZZ^c (n)), \RR) [-1] \oplus
      R\Gamma_c (G_\RR, X (\CC), \RR (n)) [-1]. \]
    Assuming the conjecture $\mathbf{B} (X,n)$, we also have a quasi-isomorphism
    \[ R\Gamma_c (G_\RR, X (\CC), \RR (n)) [-1] \cong
      \RHom (R\Gamma (X_\et, \ZZ^c (n)), \RR), \]
    so that
    \[ \dim_\RR H^i_\Wc (X,\RR(n)) =
      \dim_\RR H^{i-1}_c (X (\CC), \RR (n))^{G_\RR} +
      \dim_\RR H^{i-2}_c (X (\CC), \RR (n))^{G_\RR}. \]
    Using this, we may rewrite the sum
    \begin{multline*}
      \sum_{i \in \ZZ} (-1)^i \cdot i \cdot \rk_\ZZ H^i_\Wc (X, \ZZ (n)) =
      \sum_{i \in \ZZ} (-1)^i \cdot i \cdot \dim_\RR H^i_\Wc (X, \RR (n)) \\
      = \sum_{i \in \ZZ} (-1)^i \cdot i \cdot
      \dim_\RR H^{i-1}_c (X (\CC), \RR (n))^{G_\RR} +
      \sum_{i \in \ZZ} (-1)^i \cdot i \cdot
      \dim_\RR H^{i-2}_c (X (\CC), \RR (n))^{G_\RR} \\
      = -\sum_{i \in \ZZ} (-1)^i \, \dim_\RR H^{i-1}_c (X (\CC), \RR (n))^{G_\RR}
      = \chi (R\Gamma_c (G_\RR, X (\CC), \RR (n)).
    \end{multline*}
    Similarly,
    \[ \sum_{i \in \ZZ} (-1)^i \cdot i \cdot \rk_\ZZ H^i_\Wc (X, \ZZ (n)) =
      \chi (\RHom (R\Gamma (X_\et, \ZZ^c (n)), \RR) [1]) =
      -\chi (R\Gamma (X_\et, \ZZ^c (n))). \]
    These considerations also show that the usual Euler characteristic of
    $R\Gamma_\Wc (X, \ZZ(n))$ vanishes.
  \end{proof}
\end{proposition}

\begin{remark}
  The conjecture $\mathbf{VO} (X,n)$ is related to a conjecture of Soulé
  \cite[Conjecture~2.2]{Soule-1984-ICM}, which originally reads in terms of
  $K'$-theory
  \[ \ord_{s=n} \zeta (X,s) =
    \sum_{i \in \ZZ} (-1)^{i+1} \, \dim_\QQ K'_i (X)_{(n)}. \]
  Then, as explained in \cite[Remark~43]{Kahn-2005}, this may be rewritten in
  terms of Borel--Moore motivic homology as
  $$\sum_{i \in \ZZ} (-1)^{i+1} \, \dim_\QQ H_i^{BM} (X, \QQ (n)).$$
  In our setting, $R\Gamma (X_\et, \ZZ^c (n))$ plays the role of Borel--Moore
  homology, which explains the formula
  \[ \ord_{s=n} \zeta (X,s) =
    \sum_{i \in \ZZ} (-1)^{i+1} \rk_\ZZ H^i (X_\et, \ZZ^c (n)). \]
\end{remark}

\begin{remark}[{\cite[Proposition~5.13]{Flach-Morin-2018}}]
  \label{rmk:archimedian-euler-factor}
  As for the formula
  \[ \ord_{s=n} \zeta (X,s) =
    \sum_{i \in \ZZ} (-1)^i \dim_\RR H^i_c (X(\CC), \RR (n))^{G_\RR}, \]
  it basically means that the the vanishing order at $s = n < 0$ comes from the
  archimedian $\Gamma$-factor that appears in the (hypothetical) functional
  equation, as explained in \cite[\S\S 3,4]{Serre-1970}
  (see also \cite[\S 4]{Flach-Morin-2020}).

  Namely, assuming that $X_\CC$ is a smooth projective variety, we consider the
  Hodge decomposition
  \[ H^i (X (\CC), \CC) = \bigoplus_{p+q = i} H^{p,q}, \]
  which carries an action of $G_\RR = \{ id, \sigma \}$ such that
  $\sigma (H^{p,q}) = H^{q,p}$. We set $h^{p,q} = \dim_\CC H^{p,q}$.
  For $p = i/2$ consider the eigenspace decomposition
  $H^{p,p} = H^{p,+} \oplus H^{p,-}$, where
  \begin{align*}
    H^{p,+} & = \{ x \in H^{p,p} \mid \sigma (x) = (-1)^p\,x \},\\
    H^{p,-} & = \{ x \in H^{p,p} \mid \sigma (x) = (-1)^{p+1}\,x \},
  \end{align*}
  and set accordingly $h^{p,\pm} = \dim_\CC H^{p,\pm}$.
  It is expected that the completed zeta function
  $$\zeta (\overline{X}, s) = \zeta (X, s)\,\zeta (X_\infty, s),$$
  satisfies a functional equation of the form
  \[ A^{\frac{d-s}{2}}\,\zeta (\overline{X},d-s) =
    A^{\frac{s}{2}}\,\zeta (\overline{X},s). \]
  Here
  \begin{gather*}
    \zeta (X_\infty, s) = \prod_{i\in \ZZ} L_\infty (H^i (X),s)^{(-1)^i}, \\
    L_\infty (H^i (X), s) =
    \prod_{p = i/2} \Gamma_\RR (s - p)^{h^{p,+}}\,\Gamma_\RR (s-p+1)^{h^{p,-}} \,
    \prod_{\substack{p + q = i \\ p < q}} \Gamma_\CC (s - p)^{h^{p,q}}, \\
    \Gamma_\RR (s) = \pi^{-s/2} \, \Gamma (s/2), \quad
    \Gamma_\CC (s) = (2\pi)^{-s} \, \Gamma (s).
  \end{gather*}

  Therefore, the expected vanishing order at $s = n < 0$ is
  \begin{multline*}
    \ord_{s=n} \zeta (X,s) = -\ord_{s=n} \zeta (X_\infty,s)
    = -\sum_{i\in \ZZ} (-1)^i \ord_{s=n} L_\infty (H^i (X), s) \\
    = \sum_{i\in \ZZ} (-1)^i \Bigl(\sum_{p = i/2} h^{p,(-1)^{n-p}} +
    \sum_{\substack{p + q = i \\ p < q}} h^{p,q}\Bigr).
  \end{multline*}
  The last equality comes from the fact that $\Gamma (s)$ has simple poles at
  all $s = n \le 0$. We have
  \begin{multline*}
    \dim_\RR H^i (X (\CC), \RR (n))^{G_\RR} =
    \dim_\RR H^i (X (\CC), \RR)^{\sigma = (-1)^n} =
    \dim_\CC H^i (X (\CC), \CC)^{\sigma = (-1)^n} \\
    = \sum_{p = i/2} h^{p,(-1)^{n-p}} +
    \sum_{\substack{p + q = i \\ p < q}} h^{p,q}.
  \end{multline*}
  Here the terms $h^{p,q}$ with $p < q$ come from $\sigma (H^{p,q}) = H^{q,p}$,
  while $h^{p,(-1)^{n-p}}$ come from the action on $H^{p,p}$.
  We see that our conjectural formula gives the expected vanishing orders.
\end{remark}

Let us see some particular examples when the meromorphic continuation for
$\zeta (X,s)$ is known.

\begin{example}
  \label{example:VO(X,n)-for-number-rings}
  Suppose that $X = \Spec \mathcal{O}_F$ is the spectrum of the ring of integers
  of a number field $F/\QQ$. Let $r_1$ be the number of real embeddings
  $F \hookrightarrow \RR$ and $r_2$ the number of conjugate pairs of complex
  embeddings $F \hookrightarrow \CC$. The space $X (\CC)$ with the action of
  complex conjugation may be pictured as follows:

  \[ \begin{tikzpicture}
    \matrix(m)[matrix of math nodes, row sep=1em, column sep=1em,
    text height=1ex, text depth=0.2ex]{
      ~ & ~ & ~ & ~ & ~ & \bullet & \bullet & \cdots & \bullet \\
      \bullet & \bullet & \cdots & \bullet \\
      ~ & ~ & ~ & ~ & ~ & \bullet & \bullet & \cdots & \bullet \\};

    \draw[->] (m-2-1) edge[loop above,min distance=10mm] (m-2-1);
    \draw[->] (m-2-2) edge[loop above,min distance=10mm] (m-2-2);
    \draw[->] (m-2-4) edge[loop above,min distance=10mm] (m-2-4);

    \draw[->] (m-1-6) edge[bend left] (m-3-6);
    \draw[->] (m-1-7) edge[bend left] (m-3-7);
    \draw[->] (m-1-9) edge[bend left] (m-3-9);

    \draw[->] (m-3-6) edge[bend left] (m-1-6);
    \draw[->] (m-3-7) edge[bend left] (m-1-7);
    \draw[->] (m-3-9) edge[bend left] (m-1-9);

    \draw [decorate,decoration={brace,amplitude=5pt,mirror}] ($(m-3-1)+(-0.5em,-0.5em)$) -- ($(m-3-4)+(0.5em,-0.5em)$);
    \draw [decorate,decoration={brace,amplitude=5pt,mirror}] ($(m-3-6)+(-0.5em,-0.5em)$) -- ($(m-3-9)+(0.5em,-0.5em)$);

    \draw ($(m-3-1)!.5!(m-3-4)$) node[yshift=-2em,anchor=base] {$r_1$ points};
    \draw ($(m-3-6)!.5!(m-3-9)$) node[yshift=-2em,anchor=base] {$2 r_2$ points};
  \end{tikzpicture} \]

  The complex $R\Gamma_c (X (\CC), \RR (n))$ consists of a single $G_\RR$-module
  in degree $0$ given by
  $$\RR (n)^{\oplus r_1} \oplus (\RR (n) \oplus \RR (n))^{\oplus r_2},$$
  with the action of $G_\RR$ on the first summand
  $\RR (n)^{\oplus r_1}$ by complex conjugation and the action on the
  second summand $(\RR (n) \oplus \RR (n))^{\oplus r_2}$ via
  $(x,y) \mapsto (\overline{y}, \overline{x})$. The corresponding real
  space of fixed points has dimension
  \[ \dim_\RR H^0_c (G_\RR, X (\CC), \RR (n)) = \begin{cases}
      r_2, & n \text{ odd},\\
      r_1 + r_2, & n \text{ even},\\
    \end{cases} \]
  which indeed agrees with the vanishing order of the Dedekind zeta function
  $\zeta (X,s) = \zeta_F (s)$ at $s = n < 0$.

  On the side of motivic cohomology
  (see e.g. \cite[Proposition~4.14]{Geisser-2017}), given $n < 0$, the groups
  $H^i_\et (X, \ZZ^c (n)) = H^{i+2}_\et (X, \ZZ (1-n))$ are finite, except for
  $i = -1$, where
  \[ \rk_\ZZ H^{-1}_\et (X, \ZZ^c (n)) =
    \rk_\ZZ H^1_\et (X, \ZZ (1-n)) = \begin{cases}
      r_2, & n \text{ odd},\\
      r_1 + r_2, & n \text{ even}.
    \end{cases} \]
\end{example}

\begin{example}
  Suppose that $X$ is a variety over a finite field $\FF_q$. Then the vanishing
  order conjecture is not very interesting, as the formula gives
  \[ \ord_{s=n} \zeta (X,s) =
    \sum_{i \in \ZZ} (-1)^i \dim_\RR H^i_c (X(\CC), \RR (n))^{G_\RR} =
    \sum_{i \in \ZZ} (-1)^{i+1} \rk_\ZZ H^i (X_\et, \ZZ^c (n)) = 0, \]
  since $X (\CC) = \emptyset$, and also because $\mathbf{L}^c (X_\et, n)$
  implies $\rk_\ZZ H^i (X_\et, \ZZ^c (n)) = 0$ in case of varieties over finite
  fields \cite[Proposition~4.2]{Beshenov-Weil-etale-1}. Therefore, the
  conjecture simply asserts that $\zeta (X,s)$ does not have zeros or poles at
  $s = n < 0$.

  This is indeed the case. We have $\zeta (X,s) = Z (X,q^{-s})$, where
  $$Z (X,t) = \exp \Bigl(\sum_{k\ge 1} \frac{\# X (\FF_{q^k})}{k}\,t^k\Bigr)$$
  is the Hasse--Weil zeta function. By Deligne's work on Weil's conjectures
  \cite{Deligne-Weil-II}, the zeros and poles of $Z (X,s)$ satisfy
  $|s| = q^{-w/2}$, where $0 \le w \le 2 \dim X$ (see
  e.g. \cite[pp.\,26--27]{Katz-1994}). In particular, $q^{-s}$ for $s = n < 0$
  is not a zero or pole of $Z (X,s)$.

  We also note that our definition of $H^i_\Wc (X, \ZZ(n))$, and pretty much all
  the above, makes sense only for $n < 0$. Already for $n = 0$, for instance,
  the zeta function of a smooth projective curve $X/\FF_q$ has a simple pole
  at $s = 0$.
\end{example}

\begin{example}
  Let $X = E$ be an integral model of an elliptic curve over $\QQ$. Then as a
  consequence of the modularity theorem (Wiles--Breuil--Conrad--Diamond--Taylor)
  it is known that $\zeta (E,s)$ admits a meromorphic continuation, which
  satisfies the functional equation with the $\Gamma$-factors discussed in
  \ref{rmk:archimedian-euler-factor}. In this particular case
  $\ord_{s=n} \zeta (E,s) = 0$ for all $n < 0$. This is consistent with the fact
  that $\chi (R\Gamma_c (G_\RR, E (\CC), \RR (n))) = 0$.

  Indeed, the equivariant cohomology groups $H^i_c (E (\CC), \RR (n))^{G_\RR}$
  are the following:
  \begin{center}
    \renewcommand{\arraystretch}{1.5}
    \begin{tabular}{rccc}
      \hline
      & $i = 0$ & $i = 1$ & $i = 2$ \\
      \hline
      $n$ even: & $\RR$ & $\RR$ & $0$ \\
      $n$ odd: & $0$ & $\RR$ & $\RR$ \\
      \hline
    \end{tabular}
  \end{center}
  ---see for instance the calculation in \cite[Lemma~A.6]{Siebel-2019}.
\end{example}

%%%%%%%%%%%%%%%%%%%%%%%%%%%%%%%%%%%%%%%%%%%%%%%%%%%%%%%%%%%%%%%%%%%%%%%%%%%%%%%%

\section{Special value conjecture}
\label{sec:special-value-conjecture}

\begin{definition}
  Define a morphism of complexes
  \[ \smile\theta\colon R\Gamma_\Wc (X,\ZZ(n)) \otimes \RR \to
    R\Gamma_\Wc (X,\ZZ(n)) [1] \otimes \RR \]
  using the splitting
  \[ R\Gamma_\Wc (X, \RR (n)) \cong
    \RHom (R\Gamma (X_\et, \ZZ^c (n)), \RR) [-1] \oplus
    R\Gamma_c (G_\RR, X (\CC), \RR (n)) [-1] \]
  as follows:
  \[ \begin{tikzcd}
      R\Gamma_\Wc (X, \RR(n)) \ar{d}{\cong}\ar[dashed]{r}{\smile\theta} & R\Gamma_\Wc (X, \RR(n)) [1]\ar{d}{\cong} \\
      \RHom (R\Gamma (X_\et, \ZZ^c (n)), \RR) [-1] & \RHom (R\Gamma (X_\et, \ZZ^c (n)), \RR) \\[-2em]
      \oplus & \oplus \\[-2em]
      R\Gamma_c (G_\RR, X (\CC), \RR (n)) [-1]\ar{uur}[description]{Reg_{X,n}^\vee} & R\Gamma_c (G_\RR, X (\CC), \RR (n))
    \end{tikzcd} \]
\end{definition}

\begin{lemma}
  \label{lemma:smile-theta}
  Assuming $\mathbf{L}^c (X_\et,n)$ and $\mathbf{B} (X,n)$, the morphism
  $\smile\theta$ induces a long exact sequence of finite dimensional real vector
  spaces
  \[ \cdots \to H^{i-1}_\Wc (X, \RR (n))
    \xrightarrow{\smile\theta}
    H^i_\Wc (X, \RR (n))
    \xrightarrow{\smile\theta}
    H^{i+1}_\Wc (X, \RR (n)) \to \cdots \]

  \begin{proof}
    We obtain a sequence
    \[ \begin{tikzcd}[column sep=1em]
        \cdots\ar{r} & H^{i-1}_\Wc (X, \RR(n)) \ar{d}{\cong}\ar[dashed]{r}{\smile\theta} & H^i_\Wc (X, \RR(n))\ar{d}{\cong}\ar[dashed]{r}{\smile\theta} & H^{i+1}_\Wc (X, \RR(n))\ar{d}{\cong} \ar{r} & \cdots \\
        & \Hom (H^{-i} (X_\et, \ZZ^c (n)), \RR) & \Hom (H^{-i-1} (X_\et, \ZZ^c (n)), \RR) & \Hom (H^{-i-2} (X_\et, \ZZ^c (n)), \RR) \\[-2em]
        \cdots & \oplus & \oplus & \oplus & \cdots \\[-2em]
        & H_c^{i-2} (G_\RR, X (\CC), \RR (n))\ar{uur}[description]{\cong} & H_c^{i-1} (G_\RR, X (\CC), \RR (n))\ar{uur}[description]{\cong} & H_c^i (G_\RR, X (\CC), \RR (n))
      \end{tikzcd} \]
    Here the diagonal arrows are isomorphisms according to $\mathbf{B} (X,n)$,
    so the sequence is exact.
  \end{proof}
\end{lemma}

We recall that the Weil-étale complex $R\Gamma_\Wc (X, \ZZ(n))$ was defined in
\cite{Beshenov-Weil-etale-1} up to a \emph{non-unique} isomorphism in the
derived category $\mathbf{D} (\ZZ)$ via a distinguished triangle
\begin{equation}
  \label{eqn:triangle-defining-RGamma-Wc}
  R\Gamma_\Wc (X, \ZZ(n)) \to R\Gamma_\fg (X,\ZZ(n)) \xrightarrow{i_\infty}
  R\Gamma_c (G_\RR, X (\CC), \ZZ (n)) \to R\Gamma_\Wc (X, \ZZ (n)) [1]
\end{equation}

This is quite unpleasant, and there ought to be a better, more canonical
construction of $R\Gamma_\Wc (X, \ZZ(n))$. However, this is not a big issue for
the moment, since the special value conjecture will be formulated not in terms
of $R\Gamma_\Wc (X, \ZZ (n))$, but in terms of its determinant
$\det_\ZZ R\Gamma_\Wc (X, \ZZ(n))$ (see appendix~\ref{app:determinants}), which
is well-defined.

\begin{lemma}
  \label{lemma:determinant-of-RGamma-Wc-well-defined}
  The determinant $\det_\ZZ R\Gamma_\Wc (X, \ZZ(n))$ is well-defined up to a
  canonical isomorphism.

  \begin{proof}
    Two different choices of the mapping fiber in
    \eqref{eqn:triangle-defining-RGamma-Wc} give us an isomorphism of
    distinguished triangles
    \[ \begin{tikzcd}
        R\Gamma_\Wc (X, \ZZ(n)) \ar{r}\ar{d}{f}[swap]{\cong} & R\Gamma_\fg (X,\ZZ(n)) \ar{r}{i_\infty}\ar{d}{id} & R\Gamma_c (G_\RR, X (\CC), \ZZ (n)) \ar{r}\ar{d}{id} & R\Gamma_\Wc (X, \ZZ (n)) [1]\ar{d}{f}[swap]{\cong} \\
        R\Gamma_\Wc (X, \ZZ(n))' \ar{r} & R\Gamma_\fg (X,\ZZ(n)) \ar{r}{i_\infty} & R\Gamma_c (G_\RR, X (\CC), \ZZ (n)) \ar{r} & R\Gamma_\Wc (X, \ZZ (n))' [1]
      \end{tikzcd} \]
    Now the idea is to use the functoriality of determinants with respect to
    isomorphisms of distinguished triangles
    (see \ref{prop:det-and-isos-of-triangles}). The only technical issue is that
    whenever $X (\RR) \ne \emptyset$, the complexes $R\Gamma_\fg (X,\ZZ(n))$ and
    $R\Gamma_c (G_\RR, X (\CC), \ZZ (n))$ are not perfect, but possibly have
    finite $2$-torsion in $H^i (-)$ for arbitrarily big $i$ (in
    \cite{Beshenov-Weil-etale-1} we called such complexes
    \textbf{almost perfect}). On the other hand, the determinants are only
    defined for perfect complexes. Luckily, $H^i (i_\infty^*)$ is an isomorphism
    for $i \gg 0$, so that for $m$ big enough we may take the corresponding
    canonical truncations $\tau_{\le m}$:
    \[ \begin{tikzcd}[row sep=1.5em, column sep=1em]
        \tau_{\le m} R\Gamma_\Wc (X, \ZZ(n)) \ar{r}\ar{d}{\cong} & \tau_{\le m} R\Gamma_\fg (X,\ZZ(n)) \ar{r}\ar{d} & \tau_{\le m} R\Gamma_c (G_\RR, X (\CC), \ZZ (n)) \ar{r}\ar{d} & \tau_{\le m} R\Gamma_\Wc (X, \ZZ (n)) [1]\ar{d}{\cong} \\
        R\Gamma_\Wc (X, \ZZ(n)) \ar{r}\ar{d} & R\Gamma_\fg (X,\ZZ(n)) \ar{r}{i_\infty}\ar{d} & R\Gamma_c (G_\RR, X (\CC), \ZZ (n)) \ar{r}\ar{d} & R\Gamma_\Wc (X, \ZZ (n)) [1]\ar{d} \\
        0 \ar{r}\ar{d} & \tau_{\ge m+1} R\Gamma_\fg (X,\ZZ(n)) \ar{r}{\cong}\ar{d} & \tau_{\ge m+1} R\Gamma_c (G_\RR, X (\CC), \ZZ (n)) \ar{r}\ar{d} & 0\ar{d} \\
        \tau_{\le m} R\Gamma_\Wc (X, \ZZ(n)) [1] \ar{r} & \tau_{\le m} R\Gamma_\fg (X,\ZZ(n)) [1] \ar{r} & \tau_{\le m} R\Gamma_c (G_\RR, X (\CC), \ZZ (n)) [1] \ar{r} & \tau_{\le m} R\Gamma_\Wc (X, \ZZ (n)) [2]
      \end{tikzcd} \]

    The truncations give us (rotating the triangles for convenience)
    \[ \begin{tikzcd}[column sep=1em]
        \tau_{\le m} R\Gamma_c (G_\RR, X (\CC), \ZZ (n))[-1] \ar{r}\ar{d}{id} & R\Gamma_\Wc (X, \ZZ (n))\ar{d}{f}[swap]{\cong}\ar{r} & \tau_{\le m} R\Gamma_\fg (X,\ZZ(n)) \ar{d}{id}\ar{r} & \tau_{\le m} R\Gamma_c (G_\RR, X (\CC), \ZZ (n)) \ar{d}{id} \\
        \tau_{\le m} R\Gamma_c (G_\RR, X (\CC), \ZZ (n))[-1] \ar{r} & R\Gamma_\Wc (X, \ZZ (n))' \ar{r} & \tau_{\le m} R\Gamma_\fg (X,\ZZ(n)) \ar{r} & \tau_{\le m} R\Gamma_c (G_\RR, X (\CC), \ZZ (n))
      \end{tikzcd} \]
    Now according to \ref{prop:det-and-isos-of-triangles}, we have a commutative
    diagram
    \[ \begin{tikzcd}
        \det_\ZZ \tau_{\le m} R\Gamma_c (G_\RR, X (\CC), \ZZ (n))[-1] \otimes_\ZZ \det_\ZZ \tau_{\le m} R\Gamma_\fg (X,\ZZ(n)) \ar{r}{i}[swap]{\cong} \ar{d}{id} & \det_\ZZ R\Gamma_\Wc (X,\ZZ(n)) \ar{d}{\det_\ZZ (f)}[swap]{\cong} \\
        \det_\ZZ \tau_{\le m} R\Gamma_c (G_\RR, X (\CC), \ZZ (n))[-1] \otimes_\ZZ \det_\ZZ \tau_{\le m} R\Gamma_\fg (X, \ZZ (n)) \ar{r}{i'}[swap]{\cong} & \det_\ZZ R\Gamma_\Wc (X,\ZZ(n))
      \end{tikzcd} \]
    so that $\det_\ZZ (f) = i'\circ i^{-1}$.
  \end{proof}
\end{lemma}

\begin{lemma}
  The non-canonical splitting
  \[ R\Gamma_\Wc (X, \QQ(n)) \cong
    \RHom (R\Gamma (X_\et, \ZZ^c (n)), \QQ) [-1] \oplus
    R\Gamma_c (G_\RR, X (\CC), \QQ (n)) [-1]. \]
  Gives a canonical isomorphism of determinants
  \[ \det_\QQ R\Gamma_\Wc (X, \QQ(n)) \cong
    \det_\QQ \RHom (R\Gamma (X_\et, \ZZ^c (n)), \QQ) [-1] \otimes_\QQ
    \det_\QQ R\Gamma_c (G_\RR, X (\CC), \QQ (n)) [-1] \]

  \begin{proof}
    This is similar to the previous lemma; in fact, after tensoring with $\QQ$,
    we obtain perfect complexes of $\QQ$-vector spaces, so that the truncations
    are not needed anymore.

    We recall from \cite[Proposition~7.4]{Beshenov-Weil-etale-1} that
    $i_\infty^* \otimes \QQ = 0$, and thanks to this there is an isomorphism of
    triangles
    \begin{equation}
      \label{eqn:splitting-of-RGamma-Wc-triangles}
      \begin{tikzcd}[row sep=1em]
        R\Gamma_c (G_\RR, X (\CC), \QQ (n)) [-1] \ar{r}{id}\ar{d} & R\Gamma_c (G_\RR, X (\CC), \QQ (n)) [-1]\ar{d} \\
        R\Gamma_\Wc (X, \QQ(n)) \ar[dashed]{r}{f}[swap]{\cong}\ar{d} & \begin{array}{c} \RHom (R\Gamma (X_\et, \ZZ^c (n)), \QQ) [-1] \\ \oplus \\ R\Gamma_c (G_\RR, X (\CC), \QQ (n)) [-1] \end{array}\ar{d} \\
        R\Gamma_\fg (X, \QQ(n)) \ar{r}{g\otimes \QQ}[swap]{\cong}\ar{d} & \RHom (R\Gamma (X_\et, \ZZ^c (n)), \QQ) [-1] \ar{d} \\
        R\Gamma_c (G_\RR, X (\CC), \QQ (n)) \ar{r}{id} & R\Gamma_c (G_\RR, X (\CC), \QQ (n))
      \end{tikzcd}
    \end{equation}
    Here the third horizontal arrow comes from the triangle defining
    $R\Gamma_\fg (X, \ZZ(n))$
    \[ \RHom (R\Gamma (X_\et, \ZZ^c (n)), \QQ) [-2] \xrightarrow{\alpha_{X,n}}
      R\Gamma_c (X_\et, \ZZ (n)) \to
      R\Gamma_\fg (X, \ZZ(n)) \xrightarrow{g}
      \RHom (R\Gamma (X_\et, \ZZ^c (n)), \QQ) [-1] \]
    tensored with $\QQ$ (see \cite[\S 5]{Beshenov-Weil-etale-1}).
    The distinguished column on the right of
    \eqref{eqn:splitting-of-RGamma-Wc-triangles} is the direct sum
    \[ \begin{tikzcd}[row sep=1.5em]
        R\Gamma_c (G_\RR, X (\CC), \QQ (n)) [-1]\ar{d}{id} &[-2em] &[-2em] 0\ar{d} \\
        R\Gamma_c (G_\RR, X (\CC), \QQ (n)) [-1]\ar{d} & \oplus & \RHom (R\Gamma (X_\et, \ZZ^c (n)), \QQ) [-1]\ar{d}{id} \\
        0 \ar{d} & & \RHom (R\Gamma (X_\et, \ZZ^c (n)), \QQ) [-1] \ar{d} \\
        R\Gamma_c (G_\RR, X (\CC), \QQ (n)) & & 0
      \end{tikzcd} \]
    The splitting isomorphism $f$ in
    \eqref{eqn:splitting-of-RGamma-Wc-triangles} is by no means
    canonical. However, after taking the determinants, we obtain a commutative
    diagram (see \ref{prop:det-and-isos-of-triangles})
    \[ \begin{tikzcd}
        \begin{array}{c} \det_\QQ R\Gamma_c (G_\RR, X (\CC), \QQ (n)) [-1] \\ \otimes_\QQ \\ \det_\QQ R\Gamma_\fg (X, \QQ(n)) \end{array} \ar{r}{i}[swap]{\cong} \ar{d}{id \otimes \det_\ZZ (g\otimes \QQ)}[swap]{\cong} & \det_\QQ R\Gamma_\Wc (X, \QQ(n)) \ar{d}{\det_\ZZ (f)}[swap]{\cong}\ar[dashed]{dl}{\cong} \\
        \begin{array}{c} \det_\QQ R\Gamma_c (G_\RR, X (\CC), \QQ (n)) [-1] \\ \otimes_\QQ \\ \det_\QQ \RHom (R\Gamma (X_\et, \ZZ^c (n)), \QQ) [-1] \end{array} \ar{r}{i'}[swap]{\cong} & \det_\QQ \left(\begin{array}{c} \RHom (R\Gamma (X_\et, \ZZ^c (n)), \QQ) [-1] \\ \oplus \\ R\Gamma_c (G_\RR, X (\CC), \QQ (n)) [-1] \end{array}\right)
      \end{tikzcd} \]
    Here the dashed diagonal arrow is the desired canonical isomorphism.
  \end{proof}
\end{lemma}

\begin{definition}
  Given an arithmetic scheme $X$ and $n < 0$, assume the conjectures
  $\mathbf{L}^c (X_\et, n)$ and $\mathbf{B} (X,n)$. Consider the
  quasi-isomorphism
  \begin{equation}
    \label{eqn:definition-of-lambda}
    \begin{tikzcd}[column sep=4em]
      \begin{array}{c} R\Gamma_c (G_\RR, X (\CC), \RR (n)) [-2] \\ \oplus \\ R\Gamma_c (G_\RR, X (\CC), \RR (n)) [-1] \end{array} \ar{r}{Reg_{X,n}^\vee [-1] \oplus id}[swap]{\cong} & \begin{array}{c} \RHom (R\Gamma (X_\et, \ZZ^c (n)), \RR) [-1] \\ \oplus \\ R\Gamma_c (G_\RR, X (\CC), \RR (n)) [-1] \end{array} \ar{r}{\text{split}}[swap]{\cong} &[-2em] R\Gamma_\Wc (X, \RR (n))
    \end{tikzcd}
  \end{equation}
  Note that the first complex has determinant
  \[ \det_\RR \left(\begin{array}{c} R\Gamma_c (G_\RR, X (\CC), \RR (n)) [-2] \\ \oplus \\ R\Gamma_c (G_\RR, X (\CC), \RR (n)) [-1] \end{array}\right) \cong
    \begin{array}{c} \det_\RR R\Gamma_c (G_\RR, X (\CC), \RR (n)) \\ \otimes_\RR \\ (\det_\RR R\Gamma_c (G_\RR, X (\CC), \RR (n)))^{-1} \end{array} \cong \RR, \]
  and for the last complex, by compatibility with base change, we have a
  canonical isomorphism
  \[ \det_\RR R\Gamma_\Wc (X, \RR (n)) \cong
    (\det_\ZZ R\Gamma_\Wc (X, \ZZ (n))) \otimes_\ZZ \RR. \]
  Therefore, after taking determinants, the quasi-isomorphism
  \eqref{eqn:definition-of-lambda} induces a canonical isomorphism
  \begin{equation}
    \label{eqn:morphism-lambda}
    \lambda = \lambda_{X,n}\colon \RR \xrightarrow{\cong}
    (\det_\ZZ R\Gamma_\Wc (X, \ZZ (n))) \otimes_\ZZ \RR.
  \end{equation}
\end{definition}

\begin{remark}
  An equivalent way to define $\lambda$ is
  \begin{multline*}
    \lambda\colon \RR \xrightarrow{\cong}
    \bigotimes_{i\in \ZZ} (\det_\RR H^i_\Wc (X, \RR (n)))^{(-1)^i} \xrightarrow{\cong}
    \Bigl(\bigotimes_{i\in \ZZ} (\det_\ZZ H^i_\Wc (X, \ZZ (n)))^{(-1)^i}\Bigr) \otimes_\ZZ \RR \\
    \xrightarrow{\cong} (\det_\ZZ R\Gamma_\Wc (X, \ZZ (n))) \otimes_\ZZ \RR.
  \end{multline*}
  Here the first isomorphism comes from lemma~\ref{lemma:smile-theta}.
\end{remark}

We are ready to state the main conjecture of this paper. The determinant
$\det_\ZZ R\Gamma_\Wc (X, \ZZ (n)))$ is a free $\ZZ$-module of rank $1$ (which
does not sound very interesting), but the whole point of the determinant
business is that the isomorphism \eqref{eqn:morphism-lambda} embeds it
canonically into $\RR$. We conjecture that this embedding gives the special
value of $\zeta (X,s)$ at $s = n$ in the following sense.

\begin{conjecture}
  $\mathbf{C} (X,n)$: let $X$ be an arithmetic scheme and $n < 0$ a strictly
  negative integer. Assuming $\mathbf{L}^c (X_\et, n)$, $\mathbf{B} (X,n)$, and
  meromorphic continuation of $\zeta (X,s)$ around $s = n < 0$, the
  corresponding special value is determined up to sign by
  \[ \lambda (\zeta^* (X,n)^{-1}) \cdot \ZZ =
    \det_\ZZ R\Gamma_\Wc (X, \ZZ (n)), \]
  where $\lambda$ is the canonical isomorphism \eqref{eqn:morphism-lambda}.
\end{conjecture}

\begin{remark}
  This conjecture is similar to \cite[Conjecture~5.12]{Flach-Morin-2018}.
  When $X$ is proper and regular, the above conjecture is the same as the
  special value conjecture of Flach and Morin, which for $n \in \ZZ$ reads
  \begin{equation}
    \label{eqn:FM-special-value}
    \lambda_\infty \Bigl(\zeta^* (X,n)^{-1} \cdot C (X,n) \cdot \ZZ\Bigr) =
    \Delta (X/\ZZ, n).
  \end{equation}
  Here the fundamental line $\Delta (X/\ZZ,n)$ is defined via
  \[ \Delta (X/\ZZ,n) \dfn
    \det_\ZZ R\Gamma_\Wc (X, \ZZ(n)) \otimes_\ZZ
    \det_\ZZ R\Gamma_\dR (X/\ZZ)/\Fil^n. \]
  If $n < 0$, then
  $\Delta (X/\ZZ,n) = \det_\ZZ R\Gamma_\Wc (X, \ZZ(n))$. Moreover, $C (X,n)$ is
  some rational number, defined via $\prod_p |c_p (X,n)|_p$.
  Here $c_p (X,n) \in \QQ_p^\times/\ZZ_p^\times$ are the ``local factors''
  described in \cite[\S 5.4]{Flach-Morin-2018}, but
  \cite[Proposition~5.8]{Flach-Morin-2018} says that if $n \le 0$, then
  $c_p (X,n) \equiv 1 \pmod{\ZZ_p^\times}$ for all $p$. Therefore, $C (X,n) = 1$
  in our situation. Finally, the trivialization isomorphism ``$\lambda_\infty$''
  is defined precisely the same way as our $\lambda$. Therefore,
  \eqref{eqn:FM-special-value} coincides with our conjecture $\mathbf{C} (X,n)$
  for $n < 0$.

  Flach and Morin prove that their conjecture is consistent with the Tamagawa
  number conjecture of Bloch--Kato--Fontaine--Perrin-Riou
  \cite{Fontaine-Perrin-Riou-1994}; see \cite[\S 5.6]{Flach-Morin-2018} for the
  details.
\end{remark}

\begin{remark}
  Some canonical isomorphisms of determinants involve multiplication by $\pm 1$,
  so there is no surprise that the resulting conjecture is stated up to sign
  $\pm 1$. However, this is not a big issue, since the sign may be recovered
  from the (conjectural) functional equation.
\end{remark}

%%%%%%%%%%%%%%%%%%%%%%%%%%%%%%%%%%%%%%%%%%%%%%%%%%%%%%%%%%%%%%%%%%%%%%%%%%%%%%%%

\section{Case of varieties over finite fields}
\label{sec:finite-fields}

For varieties over finite fields, our special value conjecture corresponds to
the conjectures studied by Geisser in \cite{Geisser-2004}, \cite{Geisser-2006},
\cite{Geisser-2010-arithmetic-homology}.

\begin{proposition}
  \label{prop:C(X,n)-over-finite-fields}
  If $X/\FF_q$ is a variety over a finite field, then assuming
  $\mathbf{L}^c (X_\et,n)$, the special value conjecture
  $\mathbf{C} (X,n)$ is equivalent to
  \begin{equation}
    \label{eqn:special-value-for-X/Fq}
    \zeta^* (X,n) = \pm \prod_{i \in \ZZ} |H_\Wc^i (X, \ZZ(n))|^{(-1)^i}
    = \pm \prod_{i \in \ZZ} |H^i (X_\et, \ZZ^c (n))|^{(-1)^i}
    = \pm \prod_{i \in \ZZ} |H_i^c (X_\ar, \ZZ (n))|^{(-1)^{i+1}},
  \end{equation}
  where $H_i^c (X_\ar, \ZZ (n))$ are Geisser's arithmetic homology groups
  defined in \cite{Geisser-2010-arithmetic-homology}.

  \begin{proof}
    Assuming $\mathbf{L}^c (X_\et,n)$, thanks to
    \cite[Proposition~7.7]{Beshenov-Weil-etale-1} we have
    \[ H^i_\Wc (X, \ZZ (n)) \cong
      \Hom (H^{2-i} (X_\et, \ZZ^c (n)), \QQ/\ZZ) \cong
      \Hom (H_{i-1}^c (X_\ar, \ZZ (n)), \QQ/\ZZ). \]
    The involved cohomology groups are finite and vanish for $|i| \gg 0$
    (see \cite[Proposition~4.2]{Beshenov-Weil-etale-1}), and therefore by
    \ref{lemma:determinant-for-torsion-cohomology}, the determinant is given by
    \[ \begin{tikzcd}[row sep=0.75em, column sep=0pt]
        \det_\ZZ R\Gamma_\Wc (X, \ZZ (n)) \ar[equals]{d} & \subset & \det_\ZZ R\Gamma_\Wc (X, \ZZ (n)) \otimes_\ZZ \QQ \ar[equals]{d} \\
        \frac{1}{m}\ZZ & \subset & \QQ
      \end{tikzcd} \]
    where
    \[ m = \prod_{i \in \ZZ} |H_\Wc^i (X, \ZZ(n))|^{(-1)^i}. \qedhere \]
  \end{proof}
\end{proposition}

\begin{remark}
  Formulas similar to \eqref{eqn:special-value-for-X/Fq} were suggested long
  time ago by Lichtenbaum in \cite{Lichtenbaum-1984}.
\end{remark}

\begin{theorem}
  \label{thm:C(X,n)-over-finite-fields}
  Let $X/\FF_q$ be a variety over a finite field satisfying
  $\mathbf{L}^c (X_\et, n)$ for $n < 0$. Then the conjecture $\mathbf{C} (X,n)$
  holds.
\end{theorem}

We note that \eqref{prop:C(X,n)-over-finite-fields} is equivalent to the special
value formula that appears in
\cite[Theorem~4.5]{Geisser-2010-arithmetic-homology}. The conjecture
$\mathbf{P}_0 (X)$ in the statement of
\cite[Theorem~4.5]{Geisser-2010-arithmetic-homology} is implied by our
conjecture $\mathbf{L}^c (X_\et,n)$ thanks to
\cite[Proposition~4.1]{Geisser-2010-arithmetic-homology}. Geisser's proof
eventually reduces to \cite{Milne-1986}, but for our case of $s = n < 0$, the
situation is easier, and we can give a more direct explanation, using earlier
results from \cite{Bayer-Neukirch-1978} regarding Grothendieck's trace formula.

\begin{proof}
  In this case the conjecture reduces to
  $$\zeta (X,n) = \prod_{i \in \ZZ} |H^i (X_\et, \ZZ^c (n))|^{(-1)^i}.$$

  By duality \cite[Theorem~I]{Beshenov-Weil-etale-1}, we have
  $$|H^{2-i} (X_\et, \ZZ^c (n))| = |H^i_c (X_\et, \ZZ (n))|,$$
  where
  \[ \ZZ (n) \dfn
    \bigoplus_{\ell \ne p} \QQ_\ell/\ZZ_\ell (n) [-1] \dfn
    \bigoplus_{\ell \ne p} \mu_{\ell^\infty}^{\otimes n} [-1] \dfn
    \bigoplus_{\ell \ne p} \varinjlim_r \mu_{\ell^r}^{\otimes n} [-1]. \]
  Now $H^i_c (X_\et, \QQ_\ell (n)) = 0$ for $n < 0$, and therefore
  $H^i_c (X_\et, \ZZ_\ell (n)) \cong
  H^{i-1}_c (X_\et, \QQ_\ell/\ZZ_\ell (n))$.
  This means that our conjectural formula may be written as
  \[ \zeta (X,n) =
    \prod_{\ell \ne p} \prod_{i \in \ZZ} |H^i_c (X_\et, \ZZ_\ell (n))|^{(-1)^i}. \]

  Recall that Grothendieck's trace formula
  (see \cite{Grothendieck-FL} or \cite[Rapport]{SGA4-1-2}) reads
  \[ Z (X,t) =
    \prod_{i \in \ZZ} \det \bigl(1 - tF \bigm| H^i_c (\overline{X}, \QQ_\ell)\bigr)^{(-1)^{i+1}}. \]
  Here $\overline{X} \dfn X \times_{\Spec \FF_q} \overline{\FF}_q$, and $F$
  is the Frobenius acting on $H^i_c (\overline{X}, \QQ_\ell)$. Substituting
  $t = q^{-n}$,
  \[ \zeta (X,n) =
    \prod_{i \in \ZZ} \det \bigl(1 - q^{-n} F \bigm| H^i_c (\overline{X}, \QQ_\ell)\bigr)^{(-1)^{i+1}}. \]
  Then following the proof of \cite[Theorem~(3.1)]{Bayer-Neukirch-1978}, we
  obtain for each $\ell \ne p$
  \begin{equation}
    \label{eqn:bayer-neukirch}
    |\zeta (X,n)|_\ell =
    \prod_{i \in \ZZ} |H^i_c (X, \ZZ_\ell (n))|^{(-1)^{i+1}}.
  \end{equation}

  On the other hand, for $n < 0$ we have $|\zeta (X,n)|_p = 1$. This fact may
  be justified, without assuming that $X$ is smooth or projective,
  for instance, using the trace formula for rigid cohomology
  \cite[p.\,1446]{Kedlaya-2006}, which gives
  $Z (X,t) = \prod_i P_i (t)^{(-1)^{i+1}}$, where $P_i (t) \in \ZZ[t]$ and
  $P_i (0) = 1$. In particular, $P_i (q^{-n}) \equiv 1 \pmod{p}$.
  We conclude that the product formula gives from \eqref{eqn:bayer-neukirch}
  \[ \zeta (X,n) =
    \prod_{\ell \ne p} \prod_{i \in \ZZ} |H^i_c (X, \ZZ_\ell (n))|^{(-1)^i}, \]
  which is precisely our special value formula.
\end{proof}

\begin{remark}
  The fact that $|\zeta (X,n)|_p = 1$ observed in the above argument explains
  why Weil-étale cohomology ignores the $p$-primary part in a sense.
\end{remark}

\iffalse
To deal with singular varieties, we recall the following strong conjecture on
resolution of singularities.

\begin{conjecture}
  $\mathbf{R} (k,d)$. For a field $k$ and $d \in \NN$, for varieties $X/k$ of
  dimension $\le d$ the following conditions hold.
  \begin{itemize}
  \item For any integral variety $X/k$ of dimension $\le d$ there is a proper,
    birational map $f\colon Y \to X$ with $Y$ smooth.

  \item For every smooth variety $Y/k$ of dimension $\le d$ and every proper
    birational map $f\colon Y\to X$, there is a sequence of blowups along smooth
    centers
    $X_n \to X_{n-1} \to \cdots \to X_1 \to X$
    such that the composition $X_n \to X$ factors through $f$.
  \end{itemize}
\end{conjecture}

\begin{theorem}
  \label{thm:C(X,n)-over-finite-fields}
  Let $X$ be a smooth projective variety $X/\FF_q$. Assuming
  $\mathbf{L}^c (X_\et,n)$, the special value conjecture
  $\mathbf{C} (X,n)$ holds.

  Moreover, assuming the resolution of singularities $\mathbf{R} (\FF_q,d)$ and
  $\mathbf{L}^c (X_\et,n)$ for any smooth projective variety $X/\FF_q$ of
  dimension $\le d$, the conjecture $\mathbf{C} (X,n)$ holds for any variety
  $X/\FF_q$ of dimension $\le d$.

  \begin{proof}
    Proposition \ref{prop:C(X,n)-over-finite-fields} gives
    \[ \zeta^* (X,n) =
      \pm \prod_{i \in \ZZ} |H_i^c (X_\ar, \ZZ (n))|^{(-1)^{i+1}}. \]

    Under $\mathbf{L}^c (X_\et,n)$, the groups
    $H_i^c (X_\ar, \ZZ(n)) \cong H^{1-i} (X_\et, \ZZ^c (n))$, are finite, hence
    the conjecture $\mathbf{P}_0 (X)$ from
    \cite[\S 4]{Geisser-2010-arithmetic-homology} holds, using
    \cite[Proposition~4.1]{Geisser-2010-arithmetic-homology}.  Then the
    statement is precisely \cite[Theorem~4.5]{Geisser-2010-arithmetic-homology}.
  \end{proof}
\end{theorem}

\begin{remark}
  It is probably worth noting that Geisser's proof of the special value
  conjecture is via reduction to Milne's work \cite{Milne-1986}.

  To generalize to all varieties of dimension $\le d$ under assumption of
  $\mathbf{R} (\FF_q,d)$, one uses the dévissage lemma
  \cite[Lemma~2.7]{Geisser-2006} and compatibility of the special value
  conjecture with closed-open decompositions of schemes. In the next section
  this will be verified in full generality (for any arithmetic scheme $X$, not
  necessarily $X/\FF_q$) for the conjecture $\mathbf{C} (X,n)$.
\end{remark}
\fi

Let us consider a couple of particular examples to see how the special value
conjecture works. It is to be noted that for a general arithmetic scheme $X$,
calculating the motivic cohomology $H^i (X_\et, \ZZ^c (n))$ (and therefore our
Weil-étale cohomology $H^i_\Wc (X, \ZZ(n))$) is by no means a trivial task. The
finite generation of $H^i (X_\et, \ZZ^c (n))$ is only known for particular cases
(see \cite[\S 8]{Beshenov-Weil-etale-1}), and calculating the torsion part,
which bears the arithmetic information, is even more difficult. Similarly, an
explicit calculation of the regulator map $Reg_{X,n}$ is highly
nontrivial. Therefore, for the moment we give some toy examples over finite
fields.

\begin{example}
  \label{example:C(X,n)-for-Spec-Fq}
  If $X = \Spec \FF_q$, then $\zeta (X,s) = \frac{1}{1 - q^{-s}}$. In this case
  for $n < 0$ we obtain
  \begin{equation}
    \label{eqn:motivic-cohomology-of-Fq}
    H^i (\Spec \FF_{q,\et}, \ZZ^c (n)) = H^i (\Spec \FF_{q,\et}, \ZZ (-n)) =
    \begin{cases}
      \ZZ/(q^{-n} - 1), & i = 1, \\
      0, & i \ne 1
    \end{cases}
  \end{equation}
  (see for instance \cite[Example~4.2]{Geisser-2017}).
  Therefore, the formula \eqref{eqn:special-value-for-X/Fq} indeed recovers
  $\zeta (X,n)$ up to sign.

  Similarly, replacing $\Spec \FF_q$ with $\Spec \FF_{q^m}$, viewed as a variety
  over $\FF_q$, we have $\zeta (\Spec \FF_{q^m},s) = \zeta (\Spec \FF_q, ms)$,
  and \eqref{eqn:motivic-cohomology-of-Fq} also changes accordingly.
\end{example}

\begin{example}
  Consider $X = \PP^1_{\FF_q}/(0\sim 1)$, or equivalently, a nodal cubic.
  The zeta function is $\zeta (X,s) = \frac{1}{1 - q^{1-s}}$.  We may calculate
  the groups $H^i (X_\et, \ZZ^c (n))$ using the blowup square
  \[ \begin{tikzcd}
      \Spec \FF_q \sqcup \Spec \FF_q \ar{r}\ar{d}\tikzpb & \PP^1_{\FF_q} \ar{d} \\
      \Spec \FF_q \ar{r} & X
    \end{tikzcd} \]
  This is similar to \cite[\S 8, Example~2]{Geisser-2006}. Geisser uses
  eh-topology and long exact sequences associated to abstract blowup squares
  \cite[Proposition~3.2]{Geisser-2006}. In our case, the same works, since
  according to \cite[Theorem~I]{Beshenov-Weil-etale-1}, one has
  $H^i (X_\et, \ZZ^c (n)) \cong \Hom (H^{2-i}_c (X_\et, \ZZ (n)),\QQ/\ZZ)$,
  where $\ZZ (n) = \varinjlim_{p\nmid m} \mu_m^{\otimes n} [-1]$, and
  for such sheaves étale cohomology and eh-cohomology agree by
  \cite[Theorem~3.6]{Geisser-2006}.

  Using the projective bundle formula, we calculate from
  \eqref{eqn:motivic-cohomology-of-Fq}
  \[ H^i (\PP^1_{\FF_q,\et}, \ZZ^c (n)) = \begin{cases}
      \ZZ/(q^{1-n} - 1), & i = -1, \\
      \ZZ/(q^{-n} - 1), & i = +1, \\
      0, & i \ne \pm 1.
    \end{cases} \]
  Following the same argument from \cite[\S 8, Example~2]{Geisser-2006},
  the short exact sequences
  \[ 0 \to H^i (\PP^1_{\FF_q,\et}, \ZZ^c (n)) \to
    H^i (X_\et, \ZZ^c (n)) \to
    H^{i+1} ((\Spec \FF_q)_\et, \ZZ^c (n)) \to 0 \]
  give us
  \[ H^i (X_\et, \ZZ^c (n)) = \begin{cases}
      \ZZ/(q^{1-n} - 1), & i = -1, \\
      \ZZ/(q^{-n} - 1), & i = 0,1, \\
      0, & \text{otherwise}.
    \end{cases} \]
  The formula \eqref{eqn:special-value-for-X/Fq} recovers the correct value
  $\zeta (X,n)$.
\end{example}

\begin{example}
  In general, if $X/\FF_q$ is a curve, then the conjecture
  $\mathbf{L}^c (X_\et,n)$ holds; see for instance
  \cite[Proposition~4.3]{Geisser-2017}. The cohomology $H^i (X_\et, \ZZ^c(n))$
  is concentrated in degrees $-1, 0, +1$ by duality and the reasons of
  cohomological dimension, and the special value formula reads
  \[ \zeta^* (X,n) =
    \pm \frac{|H^0 (X_\et, \ZZ^c (n))|}{|H^{-1} (X_\et, \ZZ^c (n))|\cdot |H^1 (X_\et, \ZZ^c (n))|}. \]
\end{example}

%%%%%%%%%%%%%%%%%%%%%%%%%%%%%%%%%%%%%%%%%%%%%%%%%%%%%%%%%%%%%%%%%%%%%%%%%%%%%%%%

\section{Compatibility with operations on schemes}
\label{sec:compatibility-with-operations}

From the definition of $\zeta (X,s)$, the following basic properties follow
easily.

\begin{enumerate}
\item[1)] \textbf{Disjoint unions}: if $X = \coprod_{1 \le i \le r} X_i$ is a
  finite disjoint union of arithmetic schemes, then
  \begin{equation}
    \label{eqn:zeta-function-for-disjoint-unions}
    \zeta (X,s) = \prod_{1 \le i \le r} \zeta (X_i,s).
  \end{equation}
  In particular,
  \begin{align*}
    \ord_{s=n} \zeta (X,s) & = \sum_{1 \le i \le r} \ord_{s=n} \zeta (X_i,s), \\
    \zeta^* (X,n) & = \prod_{1 \le i \le r} \zeta^* (X_i,n).
  \end{align*}

\item[2)] \textbf{Closed-open decompositions}: if $Z \subset X$ is a closed
  subscheme and $U = X\setminus Z$ is its open complement, then we will say that
  we have a \textbf{closed-open decomposition} and write
  $Z \not\hookrightarrow X \hookleftarrow U$. In this case
  \begin{equation}
    \label{eqn:zeta-function-for-closed-open-decompositions}
    \zeta (X,s) = \zeta (Z,s) \cdot \zeta (U,s).
  \end{equation}
  In particular,
  \begin{align*}
    \ord_{s=n} \zeta (X,s) & = \ord_{s=n} \zeta (Z,s) + \ord_{s=n} \zeta (U,s), \\
    \zeta^* (X,n) & = \zeta^* (Z,n) \cdot \zeta^* (U,n).
  \end{align*}

\item[3)] \textbf{Affine bundles}: for any $r \ge 0$ the zeta function of the
  relative affine space $\AA^r_X = \AA^r_\ZZ \times X$ satisfies
  \begin{equation}
    \label{eqn:zeta-function-for-affine-space}
    \zeta (\AA^r_X, s) = \zeta (X, s-r).
  \end{equation}
  In particular,
  \begin{align*}
    \ord_{s=n} \zeta (\AA^r_X, s) & = \ord_{s=n-r} \zeta (X, s), \\
    \zeta^* (\AA^r_X, n) & = \zeta^* (X, n-r).
  \end{align*}
\end{enumerate}

This suggests that the conjectures $\mathbf{VO} (X,n)$ and $\mathbf{C} (X,n)$
should also satisfy the corresponding compatibilities. We verify in this section
that this is indeed the case.

\begin{lemma}
  \label{lemma:compatibility-of-Lc(X,n)}
  Let $n < 0$.

  \begin{enumerate}
  \item[1)] If $X = \coprod_{1 \le i \le r} X_i$ is a finite disjoint union of
    arithmetic schemes, then
    $$\mathbf{L}^c (X_\et,n) \iff \mathbf{L}^c (X_{i,\et},n)\text{ for all }i.$$

  \item[2)] For a closed-open decomposition
    $Z \not\hookrightarrow X \hookleftarrow U$, if two out of three conjectures
    \[ \mathbf{L}^c (X_\et,n), \quad
      \mathbf{L}^c (Z_\et,n), \quad
      \mathbf{L}^c (U_\et, n) \]
    hold, then the third holds as well.

  \item[3)] For an arithmetic scheme $X$ and any $r \ge 0$, one has
    $$\mathbf{L}^c (\AA^r_{X,\et}, n) \iff \mathbf{L}^c (X_\et, n-r).$$
  \end{enumerate}

  \begin{proof}
    We already verified this in \cite[Lemma~8.7]{Beshenov-Weil-etale-1}.
  \end{proof}
\end{lemma}

\begin{lemma}
  \label{lemma:compatibility-of-B(X,n)}
  ~

  \begin{enumerate}
  \item[1)] If $X = \coprod_{1 \le i \le r} X_i$ is a finite disjoint union of
    arithmetic schemes, then
    \[ Reg_{X,n} = \bigoplus_{1 \le i \le r} Reg_{X_i,n}\colon
      \bigoplus_{1 \le i \le r} R\Gamma (X_{i,\et}, \RR^c (n)) \to
      \bigoplus_{i \le i \le r} R\Gamma_\BM (G_\RR, X_i (\CC), \RR (n)) [1]. \]
    In particular,
    $$\mathbf{B} (X,n) \iff \mathbf{B} (X_i,n)\text{ for all }i.$$

  \item[2)] For a closed-open decomposition of arithmetic schemes
    $Z \not\hookrightarrow X \hookleftarrow U$, the corresponding regulators
    give a morphism of distinguished triangles
    \[ \begin{tikzcd}[column sep=0.75em]
        R\Gamma (Z_\et, \RR^c (n)) \ar{r}\ar{d}{Reg_{Z,n}} & R\Gamma (X_\et, \RR^c (n)) \ar{r}\ar{d}{Reg_{X,n}} & R\Gamma (U_\et, \RR^c (n)) \ar{r}\ar{d}{Reg_{U,n}} & \cdots [1]\ar{d}{Reg_{Z,n} [1]} \\
        R\Gamma_\BM (G_\RR, Z (\CC), \RR (n)) [1] \ar{r} & R\Gamma_\BM (G_\RR, X (\CC), \RR (n)) [1] \ar{r} & R\Gamma_\BM (G_\RR, U (\CC), \RR (n)) [1] \ar{r} & \cdots [2]
      \end{tikzcd} \]
    In particular, if two out of three conjectures
    \[ \mathbf{B} (X,n), \quad
      \mathbf{B} (Z,n), \quad
      \mathbf{B} (U,n) \]
    hold, then the third holds as well.

  \item[3)] For any $r \ge 0$, the diagram
    \[ \begin{tikzcd}
        R\Gamma (X_\et, \RR^c (n-r)) [2r] \ar{d}{Reg_{X,n-r}}\ar{r}{\cong} & R\Gamma (\AA^r_{X,\et}, \RR^c (n))\ar{d}{Reg_{\AA^r_X,n}} \\
        R\Gamma_\BM (G_\RR, X (\CC), \RR (n-r)) [2r] \ar{r}{\cong} & R\Gamma_\BM (G_\RR, \AA^r_X (\CC), \RR (n))
      \end{tikzcd} \]
    commutes. In particular, one has
    $$\mathbf{B} (\AA^r_X, n) \iff \mathbf{B} (X, n-r).$$
  \end{enumerate}

  \begin{proof}
    Part 1) is clear, since all cohomologies involved in the definition of
    $Reg_{X,n}$ decompose into direct sums over $i = 1,\ldots r$.
    Parts 2) and 3) boil down to the corresponding functoriality properties
    for the KLM morphism \eqref{eqn:KLM-morphism-1}, namely that it commutes
    with proper pushforwards and flat pullbacks. For this we refer to
    \cite[Lemma~3, Lemma~4]{Weisschuh-2017}, and it may be also verified
    directly from the KLM formula. For closed-open decompositions,
    the distinguished triangle
    \[ R\Gamma (Z_\et, \RR^c (n)) \to R\Gamma (X_\et, \RR^c (n)) \to
      R\Gamma (U_\et, \RR^c (n)) \to R\Gamma (Z_\et, \RR^c (n)) [1] \]
    comes precisely from proper pushforward along $Z \hookrightarrow X$ and flat
    pullback along $U \hookrightarrow X$ (see \cite[Corollary~7.2]{Geisser-2010}
    and \cite[\S 3]{Bloch-1986}). Similarly, the quasi-isomorphism
    $R\Gamma (X_\et, \RR^c (n-r)) [2r] \cong R\Gamma (\AA^r_{X,\et}, \RR^c (n))$
    comes from the flat pullback along $p\colon \AA^r_X \to X$.
  \end{proof}
\end{lemma}

\begin{proposition}
  \label{prop:compatibility-of-VO(X,n)}
  For each arithmetic scheme $X$ below and $n < 0$, assume
  $\mathbf{L}^c (X_\et,n)$, $\mathbf{B} (X,n)$, and the meromorphic continuation
  of $\zeta (X,s)$ around $s = n$.

  \begin{enumerate}
  \item[1)] If $X = \coprod_{1 \le i \le r} X_i$ is a finite disjoint union of
    arithmetic schemes, then
    $$\mathbf{VO} (X,n) \iff \mathbf{VO} (X_i,n)\text{ for all }i.$$

  \item[2)] For a closed-open decomposition
    $Z \not\hookrightarrow X \hookleftarrow U$, if two out of three conjectures
    \[ \mathbf{VO} (X,n), \quad
      \mathbf{VO} (Z,n), \quad
      \mathbf{VO} (U,n) \]
    hold, then the third holds as well.

  \item[3)] For any $r \ge 0$, one has
    $$\mathbf{VO} (\AA^r_X, n) \iff \mathbf{VO} (X, n-r).$$
  \end{enumerate}

  \begin{proof}
    We already observed in proposition~\ref{prop:VO(X,n)-assuming-B(X,n)} above
    that under the conjecture $\mathbf{B} (X,n)$ we can rewrite
    $\mathbf{VO} (X,n)$ as
    $$\ord_{s=n} \zeta (X,s) = \chi (R\Gamma_c (G_\RR, X(\CC), \RR (n))).$$

    In part 1), we have
    $$\ord_{s=n} \zeta (X,s) = \sum_{1 \le i \le r} \ord_{s=n} \zeta (X_i,s),$$
    and for the corresponding $G_\RR$-equivariant cohomology,
    \[ R\Gamma_c (G_\RR, X(\CC), \RR (n)) =
      \bigoplus_{1 \le i \le r} R\Gamma_c (G_\RR, X(\CC), \RR (n)). \]
    The statement follows from the additivity of Euler characteristic:
    \[ \begin{tikzcd}[column sep=5em]
        \ord_{s=n} \zeta (X,s) \ar[equals]{r}{\mathbf{VO} (X,n)}\ar[equals]{d} & \chi (R\Gamma_c (G_\RR, X(\CC), \RR (n))) \ar[equals]{d} \\
        \sum\limits_{1 \le i \le r} \ord_{s=n} \zeta (X_i,s) \ar[equals]{r}{\forall i \mathbf{VO} (X_i,n)} & \sum\limits_{1 \le i \le r} \chi (R\Gamma_c (G_\RR, X_i (\CC), \RR (n)))
      \end{tikzcd} \]

    Similarly in part 2), we may consider the distinguished triangle
    \[ R\Gamma_c (G_\RR, U (\CC), \RR (n)) \to
      R\Gamma_c (G_\RR, X (\CC), \RR (n)) \to
      R\Gamma_c (G_\RR, Z (\CC), \RR (n)) \to
      R\Gamma_c (G_\RR, U (\CC), \RR (n)) [1] \]
    and the additivity of Euler characteristic gives
    \[ \begin{tikzcd}[column sep=4em]
        \ord_{s=n} \zeta (X,s) \ar[equals]{r}{\mathbf{VO} (X,n)}\ar[equals]{d} & \chi (R\Gamma_c (G_\RR, X(\CC), \RR (n))) \ar[equals]{d} \\
        \ord_{s=n} \zeta (Z,s) \ar[equals]{r}{\mathbf{VO} (Z,n)} & \chi (R\Gamma_c (G_\RR, Z (\CC), \RR (n))) \\[-2em]
        + & + \\[-2em]
        \ord_{s=n} \zeta (U,s) \ar[equals]{r}{\mathbf{VO} (Z,n)} & \chi (R\Gamma_c (G_\RR, U (\CC), \RR (n)))
      \end{tikzcd} \]

    Finally, in part 3), assume for simplicity that $X_\CC$ is connected of
    dimension $d_\CC$. Then the Poincaré duality and homotopy invariance of the
    usual cohomology without compact support give us
    \begin{multline*}
      R\Gamma_c (G_\RR, \AA^r (\CC) \times X (\CC), \RR (n)) \stackrel{\text{P.D.}}{\cong}
      \RHom (R\Gamma (G_\RR, \AA^r (\CC) \times X (\CC), \RR (d_\CC + r - n)), \RR) [-2d_\CC - 2r] \stackrel{\text{H.I.}}{\cong} \\
      \RHom (R\Gamma (G_\RR, X (\CC), \RR (d_\CC + r - n)), \RR) [-2d_\CC - 2r] \stackrel{\text{P.D.}}{\cong}
      R\Gamma_c (G_\RR, X (\CC), \RR (n - r)) [-2r].
    \end{multline*}
    The twist $[-2r]$ is even, hence it does not affect the Euler
    characteristic, so that we obtain
    \[ \begin{tikzcd}[column sep=4em]
        \ord_{s=n} \zeta (\AA^r_X,s) \ar[equals]{r}{\mathbf{VO} (\AA^r_X,n)}\ar[equals]{d} & \chi (R\Gamma_c (G_\RR, \AA^r (\CC) \times X(\CC), \RR (n))) \ar[equals]{d} \\
        \ord_{s=n-r} \zeta (X,s) \ar[equals]{r}{\mathbf{VO} (X,n-r)} & \chi (R\Gamma_c (G_\RR, X (\CC), \RR (n-r)))
      \end{tikzcd} \]
  \end{proof}
\end{proposition}

\begin{remark}
  Recall that the formula that appears in the original statement of
  $\mathbf{VO} (X,n)$ reads
  \begin{equation}
    \label{eqn:VO(X,n)-original-formula}
    \ord_{s=n} \zeta (X,s) = \chi' (R\Gamma_\Wc (X,\ZZ(n))) \dfn
    \sum_{i\in \ZZ} (-1)^i\cdot i \cdot \rk_\ZZ H^i_\Wc (X,\ZZ(n)).
  \end{equation}
  The conjecture $\mathbf{B} (X,n)$ in the above argument is needed to rewrite
  this in terms of the usual Euler characteristic. We used
  $\chi (R\Gamma_c (G_\RR, X (\CC), \RR (n)))$, but we could do the same with
  $\chi (\RHom (R\Gamma (X_\et, \ZZ^c (n)), \RR) [1])$.

  The least interesting part 1) of the previous proposition could be proved
  directly from \eqref{eqn:VO(X,n)-original-formula}, since
  $H^i_\Wc (X,\ZZ(n)) = \bigoplus_j H^i_\Wc (X_j,\ZZ(n))$. Parts 2) and 3) would
  be problematic to prove directly from \eqref{eqn:VO(X,n)-original-formula}
  without assuming $\mathbf{B} (X,n)$, since the secondary Euler characteristic
  $\chi' (-)$ does not behave as the usual Euler characteristic $\chi (-)$.
  In particular, it is not additive for distinguished triangles.
\end{remark}

Our next goal is to prove similar compatibilities for the special value
conjecture $\mathbf{C} (X,n)$, the same way it was done in
proposition~\ref{prop:compatibility-of-VO(X,n)} for $\mathbf{VO} (X,n)$.
We will split the proof into three technical lemmas
\ref{lemma:lambda-and-disjoint-unions},
\ref{lemma:lambda-and-closed-open-decompositions},
\ref{lemma:lambda-and-affine-bundles},
each for the corresponding compatibility. We briefly recall the construction of
our Weil-étale complex. It fits in the following diagram in the derived
category $\mathbf{D} (\ZZ)$ with distinguished triangles:
\[ \begin{tikzcd}[column sep=1.5em]
    &[-3em] \RHom (R\Gamma (X_\et, \ZZ^c (n)), \QQ [-2]) \ar{d}{\alpha_{X,n}} \ar{r} &[-2.5em] 0 \ar{d} \\
    & R\Gamma_c (X_\et, \ZZ(n)) \ar{d}\ar{r}{u_\infty^*} & R\Gamma_c (G_\RR, X (\CC), \ZZ(n))\ar{d}{id} \\
    R\Gamma_\Wc (X, \ZZ (n)) \ar{r} & R\Gamma_\fg (X, \ZZ(n)) \ar[dashed]{r}{i_\infty^*}\ar{d} & R\Gamma_c (G_\RR, X (\CC), \ZZ(n)) \ar{r} \ar{d} & R\Gamma_\Wc (X, \ZZ (n)) [1] \\
    & \RHom (R\Gamma (X_\et, \ZZ^c (n)), \QQ [-1]) \ar{r} & 0
\end{tikzcd} \]
For further details, the reader may consult \cite{Beshenov-Weil-etale-1}.

\begin{lemma}
  \label{lemma:lambda-and-disjoint-unions}
  Let $n < 0$ and let $X = \coprod_{1 \le i \le r} X_i$ be a finite disjoint
  union of arithmetic schemes. Assume $\mathbf{L}^c (X_\et,n)$ and
  $\mathbf{B} (X,n)$. Then there is a quasi-isomorphism of complexes
  \[ \bigoplus_{1 \le i \le r} R\Gamma_\Wc (X_i, \ZZ(n)) \cong
    R\Gamma_\Wc (X, \ZZ(n)), \]
  which after passing to the determinants gives a commutative diagram

  \begin{equation}
    \label{eqn:lambda-and-disjoint-unions}
    \begin{tikzcd}
      \RR \otimes_\RR \cdots \otimes_\RR \RR\ar{d}{\lambda_{X_1,n}\otimes\cdots\otimes\lambda_{X_r,n}}[swap]{\cong} \ar{r}{x_1\otimes\cdots\otimes x_r \mapsto x_1\cdots x_r}[swap]{\cong} & \RR \ar{d}{\lambda_{X,n}}[swap]{\cong} \\
      \bigotimes\limits_{1 \le i \le r} (\det_\ZZ R\Gamma_\Wc (X_i, \ZZ(n))) \otimes_\ZZ \RR \ar{r}{\cong} & (\det_\ZZ R\Gamma_\Wc (X_i, \ZZ(n))) \otimes_\ZZ \RR
    \end{tikzcd}
  \end{equation}

  \begin{proof}
    From the construction of $R\Gamma_\Wc (X, \ZZ(n))$ it is clear that for
    $X = \coprod_{1 \le i \le r} X_i$ all involved cohomologies decompose into
    the corresponding direct sum over $i = 1,\ldots,r$, and at the end after
    tensoring with $\RR$ one obtains a commutative diagram

    \[ \begin{tikzcd}
        \bigoplus_i \left(\begin{array}{c} R\Gamma_c (G_\RR, X_i (\CC), \RR (n)) [-2] \\ \oplus \\ R\Gamma_c (G_\RR, X_i (\CC), \RR (n)) [-1] \end{array}\right) \ar{d}{\bigoplus_i Reg_{X_i,n}^\vee [-1] \oplus id}[swap]{\cong} \ar{r}{\cong} & \begin{array}{c} R\Gamma_c (G_\RR, X (\CC), \RR (n)) [-2] \\ \oplus \\ R\Gamma_c (G_\RR, X (\CC), \RR (n)) [-1] \end{array} \ar{d}{Reg_{X,n}^\vee [-1] \oplus id}[swap]{\cong} \\
        \bigoplus_i \left(\begin{array}{c} \RHom (R\Gamma (X_{i,\et}, \ZZ^c (n)), \RR) [-1] \\ \oplus \\ R\Gamma_c (G_\RR, X_i (\CC), \RR (n)) [-1] \end{array}\right) \ar{d}{\text{split}}[swap]{\cong} \ar{r}{\cong} & \begin{array}{c} \RHom (R\Gamma (X_\et, \ZZ^c (n)), \RR) [-1] \\ \oplus \\ R\Gamma_c (G_\RR, X (\CC), \RR (n)) [-1] \end{array} \ar{d}{\text{split}}[swap]{\cong} \\
        \bigoplus_i R\Gamma_\Wc (X_i, \RR (n)) \ar{r}{\cong} & R\Gamma_\Wc (X, \RR (n))
      \end{tikzcd} \]
    Taking the determinants, we obtain \eqref{eqn:lambda-and-disjoint-unions}.
  \end{proof}
\end{lemma}

\begin{lemma}
  \label{lemma:lambda-and-closed-open-decompositions}
  Let $n < 0$ and let $Z \not\hookrightarrow X \hookleftarrow U$ be a
  closed-open decomposition of arithmetic schemes, such that the conjectures
  \begin{gather*}
    \mathbf{L}^c (U_\et,n), ~ \mathbf{L}^c (X_\et,n), ~ \mathbf{L}^c (Z_\et,n),\\
    \mathbf{B} (U,n), ~ \mathbf{B} (X,n), ~ \mathbf{B} (Z_\et,n)
  \end{gather*}
  hold (in each case, it is enough to assume two out of three thanks to lemmas
  \ref{lemma:compatibility-of-Lc(X,n)} and \ref{lemma:compatibility-of-B(X,n)}).
  Then there is an isomorphism of determinants
  \begin{equation}
    \label{eqn:isomorphism-of-det-RGamma-Wc-for-closed-open-decompositions}
    \det_\ZZ R\Gamma_\Wc (U, \ZZ(n)) \otimes_\ZZ
    \det_\ZZ R\Gamma_\Wc (Z, \ZZ(n)) \cong
    \det_\ZZ R\Gamma_\Wc (X, \ZZ(n))
  \end{equation}
  making the following diagram commute up to signs:
  \begin{equation}
    \label{eqn:lambda-and-closed-open-decompositions}
    \begin{tikzcd}
      \RR \otimes_\RR \RR \ar{r}{x\otimes y \mapsto xy}\ar{d}{\lambda_{U,n} \otimes \lambda_{Z,n}}[swap]{\cong} & \RR\ar{d}{\lambda_{X,n}}[swap]{\cong} \\
      \begin{array}{c} (\det_\ZZ R\Gamma_\Wc (U, \ZZ(n)))\otimes_\ZZ \RR \\ \otimes_\RR \\ (\det_\ZZ R\Gamma_\Wc (Z, \ZZ(n)))\otimes_\ZZ \RR \end{array} \ar{r}{\cong} & (\det_\ZZ R\Gamma_\Wc (X, \ZZ(n))) \otimes_\ZZ \RR
    \end{tikzcd}
  \end{equation}

  \begin{proof}
    Morally, we expect in this situation a distinguished triangle of the form
    \begin{equation}
      \label{eqn:U-X-Z-triangle-for-RGamma-Wc}
      R\Gamma_\Wc (U, \ZZ(n)) \to
      R\Gamma_\Wc (X, \ZZ(n)) \to
      R\Gamma_\Wc (Z, \ZZ(n)) \to
      R\Gamma_\Wc (U, \ZZ(n)) [1].
    \end{equation}
    However, even the complex $R\Gamma_\Wc (X, \ZZ(n))$ was constructed in
    \cite{Beshenov-Weil-etale-1} up to a non-canonical isomorphism in
    the derived category $\mathbf{D} (\ZZ)$, so this is problematic. In the
    absence of a better definition, we will construct the isomorphism
    \eqref{eqn:isomorphism-of-det-RGamma-Wc-for-closed-open-decompositions} in
    an ad hoc manner.

    A closed-open decomposition $Z \not\hookrightarrow X \hookleftarrow U$ gives
    us distinguished triangles
    \[ \begin{tikzcd}[row sep=0pt,column sep=1em]
        R\Gamma (Z_\et, \ZZ^c (n)) \ar{r} & R\Gamma (X_\et, \ZZ^c (n)) \ar{r} & R\Gamma (U_\et, \ZZ^c (n)) \ar{r} & R\Gamma (Z_\et, \ZZ^c (n)) [1] \\
        R\Gamma_c (U_\et, \ZZ (n)) \ar{r} & R\Gamma_c (X_\et, \ZZ (n)) \ar{r} & R\Gamma_c (Z_\et, \ZZ (n)) \ar{r} & R\Gamma_c (U_\et, \ZZ (n)) [1] \\
        R\Gamma_c (G_\RR, U (\CC), \RR (n)) \ar{r} & R\Gamma_c (G_\RR, X (\CC), \RR (n)) \ar{r} & R\Gamma_c (G_\RR, Z (\CC), \RR (n)) \ar{r} & R\Gamma_c (G_\RR, U (\CC), \RR (n)) [1]
      \end{tikzcd} \]
    The first triangle is \cite[Corollary~7.2]{Geisser-2010}, and it means that
    $R\Gamma (-, \ZZ^c (n))$ behaves like Borel--Moore homology, while the
    following two triangles are the usual ones for cohomology with compact
    support.
    These fit together in a commutative diagram displayed on
    figure~\ref{fig:RGamma-Wc-and-closed-open-decompositions} below
    (page \pageref{fig:RGamma-Wc-and-closed-open-decompositions}).
    For brevity, we denote $\RHom (X,Y)$ by $[X,Y]$ in the diagram.
    Similarly,
    figure~\ref{fig:RGamma-Wc-and-closed-open-decompositions-otimes-Q}
    displays the same diagram tensored with $\QQ$.

    In this diagram, we start from the morphism of triangles
    $(\alpha_{U,n}, \alpha_{X,n}, \alpha_{Z,n})$, and then take the respective
    cones $R\Gamma_\fg (-, \ZZ(n))$. In fact, by
    \cite[Proposition~5.6]{Beshenov-Weil-etale-1}, these cones are well-defined
    up to a \emph{unique} isomorphism in the derived category
    $\mathbf{D} (\ZZ)$, and the same argument shows that the induced morphisms
    of complexes
    \begin{equation}
      \label{eqn:triangle-RGamma-fg}
      R\Gamma_\fg (U, \ZZ(n)) \to
      R\Gamma_\fg (X, \ZZ(n)) \to
      R\Gamma_\fg (Z, \ZZ(n)) \to
      R\Gamma_\fg (U, \ZZ(n)) [1]
    \end{equation}
    are also uniquely defined
    (see \cite[Corollary~A.3]{Beshenov-Weil-etale-1}). A priori, it does not
    have to be a distinguished triangle\footnote{Taking naively a ``cone of a
      morphism of distinguished triangles''
      $$\begin{tikzpicture}[ampersand replacement=\&]
        \matrix(m)[matrix of math nodes, row sep=1.5em, column sep=1.5em,
        text height=1.5ex, text depth=0.25ex]{
          X \& Y \& Z \& X[1] \\
          X' \& Y' \& Z' \& X'[1] \\
          X'' \& Y'' \& Z'' \& X''[1] \\
          X[1]  \& Y[1]  \& Z[1]  \& X[2]\\};

        \path[->] (m-1-1) edge (m-1-2);
        \path[->] (m-1-2) edge (m-1-3);
        \path[->] (m-1-3) edge (m-1-4);

        \path[->] (m-2-1) edge (m-2-2);
        \path[->] (m-2-2) edge (m-2-3);
        \path[->] (m-2-3) edge (m-2-4);

        \path[dashed,->] (m-3-1) edge (m-3-2);
        \path[dashed,->] (m-3-2) edge (m-3-3);
        \path[dashed,->] (m-3-3) edge (m-3-4);

        \path[->] (m-4-1) edge (m-4-2);
        \path[->] (m-4-2) edge (m-4-3);
        \path[->] (m-4-3) edge (m-4-4);

        \path[->] (m-1-1) edge (m-2-1);
        \path[->] (m-1-2) edge (m-2-2);
        \path[->] (m-1-3) edge (m-2-3);
        \path[->] (m-1-4) edge (m-2-4);

        \path[dashed,->] (m-2-1) edge (m-3-1);
        \path[dashed,->] (m-2-2) edge (m-3-2);
        \path[dashed,->] (m-2-3) edge (m-3-3);
        \path[dashed,->] (m-2-4) edge (m-3-4);

        \path[dashed,->] (m-3-1) edge (m-4-1);
        \path[dashed,->] (m-3-2) edge (m-4-2);
        \path[dashed,->] (m-3-3) edge (m-4-3);
        \path[dashed,->] (m-3-4) edge (m-4-4);

        \node[font=\small] at ($(m-3-3)!.5!(m-4-4)$) {(ac)};
      \end{tikzpicture}$$
      normally \emph{does not} give a distinguished triangle
      $X'' \to Y'' \to Z'' \to X'' [1]$,
      as discussed in \cite{Neeman-1991}. Here we are dealing with notorious
      issues associated to working with the classical derived (1-)categories.},
    but we claim that it induces a long exact sequence in cohomology.

    For this note that tensoring the diagram with $\ZZ/m\ZZ$ gives us an
    isomorphism
    \[ \begin{tikzcd}[column sep=1em,font=\small]
        R\Gamma_c (U_\et, \ZZ/m\ZZ (n)) \ar{r}\ar{d}{\cong} & R\Gamma_c (X_\et, \ZZ/m\ZZ (n)) \ar{r}\ar{d}{\cong} & R\Gamma_c (Z_\et, \ZZ/m\ZZ (n)) \ar{r}\ar{d}{\cong} & R\Gamma_c (U_\et, \ZZ/m\ZZ (n)) [1]\ar{d}{\cong} \\
        R\Gamma_\fg (U, \ZZ (n)) \otimes^\mathbf{L}_\ZZ \ZZ/m\ZZ \ar{r} & R\Gamma_\fg (X, \ZZ (n)) \otimes^\mathbf{L}_\ZZ \ZZ/m\ZZ \ar{r} & R\Gamma_\fg (Z, \ZZ (n)) \otimes^\mathbf{L}_\ZZ \ZZ/m\ZZ \ar{r} & R\Gamma_\fg (U, \ZZ (n)) \otimes^\mathbf{L}_\ZZ \ZZ/m\ZZ [1]
      \end{tikzcd} \]
    More generally, for each prime $p$ we may take the corresponding derived
    $p$-adic completions (see \cite{Bhatt-Scholze-2015} and
    \cite[Tag~091N]{Stacks-project})
    \[ R\Gamma_\fg (-, \ZZ(n))^\wedge_p \dfn
      R\varprojlim_k (R\Gamma_\fg (-, \ZZ(n)) \otimes^\mathbf{L}_\ZZ \ZZ/p^k\ZZ), \]
    and these give us a distinguished triangle for each prime $p$
    \[ R\Gamma_\fg (U, \ZZ(n))^\wedge_p \to
      R\Gamma_\fg (X, \ZZ(n))^\wedge_p \to
      R\Gamma_\fg (Z, \ZZ(n))^\wedge_p \to
      R\Gamma_\fg (U, \ZZ(n))^\wedge_p [1]. \]
    On the level of cohomology, there are natural isomorphisms
    \cite[Tag~0A06]{Stacks-project}
    \[ H^i (R\Gamma_\fg (-, \ZZ(n))^\wedge_p) \cong
      H^i_\fg (-, \ZZ(n)) \otimes_\ZZ \ZZ_p. \]
    In particular, for each $p$ there is a long exact sequence of cohomology
    groups
    \[ \cdots \to H^i_\fg (U, \ZZ(n)) \otimes_\ZZ \ZZ_p \to
      H^i_\fg (X, \ZZ(n)) \otimes_\ZZ \ZZ_p \to
      H^i_\fg (Z, \ZZ(n)) \otimes_\ZZ \ZZ_p \to
      H^{i+1}_\fg (U, \ZZ(n)) \otimes_\ZZ \ZZ_p \to \cdots \]
    induced by \eqref{eqn:triangle-RGamma-fg}. But now since the groups
    $H^i_\fg (-, \ZZ(n))$ are finitely generated, by flatness of $\ZZ_p$ this
    implies that the sequence
    \begin{equation}
      \label{eqn:RGamma-fg-long-exact-sequence}
      \cdots \to H^i_\fg (U, \ZZ(n)) \to
      H^i_\fg (X, \ZZ(n)) \to
      H^i_\fg (Z, \ZZ(n)) \to
      H^{i+1}_\fg (U, \ZZ(n)) \to \cdots
    \end{equation}
    is exact.

    Now we consider the diagram
    \[ \begin{tikzcd}[column sep=1em]
        \tau_{\le m} R\Gamma_c (G_\RR, U (\CC), \ZZ (n))[-1] \ar{r}\ar{d} & R\Gamma_\Wc (U, \ZZ (n))\ar{d}\ar{r} & \tau_{\le m} R\Gamma_\fg (U,\ZZ(n)) \ar{d}\ar{r} & \tau_{\le m} R\Gamma_c (G_\RR, U (\CC), \ZZ (n)) \ar{d} \\
        \tau_{\le m} R\Gamma_c (G_\RR, X (\CC), \ZZ (n))[-1] \ar{r}\ar{d} & R\Gamma_\Wc (X, \ZZ (n))\ar{d}\ar{r} & \tau_{\le m} R\Gamma_\fg (X,\ZZ(n)) \ar{d}\ar{r} & \tau_{\le m} R\Gamma_c (G_\RR, X (\CC), \ZZ (n)) \ar{d} \\
        \tau_{\le m} R\Gamma_c (G_\RR, Z (\CC), \ZZ (n))[-1] \ar{r}\ar{d} & R\Gamma_\Wc (Z, \ZZ (n))\ar{d}\ar{r} & \tau_{\le m} R\Gamma_\fg (Z,\ZZ(n)) \ar{d}\ar{r} & \tau_{\le m} R\Gamma_c (G_\RR, Z (\CC), \ZZ (n)) \ar{d} \\
        \tau_{\le m} R\Gamma_c (G_\RR, U (\CC), \ZZ (n)) \ar{r} & R\Gamma_\Wc (U, \ZZ (n)) [1] \ar{r} & \tau_{\le m} R\Gamma_\fg (U,\ZZ(n)) [1] \ar{r} & \tau_{\le m} R\Gamma_c (G_\RR, U (\CC), \ZZ (n)) [1]
      \end{tikzcd} \]
    Here we took truncations for $m$ big enough similarly to the proof of
    lemma~\ref{lemma:determinant-of-RGamma-Wc-well-defined}. There are
    canonical isomorphisms
    \begin{align*}
      \notag \det_\ZZ R\Gamma_\Wc (U, \ZZ(n)) & \cong \det_\ZZ (\tau_{\le m} R\Gamma_c (G_\RR, U (\CC), \ZZ (n)) [-1]) \otimes_\ZZ \det_\ZZ (\tau_{\le m} R\Gamma_\fg (U, \ZZ(n))), \\
      \notag \det_\ZZ R\Gamma_\Wc (X, \ZZ(n)) & \cong \det_\ZZ (\tau_{\le m} R\Gamma_c (G_\RR, X (\CC), \ZZ (n)) [-1]) \otimes_\ZZ \det_\ZZ (\tau_{\le m} R\Gamma_\fg (X, \ZZ(n))), \\
      \notag \det_\ZZ R\Gamma_\Wc (Z, \ZZ(n)) & \cong \det_\ZZ (\tau_{\le m} R\Gamma_c (G_\RR, Z (\CC), \ZZ (n)) [-1]) \otimes_\ZZ \det_\ZZ (\tau_{\le m} R\Gamma_\fg (Z, \ZZ(n))), \\
      \det_\ZZ (\tau_{\le m} R\Gamma_c (G_\RR, X (\CC), \ZZ(n))) & \cong \det_\ZZ (\tau_{\le m} R\Gamma_c (G_\RR, U (\CC), \ZZ (n))) \otimes_\ZZ \det_\ZZ (\tau_{\le m} R\Gamma_c (G_\RR, Z (\CC), \ZZ(n))), \\
      \det_\ZZ (\tau_{\le m} R\Gamma_\fg (X, \ZZ(n))) & \cong \det_\ZZ (\tau_{\le m} R\Gamma_\fg (U, \ZZ (n))) \otimes_\ZZ \det_\ZZ (\tau_{\le m} R\Gamma_\fg (Z, \ZZ(n))).
    \end{align*}
    Here the first four isomorphisms come from true distinguished triangles,
    while the last isomorphism comes from the cohomology long exact sequence
    \eqref{eqn:RGamma-fg-long-exact-sequence}, which gives an isomorphism
    \[ \bigotimes_{i \le m}
      \Bigl(\det_\ZZ H^i_\fg (U, \ZZ(n))^{(-1)^i} \otimes_\ZZ
      \det_\ZZ H^i_\fg (X, \ZZ(n))^{(-1)^{i+1}} \otimes_\ZZ
      \det_\ZZ H^i_\fg (Z, \ZZ(n))^{(-1)^i}\Bigr) \cong \ZZ. \]
    We may rearrange the terms \emph{at the cost of introducing a $\pm 1$ sign},
    to obtain
    \begin{multline*}
      \det_\ZZ (\tau_{\le m} R\Gamma_\fg (X, \ZZ(n))) \cong
      \bigotimes_{i \le m} \det_\ZZ H^i_\fg (X, \ZZ(n)) \cong \\
      \bigotimes_{i \le m} \det_\ZZ H^i_\fg (U, \ZZ(n)) \otimes_\ZZ
      \bigotimes_{i \le m} \det_\ZZ H^i_\fg (Z, \ZZ(n)) \cong \\
      \det_\ZZ (\tau_{\le m} R\Gamma_\fg (U, \ZZ(n))) \otimes_\ZZ
      \det_\ZZ (\tau_{\le m} R\Gamma_\fg (Z, \ZZ(n))).
    \end{multline*}

    All the above gives us the desired isomorphism of integral determinants
    \eqref{eqn:isomorphism-of-det-RGamma-Wc-for-closed-open-decompositions}.

    Now we consider the following diagram with distinguished rows:
    \[ \begin{tikzcd}[column sep=1em]
        \begin{array}{c} R\Gamma_c (G_\RR, U (\CC), \RR (n)) [-2] \\ \oplus \\ R\Gamma_c (G_\RR, U (\CC), \RR (n)) [-1] \end{array} \ar{d}{Reg_{U,n}^\vee [-1] \oplus id}[swap]{\cong} \ar{r} & \begin{array}{c} R\Gamma_c (G_\RR, X (\CC), \RR (n)) [-2] \\ \oplus \\ R\Gamma_c (G_\RR, X (\CC), \RR (n)) [-1] \end{array} \ar{d}{Reg_{X,n}^\vee [-1] \oplus id}[swap]{\cong} \ar{r} & \begin{array}{c} R\Gamma_c (G_\RR, Z (\CC), \RR (n)) [-2] \\ \oplus \\ R\Gamma_c (G_\RR, Z (\CC), \RR (n)) [-1] \end{array} \ar{d}{Reg_{Z,n}^\vee [-1] \oplus id}[swap]{\cong} \ar{r} & \cdots \ar{d} \\
        \begin{array}{c} \RHom (R\Gamma (U_\et, \ZZ^c (n)), \RR) [-1] \\ \oplus \\ R\Gamma_c (G_\RR, U (\CC), \RR (n)) [-1] \end{array} \ar{d}{\text{split}}[swap]{\cong} \ar{r} & \begin{array}{c} \RHom (R\Gamma (X_\et, \ZZ^c (n)), \RR) [-1] \\ \oplus \\ R\Gamma_c (G_\RR, X (\CC), \RR (n)) [-1] \end{array} \ar{d}{\text{split}}[swap]{\cong} \ar{r} & \begin{array}{c} \RHom (R\Gamma (Z_\et, \ZZ^c (n)), \RR) [-1] \\ \oplus \\ R\Gamma_c (G_\RR, Z (\CC), \RR (n)) [-1] \end{array} \ar{d}{\text{split}}[swap]{\cong} \ar{r} & \cdots \ar{d} \\
        R\Gamma_\Wc (U, \RR (n)) \ar{r} & R\Gamma_\Wc (X, \RR (n)) \ar{r} & R\Gamma_\Wc (Z, \RR (n)) \ar{r} & \cdots
      \end{tikzcd} \]

    Here the three squares with regulators involved commute thanks to
    lemma~\ref{lemma:compatibility-of-B(X,n)}. Taking the determinants, we
    obtain \eqref{eqn:lambda-and-closed-open-decompositions}, using the
    compatibility of determinants with distinguished triangles. We note that we
    did not construct an integral distinguished triangle
    \eqref{eqn:U-X-Z-triangle-for-RGamma-Wc}; instead we only have that the
    bottom arrow in \eqref{eqn:lambda-and-closed-open-decompositions} is induced
    by the ad hoc isomorphism of determinants
    \eqref{eqn:isomorphism-of-det-RGamma-Wc-for-closed-open-decompositions}.
  \end{proof}
\end{lemma}

\begin{lemma}
  \label{lemma:lambda-and-affine-bundles}
  For $n < 0$ and $r \ge 0$, let $X$ be an arithmetic scheme satisfying
  $\mathbf{L}^c (X_\et,n-r)$ and $\mathbf{B} (X,n-r)$. Then there is a natural
  quasi-isomorphism of complexes
  \begin{equation}
    \label{eqn:RGamma-Wc-and-affine-bundles}
    R\Gamma_\Wc (\AA^r_X, \ZZ (n)) \cong R\Gamma_\Wc (X, \ZZ (n-r)) [-2r],
  \end{equation}
  which after passing to the determinants makes the following diagram commute:
  \begin{equation}
    \label{eqn:lambda-and-affine-bundles}
    \begin{tikzcd}
    & \RR\ar{dl}{\cong}[swap]{\lambda_{\AA^r_X,n}}\ar{dr}{\lambda_{X,n-r}}[swap]{\cong} \\
      (\det_\ZZ R\Gamma_\Wc (\AA^r_X, \ZZ (n)))\otimes_\ZZ \RR \ar{rr}{\cong} & & (\det_\ZZ R\Gamma_\Wc (X, \ZZ (n-r)))\otimes_\ZZ \RR
    \end{tikzcd}
  \end{equation}

  \begin{proof}
    We refer to figure~\ref{fig:RGamma-Wc-and-affine-bundles} below (page
    \pageref{fig:RGamma-Wc-and-affine-bundles}) that shows how the flat morphism
    $p\colon \AA^r_X \to X$ induces the desired quasi-isomorphism
    \eqref{eqn:RGamma-Wc-and-affine-bundles}. Everything comes down to the
    homotopy property of motivic cohomology, namely the fact that $p$ induces a
    quasi-isomorphism
    \[ p^*\colon R\Gamma (X_\et, \ZZ^c (n-r)) [2r] \xrightarrow{\cong}
      R\Gamma (\AA^r_{X,\et}, \ZZ^c (n)) \]
    ---for this see e.g. \cite[Lemma~5.11]{Morin-2014}.
    After passing to real coefficients, we obtain the following diagram:
    \[ \begin{tikzcd}
        \begin{array}{c} R\Gamma_c (G_\RR, \AA^r_X (\CC), \RR (n)) [-2] \\ \oplus \\ R\Gamma_c (G_\RR, \AA^r_X (\CC), \RR (n)) [-1] \end{array} \ar{d}{Reg_{\AA^r_X,n}^\vee [-1] \oplus id}[swap]{\cong} \ar{r}{\cong} & \begin{array}{c} R\Gamma_c (G_\RR, X (\CC), \RR (n-r)) [-2] [-2r] \\ \oplus \\ R\Gamma_c (G_\RR, X (\CC), \RR (n-r)) [-1] [-2r] \end{array} \ar{d}{Reg_{X,n-r}^\vee [-1] [-2r] \oplus id}[swap]{\cong} \\
        \begin{array}{c} \RHom (R\Gamma (\AA^r_{X,\et}, \ZZ^c (n)), \RR) [-1] \\ \oplus \\ R\Gamma_c (G_\RR, \AA^r_X (\CC), \RR (n)) [-1] \end{array} \ar{d}{\text{split}}[swap]{\cong} \ar{r}{\cong} & \begin{array}{c} \RHom (R\Gamma (X_\et, \ZZ^c (n-r)) [2r], \RR) [-1] \\ \oplus \\ R\Gamma_c (G_\RR, X (\CC), \RR (n-r)) [-1] [-2r] \end{array} \ar{d}{\text{split}}[swap]{\cong} \\
        R\Gamma_\Wc (\AA^r_X, \RR (n)) \ar{r}{\cong} & R\Gamma_\Wc (X, \RR (n-r)) [-2r]
      \end{tikzcd} \]
    Here the first square commutes by the compatibility of the regulator with
    affine bundles (lemma~\ref{lemma:compatibility-of-B(X,n)}), and the second
    square commutes because the quasi-isomorphism
    \eqref{eqn:RGamma-Wc-and-affine-bundles} gives compatible splittings
    (again, see figure~\ref{fig:RGamma-Wc-and-affine-bundles} below).
    Taking the determinants, we obtain the desired commutative diagram
    \eqref{eqn:lambda-and-affine-bundles}.
  \end{proof}
\end{lemma}

\begin{theorem}
  \label{thm:compatibility-of-C(X,n)}
  For an arithmetic scheme $X$ and $n < 0$, assume $\mathbf{L}^c (X_\et,n)$,
  $\mathbf{B} (X,n)$, and the meromorphic continuation of $\zeta (X,s)$ around
  $s = n$.

  \begin{enumerate}
  \item[1)] If $X = \coprod_{1 \le i \le r} X_i$ is a finite disjoint union of
    arithmetic schemes, then
    $$\mathbf{C} (X,n) \iff \mathbf{C} (X_i,n)\text{ for all }i.$$

  \item[2)] For a closed-open decomposition
    $Z \not\hookrightarrow X \hookleftarrow U$, if two out of three conjectures
    \[ \mathbf{C} (X,n), \quad
      \mathbf{C} (Z,n), \quad
      \mathbf{C} (U,n) \]
    hold, then the third holds as well.

  \item[3)] For any $r \ge 0$, one has
    $$\mathbf{C} (\AA^r_X, n) \iff \mathbf{C} (X, n-r).$$
  \end{enumerate}

  \begin{proof}
    Follows from the previous lemmas
    \ref{lemma:lambda-and-disjoint-unions},
    \ref{lemma:lambda-and-closed-open-decompositions},
    \ref{lemma:lambda-and-affine-bundles},
    together with the respective identities for zeta functions
    \eqref{eqn:zeta-function-for-disjoint-unions},
    \eqref{eqn:zeta-function-for-closed-open-decompositions},
    \eqref{eqn:zeta-function-for-affine-space}.
  \end{proof}
\end{theorem}

\begin{remark}
  As a formal consequence of compatibility with closed-open decompositions, if
  we apply it to the canonical closed embedding $X_\red \hookrightarrow X$,
  then we conclude that
  $R\Gamma_\Wc (X, \ZZ(n)) \cong R\Gamma_\Wc (X_\red, \ZZ(n))$. This is not
  surprising, because Weil-étale complexes are constructed from a variant of
  cycle complexes / higher Chow groups, and these do not distinguish $X$ from
  $X_\red$ (unlike, for instance, algebraic $K$-groups).

  This is actually the desired behavior for us, since neither the zeta function
  does: $\zeta (X,s) = \zeta (X_\red,s)$.
\end{remark}

\begin{remark}
  If $X/\FF_q$ is a variety over a finite field, then the proof of
  theorem~\ref{thm:compatibility-of-C(X,n)} simplifies drastically: we can work
  with the formula \eqref{eqn:special-value-for-X/Fq}, and the following
  properties of motivic cohomology:
  \begin{enumerate}
  \item[1)] $R\Gamma (\coprod_i X_{i,\et}, \ZZ^c (n)) \cong
    \bigoplus_i R\Gamma (X_{i,\et}, \ZZ^c (n))$;

  \item[2)] triangles
    $R\Gamma (Z_\et, \ZZ^c (n)) \to
    R\Gamma (X_\et, \ZZ^c (n)) \to
    R\Gamma (U_\et, \ZZ^c (n)) \to
    R\Gamma (Z_\et, \ZZ^c (n))[1]$
    associated to closed-open decompositions;

  \item[3)] homotopy invariance
    $R\Gamma (X_\et, \ZZ^c (n-r)) [2r] \cong
    R\Gamma (\AA^r_{X,\et}, \ZZ^c (n))$.
  \end{enumerate}
  There are no regulators involved in this case, so we do not need the technical
  lemmas
  \ref{lemma:lambda-and-disjoint-unions},
  \ref{lemma:lambda-and-closed-open-decompositions},
  \ref{lemma:lambda-and-affine-bundles}.
\end{remark}

Considering the projective space $\PP_X^r = \PP_\ZZ^r \times X$,
we have a formula for the zeta function
\begin{equation}
  \label{eqn:zeta-function-for-projective-space}
  \zeta (\PP_X^r, s) = \prod_{0 \le i \le r} \zeta (X, s-i).
\end{equation}

\begin{corollary}[projective bundles]
  Let $X$ be an arithmetic scheme, $n < 0$, and $r \ge 0$.
  For $i = 0,\ldots,r$ assume the conjectures $\mathbf{L}^c (X_\et,n-i)$,
  $\mathbf{B} (X,n-i)$, and meromorphic continuation of $\zeta (X,s)$ around
  $s = n-i$. Then
  \[ \mathbf{C} (X,n-i)\text{ for }i = 0,\ldots,r \Longrightarrow
    \mathbf{C} (\PP_X^r, n). \]

  \begin{proof}
    Applied to the closed-open decomposition
    $\PP_X^{r-1} \not\hookrightarrow \PP_X^r \hookleftarrow \AA_X^r$,
    theorem~\ref{thm:compatibility-of-C(X,n)} gives
    \[ \mathbf{C} (X, n-r) \text{ and } \mathbf{C} (\PP_X^{r-1}, n)
      \Longrightarrow
      \mathbf{C} (\AA_X^r, n) \text{ and } \mathbf{C} (\PP_X^{r-1}, n)
      \Longrightarrow
      \mathbf{C} (\PP_X^r,n). \]
    The claim follows by induction on $r$.
    (Note that the same inductive argument proves the formula
    \eqref{eqn:zeta-function-for-projective-space} from
    \eqref{eqn:zeta-function-for-affine-space}.)
    \end{proof}
\end{corollary}

\begin{landscape}
  \begin{figure}
    \[ \begin{tikzcd}[font=\small]
        &[2em] &[-2.5em] &[-2.5em] &[-2.5em] R\Gamma_\Wc (U, \ZZ (n))\ar{dl} &[-2.5em] \\
        {[R\Gamma (U_\et, \ZZ^c (n)), \QQ[-2]]} \ar{r}{\alpha_{U,n}}\ar{dd} & R\Gamma_c (U_\et, \ZZ (n)) \ar{dr}{u_\infty} \ar{dd} \ar{rr} & & R\Gamma_\fg (U, \ZZ (n))\ar{dl}[swap]{i_\infty} \ar[dashed]{dd}{\exists!} \ar[crossing over]{rr} & & {[R\Gamma (U_\et, \ZZ^c (n)), \QQ[-1]]} \ar{dd} \\
        & & R\Gamma_c (G_\RR, U (\CC), \ZZ (n)) & & R\Gamma_\Wc (X, \ZZ (n))\ar{dl} \\
        {[R\Gamma (X_\et, \ZZ^c (n)), \QQ[-2]]} \ar{r}{\alpha_{X,n}}\ar{dd} & R\Gamma_c (X_\et, \ZZ (n)) \ar{dr}{u_\infty} \ar{dd} \ar{rr} & & R\Gamma_\fg (X, \ZZ (n)) \ar{dl}[swap]{i_\infty} \ar[dashed]{dd}{\exists!} \ar[crossing over]{rr} & & {[R\Gamma (X_\et, \ZZ^c (n)), \QQ[-1]]} \ar{dd} \\
        & & R\Gamma_c (G_\RR, X (\CC), \ZZ (n)) \ar[<-,near end,crossing over]{uu} & & R\Gamma_\Wc (Z, \ZZ (n))\ar{dl} \\
        {[R\Gamma (Z_\et, \ZZ^c (n)), \QQ[-2]]} \ar{r}{\alpha_{Z,n}}\ar{dd} & R\Gamma_c (Z_\et, \ZZ (n)) \ar{dr}{u_\infty} \ar{dd} \ar{rr} & & R\Gamma_\fg (Z, \ZZ (n)) \ar{dl}[swap]{i_\infty} \ar[dashed]{dd}{\exists!} \ar[crossing over]{rr} & & {[R\Gamma (Z_\et, \ZZ^c (n)), \QQ[-1]]} \ar{dd} \\
        & & R\Gamma_c (G_\RR, Z (\CC), \ZZ (n)) \ar[<-,near end,crossing over]{uu} & & R\Gamma_\Wc (U, \ZZ (n)) [1]\ar{dl} \\
        {[R\Gamma (U_\et, \ZZ^c (n)), \QQ[-1]]} \ar{r}{\alpha_{U,n} [1]} & R\Gamma_c (U_\et, \ZZ (n)) [1] \ar{dr}{u_\infty} \ar{rr} & & R\Gamma_\fg (U, \ZZ (n)) [1] \ar{dl}[swap]{i_\infty [1]} \ar{rr} & & {[R\Gamma (U_\et, \ZZ^c (n)), \QQ]} \\
        & & R\Gamma_c (G_\RR, U (\CC), \ZZ (n)) [1] \ar[<-,near end,crossing over]{uu} \\
      \end{tikzcd} \]

    \caption{Diagram induced by a closed-open decomposition
      $Z \not\hookrightarrow X \hookleftarrow U$}
    \label{fig:RGamma-Wc-and-closed-open-decompositions}
  \end{figure}
\end{landscape}

\begin{landscape}
  \begin{figure}
    \[ \begin{tikzcd}[font=\small]
        &[2em] &[-2.5em] &[-2.5em] &[-2.5em] R\Gamma_\Wc (U, \QQ (n))\ar{dl}\ar{dd} &[-2.5em] \\
        {[R\Gamma (U_\et, \ZZ^c (n)), \QQ[-2]]} \ar{r}\ar{dd} & 0 \ar{dr} \ar{dd} \ar{rr} & & R\Gamma_\fg (U, \QQ (n))\ar{dl}[swap]{0} \ar[dashed]{dd}{\exists!} \ar[crossing over,near start]{rr}{\cong} & & {[R\Gamma (U_\et, \ZZ^c (n)), \QQ[-1]]} \ar{dd} \\
        & & R\Gamma_c (G_\RR, U (\CC), \QQ (n)) & & R\Gamma_\Wc (X, \QQ (n))\ar{dl}\ar{dd} \\
        {[R\Gamma (X_\et, \ZZ^c (n)), \QQ[-2]]} \ar{r}\ar{dd} & 0 \ar{dr} \ar{dd} \ar{rr} & & R\Gamma_\fg (X, \QQ (n)) \ar{dl}[swap]{0} \ar[dashed]{dd}{\exists!} \ar[crossing over,near start]{rr}{\cong} & & {[R\Gamma (X_\et, \ZZ^c (n)), \QQ[-1]]} \ar{dd} \\
        & & R\Gamma_c (G_\RR, X (\CC), \QQ (n)) \ar[<-,near end,crossing over]{uu} & & R\Gamma_\Wc (Z, \QQ (n))\ar{dl}\ar{dd} \\
        {[R\Gamma (Z_\et, \ZZ^c (n)), \QQ[-2]]} \ar{r}\ar{dd} & 0 \ar{dr} \ar{dd} \ar{rr} & & R\Gamma_\fg (Z, \QQ (n)) \ar{dl}[swap]{0} \ar[dashed]{dd}{\exists!} \ar[crossing over,near start]{rr}{\cong} & & {[R\Gamma (Z_\et, \ZZ^c (n)), \QQ[-1]]} \ar{dd} \\
        & & R\Gamma_c (G_\RR, Z (\CC), \QQ (n)) \ar[<-,near end,crossing over]{uu} & & R\Gamma_\Wc (U, \QQ (n)) [1]\ar{dl} \\
        {[R\Gamma (U_\et, \ZZ^c (n)), \QQ[-1]]} \ar{r} & 0 \ar{dr} \ar{rr} & & R\Gamma_\fg (U, \QQ (n)) [1] \ar{dl}[swap]{0} \ar{rr}{\cong} & & {[R\Gamma (U_\et, \ZZ^c (n)), \QQ]} \\
        & & R\Gamma_c (G_\RR, U (\CC), \QQ (n)) [1] \ar[<-,near end,crossing over]{uu} \\
      \end{tikzcd} \]

    \caption{Diagram induced by a closed-open decomposition
      $Z \not\hookrightarrow X \hookleftarrow U$, tensored with $\QQ$}
    \label{fig:RGamma-Wc-and-closed-open-decompositions-otimes-Q}
  \end{figure}
\end{landscape}

\begin{landscape}
  \begin{figure}
    \[ \begin{tikzcd}[font=\small]
        &[2em] &[-2.5em] &[-2.5em] &[-2.5em] R\Gamma_\Wc (\AA^r_X, \ZZ (n))\ar{dl}\ar[dashed,near start]{dd}{\cong} &[-2.5em] \\
        {[R\Gamma (\AA^r_{X,\et}, \ZZ^c (n)), \QQ[-2]]} \ar{r}{\alpha_{\AA^r_X,n}}\ar{dd}{(p^*)^\vee}[swap]{\cong} & R\Gamma_c (\AA^r_{X,\et}, \ZZ (n)) \ar{dr}{u_\infty} \ar{dd}{p_*}[swap]{\cong} \ar{rr} & & R\Gamma_\fg (\AA^r_X, \ZZ (n))\ar{dl}[swap]{i_\infty} \ar[dashed]{dd}{\cong} \ar[crossing over]{rr} & & {[+1]} \ar{dd}{\cong} \\
        & & R\Gamma_c (G_\RR, \AA^r_X (\CC), \ZZ (n)) & & R\Gamma_\Wc (X, \ZZ (n-r)) [-2r]\ar{dl} \\
        {[R\Gamma (X_\et, \ZZ^c (n-r)) [2r], \QQ[-2]]} \ar{r}{\alpha_{X,n-r} [-2r]} & R\Gamma_c (X_\et, \ZZ (n-r)) [-2r] \ar{dr}{u_\infty} \ar{rr} & & R\Gamma_\fg (X, \ZZ (n-r)) [-2r] \ar{dl}[swap]{i_\infty} \ar{rr} & & {[+1]} \\
        & & R\Gamma_c (G_\RR, X (\CC), \ZZ (n-r)) [-2r] \ar[<-,near end,crossing over]{uu}[swap]{p_*}{\cong}\\
        & & \otimes_\ZZ \QQ = \\
        & & & & R\Gamma_\Wc (\AA^r_X, \QQ (n))\ar{dl}\ar[dashed,near start]{dd}{\cong} & \\
        {[R\Gamma (\AA^r_{X,\et}, \ZZ^c (n)), \QQ[-2]]} \ar{r}\ar{dd}{(p^*)^\vee}[swap]{\cong} & 0 \ar{dr} \ar{dd} \ar{rr} & & R\Gamma_\fg (\AA^r_X, \QQ (n))\ar{dl}[swap]{0} \ar[dashed]{dd}{\cong} \ar[crossing over]{rr} & & {[+1]} \ar{dd}{\cong} \\
        & & R\Gamma_c (G_\RR, \AA^r_X (\CC), \QQ (n)) & & R\Gamma_\Wc (X, \QQ (n-r)) [-2r]\ar{dl} \\
        {[R\Gamma (X_\et, \ZZ^c (n-r)) [2r], \QQ[-2]]} \ar{r} & 0 \ar{dr} \ar{rr} & & R\Gamma_\fg (X, \QQ (n-r)) [-2r] \ar{dl}[swap]{0} \ar{rr} & & {[+1]} \\
        & & R\Gamma_c (G_\RR, X (\CC), \QQ (n-r)) [-2r] \ar[<-,near end,crossing over]{uu}[swap]{p_*}{\cong}
      \end{tikzcd} \]

    \caption{Isomorphism
      $R\Gamma_\Wc (\AA^r_X, \ZZ (n)) \cong R\Gamma_\Wc (X, \ZZ (n-r)) [-2r]$
      and its splitting after tensoring with $\QQ$}
    \label{fig:RGamma-Wc-and-affine-bundles}
  \end{figure}
\end{landscape}

%%%%%%%%%%%%%%%%%%%%%%%%%%%%%%%%%%%%%%%%%%%%%%%%%%%%%%%%%%%%%%%%%%%%%%%%%%%%%%%%

\section{Unconditional results}
\label{sec:unconditional-results}

Now we apply theorem~\ref{thm:compatibility-of-C(X,n)} from the previous section
in order to prove the main theorem stated in the introduction: the validity of
$\mathbf{VO} (X,n)$ and $\mathbf{C} (X,n)$ for all $n < 0$ for cellular schemes
over certain $1$-dimensional bases. In fact, we will construct an even bigger
class of schemes $\mathcal{C} (\ZZ)$ whose elements satisfy the conjectures.
This approach is motivated by \cite[\S 5]{Morin-2014}.

\begin{definition}
  Let $\mathcal{C} (\ZZ)$ be the full subcategory of the category of arithmetic
  schemes generated by the following objects:
  \begin{itemize}
  \item the empty scheme $\emptyset$,
  \item $\Spec \FF_q$ for a finite field,
  \item $\Spec \mathcal{O}_F$ for an abelian number field $F/\QQ$,
  \item curves over finite fields $C/\FF_q$,
  \end{itemize}
  and the following operations.
  \begin{enumerate}
  \item[$\mathcal{C}0)$] $X$ lies in $\mathcal{C} (\ZZ)$ if and only if $X_\red$
    lies in $\mathcal{C} (\ZZ)$.

  \item[$\mathcal{C}1)$] A finite disjoint union $\coprod_{1 \le i \le r} X_i$
    lies in $\mathcal{C} (\ZZ)$ if and only if each $X_i$ lies in
    $\mathcal{C} (\ZZ)$.

  \item[$\mathcal{C}2)$] Let $Z \not\hookrightarrow X \hookleftarrow U$ be a
    closed-open decomposition such that
    $Z_{\red,\CC}$, $X_{\red,\CC}$, $U_{\red,\CC}$ are smooth and
    quasi-projective. Then if two out of three schemes $Z,X,U$ lie in
    $\mathcal{C} (\ZZ)$, then the third lies as well.

  \item[$\mathcal{C}3)$] If $X$ lies in $\mathcal{C} (\ZZ)$, then the affine
    space $\AA^r_X$ also lies in $\mathcal{C} (\ZZ)$ for each $r \ge 0$.
  \end{enumerate}
\end{definition}

We recall that the condition that $X_{\red,\CC}$ is smooth and quasi-projective
is needed to ensure that the regulator morphism exists
(see remark~\ref{rmk:regulator-is-defined-for-XC-smooth-quasi-proj}).

\begin{proposition}
  \label{prop:C(X,n)-holds-for-C(Z)}
  The conjectures $\mathbf{VO} (X,n)$ and $\mathbf{C} (X,n)$ hold for any
  $X \in \mathcal{C} (\ZZ)$ and $n < 0$.

  \begin{proof}
    Finite fields satisfy $\mathbf{C} (X,n)$ by
    example~\ref{example:C(X,n)-for-Spec-Fq}.

    If $X = \Spec \mathcal{O}_F$ for an abelian number field $F/\QQ$, then the
    conjecture $\mathbf{C} (X,n)$ is equivalent to the conjecture of Flach and
    Morin \cite[Conjecture~5.12]{Flach-Morin-2018}, which holds unconditionally
    in this particular case, via reduction to the Tamagawa number conjecture;
    see \cite[\S 5.8.3]{Flach-Morin-2018}, in particular
    [ibid., Proposition~5.35]. The condition $\mathbf{VO} (X,n)$ is also true in
    this case (see example~\ref{example:VO(X,n)-for-number-rings}).

    If $X = C/\FF_q$ is a curve over a finite field, then $\mathbf{C} (X,n)$
    holds thanks to theorem~\ref{thm:C(X,n)-over-finite-fields}. The conjecture
    $\mathbf{L}^c (X_\et,n)$ is well-known for curves and essentially goes back
    to Soulé; see for instance \cite[Proposition~4.3]{Geisser-2017}.

    Then the fact that the conjectures $\mathbf{L}^c (X_\et,n)$,
    $\mathbf{B} (X,n)$, $\mathbf{VO} (X,n)$, $\mathbf{C} (X,n)$ are closed under
    the operations $\mathcal{C}1)$--$\mathcal{C}3)$ is
    lemma~\ref{lemma:compatibility-of-Lc(X,n)},
    lemma~\ref{lemma:compatibility-of-B(X,n)},
    proposition~\ref{prop:compatibility-of-VO(X,n)},
    and theorem~\ref{thm:compatibility-of-C(X,n)}
    respectively.
  \end{proof}
\end{proposition}

\begin{lemma}
  Any $0$-dimensional arithmetic scheme $X$ lies in $\mathcal{C} (\ZZ)$.

  \begin{proof}
    Since $X$ is a noetherian scheme of dimension $0$, it is a finite disjoint
    union of $\Spec A_i$ for some artinian local rings $A_i$. Thanks to
    $\mathcal{C}1)$, we may assume that $X = \Spec A$, and thanks to
    $\mathcal{C}0)$, we may assume that $X$ is reduced. But then $A = k$ is a
    field. Since $X$ is a scheme of finite type over $\Spec \ZZ$, we conclude
    that $X = \Spec \FF_q \in \mathcal{C} (\ZZ)$.
  \end{proof}
\end{lemma}

\begin{proposition}
  \label{prop:particular-cases-1-dim-base}
  Let $B$ be a $1$-dimensional arithmetic scheme. Assume that each of the
  generic points $\eta \in B$ satisfies one of the following properties:
  \begin{enumerate}
  \item[a)] $\fchar \kappa (\eta) = p > 0$;

  \item[b)] $\fchar \kappa (\eta) = 0$, and $\kappa (\eta)/\QQ$ is an abelian
    number field.
  \end{enumerate}
  Then $B \in \mathcal{C} (\ZZ)$.

  \begin{proof}
    We will see that such a scheme can be obtained from $\Spec \mathcal{O}_F$
    for an abelian number field $F/\QQ$, or a curve over a finite field
    $C/\FF_q$, using the operations $\mathcal{C}0)$, $\mathcal{C}1)$,
    $\mathcal{C}2)$ that appear in the definition of $\mathcal{C} (\ZZ)$.

    Thanks to $\mathcal{C}0)$, we may assume that $B$ is reduced. Consider the
    normalization $\nu\colon B' \to B$. This is a birational morphism, and there
    exist open dense subsets $U' \subseteq B'$ and $U \subseteq B$ such that
    $\left.\nu\right|_{U'}\colon U' \xrightarrow{\cong} U$. Now $B\setminus U$
    is $0$-dimensional, and therefore $B\setminus U \in \mathcal{C} (\ZZ)$ by
    the previous lemma. Thanks to $\mathcal{C}2)$, it is enough to check that
    $U' \in \mathcal{C} (\ZZ)$, and this would imply $B \in \mathcal{C} (\ZZ)$.

    Now $U'$ is a finite disjoint union of normal integral schemes, so by
    $\mathcal{C}1)$ we may assume that $U'$ is integral. Consider the generic
    point $\eta \in U'$ and the residue field $F = \kappa (\eta)$. There are two
    distinct cases to consider.

    \begin{enumerate}
    \item[a)] If $\fchar F = p > 0$, then $U'$ is a curve over a finite field,
      so it lies in $\mathcal{C} (\ZZ)$ by the definition.

    \item[b)] If $\fchar F = 0$, then by our assumptions, $F/\QQ$ is an abelian
      number field.

      We note that if $V' \subseteq U'$ is an affine open neighborhood of
      $\eta$, then $U'\setminus V' \in \mathcal{C} (\ZZ)$ by the previous
      lemma. Therefore, we can assume without loss of generality that $U'$ is
      affine.

      We have $U' = \Spec \mathcal{O}$, where $\mathcal{O}$ is a finitely
      generated integrally closed domain. All this means that
      $\mathcal{O}_F \subseteq \mathcal{O} = \mathcal{O}_{F,S}$ for some finite
      set $S$. Now $U' = \Spec \mathcal{O}_F \setminus S$, and
      $S \in \mathcal{C} (\ZZ)$, so again, everything reduces to the case of
      $U' = \Spec \mathcal{O}_F$, which lies in $\mathcal{C} (\ZZ)$ by the
      definition. \qedhere
    \end{enumerate}
  \end{proof}
\end{proposition}

\begin{remark}
  Schemes as above were considered by Jordan and Poonen in
  \cite{Jordan-Poonen-2020}, where the authors write down a special value
  formula at $s = 1$, generalizing the classical class number formula. Namely,
  they consider the case of $B$ reduced and affine, albeit without requiring
  $\kappa (\eta)/\QQ$ to be abelian.
\end{remark}

\begin{example}
  If $B = \Spec \mathcal{O}$ for a nonmaximal order
  $\mathcal{O} \subset \mathcal{O}_F$, where $F/\QQ$ is an abelian number field,
  then our formalism gives a cohomological interpretation of the special values
  of $\zeta_\mathcal{O} (s)$ at $s = n < 0$. This already seems to be a new
  result.
\end{example}

\begin{definition}
  \label{dfn:B-cellular-scheme}
  Let $X \to B$ be a $B$-scheme. We say that $X$ is \textbf{$B$-cellular} if it
  admits a filtration by closed subschemes
  \begin{equation}
    \label{eqn:cellular-decomposition}
    X = Z_N \supseteq Z_{N-1} \supseteq \cdots \supseteq Z_0 \supseteq Z_{-1} = \emptyset
  \end{equation}
  such that $Z_i\setminus Z_{i-1} \cong \coprod_j \AA^{r_{i_j}}_B$ is a finite
  union of affine $B$-spaces.
\end{definition}

For instance, projective spaces $\PP^r_B$ and in general Grassmannians
$\Gr (k,\ell)_B$ are cellular. Many interesting examples of cellular schemes as
above arise from actions of algebraic groups on varieties and Białynicki-Birula
theorem; for this see \cite{Wendt-2010} and \cite{Brosnan-2005}.

\begin{proposition}
  \label{prop:cellular-schemes-in-C(Z)}
  Let $X$ be a $B$-cellular arithmetic scheme, where $B \in \mathcal{C} (\ZZ)$,
  and $X_{\red,\CC}$ is smooth and quasi-projective. Then
  $X \in \mathcal{C} (\ZZ)$.

  \begin{proof}
    Looking at the corresponding cellular decomposition
    \eqref{eqn:cellular-decomposition}, we better pass to open complements
    $U_i = X\setminus Z_i$, to obtain a filtration
    $$X = U_{-1} \supseteq U_1 \supseteq \cdots \supseteq U_{N-1} \supseteq U_N = \emptyset,$$
    with $U_{i,\CC}$ smooth and quasi-projective, being \emph{open}
    subvarieties in $X_\CC$. Now we have closed-open decompositions
    $\coprod_j \AA^{r_{i,j}}_B \not\hookrightarrow U_i \hookleftarrow U_{i+1}$,    
    and the claim follows by induction on the length of the cellular
    decomposition, using operations
    $\mathcal{C}1)$--$\mathcal{C}3)$.
  \end{proof}
\end{proposition}

As a corollary of the above, we obtain the following result, stated in the
introduction.

\begin{theorem}
  Let $B$ be a $1$-dimensional arithmetic scheme satisfying the assumptions of
  proposition~\ref{prop:particular-cases-1-dim-base}. If $X$ is a $B$-cellular
  arithmetic scheme with smooth and quasi-projective fiber $X_{\red,\CC}$, then
  the conjectures $\mathbf{VO} (X,n)$ and $\mathbf{C} (X,n)$ hold
  unconditionally for any $n < 0$.

  \begin{proof}
    Follows from propositions
    \ref{prop:C(X,n)-holds-for-C(Z)},
    \ref{prop:particular-cases-1-dim-base},
    \ref{prop:cellular-schemes-in-C(Z)}.
  \end{proof}
\end{theorem}

%%%%%%%%%%%%%%%%%%%%%%%%%%%%%%%%%%%%%%%%%%%%%%%%%%%%%%%%%%%%%%%%%%%%%%%%%%%%%%%%

\pagebreak

\begin{appendices}
\section{Determinants of complexes}
\label{app:determinants}

In this appendix we include a brief overview of determinants of complexes.
The original construction is due to Knudsen and Mumford
\cite{Knudsen-Mumford-1976}, and other useful expositions may be found in
\cite[Appendix~A]{Gelfand-Kapranov-Zelevinsky-1994} and
\cite[\S 2.1]{Kato-1993}.

For our purposes, let $R$ be a commutative ring, which is an integral domain
(we will be interested in $R = \ZZ$, $\QQ$, $\RR$). Denote by
$\mathcal{P}_\is (R)$\footnote{$\mathcal{P}$ for ``Picard''.} the category of
graded invertible $R$-modules. It has as its objects pairs $(L,r)$, where $L$ is
an invertible $R$-module (= projective of rank $1$) and $r \in \ZZ$. The
morphisms in this category are given by
\[ \Hom_{\mathcal{P}_\is (R)} ((L,r), (M,s)) = \begin{cases}
    \Isom_R (L,M), & r = s, \\
    \emptyset, & r \ne s.
  \end{cases} \]
This category is equipped with tensor products
$$(L,r) \otimes_R (M,s) = (L\otimes_R M, r + s)$$
with commutativity isomorphisms
\begin{align*}
  \psi\colon (L,r) \otimes_R (M,s) & \xrightarrow{\cong}
                                     (M,s) \otimes_R (L,r), \\
  \ell \otimes m & \mapsto (-1)^{r\,s}\,m\otimes \ell
\end{align*}
for $\ell \in L$, $m \in M$.

The unit object with respect to this product is $\bone = (R,0)$, and for each
$(L,r) \in \mathcal{P}_\is (R)$ the inverse is given by $(L^{-1}, -r)$ where
$L^{-1} = \iHom_R (L,R)$. The canonical evaluation morphism
$L \otimes_R \iHom_R (L,R) \to R$ induces an isomorphism
$$(L,r) \otimes_R (L^{-1}, -r) \cong \bone.$$

\begin{definition}
  \label{dfn:determinant-of-projective-fg-module}
  We denote by $\mathcal{C}_\is (R)$ the category whose objects are finitely
  generated projective $R$-modules and whose morphisms are isomorphisms.  For
  $A \in \mathcal{C}_\is (R)$ we define the corresponding determinant as an
  object in $\mathcal{P}_\is (R)$ given by
  $$\det_R (A) = \Bigl(\bigwedge^{\rk_R A}_R A, \rk_R A\Bigr).$$
  Here $\rk_R A$ is the rank of $A$, so that the top exterior power
  $\bigwedge^{\rk_R A}_R A$ is an invertible $R$-module.
\end{definition}

This gives a functor $\det_R\colon \mathcal{C}_\is (R) \to \mathcal{P}_\is (R)$.
The main result of \cite[Chapter~I]{Knudsen-Mumford-1976} is that this
construction may be generalized as follows. Let $\mathbf{D} (R)$ be the derived
category of the category of $R$-modules. Recall that a complex $A^\bullet$ is
\textbf{perfect} if it is quasi-isomorphic to a bounded complex of finitely
generated projective $R$-modules. We denote by $\Parf_\is (R)$ the subcategory
of $\mathbf{D} (R)$ whose objects consist of perfect complexes, and whose
morphisms are quasi-isomorphisms of complexes.

\begin{theorem}[Knudsen--Mumford]
  The determinant may be extended to perfect complexes of $R$-modules as
  follows.

  \begin{enumerate}
  \item[I)] For every ring $R$ there exists a functor
    $$\det_R\colon \Parf_\is (R) \to \mathcal{P}_\is (R)$$
    such that $\det_R (0) = \bone$.

  \item[II)] For every short exact sequence of complexes in $\Parf_\is (R)$
    \[ 0 \to A^\bullet \xrightarrow{\alpha} B^\bullet
      \xrightarrow{\beta} C^\bullet \to 0 \]
    there exists an isomorphism
    \[ i_R (\alpha,\beta)\colon
      \det_R A^\bullet \otimes_R \det_R C^\bullet
      \xrightarrow{\cong} \det_R B^\bullet. \]
    In particular, for the short exact sequence
    \[ \begin{tikzcd}[column sep=1em]
        &[-1.5em] 0 \ar{r} & A^\bullet \ar{r}{id} & A^\bullet \ar{r} & 0^\bullet \ar{r} & 0 &[-1.5em] \\[-2em]
        \text{(resp.} & 0 \ar{r} & 0^\bullet \ar{r} & A^\bullet \ar{r}{id} & A^\bullet \ar{r} & 0 & \text{)}
      \end{tikzcd} \]
    the isomorphism $i_R (id,0)$ (resp. $i_R (0,id)$) is the canonical isomorphism
    $$\det_R A^\bullet \otimes_R \bone \xrightarrow{\cong} \det_R A^\bullet$$
  \end{enumerate}

  Moreover, the following properties hold.

  \begin{enumerate}
  \item[i)] Given an isomorphism of short exact sequences of complexes
    \[ \begin{tikzcd}
        0 \ar{r} & A^\bullet \ar{r}{\alpha}\ar{d}{u}[swap]{\cong} & B^\bullet \ar{r}{\beta}\ar{d}{v}[swap]{\cong} & C^\bullet \ar{r}\ar{d}{w}[swap]{\cong} & 0 \\
        0 \ar{r} & A'^\bullet \ar{r}{\alpha'} & B'^\bullet \ar{r}{\beta'} & C'^\bullet \ar{r} & 0
      \end{tikzcd} \]
    the diagram
    \[ \begin{tikzcd}[column sep=5em]
        \det_R A^\bullet \otimes_R \det_R C^\bullet \ar{r}{i^* (\alpha,\beta)}[swap]{\cong}\ar{d}{\det_R (u) \otimes \det_R (w)}[swap]{\cong} & \det_R B^\bullet\ar{d}{\det_R (v)}[swap]{\cong} \\
        \det_R A'^\bullet \otimes_R \det_R C'^\bullet \ar{r}{i^* (\alpha',\beta')}[swap]{\cong} & \det_R B'^\bullet
      \end{tikzcd} \]
    commutes.

  \item[ii)] Given a commutative $3\times 3$ diagram with rows and columns short
    exact sequences
    \[ \begin{tikzcd}[row sep=1em, column sep=1em]
        & 0\ar{d} & 0\ar{d} & 0\ar{d} \\
        0\ar{r} & A^\bullet\ar{r}{\alpha}\ar{d}{u} & B^\bullet\ar{r}{\beta}\ar{d}{u'} & C^\bullet\ar{r}\ar{d}{u''} & 0 \\
        0\ar{r} & A'^\bullet\ar{r}{\alpha'}\ar{d}{v} & B'^\bullet\ar{r}{\beta'}\ar{d}{v'} & C'^\bullet\ar{r}\ar{d}{v''} & 0 \\
        0\ar{r} & A''^\bullet\ar{r}{\alpha''}\ar{d} & B''^\bullet\ar{r}{\beta''}\ar{d} & C''^\bullet\ar{r}\ar{d} & 0 \\
        & 0 & 0 & 0
      \end{tikzcd} \]
    the diagram
    \[ \begin{tikzcd}[column sep=7em]
        \det_R A^\bullet \otimes_R \det_R C^\bullet \otimes_R \det_R A''^\bullet \otimes_R \det_R C''^\bullet \ar{r}{i_R (\alpha,\beta) \otimes i_R (\alpha'',\beta'')}[swap]{\cong}\ar{d}{id \otimes \psi \otimes id}[swap]{\cong} & \det_R B^\bullet \otimes_R \det_R B''^\bullet\ar{dd}{i_R (u',v')}[swap]{\cong} \\
        (\det_R A^\bullet \otimes_R \det_R A''^\bullet) \otimes_R (\det_R C^\bullet \otimes_R \det_R C''^\bullet)\ar{d}[swap]{\cong}{i_R (u,v) \otimes i_R (u'',v'')} \\
        \det_R A'^\bullet \otimes_R \det_R C'^\bullet \ar{r}{i_R (\alpha',\beta')}[swap]{\cong} & \det_R B'^\bullet
      \end{tikzcd} \]
    commutes.

  \item[iii)] $\det$ and $i$ commute with base change. Namely, given a ring
    homomorphism $f\colon R\to S$, there is a natural isomorphism
    \[ \eta_A = \eta_{A^\bullet} (f)\colon
      \det_S (A^\bullet \otimes_R^\mathbf{L} S) \xrightarrow{\cong}
      (\det_R A^\bullet) \otimes_R S, \]
    such that for every short exact sequence of complexes
    \[ 0 \to A^\bullet \xrightarrow{\alpha} B^\bullet \xrightarrow{\beta}
      C^\bullet \to 0 \]
    the diagram
    \[ \begin{tikzcd}[column sep=5em]
        \det_S (A^\bullet \otimes_R^\mathbf{L} S) \otimes_S \det_S (C^\bullet \otimes_R^\mathbf{L} S) \ar{d}{\eta_A \otimes \eta_C}[swap]{\cong}\ar{r}{i_S (\alpha\otimes S, \beta\otimes S)}[swap]{\cong} & \det_S (B^\bullet \otimes_R^\mathbf{L} S) \ar{d}{\eta_B}[swap]{\cong} \\
        \Bigl((\det_R A^\bullet) \otimes_R S\Bigr) \otimes_S \Bigl((\det_R C^\bullet) \otimes_R S\Bigr) \ar{r}{i_R (\alpha,\beta) \otimes S}[swap]{\cong} & (\det_R B^\bullet) \otimes_R S
      \end{tikzcd} \]
    commutes. Similarly, there is compatibility with compositions of base
    changes along $R \xrightarrow{f} S \xrightarrow{g} T$ (we omit the
    corresponding commutative diagram).
  \end{enumerate}
\end{theorem}

\begin{remark}
  We refer to \cite{Knudsen-Mumford-1976} for the actual construction.
  In practice, the following considerations are useful; see [ibid.] for the
  proofs.

  \begin{enumerate}
  \item[1)] If $A^\bullet$ is a bounded complex where each object $A^i$ is
    perfect (i.e. admits a finite length resolution by finitely generated
    projective $R$-modules), then
    $$\det_R A^\bullet \cong \bigotimes_{i\in \ZZ} (\det_R A^i)^{(-1)^i}.$$
    In particular, if each $A^i$ is already a finitely generated projective
    $R$-module, then $\det_R A^i$ in the above formula is given by
    \ref{dfn:determinant-of-projective-fg-module}.

  \item[2)] If the cohomology modules $H^i (A^\bullet)$ are perfect, then
    \[ \det_R A^\bullet \cong
      \bigotimes_{i\in \ZZ} (\det_R H^i (A^\bullet))^{(-1)^i}. \]
  \end{enumerate}
\end{remark}

The determinants also behave well not only with short exact sequences, but with
distinguished triangles.

\begin{proposition}[{\cite[Proposition~7]{Knudsen-Mumford-1976}}]
  \label{prop:det-and-isos-of-triangles}
  For a distinguished triangle of complexes in $\Parf_\is (R)$
  \[ A^\bullet \xrightarrow{u} B^\bullet \xrightarrow{v} C^\bullet
    \xrightarrow{w} A^\bullet [1] \]
  there is a canonical isomorphism
  \[ i_R (u,v,w)\colon \det_R A^\bullet \otimes_R \det_R C^\bullet
    \xrightarrow{\cong} \det_R B^\bullet, \]
  which is functorial in the following sense: given a (quasi-)isomorphism of
  distinguished triangles
  \[ \begin{tikzcd}
      A^\bullet \ar{r}{u}\ar{d}{f}[swap]{\cong} & B^\bullet \ar{r}{v}\ar{d}{g}[swap]{\cong} & C^\bullet \ar{r}{w}\ar{d}{h}[swap]{\cong} & A^\bullet [1]\ar{d}{f[1]}[swap]{\cong} \\
      A'^\bullet \ar{r}{u} & B'^\bullet \ar{r}{v} & C'^\bullet \ar{r}{w} & A'^\bullet [1]
    \end{tikzcd} \]
  the diagram
  \[ \begin{tikzcd}[column sep=5em]
      \det_R A^\bullet \otimes_R \det_R C^\bullet \ar{r}{i_R (u,v,w)}[swap]{\cong}\ar{d}{\det_R (f) \otimes \det_R (h)}[swap]{\cong} & \det_R B^\bullet\ar{d}{\det_R (g)}[swap]{\cong} \\
      \det_R A'^\bullet \otimes_R \det_R C'^\bullet \ar{r}{i_R (u',v',w')}[swap]{\cong} & \det_R B'^\bullet
    \end{tikzcd} \]
  commutes.
\end{proposition}

\begin{remark}
  In what follows, for $(L,r) \in \mathcal{P}_\is (R)$ we will forget about
  $r$ and treat the determinant as an invertible $R$-module.
\end{remark}

A particular very simple case of interest is when all cohomology groups
$H^i (A^\bullet)$ are finite; then it is easy to understand what the determinant
means.

\begin{lemma}
  \label{lemma:determinant-for-torsion-cohomology}
  ~

  \begin{enumerate}
  \item[1)] Let $A$ be a finite abelian group. Then
    \[ (\det_\ZZ A) \subset (\det_\ZZ A) \otimes_\ZZ \QQ
      \cong \det_\QQ (A \otimes_\ZZ \QQ) = \det_\QQ (0) \cong \QQ \]
    corresponds to the fractional ideal $\frac{1}{\# A} \ZZ \subset \QQ$.

  \item[2)] In general, let $A^\bullet$ be a perfect complex of abelian groups
    such that the cohomology groups $H^i (A^\bullet)$ are all finite. Then
    $\det_\ZZ A^\bullet$ corresponds to the fractional ideal
    $\frac{1}{m}\,\ZZ \subset \QQ$, where
    $$m = \prod_{i\in \ZZ} |H^i (A^\bullet)|^{(-1)^i}.$$
  \end{enumerate}

  \begin{proof}
    Since $\det_\ZZ (A\oplus B) \cong \det_\ZZ A \otimes_\ZZ \det_\ZZ B$, in
    part 1) it would be enough to consider the case of a cyclic group
    $A = \ZZ/m\ZZ$. Then we have a quasi-isomorphism of complexes
    \[ \ZZ/m\ZZ [0] \cong \Bigl[
      \mathop{m\ZZ}_{\text{deg.\,}-1} \hookrightarrow
      \mathop{\ZZ}_{\text{deg.\,}0}
      \Bigr]. \]
    Therefore,
    $$\det_\ZZ (\ZZ/m\ZZ) \cong \ZZ \otimes_\ZZ (m\ZZ)^{-1} \cong (m\ZZ)^{-1},$$
    which corresponds to $\frac{1}{m}\,\ZZ$ inside $\QQ$.
    Part 2) follows immediately from 1) using the isomorphism
    \[ \det_\ZZ A^\bullet \cong
      \bigotimes_{i\in \ZZ} (\det_\ZZ H^i (A^\bullet))^{(-1)^i}. \qedhere \]
  \end{proof}
\end{lemma}

\begin{remark}
  The above argument works in a more general setting, assuming $R$ is a regular
  noetherian ring and $A$ is a finitely generated torsion $R$-module (replacing
  $\QQ$ with the total quotient field $Q (R)$).
\end{remark}

\end{appendices}

%%%%%%%%%%%%%%%%%%%%%%%%%%%%%%%%%%%%%%%%%%%%%%%%%%%%%%%%%%%%%%%%%%%%%%%%%%%%%%%%

\pagebreak
\bibliographystyle{amsalpha-cust}
\bibliography{weil-etale}

\end{document}
