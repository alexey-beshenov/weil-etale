\documentclass{article}

\usepackage[utf8]{inputenc}
\usepackage{fullpage}

\usepackage[titletoc]{appendix}
\usepackage[russian,english]{babel}

\usepackage{amsmath,amssymb}

\usepackage{pdflscape}

%%%%%%%%%%%%%%%%%%%%%%%%%%%%%%%%%%%%%%%%%%%%%%%%%%%%%%%%%%%%%%%%%%%%%%%%%%%%%%%%

\DeclareMathOperator{\Cone}{Cone}
\DeclareMathOperator{\coker}{coker}
\DeclareMathOperator{\Ext}{Ext}
\DeclareMathOperator{\fchar}{char}
\DeclareMathOperator{\Fil}{Fil}
\DeclareMathOperator{\Gal}{Gal}
\DeclareMathOperator{\Hom}{Hom}
\DeclareMathOperator{\im}{im}
\DeclareMathOperator{\Isom}{Isom}
\DeclareMathOperator{\ord}{ord}
\DeclareMathOperator{\Pic}{Pic}
\DeclareMathOperator{\rk}{rk}
\DeclareMathOperator{\Spec}{Spec}
\DeclareMathOperator{\Tot}{Tot}
\DeclareMathOperator{\vol}{vol}

%%%%%%%%%%%%%%%%%%%%%%%%%%%%%%%%%%%%%%%%%%%%%%%%%%%%%%%%%%%%%%%%%%%%%%%%%%%%%%%%

\newcommand{\CC}{\mathbb{C}}
\newcommand{\FF}{\mathbb{F}}
\newcommand{\NN}{\mathbb{N}}
\newcommand{\QQ}{\mathbb{Q}}
\newcommand{\RR}{\mathbb{R}}
\newcommand{\ZZ}{\mathbb{Z}}

\renewcommand{\AA}{\mathbb{A}}
\newcommand{\PP}{\mathbb{P}}

\DeclareMathOperator{\Gr}{Gr}

\newcommand{\bone}{1\!\!1}

\newcommand{\Parf}{\mathcal{P}\!\text{\it arf}}

% force \nolimits on \det:
\renewcommand{\det}{\operatorname{det}}

\renewcommand{\Re}{\operatorname{Re}}
\renewcommand{\emptyset}{\varnothing}

\newcommand{\ar}{\text{\it ar}}
\newcommand{\BM}{\text{\it BM}}
\newcommand{\DB}{{\mathcal{D}\text{-\foreignlanguage{russian}{Б}}}}
\newcommand{\dR}{\text{\it dR}}
\newcommand{\et}{\text{\it ét}}
\newcommand{\fg}{\text{\it fg}}
\newcommand{\fin}{\text{\it fin}}
\newcommand{\is}{\text{\it is}}
\newcommand{\red}{\text{\it red}}
\newcommand{\tors}{\text{\it tors}}
\newcommand{\Wc}{\text{\it W,c}}
\newcommand{\Zar}{\text{\it Zar}}

\newcommand{\dfn}{\mathrel{\mathop:}=}
\newcommand{\rdfn}{=\mathrel{\mathop:}}

\newcommand{\iHom}{\underline{\Hom}}
\newcommand{\RHom}{R\!\Hom}

%%%%%%%%%%%%%%%%%%%%%%%%%%%%%%%%%%%%%%%%%%%%%%%%%%%%%%%%%%%%%%%%%%%%%%%%%%%%%%%%

\usepackage{tikz-cd}
\usetikzlibrary{arrows}
\usetikzlibrary{calc}
\usetikzlibrary{babel}
\usetikzlibrary{decorations.pathreplacing}

\newcommand{\tikzpb}{\ar[phantom,pos=0.2]{dr}{\text{\large$\lrcorner$}}}
\newcommand{\tikzpbur}{\ar[phantom,pos=0.2]{dl}{\text{\large$\llcorner$}}}

%%%%%%%%%%%%%%%%%%%%%%%%%%%%%%%%%%%%%%%%%%%%%%%%%%%%%%%%%%%%%%%%%%%%%%%%%%%%%%%%

\usepackage[numbers]{natbib}

\usepackage[hidelinks]{hyperref}

\hypersetup{
    colorlinks,
    linkcolor={red!60!black},
    citecolor={blue!60!black},
    urlcolor={blue!80!black}
}

%%%%%%%%%%%%%%%%%%%%%%%%%%%%%%%%%%%%%%%%%%%%%%%%%%%%%%%%%%%%%%%%%%%%%%%%%%%%%%%%

\usepackage{amsthm}

\newtheoremstyle{myplain}
{\topsep}   % ABOVESPACE
{\topsep}   % BELOWSPACE
{\itshape}  % BODYFONT
{0pt}       % INDENT (empty value is the same as 0pt)
{\bfseries} % HEADFONT
{.}         % HEADPUNCT
{5pt plus 1pt minus 1pt} % HEADSPACE
{\thmnumber{#2}. \thmname{#1}\thmnote{ (#3)}}   % CUSTOM-HEAD-SPEC

\newtheoremstyle{mydefinition}
{\topsep}   % ABOVESPACE
{\topsep}   % BELOWSPACE
{\normalfont}  % BODYFONT
{0pt}       % INDENT (empty value is the same as 0pt)
{\bfseries} % HEADFONT
{.}         % HEADPUNCT
{5pt plus 1pt minus 1pt} % HEADSPACE
{\thmnumber{#2}. \thmname{#1}\thmnote{ (#3)}}   % CUSTOM-HEAD-SPEC

\theoremstyle{plain}
\newtheorem{maintheorem}{Theorem}
\renewcommand*{\themaintheorem}{\Roman{maintheorem}}
\newtheorem*{maintheorem*}{Main theorem}
\newtheorem*{thetheorem*}{Theorem}
\newtheorem*{proposition*}{Proposition}

\theoremstyle{myplain}
\newtheorem{theorem}{Theorem}[section]
\newtheorem{proposition}[theorem]{Proposition}
\newtheorem{lemma}[theorem]{Lemma}
\newtheorem{corollary}[theorem]{Corollary}

\theoremstyle{definition}
\newtheorem*{conjecture*}{Conjecture}

\theoremstyle{mydefinition}
\newtheorem{definition}[theorem]{Definition}
\newtheorem{conjecture}[theorem]{Conjecture}
\newtheorem{remark}[theorem]{Remark}
\newtheorem{example}[theorem]{Example}

%%%%%%%%%%%%%%%%%%%%%%%%%%%%%%%%%%%%%%%%%%%%%%%%%%%%%%%%%%%%%%%%%%%%%%%%%%%%%%%%

\usepackage[perpage,symbol]{footmisc}
\renewcommand{\thefootnote}{\ifcase\value{footnote}\or{*}\or{**}\or{***}\or{****}\fi}


\title{Zeta-values of one-dimensional arithmetic schemes at $n < 0$}
\author{Alexey Beshenov}

\AtEndDocument{%
  \par
  \medskip
  \begin{tabular}{@{}l@{}}%
    \\
    Alexey Beshenov \\
    Center for Research in Mathematics (CIMAT), Guanajuato, Mexico \\
    % E-mail: \texttt{alexey.beshenov@cimat.mx} \\
    URL: \url{https://cadadr.org/}
  \end{tabular}}

\numberwithin{equation}{section}

\begin{document}

\maketitle

\begin{abstract}
  Let $X \to \Spec \ZZ$ be an arithmetic scheme (separated, of finite type) of
  Krull dimension $1$. We write down a formula for the special value of
  $\zeta (X,s)$ at $s = n < 0$, in terms of étale motivic cohomology of $X$ and
  a regulator. We prove it in the case when for each generic point $\eta \in X$
  with $\fchar \kappa (\eta) = 0$, the extension $\kappa (\eta)/\QQ$ is
  abelian. Further, we conjecture that the formula holds for any one-dimensional
  arithmetic scheme.

  This is a consequence of Weil-étale formalism developed by the author in
  \cite{Beshenov-Weil-etale-1,Beshenov-Weil-etale-2}, following the work of
  Flach and Morin \cite{Flach-Morin-2018}.
\end{abstract}

\tableofcontents

%%%%%%%%%%%%%%%%%%%%%%%%%%%%%%%%%%%%%%%%%%%%%%%%%%%%%%%%%%%%%%%%%%%%%%%%%%%%%%%%

\section{Introduction}

Let $X$ be an \textbf{arithmetic scheme}, by which we will mean throughout this
text that it is separated and of finite type over $\Spec \ZZ$.
The \textbf{zeta function} associated to $X$ (see e.g. \cite{Serre-1965}) is
given by
\[ \zeta (X,s) = \prod_{\substack{x \in X \\ \text{closed pt.}}}
  \frac{1}{1 - \#\kappa (x)^{-s}}, \]
where $\kappa (x) = \mathcal{O}_{X,x}/\mathfrak{m}_{X,x}$ denotes the residue
field of a point. The above product converges for $\Re s > \dim X$, and
conjecturally admits a meromorphic continuation to the whole complex plane.
Even though this is a wide open conjecture in general, this is well-known for
$\dim X = 1$, which will be the case of interest for us.

If $\zeta (X,s)$ admits a meromorphic continuation around $s = n$, then denote
by
\begin{equation}
  \label{eqn:vanishing-order}
  d_n = \ord_{s=n} \zeta (X,s)
\end{equation}
the vanishing order of $\zeta (X,s)$ at $s = n$. The corresponding special value
of $\zeta (X,s)$ at $s = n$ is defined to be the leading nonzero coefficient of
the Taylor expansion:
$$\zeta^* (X,n) = \lim_{s \to n} (s - n)^{-d_n}\,\zeta (X,s).$$

A primordial example of special value formulas is Dirichlet's
\textbf{analytic class number formula}. Namely, for a number field $F/\QQ$, we
will denote by $\mathcal{O}_F$ the corresponding ring of integers. Then
$\zeta_F (s) = \zeta (X,s)$ for $X = \Spec \mathcal{O}_F$ is the Dedekind zeta
function attached to $F$.  It is an easy consequence of the well-known
functional equation for $\zeta_F (s)$ that it has a zero at $s = 0$ of order
$r_1 + r_2 - 1$, where $r_1$ (resp. $r_2$) is the number of real embeddings
$F \hookrightarrow \RR$ (resp. conjugate pairs of complex embeddings
$F \hookrightarrow \CC$). The corresponding special value is given by
\begin{equation}
  \label{eqn:zeta-F-at-s=0}
  \zeta^*_F (0) = -\frac{h_F}{\omega_F}\,R_F,
\end{equation}
where $h_F = \# \Pic (\mathcal{O}_F)$ is the class number,
$\omega_F = \# (\mathcal{O}_F)^\times_\tors$ is the number of roots of unity in
$F$, and $R_F$ is the regulator, which is a positive real number;
see for instance \cite[\S VII.5]{Neukirch-1999}.

It is natural to ask whether such formulas exist for $s = n \in \ZZ$ other than
$0$ and $1$. For this one has to find a suitable generalization for the numbers
$h_F$ and $\omega_F$ and the regulator $R_F$. Many special value conjectures, of
varying generality, originate from this question.

Lichtenbaum proposed in his pioneering work \cite{Lichtenbaum-1973} formulas in
terms of algebraic $K$-theory. Later this was also restated in terms of
cohomology groups $H^i (\Spec \mathcal{O}_F [1/p]_\et, \ZZ_p (n))$
for $i = 1,2$ and all primes $p$, and the corresponding formula is known as
``cohomological Lichtenbaum conjecture''; see for instance
\cite{Huber-Kings-2003}. We will not go into the details, since instead of
working with $p$-adic cohomology
$H^i (\Spec \mathcal{O}_F [1/p]_\et, \ZZ_p (n))$ for varying $p$, it will more
convenient for us to use motivic cohomology.

A suitable generalization of $R_F$ are the ``higher regulators'' that have been
considered starting from Borel's work \cite{Borel-1977}, and later on by
Beilinson \cite{Beilinson-1984}.

We will not make any attempt at giving a proper historical overview of the
subject and writing down all conjectural formulas; the interested reader may
consult lecture notes \cite{Kolster-2004} and surveys
\cite{Goncharov-2005,Kahn-2005}.

\vspace{1em}

Lichtenbaum suggested a more recent research program, known as
\textbf{Weil-étale cohomology}; see e.g.
\cite{Lichtenbaum-2005,Lichtenbaum-2009-number-rings,Lichtenbaum-2009-Euler-char}.
It suggests that for an arithmetic scheme $X$ the special value of $\zeta (X,s)$
at $s = n \in \ZZ$ should be expressible in terms of Weil-étale cohomology
$H^i_\Wc (X,\ZZ(n))$, which is a suitable modification of the (étale) motivic
cohomology of $X$, that gives finitely generated abelian groups. Flach and Morin
gave in \cite{Flach-Morin-2018} an explicit construction of Weil-étale
cohomology for a proper and regular arithmetic scheme $X$, and stated a precise
conjectural relation of $H^i_\Wc (X,\ZZ(n))$ to $\zeta^* (X,n)$.

In particular, in \cite[\S 5.8.3]{Flach-Morin-2018} they write down explicitly
their special value formula for the case of $X = \Spec \mathcal{O}_F$. For us it
will be convenient to put it as
\begin{equation}
  \label{eqn:flach-morin-zeta-F-formula}
  \zeta_F^* (n) = \pm\frac{|H^0 (X_\et, \ZZ^c (n))|}{|H^{-1} (X_\et, \ZZ^c (n))_\tors|}\,R_{F,n}
  \quad \text{for }n \le 0.
\end{equation}
The definition of groups $H^i (X_\et, \ZZ^c (n))$ is reviewed below, and the
regulator $R_{F,n} = R_{\Spec \mathcal{O}_F,n}$ is defined in
\S\ref{sec:regulators}. According to \cite[Proposition~5.35]{Flach-Morin-2018},
the above formula holds unconditionally for abelian number fields $F/\QQ$, via
reduction to the Tamagawa number conjecture of
Bloch--Kato--Fontaine--Perrin-Riou.

In particular, if we take $n = 0$, then $\ZZ^c (0) \cong \mathbb{G}_m [1]$, and
$R_{F,0}$ is the usual regulator of Dirichlet, so that the formula becomes
\[ \zeta_F^* (0) =
  \pm \frac{|H^1 (\Spec \mathcal{O}_{F,\et}, \mathbb{G}_m)|}{|H^0 (\Spec \mathcal{O}_{F,\et}, \mathbb{G}_m)_\tors|}\,R_F =
  \pm \frac{|\Pic (\mathcal{O}_F)|}{|(\mathcal{O}_F)^\times_\tors|}\,R_F, \]
which is the classical formula \eqref{eqn:zeta-F-at-s=0}.

In case of $n < 0$, the author has extended in
\cite{Beshenov-Weil-etale-1,Beshenov-Weil-etale-2} the work of Flach and Morin
to any arithmetic scheme $X$ (thus removing the assumption that $X$ is proper or
regular). In this text we would like to work out explicitly the corresponding
special value formula for one-dimensional arithmetic schemes.

\vspace{1em}

In order to state an unconditional result, it will be convenient to introduce
the following definition.

\begin{definition}
  We say that a one-dimensional arithmetic scheme $X$ is \textbf{abelian} if
  each generic point $\eta \in X$ with $\fchar \kappa (\eta) = 0$ corresponds to
  an abelian extension $\kappa (\eta)/\QQ$.
\end{definition}

In particular, if $X$ lives entirely in positive characteristic, then it is
abelian. The goal is to prove the following result.

\begin{theorem}
  \label{main-theorem}
  For an abelian one-dimensional arithmetic scheme $X$, the special value of
  $\zeta (X,s)$ at $s = n < 0$ is given by
  \begin{equation}
    \label{eqn:special-value-formula}
    \zeta^* (X,n) =
    \pm 2^\delta\,\frac{|H^0 (X_\et, \ZZ^c (n)|}{|H^{-1} (X_\et, \ZZ^c (n))_\tors| \cdot |H^1 (X_\et, \ZZ^c (n))|}\,R_{X,n}.
  \end{equation}
  Here
  \begin{itemize}
  \item $H^i (X_\et, \ZZ^c (n))$ is a version of (étale) motivic cohomology,
    reviewed below;

  \item the ``correcting factor'' $2^\delta$ is given by
    \begin{equation}
      \label{eqn:delta}
      \delta = \delta_{X,n} =
      \begin{cases}
        r_1, & n \text{ even}, \\
        0, & n \text{ odd},
      \end{cases}
    \end{equation}
    and $r_1 = |X (\RR)|$ is the number of real places of $X$,

  \item $R_{X,n}$ is a positive real number, defined in \S\ref{sec:regulators}
    via Beilinson's regulator map.
  \end{itemize}
\end{theorem}

We further conjecture that the special value formula still holds for non-abelian
one-dimensional schemes. This is equivalent to Tamagawa number conjecture for
non-abelian number fields.

\vspace{1em}

We will give two proofs of \eqref{eqn:special-value-formula}. The first proof
can be found in \S\ref{sec:special-value-formula} after some preliminary
calculations of motivic and Weil-étale cohomology. We observe that the special
value formula is the same as conjecture $\mathbf{C} (X,n)$ that was stated in
\cite{Beshenov-Weil-etale-2}, specialized to $\dim X = 1$ and written out more
explicitly. The second proof in \S\ref{sec:demystification} is more elementary
and self-contained. Essentially it is the same argument, only written out in a
more explicit manner.

The purpose of this text is twofold. Firstly, we establish a new special value
formula, generalizing several formulas that can be found in the
literature. Secondly, we go through the construction of Weil-étale cohomology
$H^i_\Wc (X, \ZZ(n))$ from \cite{Beshenov-Weil-etale-1}, explaining it in the
case of one-dimensional schemes. It is not very surprising that such a special
value formula exists, but the right cohomological invariants to state it are
suggested by the Weil-étale framework.

This text was inspired by the work of Jordan and Poonen
\cite{Jordan-Poonen-2020}, where the authors write down a formula for
$\zeta^* (X,1)$, where $X$ is an affine reduced one-dimensional arithmetic
scheme. The affine and reduced restriction does not appear in our case, since we
work with different invariants. In particular, as
$\zeta (X,s) = \zeta (X_\red, s)$, the ``right'' invariants should not
distinguish between $X$ and $X_\red$, and motivic cohomology satisfies this
property.

\subsection*{Notation and conventions}

Throughout this text $X$ will always denote a one-dimensional
\textbf{arithmetic scheme}; that is, separated scheme of finite type
$X \to \Spec \ZZ$ of Krull dimension $1$. The number $n$ will be a fixed
\emph{strictly negative} integer.

\paragraph{Motivic cohomology.}
We will work with étale motivic cohomology defined in terms of Bloch's cycle
complexes. These were introduced by Bloch in \cite{Bloch-1986} for varieties
over fields, and for the version over $\Spec \ZZ$ that we will use, see
\cite{Geisser-2004-Dedekind,Geisser-2005}.

Namely, we let $\Delta^i = \Spec \ZZ [t_0,\ldots,t_i] / (1 - \sum_i t_i)$ be the
algebraic simplex. Denote by $z_n (X, i)$ the group freely generated by
algebraic cycles $Z \subset X \times \Delta^i$ of dimension $n+i$. Then for
$n < 0$ we let $\ZZ^c (n)$ be the complex of étale sheaves
$U \rightsquigarrow z_n (U, -\bullet) [2n]$. The corresponding (hyper)cohomology
$H^i (X_\et, \ZZ^c (n))$ is what we will call motivic cohomology throughout this
note. For a proper regular arithmetic scheme $X$ of dimension $d$ we have
$\ZZ^c (n) = \ZZ (d-n) [2d]$, where $\ZZ (n)$ is the other motivic complex that
usually appears in the literature.

We recall from \cite[Corollary~7.2]{Geisser-2010} that our motivic cohomology
satisfies the localization property: if $Z \subset X$ is a closed subscheme and
$U = X\setminus Z$ is its closed complement, then there is a distinguished
triangle
\[ R\Gamma (Z_\et, \ZZ^c (n)) \to
  R\Gamma (X_\et, \ZZ^c (n)) \to
  R\Gamma (U_\et, \ZZ^c (n)) \to 
  R\Gamma (Z_\et, \ZZ^c (n)) [1], \]
which gives a long exact sequence
\begin{equation}
  \label{eqn:localization-les}
  \cdots \to H^i (Z_\et, \ZZ^c (n)) \to
  H^i (X_\et, \ZZ^c (n)) \to
  H^i (U_\et, \ZZ^c (n)) \to 
  H^{i+1} (Z_\et, \ZZ^c (n)) \to \cdots
\end{equation}
This means that $H^i (-, \ZZ^c (n))$ behaves like (motivic) Borel--Moore
homology. For the corresponding zeta functions, we have
$$\zeta (X,s) = \zeta (Z,s)\,\zeta (U,s)$$
---this is clear from the definition.

In general, the groups $H^i (X_\et, \ZZ^c (n))$ are very difficult to
compute. However, these are rather well-understood for one-dimensional
arithmetic schemes $X$; see \S\ref{sec:motivic-cohomology-structure} below.

\paragraph{Real and complex places.}
Consider the finite discrete space of complex points
$X (\CC) = \Hom (\Spec \CC, X)$. There is a canonical action of complex
conjugation $G_\RR \dfn \Gal (\CC/\RR)$ on $X (\CC)$. The fixed points of this
action correspond to the real points $X (\RR)$. We will denote
$r_1 = |X (\RR)|$. The non-real points come in conjugate pairs, and we will
denote their number by $2 r_2$.

\[ \begin{tikzpicture}
    \matrix(m)[matrix of math nodes, row sep=1em, column sep=1em,
    text height=1ex, text depth=0.2ex]{
      ~ & ~ & ~ & ~ & ~ & \bullet & \bullet & \cdots & \bullet \\
      \bullet & \bullet & \cdots & \bullet \\
      ~ & ~ & ~ & ~ & ~ & \bullet & \bullet & \cdots & \bullet \\};

    \draw[->] (m-2-1) edge[loop above,min distance=10mm] (m-2-1);
    \draw[->] (m-2-2) edge[loop above,min distance=10mm] (m-2-2);
    \draw[->] (m-2-4) edge[loop above,min distance=10mm] (m-2-4);

    \draw[->] (m-1-6) edge[bend left] (m-3-6);
    \draw[->] (m-1-7) edge[bend left] (m-3-7);
    \draw[->] (m-1-9) edge[bend left] (m-3-9);

    \draw[->] (m-3-6) edge[bend left] (m-1-6);
    \draw[->] (m-3-7) edge[bend left] (m-1-7);
    \draw[->] (m-3-9) edge[bend left] (m-1-9);

    \draw ($(m-3-1)!.5!(m-3-4)$) node[yshift=-2em,anchor=base] {$r_1$ points};
    \draw ($(m-3-6)!.5!(m-3-9)$) node[yshift=-2em,anchor=base] {$2 r_2$ points};
  \end{tikzpicture} \]

Equivalently, for a number field $F/\QQ$, denote by $r_1 (F)$ the number of real
embeddings $F \hookrightarrow \RR$ and and by $r_2 (F)$ the number of pairs of
complex embeddings $F \hookrightarrow \CC$. Then $r_1 (F) = r_1$ and
$r_2 (F) = r_2$ for $X = \Spec \mathcal{O}_F$. In general, for a one-dimensional
arithmetic scheme $X$ we have
\begin{align*}
  r_1 & = \sum_{\fchar (\kappa (\eta)) = 0} r_1 (\kappa (\eta)), \\
  r_2 & = \sum_{\fchar (\kappa (\eta)) = 0} r_2 (\kappa (\eta)),
\end{align*}
where the sums are over generic points $\eta \in X$ with residue field
$\kappa (\eta)$ of characteristic $0$.

\vspace{1em}

It will be convenient to introduce the following notation.
\begin{definition}
  \label{dfn:dn}
  If $X$ is a one-dimensional arithmetic scheme with $r_1$ real and $2r_2$
  complex places, we define for $n < 0$
  \[ d_n =
    \begin{cases}
      r_1 + r_2, & n\text{ even}, \\
      r_2, & n\text{ odd}.
    \end{cases} \]
\end{definition}

There is no coincidence that this notation coincides with
\eqref{eqn:vanishing-order}; we will prove this
in proposition~\ref{prop:vanishing-order-equals-dn} below.

\paragraph{Abelian groups.}
We will say that an abelian group $A$ is of \textbf{cofinite type} if
$A \cong \Hom (B, \QQ/\ZZ)$ for some finitely generated abelian group $B$.  If
$B \cong \ZZ^{\oplus n} \oplus T$, where $T$ is finite torsion, then we have
$A \cong (\QQ/\ZZ)^{\oplus n} \oplus \Hom (T, \QQ/\ZZ)$. Here
\[ A_\div \dfn (\QQ/\ZZ)^{\oplus n}, \quad
  A_\codiv \dfn A/A_\div \cong \Hom (T, \QQ/\ZZ) \]
are respectively the divisible and \textbf{codivisible} part of $A$.

\subsection*{Outline of the paper}

In \S\ref{sec:devissage} we prove a dévissage lemma, which shows how a property
that holds for curves over finite fields and for number rings may be generalized
to any one-dimensional arithmetic scheme. This is an elementary argument, but it
is isolated in order not to repeat the same reasoning in several proofs.

Then in \S\ref{sec:motivic-cohomology-structure} we put together various
well-known results in order to to describe motivic cohomology groups
$H^i (X_\et, \ZZ^c(n))$ for one-dimensional arithmetic schemes.

The following \S\ref{sec:Weil-etale-cohomology-of-X} is dedicated to an explicit
description of Weil-étale cohomology groups $H^i_\Wc (X, \ZZ(n))$ that were
defined in \cite{Beshenov-Weil-etale-1}, again for the particular case of
one-dimensional $X$. In \S\ref{sec:regulators} we define the regulator, and in
\S\ref{sec:special-value-formula} the calculations of Weil-étale cohomology are
used to write down an explicit formula for the special value $\zeta^* (X,n)$,
which corresponds to the conjecture $\mathbf{C} (X,n)$ from
\cite{Beshenov-Weil-etale-2}. The main theorem is then deduced from the results
of \cite{Beshenov-Weil-etale-2}.

In \S\ref{sec:demystification} it is explained how to prove the formula
directly, using localization. This is essentially the argument from
\cite{Beshenov-Weil-etale-2}, only written out explicitly for one-dimensional
$X$.

We conclude in \S\ref{sec:examples} with a couple of examples that illustrate
how the special value formula works.

\subsection*{Acknowledgments}

I am grateful to Baptiste Morin for various discussions that led to this text.

%%%%%%%%%%%%%%%%%%%%%%%%%%%%%%%%%%%%%%%%%%%%%%%%%%%%%%%%%%%%%%%%%%%%%%%%%%%%%%%% 

\section{Dévissage lemma}
\label{sec:devissage}

The main idea of this text is to consider a property that holds for spectra of
number rings $X = \Spec \mathcal{O}_F$ and curves over finite fields $X/\FF_q$,
and then formally generalize it to any one-dimensional arithmetic scheme.
For this we isolate in this section a dévissage argument, which will be used
repeatedly in everything that follows.

\begin{lemma}
  \label{lemma:devissage}
  Let $\mathcal{P}$ be a property of arithmetic schemes of Krull dimension
  $\le 1$. Assume that it satisfies the following compatibilities.
  \begin{enumerate}
  \item[a)] $\mathcal{P} (X)$ holds if and only if $\mathcal{P} (X_\red)$ does.

  \item[b)] If $X = \coprod_i X_i$ is a finite disjoint union, then
    $\mathcal{P} (X)$ is equivalent to the conjunction of $\mathcal{P} (X_i)$ for
    all $i$.

  \item[c)] If $U \subset X$ is a dense open subscheme, then $\mathcal{P} (X)$
    is equivalent to $\mathcal{P} (U)$.
  \end{enumerate}
  Suppose that
  \begin{enumerate}
  \item[0)] $\mathcal{P} (\Spec \FF_q)$ holds for any finite field $\FF_q$,

  \item[1)] $\mathcal{P} (X)$ holds for any smooth curve $X/\FF_q$,

  \item[2)] $\mathcal{P} (\Spec \mathcal{O}_F)$ holds for any number field
    $F/\QQ$.
  \end{enumerate}
  Then $\mathcal{P} (X)$ holds for any one-dimensional arithmetic scheme $X$.

  \begin{proof}
    First suppose that $\dim X = 0$. Then thanks to a), we may suppose that $X$
    is reduced, and then $X = \coprod_i \Spec \FF_{q,i}$ is a finite disjoint
    union of spectra of finite fields, so that $\mathcal{P} (X)$ holds thanks to
    0) and b).

    Now consider the case of $\dim X = 1$. Again, we may assume that $X$ is
    reduced. Consider the normalization $\nu\colon X' \to X$. This is a
    birational morphism: there exist dense open subsets $U \subset X$ and
    $U' \subset X'$ such that
    $\left.\nu\right|_{U'}\colon U' \xrightarrow{\cong} U$ is an
    isomorphism. Thanks to c), we have
    \[ \mathcal{P} (X) \iff
      \mathcal{P} (U) \iff
      \mathcal{P} (U') \iff
      \mathcal{P} (X'). \]
    Therefore, we may assume that $X$ is regular. Now $X = \coprod_i X_i$ is a
    finite disjoint union of normal integral schemes, so thanks to b), we may
    assume that $X$ is integral. There are two cases.

    \begin{itemize}
    \item If $X \to \Spec \ZZ$ lives over a closed point, then it is a smooth
      curve over $\FF_q$, and the claim holds thanks to 1).

    \item If $X \to \Spec \ZZ$ is a dominant morphism, consider an open affine
      neighborhood of the generic point $U \subset X$. Again, $\mathcal{P} (X)$
      is equivalent to $\mathcal{P} (U)$, so it will be enough to prove the
      claim for $U$. We have $U = \Spec \mathcal{O}_{F,S}$ for some number field
      $F/\QQ$ and a finite set of places $S$, hence everything again reduces to
      $\mathcal{P} (\Spec \mathcal{O}_F)$. \qedhere
    \end{itemize}
  \end{proof}
\end{lemma}

%%%%%%%%%%%%%%%%%%%%%%%%%%%%%%%%%%%%%%%%%%%%%%%%%%%%%%%%%%%%%%%%%%%%%%%%%%%%%%%%

\section{Calculations of motivic cohomology}
\label{sec:motivic-cohomology-structure}

In this section we review some results regarding the structure of étale motivic
cohomology $H^i (X_\et, \ZZ^c(n))$. What follows is rather well-known, but we
put together the references and state the corresponding results for a general
one-dimensional arithmetic scheme.

\subsection*{Vanishing order of $\zeta (X,s)$ at $s = n < 0$}

In the subsequent calculations the number $d_n$ from definition~\ref{dfn:dn}
will appear frequently, so we make a digression to explain its arithmetic
meaning.

\begin{proposition}
  \label{prop:vanishing-order-equals-dn}
  We have $d_n = \ord_{s = n} \zeta (X,s)$.

  \begin{proof}
    For $X = \Spec \mathcal{O}_F$ the claim is a well-known consequence of the
    functional equation for the Dedekind zeta function
    \cite[\S VII.5]{Neukirch-1999}. It is also true for $X/\FF_q$ since in this
    case $\zeta (X,s)$ has no zeros or poles at $s = n < 0$
    \cite[pp.\,26--27]{Katz-1994}. We proceed with dévissage
    lemma~\ref{lemma:devissage}.

    We have $\zeta (X,s) = \zeta (X_\red,s)$ and
    $r_{1,2} (X) = r_{1,2} (X_\red)$.
    If $X = \coprod_i X_i$ is a finite disjoint union, then
    \[ \ord_{s = n} \zeta (X,s) = \sum_i \ord_{s = n} \zeta (X_i,s),
      \quad
      r_{1,2} (X) = \sum_i r_{1,2} (X_i), \]
    so that the property is compatible with disjoint unions. Finally, if
    $U \subset X$ is a dense open subscheme, then $Z = X\setminus U$ is
    a zero-dimensional scheme, and
    \[ \ord_{s = n} \zeta (X,s) = \ord_{s = n} \zeta (U,s),
      \quad
      r_{1,2} (X) = r_{1,2} (U), \]
    so that the property is compatible with taking dense open subschemes.
    We conclude that lemma~\ref{lemma:devissage} applies.
  \end{proof}
\end{proposition}

\subsection*{$G_\RR$-equivariant cohomology of $X (\CC)$}

Viewing $\ZZ (n) = (2\pi i)^n\,\ZZ$ as a constant $G_\RR$-equivariant sheaf on
$X (\CC)$, we consider the $G_\RR$-equivariant cohomology groups (resp. Tate
cohomology)
\begin{align*}
  H^i_c (G_\RR, X (\CC), \ZZ(n)) & \dfn H^i \Bigl(R\Gamma (G_\RR, R\Gamma_c (X (\CC), \ZZ(n)))\Bigr), \\
  \widehat{H}^i_c (G_\RR, X (\CC), \ZZ(n)) & \dfn H^i \Bigl(R\widehat{\Gamma} (G_\RR, R\Gamma_c (X (\CC), \ZZ(n)))\Bigr).
\end{align*}
Of course, $X (\CC)$ is just a finite discrete space, so that there is no need
to use cohomology with compact support $H^i_c$ and $\widehat{H}^i_c$, but we
will use this notation for consistency with the general case, treated in
\cite{Beshenov-Weil-etale-1}. Since $\dim X (\CC) = 0$, we have
\begin{align*}
  H^i_c (G_\RR, X (\CC), \ZZ(n)) & \cong H^i (G_\RR, H^0 (X (\CC), \ZZ(n))), \\
  \widehat{H}^i_c (G_\RR, X (\CC), \ZZ(n)) & \cong \widehat{H}^i (G_\RR, H^0 (X (\CC), \ZZ(n))).
\end{align*}

\begin{proposition}
  Let $X$ be a one-dimensional arithmetic scheme with $r_1$ real and $2r_2$
  complex places. Then the $G_\RR$-equivariant cohomology of $X (\CC)$ is given
  by
  \begin{align*}
    \widehat{H}^i (G_\RR, X (\CC), \ZZ (n)) & \cong
                                              \begin{cases}
                                                (\ZZ/2\ZZ)^{\oplus r_1}, & i \equiv n ~ (2), \\
                                                0, & i \not\equiv n ~ (2);
                                              \end{cases} \\
    H^i (G_\RR, X (\CC), \ZZ (n)) & \cong
                                    \begin{cases}
                                      0, & i < 0, \\
                                      \ZZ^{\oplus d_n}, & i = 0, ~ n\text{ even}, \\
                                      \widehat{H}^i (G_\RR, X (\CC), \ZZ (n)), & i \ge 1.
                                    \end{cases}
  \end{align*}

  \begin{proof}
    We have
    \[ H^0 (X (\CC), \ZZ(n)) \cong
      \ZZ (n)^{\oplus r_1} \oplus (\ZZ (n) \oplus \ZZ (n))^{\oplus r_2}, \]
    and the $G_\RR$-action on the two summands is given by
    $x \mapsto \overline{x}$ and $(x,y) \mapsto (\overline{y}, \overline{x})$
    respectively. We recall that Tate cohomology of a finite cyclic group is
    $2$-periodic:
    \[ \widehat{H}^i (G,A) \cong
      \begin{cases}
        \widehat{H}^0 (G,A), & i\text{ even}, \\
        \widehat{H}_0 (G,A), & i\text{ odd},
      \end{cases} \]
    and the groups $\widehat{H}^0 (G,A)$ and $\widehat{H}_0 (G,A)$ are given by
    the exact sequence
    \[ 0 \to \widehat{H}_0 (G,A) \to
      A_G \xrightarrow{N} A^G \to
      \widehat{H}^0 (G,A) \to 0 \]
    where $N\colon A_G\to A^G$ is the norm map, induced by the action of
    $\sum_{g\in G} g$.

    \vspace{1em}

    Therefore, we may consider two cases.
    \begin{enumerate}
    \item[1)] $G_\RR$-module $A = \ZZ (n)$ with action via
      $x \mapsto \overline{x}$.  In this case we see that
      \[ A^{G_\RR} \cong
        \begin{cases}
          \ZZ, & n\text{ even}, \\
          0, & n\text{ odd}.
        \end{cases} \]
      Similarly it is easy to calculate the coinvariants $A_{G_\RR}$, and
      \[ \widehat{H}^0 (G_\RR, A) \cong
        \begin{cases}
          \ZZ/2\ZZ, & n\text{ even},\\
          0, & n\text{ odd},
        \end{cases} \quad
        \widehat{H}_0 (G_\RR, A) \cong
        \begin{cases}
          0, & n\text{ even},\\
          \ZZ/2\ZZ, & n\text{ odd}.
        \end{cases} \]

    \item[2)] $G_\RR$-module $A = \ZZ (n) \oplus \ZZ (n)$ with action via
      $(x,y) \mapsto (\overline{y}, \overline{x})$. In this case
      $A^{G_\RR} \cong \ZZ$ and
      $\widehat{H}^0 (G_\RR,A) = \widehat{H}_0 (G_\RR,A) = 0$.
    \end{enumerate}

    Putting together these two calculations, we obtain Tate cohomology groups
    $\widehat{H}^i (G_\RR, X (\CC), \ZZ (n))$. For the usual cohomology, we have
    \begin{align*}
      H^0 (G_\RR, X (\CC), \ZZ (n)) & \cong H^0 (X (\CC), \ZZ (n))^{G_\RR}, \\
      H^i (G_\RR, X (\CC), \ZZ (n)) & \cong \widehat{H}^i (G_\RR, X (\CC), \ZZ (n)) \quad \text{for }i \ge 1. \qedhere
    \end{align*}
  \end{proof}
\end{proposition}

\subsection*{Étale motivic cohomology}

We now describe motivic cohomology groups $H^i (X_\et, \ZZ^c (n))$ for
one-dimensional arithmetic schemes.

\begin{proposition}
  \label{prop:structure-of-motivic-cohomology}
  If $X$ is a one-dimensional arithmetic scheme and $n < 0$, then
  \begin{equation}
    \label{eqn:structure-of-motivic-cohomology}
    H^i (X_\et, \ZZ^c (n)) \cong
    \begin{cases}
      0, & i < -1, \\
      \text{finitely generated of rk } d_n, & i = -1, ~ n\text{ even}, \\
      \text{finite}, & i = 0,1, \\
      (\ZZ/2\ZZ)^{\oplus r_1}, & i \ge 2, ~ i\not\equiv n ~ (2), \\
      0, & i \ge 2, ~ i\equiv n ~ (2).
    \end{cases}
  \end{equation}

  Further, if $X = \Spec \mathcal{O}_F$ for a number field $F/\QQ$, then
  \begin{equation}
    \label{eqn:2-torsion-in-H1-Zc-for-Spec-OF}
    H^1 (X_\et, \ZZ^c (n)) \cong
    \begin{cases}
      (\ZZ/2\ZZ)^{\oplus r_1}, & n\text{ even}, \\
      0, & n \text{ odd}.
    \end{cases}
  \end{equation}
\end{proposition}

In particular, this explains why the only relevant group for the regulator is
$H^{-1} (X_\et, \ZZ^c (n))$.

An important ingredient of the proof, and other arguments below, will be an
arithmetic duality \cite[Theorem~I]{Beshenov-Weil-etale-1}, which states that if
$H^i (X_\et, \ZZ^c (n))$ are finitely generated groups for all $i \in \ZZ$, then
\begin{equation}
  \label{eqn:arithmetic-duality}
  \widehat{H}^i_c (X_\et, \ZZ (n)) \cong
  \Hom (H^{2-i} (X_\et, \ZZ^c (n)), \QQ/\ZZ),
\end{equation}
where
\[ \ZZ (n) \dfn \QQ/\ZZ (n) [-1] \dfn
  \bigoplus_p \varinjlim_r j_{p!} \mu_{p^r}^{\otimes n} [-1]. \]
Here $\widehat{H}^i_c (X_\et, \ZZ (n))$ is the modified cohomology with compact
support, for which we refer to \cite[Appendix~B]{Beshenov-Weil-etale-1} or
\cite[\S 2]{Geisser-Schmidt-2018}. In particular,
$\widehat{H}^i_c (X_\et, \ZZ (n)) = H^i_c (X_\et, \ZZ (n))$ whenever
$X (\RR) = \emptyset$.
We note that \eqref{eqn:arithmetic-duality} is a powerful result, and it is
deduced in \cite{Beshenov-Weil-etale-1} from Geisser's work \cite{Geisser-2010}.

\begin{proof}[Proof of proposition~\ref{prop:structure-of-motivic-cohomology}]
  We will use the dévissage lemma~\ref{lemma:devissage}. We will say that
  $\mathcal{P} (X)$ holds if the motivic cohomology of $X$ has structure
  \eqref{eqn:structure-of-motivic-cohomology}.

  \vspace{1em}

  \textbf{First consider the case of a finite field $X = \Spec \FF_q$}.
  We have
  \begin{equation}
    \label{eqn:motivic-cohomology-finite-fields}
    H^i (\Spec \FF_{q,\et}, \ZZ^c (n)) \cong
    \begin{cases}
      \ZZ/(q^{-n} - 1), & i = 1, \\
      0, & i \ne 1.
    \end{cases}
  \end{equation}
  ---see for instance \cite[Example~4.2]{Geisser-2017}. This is related to
  Quillen's calculation of $K$-theory of finite fields \cite{Quillen-1972}.

  In general, if $X$ is a zero-dimensional arithmetic scheme, then the motivic
  cohomology coincides for $X$ and $X_\red$, hence we may assume that $X$ is
  reduced. Then $X$ is a finite disjoint union of $X_i = \Spec \FF_{q_i}$, and
  \begin{equation}
    H^i (X, \ZZ^c (n)) = \begin{cases}
      \text{finite}, & i = 1, \\
      0, & i \ne 1.
    \end{cases}
  \end{equation}
  In particular, $\mathcal{P} (X)$ holds if $\dim X = 0$.

  \vspace{1em}

  \textbf{Now we check the compatibility properties for $\mathcal{P}$}.
  If $X = \coprod_i X_i$ is a finite disjoint union, then
  $H^i (X_\et, \ZZ^c (n)) \cong \bigoplus_i H^i (X_{i,\et}, \ZZ^c (n))$,
  so that the property $\mathcal{P}$ is compatible with disjoint unions.

  Similarly, if $U \subset X$ is a dense open subscheme, and $Z = X\setminus U$
  its closed complement, then $\dim Z = 0$. We consider the long exact sequence
  \eqref{eqn:localization-les}. Since cohomology of $Z$ is concentrated in
  $i = 1$, we have $H^i (X_\et, \ZZ^c (n)) \cong H^i (U_\et, \ZZ^c (n))$ for
  $i \ne 0,1$, and what remains is an exact sequence
  \[ 0 \to H^0 (X_\et, \ZZ^c (n)) \to
    H^0 (U_\et, \ZZ^c (n)) \to
    H^1 (Z_\et, \ZZ^c (n)) \to
    H^1 (X_\et, \ZZ^c (n)) \to
    H^1 (U_\et, \ZZ^c (n)) \to 0 \]
  Moreover, $d_n (X) = d_n (U)$. These considerations show that
  $\mathcal{P} (X)$ and $\mathcal{P} (U)$ are equivalent, and therefore
  lemma~\ref{lemma:devissage} works, and it remains to establish
  $\mathcal{P} (X)$ for a curve $X/\FF_q$ or $X = \Spec \mathcal{O}_F$.

  \vspace{1em}

  \textbf{Suppose that $X/\FF_q$ is a smooth curve}. The groups
  $H^i (X_\et, \ZZ^c (n))$ are finitely generated by
  \cite[Proposition~4.3]{Geisser-2017}, so that the duality
  \eqref{eqn:arithmetic-duality} holds. The $\QQ/\ZZ$-dual groups
  \[ H^i_c (X_\et, \ZZ(n)) =
    \bigoplus_\ell H^{i-1}_c (X_\et, \QQ_\ell/\ZZ_\ell (n)) \]
  are finite by \cite[Theorem~3]{Kahn-2003}, and concentrated in
  $i = 1,2,3$ for dimension reasons. It follows that in this case
  $H^i (X_\et, \ZZ^c (n))$ are finite groups concentrated in $i = -1,0,1$,
  and the property $\mathcal{P} (X)$ holds.

  \vspace{1em}

  \textbf{It remains to consider the case of $X = \Spec \mathcal{O}_F$}.
  In this case finite generation of $H^i (X_\et, \ZZ^c (n))$ is also known; see
  for instance \cite[Proposition~4.14]{Geisser-2017}. Therefore, the duality
  \eqref{eqn:arithmetic-duality} holds. We have
  $\widehat{H}^i_c (\Spec \mathcal{O}_F [1/p], \mu_{p^r}^{\otimes n}) = 0$ for
  $i \ge 3$ by Artin--Verdier duality
  \cite[Chapter~II, Corollary~3.3]{Milne-ADT}, or by
  \cite[p.\,268]{Soule-1979}. Therefore, it follows that
  $\widehat{H}^i_c (X_\et, \ZZ (n)) = 0$ for $i \ge 4$, and hence by duality
  \eqref{eqn:arithmetic-duality}, $H^i (X_\et, \ZZ^c (n)) = 0$ for $i \le -2$.

  Now we identify the finite $2$-torsion in $H^i (X_\et, \ZZ^c (n))$ for
  $i \ge 2$. By \cite[Lemma~6.14]{Flach-Morin-2018}, there is an exact
  sequence
  \[ \cdots \to H^{i-1}_c (X_\et, \ZZ (n)) \to
    \widehat{H}^{i-1} (G_\RR, X (\CC), \ZZ (n)) \to
    \widehat{H}^i_c (X_\et, \ZZ (n)) \to
    H^i_c (X_\et, \ZZ (n)) \to \cdots \]
  For $i \le 0$ we have $H^i_c (X_\et, \ZZ (n)) = 0$, and therefore
  \[ \widehat{H}^i_c (X_\et, \ZZ (n)) \cong
    \widehat{H}^{i-1} (G_\RR, X (\CC), \ZZ (n)) \cong
    \begin{cases}
      (\ZZ/2\ZZ)^{\oplus r_1}, & i\not\equiv n ~ (2), \\
      0, & i\equiv n ~ (2).
    \end{cases} \]
  By duality, we have for $i \ge 2$
  \[ H^i (X_\et, \ZZ^c (n)) \cong
    \begin{cases}
      (\ZZ/2\ZZ)^{\oplus r_1}, & i\not\equiv n ~ (2), \\
      0, & i\equiv n ~ (2).
    \end{cases} \]

  Now we determine the ranks of $H^i (X_\et, \ZZ^c (n))$ for $i = -1,0,1$.
  Put $m = 1-n$, so that $m \ge 2$. We have for $i = -1,0,1$
  \[ H^i (X_\et, \ZZ^c(n)) \cong
    H^{2+i} (X_\et, \ZZ (m)) \cong
    H^{2+i} (X_\Zar, \ZZ(m)). \]
  Here the last isomorphism is Beilinson--Lichtenbaum conjecture,
  which is now a theorem \cite[Theorem~1.2]{Geisser-2004-Dedekind}.
  We will use this identification to analyze
  the groups $H^i (X_\et, \ZZ (m))$ for $i = 1,2,3$. According to
  \cite[Proposition~2.1]{Kolster-Sands-2008}, for $i = 1,2$ the Chern
  character
  $K_{2m - i} (X) \to H^i (X_\Zar, \ZZ(m))$
  has finite $2$-torsion kernel and cokernel. In particular,
  $\rk_\ZZ H^i (X_\Zar, \ZZ(m)) = \rk_\ZZ K_{2m - i} (X)$.
  The ranks of $K$-theory of rings of integers were calculated by Borel
  \cite{Borel-1974}:
  \[ \rk K_{2m-i} (X) = \begin{cases}
      0, & 2m - i \text{ even}, \\
      r_1 + r_2, & 2m - i \equiv 1 ~ (4), \\
      r_2, & 2m - i \equiv 3 ~ (4).
    \end{cases} \]
  This implies that $H^2 (X_\Zar, \ZZ(m))$ is a finite group, while
  \[ \rk_\ZZ H^1 (X_\Zar, \ZZ(m)) =
    \begin{cases}
      r_2, & m \text{ even}, \\
      r_1 + r_2, & m \text{ odd}.
    \end{cases} \]
  Finally, for $i = 3$ we have by \cite[p.\,179]{Kolster-Sands-2008}
  \[ H^i (X_\Zar, \ZZ (m)) \cong
    \begin{cases}
      0, & m\text{ even}, \\
      (\ZZ/2\ZZ)^{\oplus r_1}, & m\text{ odd}.
    \end{cases} \]
  This concludes the proof.
\end{proof}

%%%%%%%%%%%%%%%%%%%%%%%%%%%%%%%%%%%%%%%%%%%%%%%%%%%%%%%%%%%%%%%%%%%%%%%%%%%%%%%%

\section{Calculations of Weil-étale cohomology}
\label{sec:Weil-etale-cohomology-of-X}

In this section we calculate Weil-étale cohomology groups $H^i_\Wc (X, \ZZ(n))$,
as defined in \cite{Beshenov-Weil-etale-1}. We briefly recall the
construction. It is performed in two steps: first we define cohomology groups
$H^i_\fg (X, \ZZ(n))$, and then these are modified to give Weil-étale cohomology
with compact support $H^i_\Wc (X, \ZZ(n))$.

We consider the morphism of complexes
\[ \alpha_{X,n}\colon \RHom (R\Gamma (X_\et, \ZZ^c(n)), \QQ [-2]) \to
  R\Gamma_c (X_\et, \ZZ(n)) \]
determined on the level of cohomology, using the arithmetic
duality \eqref{eqn:arithmetic-duality}, via
\begin{multline*}
  H^i (\alpha_{X,n})\colon \Hom (H^{2-i} (X_\et, \ZZ^c(n)), \QQ)
  \xrightarrow{\QQ \twoheadrightarrow \QQ/\ZZ}
  \Hom (H^{2-i} (X_\et, \ZZ^c(n)), \QQ/\ZZ)\\
  \xleftarrow{\cong} \widehat{H}^i_c (X_\et, \ZZ(n)) \to H^i_c (X_\et, \ZZ(n)).
\end{multline*}
We note that
\begin{multline*}
  \ker H^i (\alpha_{X,n}) \cong
  \Hom (H^{i-2} (X_\et, \ZZ^c(n)), \ZZ) \cong \\
  \ker \Bigl(\Hom (H^{i-2} (X_\et, \ZZ^c(n)), \QQ) \to
  \Hom (H^{i-2} (X_\et, \ZZ^c(n)), \QQ/\ZZ)\Bigr).
\end{multline*}
We note that
\begin{align*}
  \ker H^i (\alpha_{X,n}) & \cong \Hom (H^{2-i} (X_\et, \ZZ^c(n)), \ZZ), \\
  \coker H^i (\alpha_{X,n}) & \cong H^i_c (X_\et, \ZZ(n))_\codiv.
\end{align*}
For the last isomorphism, note that the image of $H^i (\alpha_{X,n})$
corresponds to the maximal divisible subgroup of $H^i_c (X_\et, \ZZ(n))$,
i.e. $\QQ/\ZZ$-dual of the free part of $H^{2-i} (X_\et, \ZZ^c(n))$.

The complex $R\Gamma_\fg (X, \ZZ(n))$ is defined as a cone of
$\alpha_{X,n}$:
\[ \RHom (R\Gamma (X_\et, \ZZ^c(n)), \QQ [-2]) \xrightarrow{\alpha_{X,n}}
  R\Gamma_c (X_\et, \ZZ(n)) \to
  R\Gamma_\fg (X_\et, \ZZ(n)) \to
  \RHom (R\Gamma (X_\et, \ZZ^c(n)), \QQ [-1]) \]
For a general arithmetic scheme $X$ with finitely generated motivic cohomology
$H^i (X_\et, \ZZ^c(n))$, the groups
$H^i_\fg (X, \ZZ(n)) \dfn H^i (R\Gamma_\fg (X, \ZZ(n)))$ are finitely generated
for all $i \in \ZZ$, vanish for $i \ll 0$, and finite $2$-torsion for $i \gg 0$.
We refer to \cite[\S 5]{Beshenov-Weil-etale-1} for the details.

\begin{proposition}
  \label{prop:calculation-of-H-fg}
  For a one-dimensional arithmetic scheme $X$ and $n < 0$,

  \begin{enumerate}
  \item[0)] $H^i_\fg (X, \ZZ(n)) = 0$ for $i \le 0$,

  \item[1)] there is an isomorphism of finite groups
    $H^1_c (X_\et, \ZZ(n)) \xrightarrow{\cong} H^1_\fg (X, \ZZ(n))$,

  \item[2)] there is an isomorphism of finitely generated groups
    \[ H^2_\fg (X, \ZZ(n)) \cong
      \underbrace{\Hom (H^{-1} (X_\et, \ZZ^c(n)), \ZZ)}_{\cong \ZZ^{\oplus d_n}}
      \oplus
      \underbrace{H^2 (X_\et, \ZZ(n))}_{\text{finite}}, \]
    and under this identification,
    $H^2_c (X_\et, \ZZ(n)) \to H^2_\fg (X, \ZZ(n))$ is the natural inclusion.

  \item[3)] there is an isomorphism of finite groups
    $H^3_\fg (X, \ZZ(n)) \cong H^3_c (X_\et, \ZZ(n))_\codiv$,
    and under this identification,
    $H^3_c (X_\et, \ZZ(n)) \to H^3_\fg (X, \ZZ(n))$ is the natural projection.

  \item[4)] there is an isomorphism of finite $2$-torsion groups
    $H^i_c (X_\et, \ZZ(n)) \xrightarrow{\cong} H^i_\fg (X, \ZZ(n))$
    for $i \ge 4$.
  \end{enumerate}

  \begin{proof}
    Consider the exact sequence
    \[ \cdots \to \Hom (H^{2-i} (X_\et, \ZZ^c (n)), \QQ) \to
      H^i_c (X_\et, \ZZ(n)) \to
      H^i_\fg (X, \ZZ(n)) \to
      \Hom (H^{1-i} (X_\et, \ZZ^c (n)), \QQ) \to \cdots \]
    Here we have $\Hom (H^{2-i} (X_\et, \ZZ^c (n)), \QQ) = 0$ for all
    $i \ne 3$, and $H^i_c (X_\et, \ZZ(n))$ for $i \le 0$. From this 0), 1), 4)
    readily follow. For 2) and 3), consider the exact sequence
    \begin{multline*}
      0 \to \underbrace{H^2_c (X_\et, \ZZ(n))}_{\text{finite}} \to
      \underbrace{H^2_\fg (X, \ZZ(n))}_{\text{finitely generated}} \to
      \underbrace{\Hom (H^{-1} (X_\et, \ZZ^c (n)), \QQ)}_{\text{finite dimension}} \xrightarrow{H^3 (\alpha_{X,n})} \\
      \underbrace{H^3_c (X_\et, \ZZ(n))}_{\text{cofinite}} \to
      \underbrace{H^3_\fg (X, \ZZ(n))}_{\text{finite}} \to 0
    \end{multline*}
    We have a split short exact sequence
    \[ 0 \to H^2_c (X_\et, \ZZ(n)) \to
      H^2_\fg (X, \ZZ(n)) \to
      \ker H^3 (\alpha_{X,n}) \to 0 \]
    where
    $\ker H^3 (\alpha_{X,n}) \cong \Hom (H^{-1} (X_\et, \ZZ^c(n)), \ZZ)$
    is a free group. Moreover,
    $H^3_\fg (X, \ZZ(n)) \cong \coker H^3 (\alpha_{X,n}) \cong
    H^3_c (X_\et, \ZZ(n))_\codiv$.
  \end{proof}
\end{proposition}

Then one defines a canonical morphism in the derived category $i^*_\infty$ that
is torsion and gives a commutative diagram
\[ \begin{tikzcd}
    R\Gamma_c (X_\et, \ZZ(n)) \ar{r}\ar{d}[swap]{u^*_\infty} & R\Gamma_\fg (X, \ZZ(n)) \ar{dl}{i^*_\infty} \\
    R\Gamma_c (G_\RR, X(\CC), \ZZ(n))
  \end{tikzcd} \]
---see \cite[\S\S 6,7]{Beshenov-Weil-etale-1} for further details.

\begin{lemma}
  \label{lemma:i-infty}
  For a one-dimensional arithmetic scheme $X$ and $n < 0$,

  \begin{enumerate}
  \item[0)] $H^i (i^*_\infty) = 0$ for $i \le 0$.

  \item[1)] $H^1 (i^*_\infty) = H^1 (u^*_\infty)$ under the identification
    $H^1_c (X_\et, \ZZ(n)) \cong H^1_\fg (X, \ZZ(n))$:
    \[ \begin{tikzcd}
        H^1_c (X_\et, \ZZ(n)) \ar{r}{\cong}\ar{d}[swap]{H^1 (u^*_\infty)} & H^1_\fg (X, \ZZ(n)) \ar{dl}{H^1 (i^*_\infty)} \\
        H^1_c (G_\RR, X(\CC), \ZZ(n))
      \end{tikzcd} \]

  \item[2)] $H^2 (i^*_\infty)$ is trivial on the torsion free part of
    $H^2_\fg (X, \ZZ(n))$, and it coincides with $H^2 (u^*_\infty)$ on
    $H^2_\fg (X, \ZZ(n))_\tors \cong H^2_c (X_\et, \ZZ(n))$:
    \[ \begin{tikzcd}
        H^2_c (X_\et, \ZZ(n)) \ar[right hook->]{r}\ar{d}[swap]{H^2 (u^*_\infty)} &
        \begin{array}{c}
          \Hom (H^{-1} (X_\et, \ZZ^c(n)), \ZZ) \\
          \oplus \\
          H^2_c (X_\et, \ZZ(n))
        \end{array} \ar{dl}{H^2 (i^*_\infty)} \\
        H^2_c (G_\RR, X(\CC), \ZZ(n))
      \end{tikzcd} \]

  \item[3)] $H^3 (i^*_\infty) = H^3 (u^*_\infty)_\codiv$, and this morphism is
    surjective:
    \[ \begin{tikzcd}
        H^3_c (X_\et, \ZZ(n)) \ar[->>]{r}\ar[->>]{d}[swap]{H^3 (u^*_\infty)} &
        H^3_c (X_\et, \ZZ(n))_\codiv \ar{dl}{H^3 (i^*_\infty)} \\
        H^3_c (G_\RR, X(\CC), \ZZ(n))
      \end{tikzcd} \]
    where
    $H^3_c (X_\et, \ZZ(n)) \cong H^3_c (X_\et, \ZZ(n))_\div \oplus H^3_c (X_\et, \ZZ(n))_\codiv$,

    \item[4)] for $i \ge 4$ there is a commutative diagram with isomorphisms of
    finite $2$-torsion groups
    \[ \begin{tikzcd}
        H^i_c (X_\et, \ZZ(n)) \ar{r}{\cong}\ar{d}{\cong}[swap]{H^i (u^*_\infty)} &
        H^i_\fg (X, \ZZ(n)) \ar{dl}{H^i (i^*_\infty)} \\
        H^i_c (G_\RR, X(\CC), \ZZ(n))
      \end{tikzcd} \]
  \end{enumerate}

  \begin{proof}
    Clear from the above calculations of $H^i_\fg (X, \ZZ(n))$, and the fact
    that $i^*_\infty$ is torsion. Surjectivity in part 3) follows from the exact
    sequence
    \[ \widehat{H}^3_c (X_\et, \ZZ(n)) \to
      H^3_c (X_\et, \ZZ(n)) \xrightarrow{H^3 (u^*_\infty)}
      H^3_c (G_\RR, X(\CC), \ZZ(n)) \to
      \underbrace{\widehat{H}^4_c (X_\et, \ZZ(n))}_{= 0} \qedhere \]
  \end{proof}
\end{lemma}

Weil-étale cohomology is defined as a mapping fiber of $i^*_\infty$:
\[ R\Gamma_\Wc (X, \ZZ(n)) \to
  R\Gamma_\fg (X, \ZZ(n)) \xrightarrow{i^*_\infty}
  R\Gamma_c (G_\RR, X(\CC), \ZZ(n)) \to
  R\Gamma_\Wc (X, \ZZ(n)) [1] \]
For a general arithmetic scheme $X$ with finitely generated motivic cohomology
$H^i (X_\et, \ZZ^c(n))$, the resulting groups
$H^i_\Wc (X, \ZZ(n)) \dfn H^i (R\Gamma_\Wc (X, \ZZ(n)))$ are finitely generated
and vanish for $i \notin [0,2d+1]$. Here we calculate these for one-dimensional $X$.

\begin{proposition}
  \label{prop:calculation-of-H-Wc}
  Let $X$ be a one-dimensional arithmetic scheme and $n < 0$.

  \begin{enumerate}
  \item[0)] $H^i_\Wc (X, \ZZ(n)) = 0$, unless $i = 1,2,3$.

  \item[1)] There is a short exact sequence
    \[ 0 \to \underbrace{H^0_c (G_\RR, X (\CC), \ZZ(n))}_{\cong \ZZ^{\oplus d_n}} \to H^1_\Wc (X, \ZZ(n)) \to T_1 \to 0 \]
    where $T_1$ sits in a short exact sequence of finite groups
    \[ 0 \to \widehat{H}^0_c (G_\RR, X(\CC), \ZZ(n)) \to
      \widehat{H}^1_c (X_\et, \ZZ(n)) \to
      T_1 \to 0 \]
    In particular, $H^1_\Wc (X, \ZZ(n))$ is finitely generated of rank $d_n$,
    and
    $$|T_1| = \frac{1}{2^\delta}\cdot |H^1 (X_\et, \ZZ^c (n))|,$$
    where $\delta$ is defined by \eqref{eqn:delta}.

  \item[2)] There is an isomorphism of finitely generated groups
    \[ H^2_\Wc (X, \ZZ(n)) \cong
      \underbrace{\Hom (H^{-1} (X_\et, \ZZ^c(n)), \ZZ)}_{\cong \ZZ^{\oplus d_n}}
      \oplus
      T_2, \]
    where $|T_2| = |H^0 (X_\et, \ZZ^c(n))|$.

  \item[3)] $H^3_\Wc (X, \ZZ(n))$ is a finite group of order
    $|H^{-1} (X_\et, \ZZ^c(n))_\tors|$.
  \end{enumerate}

  \begin{proof}
    Note that since $H^i_\fg (X,\ZZ(n)) = 0$ for $i \le 0$ and
    $H^i_c (G_\RR, X(\CC), \ZZ(n)) = 0$ for $i < 0$, we see from the exact
    sequence
    \[ \cdots \to H^i_\Wc (X, \ZZ(n)) \to
      H^i_\fg (X,\ZZ(n)) \xrightarrow{H^i (i^*_\infty)}
      H^i_c (G_\RR, X(\CC), \ZZ(n)) \to
      H^{i+1}_\Wc (X, \ZZ(n)) \]
    that $H^i_\Wc (X, \ZZ(n)) = 0$ for $i \le 0$.

    \vspace{1em}

    For $i = 1$, we consider short exact sequences
    \[ \begin{tikzcd}[column sep=1em,row sep=0pt]
        0 \ar{r} & H^0_c (G_\RR, X(\CC), \ZZ(n)) \ar{r} & H^1_\Wc (X, \ZZ(n)) \ar{r} & \ker H^1 (i^*_\infty) \ar{r} & 0 \\
        0 \ar{r} & \widehat{H}^0_c (G_\RR, X(\CC), \ZZ(n)) \ar{r} & \widehat{H}^1_c (X_\et, \ZZ(n)) \ar{r} & \ker H^1 (u_\infty^*) \ar{r} & 0
      \end{tikzcd} \]
    and use the identification $H^1 (u^*_\infty) = H^1 (i^*_\infty)$.

    The calculation of $H^i_\Wc (X, \ZZ(n))$ for $i = 2,3$ follows from the
    above calculations of $H^i_\fg (X, \ZZ(n))$ and $H^i (i^*_\infty)$, and long
    exact sequences
    \[ \begin{tikzcd}[column sep=1em,row sep=0pt]
        \cdots \ar{r} & H^i_\Wc (X, \ZZ(n)) \ar{r} & H^i_\fg (X, \ZZ(n)) \ar{r}{H^i (i^*_\infty)} &[2em] H^i_c (G_\RR, X(\CC), \ZZ(n)) \ar{r} & H^{i+1}_\Wc (X, \ZZ(n)) \ar{r} & \cdots \\
        \cdots \ar{r} & \widehat{H}^i_c (X, \ZZ(n)) \ar{r} & H^i_c (X_\et, \ZZ(n)) \ar{r}{H^i (u^*_\infty)} & \widehat{H}^i_c (G_\RR, X(\CC), \ZZ(n)) \ar{r} & \widehat{H}^{i+1}_c (X, \ZZ(n)) \ar{r} & \cdots
      \end{tikzcd} \]
    Namely, we have from lemma~\ref{lemma:i-infty}
    \begin{align*}
      \ker H^2 (i^*_\infty) & \cong \Hom (H^{-1} (X_\et, \ZZ^c(n)), \ZZ) \oplus \ker H^2 (u^*_\infty), \\
      \coker H^2 (i^*_\infty) & \cong \coker H^2 (u^*_\infty), \\
      \ker H^3 (i^*_\infty) & \cong \ker H^3 (u^*_\infty)_\codiv, \\
      \coker H^3 (i^*_\infty) & = 0.
    \end{align*}

    For $i = 2$ we consider extensions
    \[ \begin{tikzcd}[column sep=1em]
        0 \ar{r} & \coker H^1 (i^*_\infty) \ar{r}\ar{d}{\cong} & H^2_\Wc (X, \ZZ(n)) \ar{r} & \ker H^2 (i^*_\infty) \ar{r}\ar{d}{\cong} & 0 \\
        0 \ar{r} & \coker H^1 (u_\infty^*) \ar{r} & \widehat{H}^2 (X_\et, \ZZ(n)) \ar{r} & \ker H^2 (u_\infty^*) \ar{r} & 0 \\[-2em]
        & \oplus & \oplus & \oplus \\[-2em]
        0 \ar{r} & 0 \ar{r} & \Hom (H^{-1} (X_\et, \ZZ^c(n)), \ZZ) \ar{r} & \Hom (H^{-1} (X_\et, \ZZ^c(n)), \ZZ) \ar{r} & 0
      \end{tikzcd} \]
    Here
    $\widehat{H}^2 (X_\et, \ZZ(n)) \cong \Hom (H^0 (X_\et, \ZZ^c(n)), \QQ/\ZZ)$
    by duality. Similarly for $i = 3$,
    \[ \begin{tikzcd}[column sep=1em]
        0 \ar{r} & \coker H^2 (i^*_\infty) \ar{r}\ar{d}{\cong} & H^3_\Wc (X, \ZZ(n)) \ar{r} & \ker H^3 (i^*_\infty) \ar{r}\ar{d}{\cong} & 0 \\
        0 \ar{r} & \coker H^2 (u_\infty^*) \ar{r} & \widehat{H}^3 (X_\et, \ZZ(n))_\codiv \ar{r} & \ker H^3 (u_\infty^*)_\codiv \ar{r} & 0
      \end{tikzcd} \]
    where
    $\widehat{H}^3 (X_\et, \ZZ(n))_\codiv \cong \Hom (H^{-1} (X_\et, \ZZ^c(n))_\tors, \QQ/\ZZ)$.

    \vspace{1em}

    Finally, for $i \ge 4$ we have
    $\ker H^i (i^*_\infty) = \coker H^i (i^*_\infty) = 0$. Together with the
    fact that $\coker H^3 (i^*_\infty) = 0$, this implies that
    $H^i_\Wc (X, \ZZ(n)) = 0$ for $i \ge 4$.
  \end{proof}
\end{proposition}

\begin{remark}
  Part 0) of the above proposition is a consequence of general results on
  boundedness of $H^i_\Wc (X,\ZZ(n))$ that can be found in
  \cite[\S 7]{Beshenov-Weil-etale-1}.
\end{remark}

\begin{remark}
  The groups $H^i_\Wc (X, \ZZ(n))$ for $X = \Spec \mathcal{O}_F$ are calculated
  in \cite[\S 5.8.3]{Flach-Morin-2018}. The result is (rewriting in terms of
  $H^i (X_\et, \ZZ^c(n)) \cong H^{2+i} (X_\et, \ZZ (1-n))$):
  \begin{equation}
    \label{eqn:Weil-etale-cohomology-of-Spec-OF}
    H^i_\Wc (\Spec \mathcal{O}_F, \ZZ(n)) \cong
    \begin{cases}
      0, & i < 1, \\
      \ZZ^{\oplus d_n}, & i = 1, \\
      \Hom (H^{-1} (X_\et, \ZZ^c (n)), \ZZ) \oplus \Hom (H^0 (X_\et, \ZZ^c (n)), \QQ/\ZZ), & i = 2, \\
      \Hom (H^{-1} (X_\et, \ZZ^c (n))_\tors, \QQ/\ZZ), & i = 3, \\
      0, & i > 3.
    \end{cases}
  \end{equation}
  Our calculation generalizes this. What may look puzzling is the general answer
  for $H^1_\Wc (X,\ZZ(n))$ given by
  proposition~\ref{prop:calculation-of-H-Wc}. In case of
  $X = \Spec \mathcal{O}_F$, we have according to
  \eqref{eqn:2-torsion-in-H1-Zc-for-Spec-OF} that
  $H^1 (X_\et, \ZZ^c(n)) \cong (\ZZ/2\ZZ)^{\oplus r_1}$ for even $n$, and
  therefore $T_1 = 0$, which agrees with
  \eqref{eqn:Weil-etale-cohomology-of-Spec-OF}.

  Intuitively, the arithmetically interesting cohomology $H^i (X_\et, \ZZ^c(n))$
  for $X = \Spec \mathcal{O}_F$ is concentrated in degrees $i = -1,0$. The
  groups $H^i (X_\et, \ZZ^c (n))$ for $i \ge 1$ bear no interesting information:
  these are finite $2$-torsion, coming from the real places of $F$. Passing to
  Weil-étale cohomology removes this $2$-torsion. On the other hand, for a curve
  over a finite field $X/\FF_q$, the group $H^1 (X_\et, \ZZ^c (n))$ is not
  trivial and bears arithmetic information. The finite group $T_1$ that appears
  in the statement corresponds to $H^1 (X_\et, \ZZ^c (n))$, removing the
  $2$-torsion coming from the real places of $X$.
\end{remark}

\begin{remark}
  For a curve over a finite field $X/\FF_q$, all groups $H^i (X_\et, \ZZ^c(n))$
  are finite, and our calculation gives
  \begin{align*}
    H^1_\Wc (X, \ZZ(n)) & \cong \Hom (H^1 (X_\et, \ZZ^c(n)), \QQ/\ZZ), \\
    H^2_\Wc (X, \ZZ(n)) & \cong \Hom (H^0 (X_\et, \ZZ^c(n)), \QQ/\ZZ), \\
    H^3_\Wc (X, \ZZ(n)) & \cong \Hom (H^{-1} (X_\et, \ZZ^c(n)), \QQ/\ZZ).
  \end{align*}
  In general, for any variety over a finite field $X/\FF_q$, assuming finite
  generation of motivic cohomology $H^i (X_\et, \ZZ^c(n))$, one has
  $H^i_\Wc (X, \ZZ(n)) \cong \Hom (H^{2-i} (X_\et, \ZZ^c(n)), \QQ/\ZZ)$; see
  \cite[Proposition~7.7]{Beshenov-Weil-etale-1}.
\end{remark}

\begin{remark}
  It is conjectured in \cite[\S 3]{Beshenov-Weil-etale-2} that
  \[ \ord_{s = n} \zeta (X,s) =
    \sum_{i\in \ZZ} (-1)^i\cdot i\cdot\rk_\ZZ H^i_\Wc (X,\ZZ(n)). \]
  In this case $\rk_\ZZ H^1_\Wc (X,\ZZ(n)) = \rk_\ZZ H^2_\Wc (X,\ZZ(n)) = d_n$
  and $\rk_\ZZ H^3_\Wc (X,\ZZ(n)) = 0$, so that the conjecture holds
  (see proposition~\ref{prop:vanishing-order-equals-dn}).
\end{remark}

%%%%%%%%%%%%%%%%%%%%%%%%%%%%%%%%%%%%%%%%%%%%%%%%%%%%%%%%%%%%%%%%%%%%%%%%%%%%%%%%

\section{Regulators}
\label{sec:regulators}

Now we explain what will be intended by the regulator in our setting.

\begin{definition}
  We let the \textbf{regulator morphism} be
  \[ \varrho_{X,n}\colon
    H^{-1} (X_\et, \ZZ^c (n)) \to
    H^{-1} (X_\et, \ZZ^c (n)) \otimes \RR \xrightarrow{Reg_{X,n}}
    H^0_\BM (G_\RR, X(\CC), \RR(n)), \]
  where the map $Reg_{X,n}$ is defined in \cite[\S 2]{Beshenov-Weil-etale-2}.
\end{definition}

The right hand side is Borel--Moore cohomology, defined via
\[ H^0_\BM (G_\RR, X(\CC), \RR(n)) \dfn
  \Hom (H^0_c (G_\RR, X(\CC), \RR(n)), \RR). \]
In general, the regulator takes values in Deligne--Beilinson cohomology, but the
target simplifies in case of $n < 0$, as explained in
\cite[\S 2]{Beshenov-Weil-etale-2}.

\begin{remark}
  If $X = \Spec \mathcal{O}_F$, then $\varrho_{X,n}$ is the usual Beilinson
  regulator map
  \[ H^1 (X_\et, \ZZ (1-n)) \to H^1_\BM (G_\RR, X(\CC), \RR(1-n)) \]
  that also appears in the special value conjecture in
  \cite[\S 5.8.3]{Flach-Morin-2018}.
\end{remark}

\begin{lemma}
  \label{lemma:regulator-isomorphism}
  For any $1$-dimensional scheme $X$ and $n < 0$, the $\RR$-dual to the
  regulator
  \[ Reg_{X,n}^\vee\colon H^0_c (G_\RR, X(\CC), \RR(n)) \to
    \Hom (H^{-1} (X_\et, \ZZ^c (n)), \RR) \]
  is an isomorphism.

  \begin{proof}
    If $X/\FF_q$, then the claim is trivial. For $X = \Spec \mathcal{O}_F$, this
    is a well-known property of Beilinson's regulator. To apply dévissage
    lemma~\ref{lemma:devissage}, we need to check the compatibility with
    disjoint unions and taking a dense open subset $U \subset X$. For disjoint
    unions, this is clear. For a dense open subset $U \subset X$, the closed
    complement $Z = X\setminus U$ has dimension $0$, and the localization exact
    sequence \eqref{eqn:localization-les} with the long exact sequence for
    cohomology with compact support give integral isomorphisms
    \begin{align*}
      H^{-1} (X_\et, \ZZ^c (n)) & \xrightarrow{\cong} H^{-1} (U_\et, \ZZ^c (n)), \\
      H^0_c (G_\RR, U(\CC), \ZZ(n)) & \xrightarrow{\cong} H^0_c (G_\RR, Z(\CC), \ZZ(n)).
    \end{align*}

    We now have a commutative diagram
    \[ \begin{tikzcd}
        H^0_c (G_\RR, U(\CC), \RR(n)) \ar{r}{Reg_{U,n}^\vee} \ar{d}{\cong} &
        \Hom (H^{-1} (U_\et, \ZZ^c (n)), \RR) \ar{d}{\cong} \\
        H^0_c (G_\RR, X(\CC), \RR(n)) \ar{r}{Reg_{X,n}^\vee} &
        \Hom (H^{-1} (X_\et, \ZZ^c (n)), \RR)
      \end{tikzcd} \]
    The top arrow is an isomorphism if and only if the bottom arrow is.
  \end{proof}
\end{lemma}

\begin{definition}
  \label{dfn:regulator}
  For a one-dimensional arithmetic scheme $X$, we define the \textbf{regulator}
  as a positive real number
  \[ R_{X,n} \dfn \vol \Bigl(\coker \bigl(
    H^{-1} (X_\et, \ZZ^c (n)) \xrightarrow{\varrho_{X,n}}
    H^0_\BM (G_\RR, X(\CC), \RR(n))
    \bigr)\Bigr) \]
  where the volume is taken with respect to the canonical integral structure.
\end{definition}

If $X (\CC) = \emptyset$, or $n$ is odd and $r_2 = 0$, then
$H^0_\BM (G_\RR, X(\CC), \RR (n)) = 0$, and we set $R_{X,n} = 1$.

\begin{lemma}
  \label{lemma:regulator-dense-open-subset}
  Let $X$ be a one-dimensional scheme and $n < 0$. For any dense open subset
  $U \subset X$, one has $R_{X,n} = R_{U,n}$.

  \begin{proof}
    Follows from the proof of lemma~\ref{lemma:regulator-isomorphism}.
  \end{proof}
\end{lemma}

\begin{proposition}
  \label{prop:trivialization-of-free-part}
  Given an arithmetic scheme $X$ and $n < 0$, consider the two-term acyclic
  complex of real vector spaces
  \[ C^\bullet\colon
    0 \to
    \underbrace{H^0_c (G_\RR, X(\CC), \RR(n))}_{\deg 0}
    \xrightarrow{Reg_{X,n}^\vee}
    \underbrace{\Hom (H^{-1} (X_\et, \ZZ^c (n)), \RR)}_{\deg 1}
    \to 0 \]
  Then taking the determinant $\det_\RR (C^\bullet)$ in the sense of
  \cite{Knudsen-Mumford-1976}, the image of the canonical map
  \begin{multline*}
    \det_\ZZ H^0_c (G_\RR, X(\CC), \ZZ(n)) \otimes_\ZZ
    \det_\ZZ \Hom (H^{-1} (X_\et, \ZZ^c (n)), \ZZ)^{-1} \to \\
    \det_\RR H^0_c (G_\RR, X(\CC), \RR (n)) \otimes_\RR
    \det_\RR \Hom (H^{-1} (X_\et, \ZZ^c (n)), \RR)^{-1}
    \xrightarrow{\cong} \RR
  \end{multline*}
  corresponds to $R_{X,n}\,\ZZ \subset \RR$.

  \begin{proof}
    In general, if $F$ and $F'$ are free groups of finite rank $d$, and
    $$C^\bullet\colon 0 \to F\otimes_\ZZ \RR \xrightarrow{\phi} F'\otimes_\ZZ \RR \to 0$$
    is a two-term acyclic complex of real vector spaces, then the image of
    \[ \ZZ \cong \det_\ZZ F \otimes_\ZZ (\det_\ZZ F')^{-1} \to
      \det_\RR (F\otimes_\ZZ \RR) \otimes_\RR \det_\RR (F' \otimes_\ZZ \RR)^{-1}
      = \det_\RR (C^\bullet) \xrightarrow{\cong} \RR \]
    corresponds to $D\ZZ \subset \RR$, where $D$ is the determinant of $\phi$ with
    respect to the bases induced by $\ZZ$-bases of $F$ and $F'$.
    This follows from the explicit description of the isomorphism
    $\det_\RR (C^\bullet) \xrightarrow{\cong} \RR$ from
    \cite[p.\,33]{Knudsen-Mumford-1976}: it is
    \[ \det_\RR (F \otimes_\ZZ \RR) \otimes_\RR
      \det_\RR (F'\otimes_\ZZ \RR)^{-1} \xrightarrow{\det_\RR (\phi)}
      \det_\RR (F' \otimes_\ZZ \RR) \otimes_\RR
      \det_\RR (F' \otimes_\ZZ \RR)^{-1} \xrightarrow{\cong} \RR \]
    where the last arrow is the canonical pairing.

    Therefore, in our situation, the image of
    \[ \det_\ZZ H^0_c (G_\RR, X(\CC), \ZZ(n)) \otimes_\ZZ
      \det_\ZZ \Hom (H^{-1} (X_\et, \ZZ^c (n)), \ZZ)^{-1} \]
    is $D \ZZ \subset \RR$, where $D$ is the determinant of $Reg_{X,n}^\vee$
    considered with respect to bases induced by $\ZZ$-bases of
    $H^0_c (G_\RR, X(\CC), \ZZ(n))$ and
    $\Hom (H^{-1} (X_\et, \ZZ^c (n)), \ZZ)$. Dually, $D = R_{X,n}$.
  \end{proof}
\end{proposition}

%%%%%%%%%%%%%%%%%%%%%%%%%%%%%%%%%%%%%%%%%%%%%%%%%%%%%%%%%%%%%%%%%%%%%%%%%%%%%%%% 

\section{The special value formula}
\label{sec:special-value-formula}

Now we write down explicitly the special value conjecture $\mathbf{C} (X,n)$
stated in \cite[\S 4]{Beshenov-Weil-etale-2}. For this one considers the
canonical isomorphism
\begin{multline*}
  \lambda\colon \RR \xrightarrow{\cong}
  \bigotimes_{i\in \ZZ} (\det_\RR H^i_\Wc (X, \RR (n)))^{(-1)^i} \xrightarrow{\cong}
  \Bigl(\bigotimes_{i\in \ZZ} (\det_\ZZ H^i_\Wc (X, \ZZ (n)))^{(-1)^i}\Bigr) \otimes_\ZZ \RR \\
  \xrightarrow{\cong} (\det_\ZZ R\Gamma_\Wc (X, \ZZ (n))) \otimes_\ZZ \RR,
\end{multline*}
where the first isomorphism
$\RR \cong \bigotimes_{i\in \ZZ} (\det_\RR H^i_\Wc (X, \RR (n)))^{(-1)^i}$
comes from the regulator, as will be explained below.

\vspace{1em}

In our case, we are interested in the determinant of Weil-étale complex
\begin{multline*}
  \det_\ZZ R\Gamma_\Wc (X, \ZZ(n)) \cong
  \bigotimes_{i \in \ZZ} \det_\ZZ H^i_\Wc (X, \ZZ(n))^{(-1)^i} \\
  =
  \det_\ZZ H^1_\Wc (X, \ZZ(n))^{-1} \otimes_\ZZ
  \det_\ZZ H^2_\Wc (X, \ZZ(n)) \otimes_\ZZ
  \det_\ZZ H^3_\Wc (X, \ZZ(n))^{-1}.
\end{multline*}
Using calculations from proposition~\ref{prop:calculation-of-H-Wc}, we have
\begin{align*}
  \det_\ZZ H^1_\Wc (X, \ZZ(n)) & \cong \det_\ZZ H^0_c (G_\RR, X (\CC), \ZZ(n)) \otimes_\ZZ \det_\ZZ T_1, \\
  \det_\ZZ H^2_\Wc (X, \ZZ(n)) & \cong \det_\ZZ \Hom (H^{-1} (X_\et, \ZZ^c(n)), \ZZ) \otimes_\ZZ \det_\ZZ T_2.
\end{align*}
Therefore, we have an isomorphism (up to sign $\pm 1$, after rearranging the
terms)
\begin{multline*}
  \det_\ZZ R\Gamma_\Wc (X, \ZZ(n)) \cong \\
  =
  \det_\ZZ H^0_c (G_\RR, X (\CC), \ZZ(n))^{-1} \otimes_\ZZ
  \det_\ZZ \Hom (H^{-1} (X_\et, \ZZ^c(n)), \ZZ) \otimes_\ZZ \\
  \det_\ZZ (T_1)^{-1} \det_\ZZ  \det_\ZZ (T_2) \otimes_\ZZ \det_\ZZ H^3_\Wc (X, \ZZ(n))^{-1}.
\end{multline*}
Recall that $T_1$, $T_2$, $H^3_\Wc (X, \ZZ(n))$ are finite groups, and
$H^0_c (G_\RR, X (\CC), \ZZ(n))$, $\Hom (H^{-1} (X_\et, \ZZ^c(n)), \ZZ)$
are free of rank $d_n$. Now we consider the canonical trivialization
\[ (\det_\ZZ R\Gamma_\Wc (X, \ZZ(n))) \otimes_\ZZ \RR \cong
  \bigotimes_{i\in \ZZ} \det_\RR (H^i_\Wc (X, \ZZ(n))\otimes_\ZZ \RR)
  \cong \RR \]
via the regulator morphism
\[ \begin{tikzcd}[column sep=3em]
    H^0_c (G_\RR, X (\CC), \ZZ(n)) \otimes \RR\ar{d}{\cong} & \Hom (H^{-1} (X_\et, \ZZ^c (n)), \ZZ) \otimes \RR\ar{d}{\cong} \\
    H^0_c (G_\RR, X (\CC), \RR (n)) \ar{r}{Reg^\vee_{X,n}}[swap]{\cong} & \Hom (H^{-1} (X_\et, \ZZ^c (n)), \RR)
  \end{tikzcd} \]

\begin{proposition}
  Under the above trivialization, $\det_\ZZ R\Gamma_\Wc (X, \ZZ(n)) \subset \RR$
  corresponds to $\alpha^{-1}\,\ZZ \subset \RR$, where
  \[ \alpha = \frac{|T_2|}{|T_1| \cdot |R\Gamma_\Wc (X, \ZZ(n))|}\,R_{X,n}
    = 2^\delta\,\frac{|H^0 (X_\et, \ZZ^c (n))|}{|H^{-1} (X_\et, \ZZ^c(n))_\tors|\cdot |H^1 (X_\et, \ZZ^c(n))|}\,R_{X,n}, \]
  the number $\delta$ is given by \eqref{eqn:delta}, and $R_{X,n}$ is
  the regulator from definition~\ref{dfn:regulator}.

  \begin{proof}
    For finite groups $T_1$, $T_2$, $H^3_\Wc (X, \ZZ(n))$, this is
    \cite[Lemma~A.6]{Beshenov-Weil-etale-2}. On the other hand, for free groups
    $H^0_c (G_\RR, X (\CC), \ZZ(n))$ and
    $\Hom (H^{-1} (X_\et, \ZZ^c (n)), \ZZ)$, this is
    proposition~\ref{prop:trivialization-of-free-part}
    (the corresponding groups sit in degrees $1$ and $2$ in this case, so the
    determinant gets inverted).
  \end{proof}
\end{proposition}

We recall that conjecture $\mathbf{C} (X,n)$ from
\cite[\S 4]{Beshenov-Weil-etale-2} asserts that
$\det_\ZZ R\Gamma_\Wc (X, \ZZ(n)) \subset \RR$ corresponds to
$\zeta^* (X,n)^{-1}\,\ZZ \subset \RR$. This leads to the following conclusion.

\begin{proposition}
  Let $X$ be a one-dimensional arithmetic scheme and $n < 0$. Then the special
  value conjecture $\mathbf{C} (X,n)$ stated in \cite{Beshenov-Weil-etale-2} is
  equivalent to the formula \eqref{eqn:special-value-formula}.
\end{proposition}

It is proved in \cite[\S 7]{Beshenov-Weil-etale-2} that $\mathbf{C} (X,n)$ holds
unconditionally for an abelian one-dimensional arithmetic scheme $X$. Together
with the above proposition, this proves theorem~\ref{main-theorem} stated in the
introduction.

%%%%%%%%%%%%%%%%%%%%%%%%%%%%%%%%%%%%%%%%%%%%%%%%%%%%%%%%%%%%%%%%%%%%%%%%%%%%%%%%

\section{Direct proof of the formula}
\label{sec:demystification}

The derivation of the special value formula from our Weil-étale machinery might
look rather contrived. In this section we explain how to prove it directly,
combining via localization the known special value formulas for
$X = \Spec \mathcal{O}_F$ and curves over finite fields $X/\FF_q$. This will
establish our result, but the two previous sections also serve their purpose:
these show how we came up with the formula \eqref{eqn:special-value-formula}
in the first place.

\begin{lemma}
  \label{lemma:elementary-proof-1}
  Let $n < 0$.

  \begin{enumerate}
  \item[0)] If $X$ is a zero-dimensional arithmetic scheme, then
    $$\zeta (X,n) = \pm \frac{1}{|H^1 (X_\et, \ZZ^c (n))|}.$$

  \item[1)] If $X/\FF_q$ be a curve over a finite field, then
    \[ \zeta (X,n) =
      \pm\frac{|H^0 (X_\et, \ZZ^c(n))|}{|H^{-1} (X_\et, \ZZ^c(n))|\cdot |H^1 (X_\et, \ZZ^c(n))|}. \]

  \item[2)] If $X = \Spec \mathcal{O}_F$ for an abelian number field $F/\QQ$,
    then
    \[ \zeta (X,n) = \pm\frac{|H^0 (X_\et, \ZZ^c(n))|}{|H^{-1} (X_\et, \ZZ^c(n))|}\,R_{X,n}. \]
  \end{enumerate}

  In particular, the formula \eqref{eqn:special-value-formula} holds in these
  cases.

  \begin{proof}
    In part 0), the motivic cohomology and zeta function do not distinguish
    between $X$ and $X_\red$, so that we may suppose that $X$ is a finite
    disjoint union of $\Spec \FF_{q_i}$. Thanks to
    \eqref{eqn:motivic-cohomology-finite-fields},
    \[ \zeta (X,n) = \prod_i \frac{1}{1 - q_i^{-n}} =
      \pm \prod_i \frac{1}{|H^1 (X_{i,\et}, \ZZ^c (n))|} =
      \pm \frac{1}{|H^1 (X_\et, \ZZ^c (n))|}. \]
    Note that this is the formula \eqref{eqn:special-value-formula},
    since $\delta = 0$ in this case, and
    $H^{-1} (X_\et, \ZZ^c (n)) = H^0 (X_\et, \ZZ^c (n)) = 0$ by
    \eqref{eqn:motivic-cohomology-finite-fields}.

    \vspace{1em}

    For part 1), we refer to \cite[\S 5]{Beshenov-Weil-etale-2}.
    Part 2) follows from \cite[\S 5.8.3]{Flach-Morin-2018}, and in particular
    \cite[Proposition~5.35]{Flach-Morin-2018}\footnote{The latter uses reduction
      to the Tamagawa number conjecture.}.
    The formula is equivalent to \eqref{eqn:special-value-formula}, since
    $2^\delta = |H^1 (X_\et, \ZZ (n))|$ by
    \eqref{eqn:2-torsion-in-H1-Zc-for-Spec-OF}.
  \end{proof}
\end{lemma}

\begin{remark}
  The special value at $s = 0$ is not necessarily a rational number:
  \[ \zeta^* (\Spec \FF_q, 0) =
    \lim_{s \to 0} \frac{s}{1 - q^{-s}} =
    \frac{1}{\log q}. \]
  Moreover,
  \[ H^i (\Spec \FF_{q,\et}, \ZZ^c (0)) =
    \begin{cases}
      \ZZ, & i = 1, \\
      \QQ/\ZZ, & i = 2, \\
      0, & i \ne 1,2.
    \end{cases} \]
  This toy example already shows that it is important that we focus on the case
  of $n < 0$.
\end{remark}

\begin{lemma}
  \label{lemma:elementary-proof-2}
  Let $X$ be a one-dimensional arithmetic scheme and let $U \subset X$ be a
  dense open subset. Then the special value formula
  \eqref{eqn:special-value-formula} for $X$ is equivalent to the corresponding
  formula for $U$.

  \begin{proof}
    Let $Z = X\setminus U$ be the zero-dimensional complement. We have
    $$\zeta (X,n) = \zeta (U,n)\,\zeta (Z,n),$$
    where
    \begin{align}
      \label{eqn:special-value-for-X} \zeta (X,n) & \stackrel{?}{=} \pm 2^\delta\,\frac{|H^0 (X_\et, \ZZ^c (n)|}{|H^{-1} (X_\et, \ZZ^c (n))_\tors| \cdot |H^1 (X_\et, \ZZ^c (n))|}\,R, \\
      \label{eqn:special-value-for-U} \zeta (U,n) & \stackrel{?}{=} \pm 2^\delta\,\frac{|H^0 (U_\et, \ZZ^c (n)|}{|H^{-1} (U_\et, \ZZ^c (n))_\tors| \cdot |H^1 (U_\et, \ZZ^c (n))|}\,R, \\
      \notag \zeta (Z,n) & = \pm\frac{1}{|H^1 (Z_\et, \ZZ^c (n))|}.
    \end{align}
    Here $\delta = \delta_{X,n} = \delta_{U,n}$, and $R = R_{X,n} = R_{U,n}$
    (see lemma~\ref{lemma:regulator-dense-open-subset}). We note that
    $|H^{-1} (X_\et, \ZZ^c (n))_\tors| = |H^{-1} (U_\et, \ZZ^c (n))_\tors|$.
    On the other hand, the exact sequence of finite groups
    \[ 0 \to H^0 (X_\et, \ZZ^c (n)) \to
      H^0 (U_\et, \ZZ^c (n)) \to
      H^1 (Z_\et, \ZZ^c (n)) \to
      H^1 (X_\et, \ZZ^c (n)) \to
      H^1 (U_\et, \ZZ^c (n)) \to 0 \]
    gives
    \[ \frac{|H^0 (X_\et, \ZZ^c (n))|}{|H^1 (X_\et, \ZZ^c (n))|} =
      \frac{|H^0 (U_\et, \ZZ^c (n))|}{|H^1 (U_\et, \ZZ^c (n))|}\cdot
      \frac{1}{|H^1 (Z_\et, \ZZ^c (n))|}. \]
    From this we see that \eqref{eqn:special-value-for-X} and
    \eqref{eqn:special-value-for-U} are equivalent.
  \end{proof}
\end{lemma}

Now lemmas~\ref{lemma:elementary-proof-1} and \ref{lemma:elementary-proof-2}
above, together with dévissage lemma~\ref{lemma:devissage} prove
theorem~\ref{main-theorem} stated in the introduction.

\begin{remark}
  Note that $\zeta (\Spec \FF_q, n) = \frac{1}{1 - q^{-n}} < 0$, and therefore
  removing $m$ points from $X$ changes the sign of $\zeta^* (X,n)$ by
  $(-1)^m$. It is not difficult to figure out the sign in any specific example;
  however, it is not so clear in which terms one may write the general
  expression for the sign.
\end{remark}

%%%%%%%%%%%%%%%%%%%%%%%%%%%%%%%%%%%%%%%%%%%%%%%%%%%%%%%%%%%%%%%%%%%%%%%%%%%%%%%%

\section{A couple of examples}
\label{sec:examples}

We finish with two examples that illustrate how the localization arguments
work. The first will be rather generic, and consists in specifying the previous
section to the case of a nonmaximal order in a number field.

\begin{example}
  Let $\mathcal{O} \subset \mathcal{O}_F$ be a nonmaximal order in a number
  field $F/\QQ$. Denote $X = \Spec \mathcal{O}$ and
  $X' = \Spec \mathcal{O}_F$. Geometrically, $\nu\colon X' \to X$ is the
  normalization. There exist open dense subsets $U \subset X$ and
  $U' \subset X'$ such that $\nu$ induces an isomorphism $U' \cong U$. If we
  denote the corresponding closed complements by $Z = X\setminus U$ and
  $Z' = X'\setminus U'$, then we have
  $$\zeta_\mathcal{O} (s) = \frac{\zeta (Z,s)}{\zeta (Z',s)}\,\zeta_F (s).$$
  For this identity formulated in classical terms of algebraic number theory,
  see for instance \cite{Jenner-1969}. In particular,
  \[ \zeta^*_\mathcal{O} (n) =
    \pm \frac{|H^1 (Z'_\et, \ZZ^c(n))|}{|H^1 (Z_\et, \ZZ^c(n))|}\,\zeta^*_F (n). \]

  Now our special value conjectures for $\zeta^*_\mathcal{O} (n)$ and
  $\zeta^*_F (n)$ take form
  \begin{align}
    \label{eqn:special-value-nonmax-order}
    \zeta^*_\mathcal{O} (n) & \stackrel{?}{=}
                              \pm 2^\delta\,\frac{|H^0 (X_\et, \ZZ^c (n)|}{|H^{-1} (X_\et, \ZZ^c (n))_\tors|\cdot |H^1 (X_\et, \ZZ^c (n)|}\,R, \\
    \label{eqn:special-value-max-order}
    \zeta^*_F (n) & \stackrel{?}{=}
                    \pm 2^\delta\,\frac{|H^0 (X'_\et, \ZZ^c (n)|}{|H^{-1} (X'_\et, \ZZ^c (n))_\tors|\cdot |H^1 (X'_\et, \ZZ^c (n)|}\,R.
  \end{align}
  Here
  $|H^{-1} (X_\et, \ZZ^c (n))_\tors| = |H^{-1} (X'_\et, \ZZ^c (n))_\tors|$, and
  the exact sequences of finite groups
  \[ \begin{tikzcd}[row sep=0pt, column sep=1em]
      0 \ar{r} & H^0 (X_\et, \ZZ^c (n)) \ar{r} & H^0 (U_\et, \ZZ^c (n)) \ar{r} & H^1 (Z_\et, \ZZ^c (n)) \ar{r} & H^1 (X_\et, \ZZ^c (n)) \ar{r} & H^1 (U_\et, \ZZ^c (n)) \ar{r} & 0 \\
      0 \ar{r} & H^0 (X'_\et, \ZZ^c (n)) \ar{r} & H^0 (U'_\et, \ZZ^c (n)) \ar{r} & H^1 (Z'_\et, \ZZ^c (n)) \ar{r} & H^1 (X'_\et, \ZZ^c (n)) \ar{r} & H^1 (U'_\et, \ZZ^c (n)) \ar{r} & 0
    \end{tikzcd} \]
  give us
  \[ \frac{|H^1 (Z'_\et, \ZZ^c (n))|}{|H^1 (Z_\et, \ZZ^c (n))|} =
    \frac{|H^1 (X'_\et, \ZZ^c (n))|}{|H^1 (X_\et, \ZZ^c (n))|}\cdot
    \frac{|H^0 (X_\et, \ZZ^c (n))|}{|H^0 (X'_\et, \ZZ^c (n))|}, \]
  which implies that the formulas \eqref{eqn:special-value-nonmax-order} and
  \eqref{eqn:special-value-max-order} are equivalent.
\end{example}

The second example is suggested by \cite[\S 7]{Jordan-Poonen-2020}.

\begin{example}
  Let $p$ be an odd prime. Consider the affine scheme
  \[ X = \Spec (\ZZ [1/2] \times_{\FF_p} \FF_p [t]) =
    \Spec \ZZ [1/2] \mathop{\sqcup}_{\Spec \FF_p} \AA^1_{\FF_p}, \]
  which is obtained from $\Spec \ZZ [1/p]$ and $\AA^1_{\FF_p} = \Spec \FF_p [t]$ by
  gluing together the points corresponding to prime ideals
  $(p) \subset \ZZ [1/2]$ and $(t) \subset \FF_p [t]$:
  \[ \ZZ [1/2] \times_{\FF_p} \FF_p [t] =
    \{ (a,f) \in \ZZ [1/2] \times \FF_p [t] \mid a \equiv f (0) \pmod{p} \}. \]

  If we take odd $n < 0$, then there's no regulator. We consider $n = -3$.

  First we recall some calculations of motivic cohomology of $\Spec \ZZ$.
  According to \cite[Proposition~2.1]{Kolster-Sands-2008}, there is a short
  exact sequence
  $$0 \to K_6 (\ZZ) \to H^2 (\Spec \ZZ_\et, \ZZ (4)) \to \ZZ/2\ZZ \to 0$$
  and an isomorphism
  $H^1 (\Spec \ZZ_\et, \ZZ (4)) \cong K_7 (\ZZ)$.
  We have $K_6 (\ZZ) = 0$ and $K_7 (\ZZ) \cong \ZZ/240\ZZ$;
  see for instance Weibel's survey \cite{Weibel-2005}\footnote{Strictly
    speaking, this reasoning is \emph{backward}: as explained in
    \cite{Weibel-2005}, in fact one calculates motivic cohomology of
    $\Spec \mathcal{O}_F$ in order to obtain the $K$-theory of $\mathcal{O}_F$,
    not vice versa. Our explanation just serves to reduce everything to the
    well-known tables of $K$-groups of $\ZZ$.}.  We conclude that
  \begin{align*}
    H^{-1} (\Spec \ZZ_\et, \ZZ^c (-3)) & \cong H^1 (\Spec \ZZ_\et, \ZZ (4)) \cong \ZZ/240\ZZ, \\
    H^0 (\Spec \ZZ_\et, \ZZ^c (-3)) & \cong H^2 (\Spec \ZZ_\et, \ZZ (4)) \cong \ZZ/2\ZZ, \\
    H^1 (\Spec \ZZ_\et, \ZZ^c (-3)) & \cong H^3 (\Spec \ZZ_\et, \ZZ (4)) = 0.
  \end{align*}
  We note that, as expected,
  \[ \zeta (\Spec \ZZ, -3) = \zeta (-3) = -\frac{B_4}{4} = \frac{1}{120} =
    \frac{|H^0 (\Spec \ZZ_\et, \ZZ^c (-3))|}{|H^{-1} (\Spec \ZZ_\et, \ZZ^c (-3))|}. \]

  Localization gives
  \begin{align*}
    H^{-1} (\Spec \ZZ [1/2]_\et, \ZZ^c (-3)) & \cong H^{-1} (\Spec \ZZ_\et, \ZZ^c (-3)) \cong \ZZ/240\ZZ, \\
    H^0 (\Spec \ZZ [1/2]_\et, \ZZ^c (-3)) & \cong \ZZ/2\ZZ \oplus \ZZ/7\ZZ, \\
    H^1 (\Spec \ZZ [1/2]_\et, \ZZ^c (-3)) & = H^1 (\Spec \ZZ_\et, \ZZ^c (-3)) = 0.
  \end{align*}
  Arithmetically, this corresponds to the fact that the zeta function of
  $\Spec \ZZ [1/2]$ has the same Euler product as $\zeta (s)$, with the factor
  $\frac{1}{1-2^{-s}}$ removed. Therefore, the value at $s = -3$ should be
  corrected by $2^3 - 1 = 7$.

  Now for $\AA^1_{\FF_p}$, we have
  \[ H^i (\AA^1_{\FF_p, \et}, \ZZ^c (n)) \cong
    H^{i+2} (\Spec \FF_{p,\et}, \ZZ^c (n-1)) \cong
    \begin{cases}
      \ZZ/(p^{1-n} - 1)\ZZ, & i = -1, \\
      0, & i \ne -1.
    \end{cases} \]
  In particular, the motivic cohomology of $\AA^1_{\FF_p}$ is concentrated in
  \[ H^{-1} (\AA^1_{\FF_p, \et}, \ZZ^c (-3)) \cong \ZZ/(p^4-1)\ZZ. \]

  Consider the normalization of $X$, which is given by
  $X' = \Spec \ZZ [1/2] \sqcup \AA^1_{\FF_p}$:
  \[ \begin{tikzcd}
      Z' \ar[right hook->]{r}\ar{d} & X' \ar{d} \\
      Z \ar[right hook->]{r} & X
    \end{tikzcd} \]
  Here $Z = \{ \mathfrak{p} \}$, $Z' = \{ \mathfrak{P}, \mathfrak{P}' \}$,
  and
  \begin{align*}
    \mathfrak{p} & \dfn \{ (a,f) \in \ZZ [1/2] \times \FF_p [t] \mid a \equiv f(0) \equiv 0 \pmod{p} \}, \\
    \mathfrak{P} & \dfn \{ (a,f) \in \ZZ [1/2] \times \FF_p [t] \mid a \equiv 0 \pmod{p} \}, \\
    \mathfrak{P}' & \dfn \{ (a,f) \in \ZZ [1/2] \times \FF_p [t] \mid f(0) \equiv 0 \pmod{p} \}.
  \end{align*}
  The canonical morphism $X' \to X$ induces an isomorphism
  \[ X'\setminus Z' \cong X\setminus Z \cong
    (\Spec \ZZ \setminus \{ (2), (p) \}) \sqcup
    (\Spec \FF_p [t] \setminus (t)). \]

  We may calculate via localizations that
  \begin{align*}
    H^{-1} (X_\et, \ZZ^c (-3)) & \cong H^{-1} ((X\setminus Z)_\et, \ZZ^c (-3)) \cong
                                 \ZZ/240\ZZ \oplus \ZZ/(p^4 - 1)\ZZ, \\
    H^0 (X_\et, \ZZ^c (-3)) & \cong \ZZ/2\ZZ \oplus \ZZ/7\ZZ \oplus \ZZ/(p^3 - 1)\ZZ, \\
    H^1 (X_\et, \ZZ^c (-3)) & = 0.
  \end{align*}
  Consequently,
  \[ \frac{|H^0 (X_\et, \ZZ^c (-3))|}{|H^{-1} (X_\et, \ZZ^c (-3))|\cdot |H^1 (X_\et, \ZZ^c (-3))|} =
    \frac{7}{120}\,\frac{p^3-1}{p^4 - 1}. \]

  On the level of zeta-functions,
  \begin{multline*}
    \zeta (X,s) =
    \zeta (Z,s)\,\zeta (X\setminus Z,s) =
    \frac{\zeta (Z,s)}{\zeta (Z',s)}\,\zeta (X',s) =
    \frac{1}{\zeta (\Spec \FF_p,s)}\,\zeta (\Spec \ZZ[1/2],s)\,\zeta (\AA^1_{\FF_p},s) = \\
    (1-p^{-s})\,(1 - 2^{-s})\,\zeta (s)\,\frac{1}{1 - p^{1-s}}.
  \end{multline*}

  In particular, substituting $s = -3$, we obtain
  $\zeta (X,-3) = -\frac{7}{120}\,\frac{p^3 - 1}{p^4 - 1}$.
\end{example}

%%%%%%%%%%%%%%%%%%%%%%%%%%%%%%%%%%%%%%%%%%%%%%%%%%%%%%%%%%%%%%%%%%%%%%%%%%%%%%%%

\pagebreak
\bibliographystyle{amsalpha-cust}
\bibliography{weil-etale}

\end{document}
